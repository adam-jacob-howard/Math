\chapter{Framed Immersions of the Torus}\label{CH:F}

The main result of this chapter identifies the continuous group of \bit{framed diffeomorphisms}, and the continuous monoid of \bit{framed local-diffeomorphisms}, of a framed torus.


We state and contextualize this result right away as Theorem~\ref{Theorem A} and direct a reader to the body of the chapter for definitions and proofs.  





\begin{conventions*}
{\small
\begin{itemize}
\item[~]

\item
We work in the $\infty$-category $\Spaces$ of spaces, or $\infty$-groupoids, an object in which is a \bit{space}.  
This $\infty$-category can be presented as the $\infty$-categorical localization of the ordinary category of compactly-generated Hausdorff topological spaces that are homotopy-equivalent with a CW complex, localized on the weak homotopy-equivalences.  
So we present some objects in $\Spaces$ by naming a topological space.  

\item
By a pullback square among spaces we mean a pullback square in the $\infty$-category $\Spaces$.
Should the square be presented by a homotopy-commutative square among topological spaces, then the canonical map from the initial term in the square to the homotopy-pullback is a weak homotopy-equivalence.  

\item
By a \bit{continuous group} (resp. \bit{continuous monoid}) we mean a group-object (resp. monoid-object) in $\Spaces$.
A continuous monoid $N$ determines a pointed $(\infty,1)$-category $\fB N$, which can be presented by the Segal space $\bDelta^{\op}\xra{\bBar_\bullet(N)}\Spaces$ which is the bar construction of $N$.
For $X \in \cX$ an object in an $\infty$-category, and for $N$ a continuous monoid, 
an \bit{action of $N$ on $X$}, denoted $N \lacts X$, is an extension
$
\lag X \rag 
\colon 
\ast 
\to 
\fB N
\xra{ \lag N \lacts X \rag}
\cX
$.
The $\infty$-category of \bit{(left) $N$-modules in $\cX$} is
\[
\Mod_N(\cX)
~:=~
\Fun( \fB N , \cX)
~.
\]

\item
For $G\lacts X$ an action of a continuous group on a space, the space of \bit{coinvariants} is the colimit
\[
X_{/G}
~:=~
\colim\bigl(
\sB G
\xra{~\lag G \lacts X \rag~}
\Spaces
\bigr)
~\in~
\Spaces
~.
\]
Should the action $G\lacts X$ be presented by a continuous action of a topological group on a topological space, then this space of coinvariants can be presented by the homotopy-coinvariants.  




\item
We work with $\infty$-operads, as developed in~\cite{HA}.
As so, they are implicitly symmetric.
Some $\infty$-operads are presented as discrete operads, such as ${\sf Assoc}$, while some are presented as topological operads, such as the little 2-disks operad $\cE_2$.




\end{itemize}

}



\end{conventions*}






Here we state our first result, which identifies the entire symmetries of a framed torus.  



The \bit{braid group on 3 strands} can be presented as
\begin{equation}
\label{e67}
\Braid 
~\cong~
\Bigl \lag~ \tau_{1}~,~ \tau_{2}~ \mid ~ \tau_{1}\tau_{2}\tau_{1} ~=~ \tau_{2} \tau_{1} \tau_{2} ~\Bigr\rag
~.
\end{equation}
Through this presentation, there is a standard representation
\begin{equation}
\label{e63}
\Phi
\colon
\Braid
\xra{\bigl\lag~\tau_{1} ~\mapsto ~U_1~,~\tau_{2} ~ \mapsto ~U_2  \bigr\rag}
\GL_2(\ZZ)
~,
\hspace{7pt}
\text{ where }
U_1
~:=~
\begin{bmatrix} 
1 & 1
\\
0 & 1
\end{bmatrix}
\hspace{7pt}
\text{and}
\hspace{7pt}
U_2~:=~
\begin{bmatrix} 
1 & 0
\\
-1 & 1
\end{bmatrix}
~.
\end{equation} 
The homomorphism $\Phi$ defines an action $\Braid \xra{\Phi} \GL_2(\ZZ) \lacts \TT^2$
as a topological group.  
This action defines a topological group:
\[
\TT^2 \rtimes \Braid 
~.
\]



The following result, which is essentially due to Milnor, is the starting point of this paper.  
\begin{prop}
[see~\S10 of \cite{mil}]
\label{t32}
The image of $\Phi$ is the subgroup $\SL_2(\ZZ)$; the kernel of $\Phi$ is central, and is freely generated by the element $(\tau_{1}\tau_{2})^6  \in \Braid$.
In other words, $\Phi$ fits into a central extension among groups:
\begin{equation}
\label{e46}
1
\longrightarrow
\ZZ
\xra{~\bigl\lag (\tau_{1}\tau_{2})^6 \bigr\rag~}
\Braid
\xra{~\Phi~}
\SL_2(\ZZ)
\longrightarrow  
1
~.
\end{equation}
Furthermore, this central extension~(\ref{e46}) is classified by the element
\[
\Bigl[{\sf BSL}_2(\ZZ) 
{\xra{\sB \bigl( \RR\underset{\ZZ}\ot \bigr)}} {\sf BSL}_2(\RR) \simeq \sB^2 \ZZ
\Bigr] 
~\in~ 
\sH^2\bigl( \SL_2(\ZZ) ; \ZZ \bigr)
~,
\]
which is to say there is a canonical top horizontal homomorphism making a pullback among groups:
\[
\xymatrix{
\Braid
\ar@{-->}[rr]
\ar[d]_-{\Phi}
&&
\w{\SL}_2(\RR)  \ar[d]^-{\rm universal~cover}
\\
\SL_2(\ZZ)
\ar[rr]_-{\rm standard}^-{\RR\underset{\ZZ}\ot}
&&
\SL_2(\RR)
.
}
\]


\end{prop}






Consider the subgroup $\GL^+_2(\RR)\subset \GL_2(\RR)$ consisting of those $2\times 2$ matrices with positive determinant -- it is the connected component of the identity matrix.  
Consider the submonoid
\[
\RR\underset{\ZZ}\ot
\colon
\EpZ
~\subset~
\GL^+_2(\RR)
\]
consisting of those $2\times 2$ matrices with positive determinant whose entries are integers.
Consider the pullback among monoids:
\begin{equation}
\label{e77}
\xymatrix{
\Ebraid
\ar[d]_-{\Psi} \ar[rr]
&&
\w{\GL}^+_2(\RR)  \ar[d]^-{\rm universal~cover}
\\
\EpZ
\ar[rr]^-{\RR \underset{\ZZ}\ot}
&&
\GL^+_2(\RR)
.
}
\end{equation}
This morphism $\Psi$ supplies a canonical action $\Ebraid \xra{\Psi} \EpZ \lacts \TT^2$ as a topological group.
This action defines a topological monoid
\[
\TT^2 \rtimes \Ebraid
~.
\]
\begin{convention}
\label{r6}
By way of~\S\ref{ambidex}, in particular Corollary~\ref{t61}, we regard all actions of $\Braid$ and $\Ebraid$ as \bit{left}-actions.  
\end{convention}





\begin{notation}
\label{d6}
Let $\vec{x} = \mbox{\scriptsize$\begin{bmatrix} u \\ v \end{bmatrix}$} \in \ZZ^2$ and $r\in \NN$.
Denote the matrices
\[
A_{\vec{x}} ~:=~
\begin{bmatrix}
1+cd
&
d^2
\\
-c^2
&
1-cd
\end{bmatrix}
\qquad
\text{ and }
\qquad
D_{\vec{x},r} ~:=~
\begin{bmatrix}
1 + (r-1) bc
& 
-(r-1)bd
\\
(r-1)ac
&
1+(r-1)ad
\end{bmatrix}
~,
\]
for some $a,b,c,d\in \ZZ$ that solve
\begin{eqnarray}
\label{q12}
au+bv
&
=
&
{\sf gcd}(u,v) \geq 0
\\
\nonumber
cu+dv
&
=
&
0
\\
\nonumber
ad-bc
&
=
&
1
~.
\end{eqnarray}
Denote the semi-direct continuous group, and continuous monoid, 
\[
\TT^2 
\underset{A_{\vec{x}}}
\rtimes 
\ZZ
\qquad
\text{ and }
\qquad
\TT^2 
\underset{D_{\vec{x}}, A_{\vec{x}}}
\rtimes 
(\NN^\times \ltimes \ZZ)
\]
given through the actions on the continuous group $\TT^2$:
\[
\ZZ 
\xra{b \mapsto {A_{\vec{x}}^b} }
\SL_2(\ZZ)
\lacts
\TT^2
\qquad
\text{ and }
\qquad
\NN^\times \ltimes \ZZ 
\xra{(d,b) \mapsto {D_{\vec{x},d}} {A_{\vec{x}}^b} }
\EZ
\lacts 
\TT^2
~.
\]


\end{notation}


\begin{remark}
\label{q13}
Observation~\ref{q11} ensures the existence of a solution to~(\ref{q12}).
Observation~\ref{q11} also implies, for $A'_{\vec{x}}$ and $D'_{\vec{x},r}$ defined by another choice of solution to~(\ref{q12}), then $A'_{\vec{x}}$ and $D'_{\vec{x},r}$ are respectively canonically conjugate with $A_{\vec{x}}$ and $D_{\vec{x},r}$, and therefore the continuous groups and continuous monoids are respectively canonically identified:
\[
\TT^2 
\underset{A_{\vec{x}}}
\rtimes 
\ZZ
~\simeq~
\TT^2 
\underset{A'_{\vec{x}}}
\rtimes 
\ZZ
\qquad
\text{ and }
\qquad
\TT^2 
\underset{D_{\vec{x}}, A_{\vec{x}}}
\rtimes 
(\NN^\times \ltimes \ZZ)
~\simeq~
\TT^2 
\underset{D'_{\vec{x}}, A'_{\vec{x}}}
\rtimes 
(\NN^\times \ltimes \ZZ)
~.
\]
\end{remark}







For $\varphi \colon  \tau_{\TT^2} \cong \epsilon^2_{\TT^2}$ a framing of the torus, we introduce as Definition~\ref{d1} the continuous group of \bit{framed diffeomorphisms}, and the continuous monoid of \bit{framed local-diffeomorphisms} of the torus:
\[
\Diff^{\fr}(\TT^2,\varphi)
\qquad
\text{ and }
\qquad
\Imm^{\fr}(\TT^2,\varphi)
~.
\]
For $\varphi_0$ the \bit{standard framing} of $\TT^2$, we simply write 
\[
\Diff^{\fr}(\TT^2)
~:=~
\Diff^{\fr}(\TT^2,\varphi_0)
\qquad
\text{ and }
\qquad
\Imm^{\fr}(\TT^2)
~:=~
\Imm^{\fr}(\TT^2,\varphi_0)
~.
\]






\begin{mythm}{X}
\label{Theorem A}


\begin{enumerate}

\item[~]

\item
The map from the set of homotopy-classes of framings of $\TT^2$ to the set of framed-diffeomorphism-types of tori,
\[
\pi_0 \Fr(\TT^2)
\longrightarrow
\pi_0 \bcM^{\fr}_1
~,
\]
is canonically identified as the map
\[
\ZZ^2 
\times
\ZZ_{/2\ZZ}
\xra{~\pr~}
\ZZ_{\geq 0}
~,\qquad
\left(
~
\mbox{\scriptsize$\begin{bmatrix}
u \\ v
\end{bmatrix}$}
~,~
\sigma
~
\right)
\longmapsto
{\sf gcd}(u,v)
~.
\]
In particular, each framing $\varphi\colon \tau_{\TT^2} \cong \epsilon_{\TT^2}$ of the torus determines an element $\vec{\varphi} \in \ZZ^2$.
A framing $\varphi$ is homotopic to a translation-invariant framing if and only if $\vec{\varphi} = \vec{0}$.




\item
Let $\varphi\colon \tau_{\TT^2} \cong \epsilon_{\TT^2} $ be a framing of the torus.

\begin{enumerate}

\item
There is a canonical identification of the continuous group of framed diffeomorphisms of $(\TT^2,\varphi)$
\[
\Diff^{\fr}(\TT^2,\varphi)
~\simeq~
\begin{cases}
\TT^2 \rtimes \Braid
&
,~
\text{ if } \vec{\varphi} = \vec{0}
\\
( \TT^2 \underset{A_{\vec{\varphi}}}\rtimes \ZZ ) \times \ZZ
&
,~
\text{ if } \vec{\varphi} \neq \vec{0}
\end{cases}
~.
\]
The set of framed-diffeomorphism-types of tori is in canonical bijection with $\ZZ_{\geq 0}$.








\item
There is a canonical identification of the continuous monoid of framed local-diffeomorphisms of $(\TT^2,\varphi)$:
\[
\Imm^{\fr}(\TT^2,\varphi)
~\simeq~
\begin{cases}
\TT^2 \rtimes \Ebraid
&
,~
\text{ if } \vec{\varphi} = \vec{0}
\\
\left( \TT^2 \underset{D_{\vec{\varphi}} , A_{\vec{\varphi}}} \rtimes (\NN^\times \ltimes \ZZ ) \right) \times \ZZ
&
,~
\text{ if } \vec{\varphi} \neq \vec{0}
\end{cases}
~,
\]


\end{enumerate}





\end{enumerate}



\end{mythm}





Taking path-components, Theorem~\ref{Theorem A}(2a) has the following immediate consequence.
\begin{cor}
\label{t38}
Let $\varphi$ be a framing of the torus.
There is a canonical identification of the framed mapping class group of $(\TT^2,\varphi)$ as a subgroup of the braid group on 3 strands:
\[
{\sf MCG}^{\fr}(\TT^2,\varphi)
~\subset~
\Braid
~.
\]
If $\varphi$ is homotopic with a translation-invariant framing, this subgroup is entire.
If $\varphi$ is not homotopic with a translation-invariant framing, this subgroup is conjugate with a standard subgroup,
\[
{\sf MCG}^{\fr}(\TT^2,\varphi)
~\underset{\rm conjugate}\cong~
\bigl\lag \tau_1 , (\tau_1 \tau_2)^6 \bigr\rag
~\cong~
\ZZ \times \ZZ
~,
\]
which is abstractly isomorphic with $\ZZ\times \ZZ$.

\end{cor}



\begin{remark}
Consider the moduli space $\bcM^{\fr}_1$ of framed tori.  
Theorem~\ref{Theorem A}(1) \& (2a) can be phrased as the assertion that $\bcM^{\fr}_1$ has $\ZZ_{\geq 0}$-many path-components, with the $0$-path-component the space of homotopy-coinvariants ${(\CC\PP^\infty)^2}_{/ \Braid}$ with respect to the action $\Braid \xra{\Phi}\GL_2(\ZZ) \lacts \sB^2 \ZZ^2 \simeq (\CC\PP^\infty)^{\times 2}$, and each other path-component the space ${(\CC\PP^\infty)^2}_{/\ZZ} \times \sB \ZZ$ in which the coinvariants are with respect to the action $\ZZ \xra{\lag U_1\rag }\GL_2(\ZZ) \lacts \sB^2 \ZZ^2 \simeq (\CC\PP^\infty)^{\times 2}.$
A neat result of Milnor (see~\S10 of~\cite{mil}) gives an isomorphism between groups: 
\[
\Braid
~\cong~ 
\pi_1 (\SS^3 \smallsetminus {\sf Trefoil} )
~.
\]
Using that $\SS^3 \smallsetminus {\sf Trefoil}$ is a path-connected 1-type, this isomorphism reveals that, the $0$-path-component $(\bcM^{\fr}_1)_{0} \subset \bcM^{\fr}_1$ fits into a fiber sequence of spaces:
\[
(\CC\PP^\infty)^2
\longrightarrow
(\bcM^{\fr}_1)_{0}
\longrightarrow
\bigl( \SS^3 \smallsetminus {\sf Trefoil} \bigr)
~.
\]


\end{remark}





%\begin{remark}
%Theorem~\ref{Theorem A}(2a) implies the equivalence-type of the continuous group $\Diff^{\fr}(\TT^2,\varphi)$ is independent of the framing $\varphi$ of the torus.  
%
%\end{remark}



A generalization of Smale's conjecture, proved by Hatcher (see \cite{sm} and~\cite{hatcher}) implies the standard inclusion is an equivalence between continuous groups:
\[
\TT^3 \rtimes \GL_3(\ZZ)
\xra{~\simeq~}
\Diff(\TT^3)
~.
\]
In particular, there is an identification of the mapping class group,
$
{\sf MCG}(\TT^3)
\cong
\GL_3(\ZZ)
$.
We expect our methods could be used to prove the following.  
\begin{conj}
Let $\varphi_0$ be a translation-invariant framing of $\TT^3$.
There is a canonical identification between continuous groups:
\[
\Diff^{\fr}(\TT^3,\varphi_0)
~\simeq~
\Bigl(
\TT^3
\rtimes 
\Omega \bigl( \SL_3(\RR)_{/\SL_3(\ZZ)} \bigr)
\Bigr) 
\times
\Bigl(
\Omega^2 \SS^3 \times \Omega^3 \SS^3 
\Bigr)^3
\times
\Omega^4 \SS^3
~,
\]
in which the semi-direct product is with respect to the action $\Omega \bigl( \SL_3(\RR)_{/\SL_3(\ZZ)} \bigr) \xra{\rm Puppe} \SL_3(\ZZ) \lacts \TT^3$.
In particular, there is a central extension among groups:
\[
1
\longrightarrow
\ZZ^3 \times {\ZZ_{/2\ZZ}}^2
\longrightarrow
{\sf MCG}^{\fr}(\TT^3,\varphi_0) 
\longrightarrow
\SL_3(\ZZ)
\longrightarrow
1
~.
\]

\end{conj}





In~\S6 of~\cite{Dehn}, 
Dehn
identifies the oriented mapping class group of a punctured torus with parametrized boundary as the braid group on 3 strands, as it is equipped with a homomorphism to the oriented mapping class group of the torus.  
Through Corollary~\ref{t38}, this results in an identification between these mapping class groups.  
The next result lifts this identification to continuous groups; it is proved in a later section.  




\begin{cor}
\label{t40}
Fix a smooth framed embedding from the closed 2-disk $\DD^2 \hookrightarrow \TT^2$ extending the inclusion $\{0\} \hookrightarrow \TT^2$ of the identity element.
There are canonical identifications among continuous groups over $\Diff(\TT^2)$:
\[
\Diff^{\fr}(\TT^2~{\sf rel}~0)
~
\simeq
~
\Braid
~\simeq~
\Diff(\TT^2 ~{\sf rel}~ \DD^2)
~.
\]
In particular, there are canonical isomorphisms among groups over ${\sf MCG}(\TT^2)$:
\[
{\sf MCG}^{\fr}(\TT^2)
~\cong~
\Braid
~\cong~
{\sf MCG}(\TT^2 \smallsetminus \BB^2 ~{\sf rel}~ \partial)
~,
\]
where $\BB^2 \subset \DD^2$ is the open 2-ball. 


\end{cor}

























































\section{Moduli and Isogeny of Framed Tori}




Vector addition, as well as the standard vector norm, gives $\RR^2$ the structure of a topological abelian group.
Consider its closed subgroup $\ZZ^2\subset \RR^2$.  
The \bit{torus} is the quotient in the short exact sequence of topological abelian groups:
\[
0
\longrightarrow
\ZZ^2
\xra{\rm inclusion}
\RR^2 
\xra{~\quot~}
\TT^2
\longrightarrow
0
~.
\] 
Because $\RR^2$ is connected, and because $\ZZ^2$ acts cocompactly by translations on $\RR^2$, the torus $\TT^2$ is connected and compact.  
The quotient map $\RR^2 \xra{\quot} \TT^2$ endows the torus with the structure of a Lie group, and in particular a smooth manifold.
Consider the submonoid
\begin{equation*}
\EZ
~:=~
\Bigl\{
\ZZ^2 \xra{A} \ZZ^2 \mid {\sf det}(A) \neq 0
\Bigr\}
~\subset~
\End_{\sf Groups}(\ZZ^2)
~,
\end{equation*}
consisting of the cofinite endomorphisms of the group $\ZZ^2$.  
Using that the smooth map $\RR^2 \xra{\quot} \TT^2$ is a covering space and $\TT^2$ is connected, there is a canonical continuous action on the topological group:
\begin{equation}
\label{e2}
\EZ
~\lacts~
\TT^2
~,\qquad
A \cdot q
~:=~
\quot( A\w{q} )
\qquad
{ \bigl(\text{for any }\w{q} \in \quot^{-1}(q) \bigr) }
~.
\footnote{
Note that (\ref{e2}) indeed does not depend on $\w{q} \in \quot^{-1}(q)$.
}
\end{equation}
This homomorphism~(\ref{e2}) defines a semi-direct product topological monoid:
\[
\TT^2 \rtimes \EZ
~.
\]


Consider the topological monoid of smooth local-diffeomorphisms of the torus:
\[
\Imm(\TT^2)
~\subset~
\Map(\TT^2 , \TT^2)
~,
\] 
which is endowed with the subspace topology of the $\sC^\infty$-topology on the set of smooth self-maps of the torus.

\begin{observation}
\label{t21}
\begin{enumerate}

\item[~]

\item
The standard inclusion $\GL_2(\ZZ) \hookrightarrow \EZ$ witnesses the maximal subgroup.
It follows that the standard inclusion $ \TT^2 \rtimes \GL_2(\ZZ)  \hookrightarrow \TT^2 \rtimes \EZ$ witnesses the maximal subgroup, both as topological monoids and as continuous monoids.


\item
The standard monomorphism $\Diff(\TT^2) \hookrightarrow \Imm(\TT^2)$ witnesses the maximal subgroup, both as topological monoids and as continuous monoids.

\end{enumerate}
\end{observation}






Consider the morphism between topological monoids:
\begin{equation}
\label{e7}
\Aff: \TT^2 \rtimes \EZ 
\longrightarrow
\Imm(\TT^2)
~,\qquad
(p, A)
\mapsto
\Bigl(
q\mapsto 
Aq + p
\Bigr)
~.
\end{equation}


We record the following classical result.
\begin{lemma}\label{t1}
The restriction of the morphism~(\ref{e7}) to maximal subgroups is a homotopy-equivalence:
\[
\Aff \colon
\TT^2 \rtimes \GL_2(\ZZ)
\xra{~\simeq~}
\Diff(\TT^2)
~,\qquad
(p, A)
\mapsto
\Bigl(
q\mapsto 
Aq + p
\Bigr)
~.
\]
\end{lemma}


\begin{proof}
Let $G$ be a locally path-connected topological group, which we regard as a continuous group.
Denote by $G_{\uno} \subset G$ the path-component containing the identity element in $G$.
This subspace $G_{\uno}\subset G$ is a normal subgroup, and the sequence of continuous homomorphisms
\[
1
\longrightarrow
G_{\uno}
\xra{~\rm inclusion~}
G
\xra{~\rm quotient~}
\pi_0(G)
\longrightarrow
1
\]
is a fiber-sequence among continuous groups.
This fiber sequence is evidently functorial in the argument $G$.
In particular, there is a commutative diagram among topological groups,
\[
\xymatrix{
1 \ar[d]_-= \ar[r]
&
\TT^{2} = \bigl( \TT^{2} \rtimes \GL_2(\ZZ) \bigr)_{\uno}
\ar[d]_-{\Aff_{\uno}}
\ar[r]^-{\rm inc}
&
\TT^{2} \rtimes \GL_2(\ZZ)  
\ar[d]_-{\Aff} 
\ar[r]^-{\rm quot}
&
\pi_0 \bigl( \TT^{2} \rtimes  \GL_2(\ZZ) \bigr)
=
\GL_2(\ZZ)
\ar[d]_-{\pi_0(\Aff)}
\ar[r]
&
1 \ar[d]_-=
\\
1
\ar[r]
&
\Diff(\TT^2)_{\uno}
\ar[r]^-{\rm inc}
&
\Diff(\TT^2) 
\ar[r]^-{\rm quot}
&
\pi_0 \bigl( \Diff(\TT^2) \bigr)
\ar[r]
&
1
,
}
\] 
in which the horizontal sequences are fiber sequences.  
By the 5-lemma applied to homotopy groups, we are reduced to showing the vertical homomorphisms $\Aff_{\uno}$ and $\pi_0(\Aff)$ are homotopy equivalences.



Theorem 2.D.4 of~\cite{rolf}, along with Theorem B of \cite{TT}, implies $\pi_0(\Aff)$ is an isomorphism.  
So it remains to show $\Aff_{\uno}$ is a homotopy equivalence. 
With respect to the canonical continuous action $\Diff(\TT^2)_{\uno} \lacts \TT^2$, 
the orbit of the identity element $0\in \TT^2$ is the evaluation map 
\[
\ev_0
\colon 
\Diff(\TT^2)_{\uno}
\longrightarrow
\TT^2
~.
\]
Note that the composition,
\[
\id
\colon
\TT^2
\xra{~\Aff_{\uno}~}
\Diff(\TT^2)_{\uno}
\xra{~\ev_0~}
\TT^2
~,
\]
is the identity map.
So it remains to show that the homotopy-fiber of $\ev_0$ is weakly-contractible.  
The isotopy-extension theorem implies $\ev_0$ is a Serre fibration.  
So it is sufficient to show the fiber of $\ev_0$, which is the stabilizer
${\sf Stab}_0\bigl(\Diff(\TT^2)_{\uno}\bigr)$,
is weakly-contractible.  
Finally, Theorem 1b of~\cite{ee} states that this stabilizer is contractible.  

\end{proof}






\begin{remark}
\label{r10}
By the classification of compact surfaces, the moduli space $\bcM_1$ of smooth tori is path-connected, and as so is 
\[
\bcM_1
~\simeq~
{\sf BDiff}(\TT^2)
\underset{\rm Lem~\ref{t1}}{~\simeq~}
\sB\bigl(
\TT^2 \rtimes \GL_2(\ZZ)
\bigr)
~.
\]
In particular, this path-connected moduli space fits into a fiber sequence
\[
(\CC\PP^\infty)^2
\longrightarrow
\bcM_1
\longrightarrow
{\sf BGL}_2(\ZZ)
~,
\]
which is classified by the standard action $\GL_2(\ZZ) \lacts \sB^2 \ZZ^2 \simeq (\CC\PP^\infty)^2$.

\end{remark}





Consider the set of \bit{cofinite subgroups} of $\ZZ^2$:
\begin{equation*}
\bcL(2) 
~:=~
\Bigl\{
\Lambda
\underset{\rm cofin}
\subset
\ZZ^2
\Bigr\}
~.
\end{equation*}
\begin{observation} \label{t99}

\begin{enumerate}

\item[~]

\item
The orbit-stabilizer theorem immediately implies the composite map 
$
\TT^2 \rtimes \EZ
\xra{\pr}
\EZ
\xra{\rm Image} 
\bcL(2)
$
witnesses the quotient:
\[
\bigl(
\TT^2 \rtimes \EZ
\bigr)_{/\TT^2 \rtimes \GL_2(\ZZ)}
\xra{~\cong~}
\EZ_{/\GL_2(\ZZ)}
\xra{~\cong~}
\bcL(2)
~.
\]


\item
Using that each finite-sheeted cover over $\TT^2$ is diffeomorphic with $\TT^2$, the classification of covering spaces implies the map given by taking the image of homology
$
\Imm(\TT^2)
\xra{{\sf Image}(\sH_1)}
\bcL(2)
$
witnesses the quotient:
\[
\Imm(\TT^2)_{/\Diff(\TT^2)}
\xra{~\cong~}
\bcL(2)
~.
\]


\item
The diagram
\[
\xymatrix{
\TT^2 \rtimes \EZ 
\ar[rr]^-{\Aff}
\ar[d]_-{\pr}
&&
\Imm(\TT^2) 
\ar[d]^-{{\rm Image}(\sH_1)}
\ar[dll]_-{\sH_1}
\\
\EZ
\ar[rr]^-{\rm Image}
&&
\bcL(2)
}
\]
commutes.

\end{enumerate}


\end{observation}




\begin{cor}
\label{t22}
The morphism (\ref{e7}) between topological monoids is a homotopy-equivalence:
\[
\Aff
\colon
\TT^2 \rtimes \EZ
\xra{~\simeq~}
\Imm(\TT^2)
~.
\]

\end{cor}

\begin{proof}

Consider the morphism between fiber sequences in the $\infty$-category $\Spaces$:
\[
\xymatrix{
 \TT^2 \rtimes\EZ
\ar[rr]^-{\rm quotient}
\ar[d]_-{\Aff}
&&
\bigl(
\TT^2 \rtimes \EZ
\bigr)_{\TT^2 \rtimes \GL_2(\ZZ)}
\ar[rr]
\ar[d]^-{\Aff_{\Aff}}
&&
\sB\bigl( \TT^2 \rtimes \GL_2(\ZZ) \bigr)
\ar[d]^-{\sB \Aff}
\\
\Imm(\TT^2)
\ar[rr]^-{\rm quotient}
&&
\Imm(\TT^2)_{/\Diff(\TT^2)}
\ar[rr]
&&
\sB \Diff(\TT^2)
.
}
\]
Lemma~\ref{t1} implies the right vertical map is an equivalence.
Observation~\ref{t99} implies the middle vertical map is an equivalence.
It follows that the left vertical map is an equivalence, as desired.



\end{proof}








\section{Framings}



A \bit{framing} of the torus is a trivialization of its tangent bundle: $\varphi\colon  \tau_{\TT^2} \cong  \epsilon^2_{\TT^2}$.
Consider the topological \bit{space of framings} of the torus:
\[
\Fr(\TT^2)
~:=~
\Iso_{\Bdl_{\TT^2}}\bigl( \tau_{\TT^2} , \epsilon^2_{\TT^2}  \bigr)
~\subset~
\Map(  \sT \TT^2  , \TT^2 \times \RR^2 )
~,
\]
which is endowed with the subspace topology of the $\sC^\infty$-topology on the set of smooth maps between total spaces.  
The quotient map $\RR^2\xra{\quot}\TT^2$ endows the smooth manifold $\TT^2$ with a \bit{standard framing} $\varphi_0$:
for 
\[
\trans\colon \TT^2 \times \TT^2 
\xra{~(p,q)\mapsto \trans_p(q) := p+q~} 
\TT^2
\]
the abelian multiplication rule of the Lie group $\TT^2$, 
\[
(\varphi_0)^{-1}
\colon
\epsilon^2_{\TT^2}
\xra{~\cong~}
\tau_{\TT^2}
~,\qquad
\TT^2 \times \RR^2 \ni (p,v)
\mapsto
\bigl(p,\sD_{0}(\trans_p \circ \quot) (v) \bigr)
\in \sT \TT^2
~.
\]




The next sequence of observations culminates as an identification of this space of framings.
\begin{observation}
\label{t20}


\begin{enumerate}
\item[~]


\item
Postcomposition gives the topological space $\Fr(\TT^2)$ the structure of a torsor for the topological group $\Iso_{\Bdl_{\TT^2}}\bigl( \epsilon^2_{\TT^2} , \epsilon^2_{\TT^2} \bigr)$.
In particular, the orbit map of a framing $\varphi \in \Fr(\TT^2)$ is a homeomorphism:
\begin{equation}
\label{e41}
\Iso_{\Bdl_{\TT^2}}\bigl( \epsilon^2_{\TT^2} , \epsilon^2_{\TT^2} \bigr)
\xra{~\cong~}
\Fr(\TT^2)
~,\qquad
\alpha \mapsto 
\alpha \circ \varphi 
~.
\end{equation}


\item
Consider the topological space $\Map \bigl( \TT^2 , \GL_2(\RR) \bigr)$ of smooth maps from the torus to the standard smooth structure on $\GL_2(\RR)$, which is endowed with the $\sC^\infty$-topology. 
The map
\begin{equation}
\label{e43}
\Map \bigl( \TT^2 , \GL_2(\RR) \bigr)
\xra{~\cong~}
\Iso_{\Bdl_{\TT^2}}\bigl( \epsilon^2_{\TT^2} , \epsilon^2_{\TT^2} \bigr)
~,\qquad
a
\mapsto 
\Bigl(
\TT^2\times \RR^2
\xra{\bigl( p,v \bigr) \mapsto \bigl( p,a_p(v) \bigr) }
\TT^2 \times \RR^2
\Bigr)
~,
\end{equation}
is a homeomorphism.

\item
The map to the product with based maps,
\begin{equation}
\label{e44}
\Map \bigl( \TT^2 , \GL_2(\RR) \bigr)
\xra{~\cong~}
\Map\Bigl( ( 0\in \TT^2) , ( \uno \in \GL_2(\RR) \bigr) \Bigr)
\times
\GL_2(\RR),
\end{equation}
\[
a
\mapsto 
\bigl(
~
a(0)^{-1} a~ ,~ a(0)
~
\bigr)
~,
\]
is a homeomorphism.


\item
Because both of the spaces $\TT^2$ and $\GL_2(\RR)$ are 1-types with the former path-connected, 
the map,
\[
\pi_1
\colon
\Map\Bigl( ( 0\in \TT^2) , ( \uno \in \GL_2(\RR) \bigr) \Bigr)
\xra{~\simeq~}
{\sf Homo}\Bigl( \pi_1\bigl( 0 \in \TT^2 \bigr) , \pi_1\bigl( \uno \in \GL_2(\RR) \bigr) \Bigr) 
~,
\]
is a homotopy-equivalence.

\item
Evaluation on the standard basis for $\pi_1(0\in \TT^2) \xra{\cong} \pi_1(0\in \TT)^2 \cong \ZZ^2$ defines a homeomorphism:
\begin{equation}
\label{e45}
{\sf Homo}\Bigl( \pi_1\bigl( 0 \in \TT^2 \bigr) , \pi_1\bigl( \uno \in \GL_2(\RR) \bigr) \Bigr) 
\xra{~\cong~}
\pi_1\bigl( \uno \in \GL_2(\RR)^2 \bigr)
~\cong~
\ZZ^2
~.
\end{equation}

\end{enumerate}

\end{observation}




Observation~\ref{t20}, together with the Gram-Schmidt homotopy-equivalence ${\sf GS}\colon\sO(2) \xra{\simeq} \GL_2(\RR)$, yields the following.
\begin{cor}
\label{t25}
A choice of framing $\varphi \in \Fr(\TT^2)$ determines a composite homotopy-equivalence:
\begin{eqnarray*}
\Fr(\TT^2) 
&
\underset{\simeq}{
\xla{~(\ref{e43})\circ (\ref{e41})~}
}
&
\Map\bigl( \TT^2 , \GL_2(\RR) \bigr)
\\
\nonumber
&
\underset{\simeq}{
\xra{~(\ref{e44})~}
}
&
\Map \Bigl( \bigl( 0 \in \TT^2 \bigr) , \bigl( \uno \in \GL_2(\RR) \bigr) \Bigr) \times \GL_2(\RR) 
\\
\nonumber
&
\underset{\simeq}{
\xra{~\pi_1 \times \id ~}
}
&
{\sf Homo}\Bigl( \pi_1\bigl( 0 \in \TT^2 \bigr) , \pi_1\bigl( \uno \in \GL_2(\RR) \bigr) \Bigr) \times \GL_2(\RR) 
\\
\nonumber
&
\underset{\simeq}{
\xra{(\ref{e45}) \times \id }
}
&
\ZZ^2 \times \GL_2(\RR)
\\
\nonumber
&
\underset{\simeq}{
\xla{ \id \times {\sf GS}}
}
&
\ZZ^2 \times \sO(2)
~.
\end{eqnarray*}

\end{cor}




















\section{Moduli of Framed Tori}


Consider the map:
\[
\Act\colon
\Fr(\TT^2)
\times
\Imm(\TT^2)
\longrightarrow
\Fr(\TT^2)
~,
\]
\[
( \varphi , f )
\mapsto 
\Bigl(
~
\tau_{\TT^2} 
\underset{\cong}{\xra{\sD f}}
f^\ast \tau_{\TT^2}
\underset{\cong}{ \xra{f^\ast \varphi} }
f^\ast \epsilon^2_{\TT^2}
=
\epsilon^2_{\TT^2}
~
\Bigr)
~.
\]

\begin{lemma}
\label{t50}
The map $\Act$ is a continuous right-action of the topological monoid $\Imm(\TT^2)$ on the topological space $\Fr(\TT^2)$.
In particular, there is a continuous action of the topological group $\Diff(\TT^2)$ on the topological space $\Fr(\TT^2)$. 

\end{lemma}

\begin{proof}
Consider the topological subspace of the topological space of smooth maps between total spaces of tangent bundles, which is endowed with the $\sC^\infty$-topology,
\[
{\sf Bdl}^{\sf fw.iso}(\tau_{\TT^2} , \tau_{\TT^2})
~\subset~
\Map\bigl( \sT \TT^2 , \sT \TT^2 \bigr)
~,
\]
consisting of the smooth maps between tangent bundles that are fiberwise isomorphisms.
Notice the factorization
\[
\Act\colon
\Fr(\TT^2)
\times
\Imm(\TT^2)
\xra{\id \times \sD}
\Fr(\TT^2)
\times 
{\sf Bdl}^{\sf fw.iso}(\tau_{\TT^2} , \tau_{\TT^2})
\xra{\circ}
\Fr(\TT^2)
\]
as first taking the derivative, followed by composition of bundle morphisms.  
The definition of the $\sC^\infty$-topology is so that the first map in this factorization is continuous. 
The second map in this factorization is continuous because composition is continuous with respect to $\sC^\infty$-topologies.
We conclude that $\Act$ is continuous.  

We now show that $\Act$ is an action.
Clearly, for each $\varphi \in \Fr(\TT^2)$, there is an equality $\Act( \varphi  , \id) = \varphi$.
Next, let $g,f\in \Imm(\TT^2)$, and let $\varphi \in \Fr(\TT^2)$.
The chain rule, together with universal properties for pullbacks, gives that the diagram among smooth vector bundles

\[
\xymatrix{
\tau_{\TT^2}
\ar@(u,u)[rrrrrr]^-{\sD (g\circ f)}
\ar[rr]_-{\sD g}
&&
g^\ast 
\tau_{\TT^2}
\ar[rr]_-{g^\ast \sD f}
&&
f^\ast g^\ast \tau_{\TT^2}
\ar[d]^-{f^\ast g^\ast \varphi}
\ar[rr]_-=
&&
(g\circ f)^\ast \tau_{\TT^2}
\ar[d]^-{(g \circ f)^\ast \varphi}
\\
\epsilon^2_{\TT^2}
&&
g^\ast \epsilon^2_{\TT^2}
\ar[ll]_-=
&&
f^\ast g^\ast \epsilon^2_{\TT^2}
\ar[ll]_-=
&&
(g \circ f)^\ast \epsilon^2_{\TT^2}
\ar[ll]_-=
\ar@(d,d)[llllll]^=
}
\]
commutes. 
Inspecting the definition of $\Act$, the commutativity of this diagram implies the equality $\Act\bigl( \Act(\varphi, g) , f \bigr) = \Act( \varphi , g\circ f)$, as desired.  



\end{proof}




\begin{definition}
\label{r8}
The \bit{moduli space of framed tori} is the space of homotopy-coinvariants with respect to this conjugation action $\Act$:
\[
\bcM_1^{\fr}
~:=~
\Fr(\TT^2)_{/\Diff(\TT^2)}
~.
\]


\end{definition}





\begin{observation}\label{t4}
Through Corollary~\ref{t25} applied to the standard framing $\varphi_0 \in \Fr(\TT^2)$, the action $\Act$ is compatible with familiar actions.
Specifically, $\Act$ fits into a commutative diagram among topological spaces:
\[
\xymatrix@C=1em{
\Fr(\TT^2)
\times
\Imm(\TT^2)
\ar[rrr]^-{\Act}
&
&&
\Fr(\TT^2)
\\
\Map\bigl( \TT^2 , \GL_2(\RR) \bigr)
\times
\bigl(
\TT^2 \rtimes \EZ
\bigr)
\ar[u]^-{ {\rm Cor}~\ref{t25} \times \Aff}_-{\simeq}
\ar[r]^-{ \id \times \pr }
\ar[d]_-{ {\rm Cor}~\ref{t25} \times \id}^-{\simeq}
&
\Map\bigl( \TT^2 , \GL_2(\RR) \bigr)
\times
\EZ 
\ar[rr]^-{\rm value-wise}_-{\rm multiply}
\ar[d]^-{{\rm Cor}~\ref{t25} \times \id}_-{\simeq}
&&
\Map\bigl( \TT^2 , \GL_2(\RR) \bigr)
\ar[d]^-{{\rm Cor}~\ref{t25}}_-{\simeq}
\ar[u]_-{{\rm Cor}~\ref{t25}}^-{\cong}
\\
\bigl( \ZZ^2 \times \GL_2(\RR) \bigr)
\times
\bigl(
\TT^2 \rtimes \EZ
\bigr)
\ar[r]^-{ \id \times \pr }
&
\bigl( \ZZ^2\times  \GL_2(\RR) \bigr)
\times
\EZ 
\ar[rr]^-{(\vec{x},A;B)\mapsto (B^T \vec{x},AB)}
&&
\ZZ^2 \times \GL_2(\RR) 
.
}
\]


\end{observation}









We record the following basic application of group theory.
\begin{observation}
\label{q11}
Let $\vec{x} = \mbox{\scriptsize$\begin{bmatrix} u \\ v \end{bmatrix} $}\in \ZZ^2$.
Consider the subset \[T_{\vec{x}} := \left\{ P \in \GL_2(\ZZ) \mid P \vec{x}= {\sf gcd}(u,v) \cdot \vec{e}_1 \right\}\subset \GL_2(\ZZ)~.\]
\begin{enumerate}

\item
In the case that $u\geq 0$ and $v=0$, the set $T_{\vec{x}}$ is identical with the stabilizer subgroup:
\[
T_{\vec{x}}
=
{\sf Stab}_{\GL_2}(\ZZ)( {\sf gcd}(u,v) \cdot \vec{e}_1 )
=
\]
\[
\begin{cases}
\GL_2(\ZZ)
&
~,
\text{ if } u=0
\\
\left\{ \mbox{\scriptsize$ \begin{bmatrix} 1 & b \\ 0 & d \end{bmatrix} $}\right\}
=
\left\lag  
\mbox{\scriptsize$
\begin{bmatrix}
1
&
0
\\
0
&
-1
\end{bmatrix} $}
,

\mbox{\scriptsize$\begin{bmatrix}
1
&
1
\\
0
&
1
\end{bmatrix} $}
\right\rag
\cong
\sO(1) \ltimes \ZZ
&
~,
\text{ if } u>0
\end{cases}
~,
\]
the latter case which is isomorphic with a semi-direct product of $\sO(1)$ and $\ZZ$ with respect to the standard action $\sO(1) \xra{\cong} \Aut(\ZZ)$.




\item
The set $T_{\vec{x}}$ is not empty.
Left multiplication defines a free transitive action of this stabilizer subgroup:
\[
\GL_2(\ZZ)
\lacts
T_{\vec{x}}
\qquad 
\text{for $\vec{x}= \vec{0}$}
~,
\qquad
\text{ and }
\qquad
\sO(1) \ltimes \ZZ
\lacts
T_{\vec{x}}
\qquad 
\text{for $\vec{x}\neq \vec{0}$}
~.
\]

\item
An element $P \in T_{\vec{x}}$ determines an isomorphism between groups:
\[
{\sf Stab}_{\GL_2}(\ZZ)(\vec{x}) 
=
P^{-1}
{\sf Stab}_{\GL_2}(\ZZ)( {\sf gcd}(u,v) \cdot \vec{e}_1 )
P
\]
\[
=
\begin{cases}
\GL_2(\ZZ)
&
~,
\text{ if } \vec{x}= \vec{0}
\\
\left\lag  
P^{-1}
\mbox{\scriptsize$\begin{bmatrix}
1
&
0
\\
0
&
-1
\end{bmatrix} $}
P
,
P^{-1}
\mbox{\scriptsize$\begin{bmatrix}
1
&
1
\\
0
&
1
\end{bmatrix} $}
P
\right\rag
\cong
\sO(1) \ltimes \ZZ
&
~,
\text{ if } \vec{x}\neq \vec{0}
\end{cases}
~.
\]


\item
An element $P=\mbox{\scriptsize$\begin{bmatrix} a & b \\ c & d \end{bmatrix}$} \in T_{\vec{x}} \cap \SL_2(\ZZ)$ determines an isomorphism
\[
{\sf Stab}_{\SL_2(\ZZ)}(\vec{x}) 
=
\begin{cases}
\SL_2(\ZZ)
&
~,
\text{ if } \vec{x} = \vec{0}
\\
\left\lag  
\mbox{\scriptsize$\begin{bmatrix}
1+cd
&
d^2
\\
-c^2
&
1-cd
\end{bmatrix} $}
\right\rag
=
\lag P^{-1} U_1 P \rag
\cong
\ZZ
&
~,
\text{ if } \vec{x} \neq \vec{0}
\end{cases}
~.
\]


\end{enumerate}


\end{observation}










The next result is phrased in terms of spaces fitting into the diagram in which each square is a pullback:
\begin{equation}
\label{q10}
\xymatrix{
{(\CC\PP^\infty)^2}_{/\ZZ
}
\times
\sB\ZZ
\ar[rr]
\ar[d]
&&
{(\CC\PP^\infty)^2}_{/\Braid
}
\ar[rr]
\ar[d]
&&
{(\CC\PP^\infty)^2}_{/\GL_2(\ZZ)}
\ar[dd]
\\
\sB \ZZ
\times
\sB \ZZ
\ar[rr]^-{\lag \tau_1 , (\tau_1\tau_2)^6 \rag}
\ar[d]_-{\pr}
&&
\sB \Braid
\ar[d]^-{\Phi}
&&
\\
\sB \ZZ
\ar[rr]^-{\lag U_1\rag}
&&
\sB \SL_2(\ZZ)
\ar[rr]
&&
\sB \GL_2(\ZZ)
.
}
\end{equation}


\begin{prop}
\label{t26}
The standard framing $\varphi_0 \in \Fr(\TT^2)$ determines an identification between spaces:
\[
\bcM_1^{\fr} 
\xra{~\simeq~}
\Bigl(
{(\CC\PP^\infty)^2}_{/ \Braid
}
\Bigr)
~\coprod~
\Bigl(
{(\CC\PP^\infty)^2}_{/ \ZZ
}
\times 
\sB \ZZ
\Bigr)^{ \amalg \NN}
~,
\]
through which $\varphi_0$ selects the distinguished path-component. 
Furthermore, the resulting map $\pi_0 \Fr(\TT^2) \to \pi_0 \bcM_1^{\fr} \xra{\cong} \{0\} \amalg \NN = \ZZ_{\geq 0}$ factors as a composition:
\[
\xymatrix{
\pi_0 \Fr(\TT^2) 
\ar[rr]^-{\cong} 
\ar[dr]
&&
\ZZ_{\geq 0}
\\
&
\ZZ^2
\ar[ur]_-{\sf gcd}
&
,
}
\]
in which the second map takes the \bit{greatest common divisor}, and the first map is
\[
[\varphi]
\mapsto
\Bigl[
\TT \vee \TT = \sk_1(\TT^2)
\xra{ \varphi_0^{-1} \varphi _{|\sk_1(\TT^2)} }
\GL_2(\RR) 
\Bigr]
\in \pi_1\bigl(\uno \in \GL_2(\RR)\bigr)^2 \cong \ZZ^2
~.
\]



\end{prop}




\begin{proof}


The result follows upon explaining the following sequences of identifications in the $\infty$-category $\Spaces$:
\begin{eqnarray}
\nonumber
%\label{q1}
\bcM_1^{\fr} 
&
\underset{\rm Obs~\ref{t4}}
{~\simeq~}
&
\Bigl(
\ZZ^2
\times
\GL_2(\RR) 
\Bigr)_{/\TT^2 \rtimes \GL_2(\ZZ)}
\\
\label{q2}
&
\underset{\rm iterate~quotient}
{~\simeq~}
&
\Bigl(
\bigl(
\ZZ^2
\times
\GL_2(\RR) 
\bigr)_{/\TT^2}
\Bigr)_{/\GL_2(\ZZ)}
\\
\label{q3}
&
\underset{\rm trivial~\TT^2~action}
{~\simeq~}
&
\Bigl(
\ZZ^2
\times
\sB \TT^2
\times
\GL_2(\RR) 
\Bigr)_{/\GL_2(\ZZ)}
\\
\label{q4}
&
\underset{\rm groupoids~are~effective}
{~\simeq~}
&
{\ZZ^2} _{/\GL_2(\ZZ)}
\underset{ \sB \GL_2(\ZZ) }
\times
\bigl(
(\CC\PP^\infty)^2\times \GL_2(\RR) 
\bigr)_{/\GL_2(\ZZ)}
\\
\label{q5}
&
\underset{\rm explicit~quotient}
{~\simeq~}
&
\Bigl(
\sB \GL_2(\ZZ) \amalg \sB (\sO(1) \ltimes \ZZ)^{\amalg \NN}
\Bigr)
\underset{ \sB \GL_2(\ZZ) }
\times
\bigl(
(\CC\PP^\infty)^2\times \GL_2(\RR) 
\bigr)_{/\GL_2(\ZZ)}
\\
\label{q6}
&
\underset{\rm distribute~\times~over~\amalg}
{~\simeq~}
&
\left(
\sB \GL_2(\ZZ)
\underset{ \sB \GL_2(\ZZ) }
\times
\bigl(
(\CC\PP^\infty)^2\times \GL_2(\RR) 
\bigr)_{/\GL_2(\ZZ)}
\right)
\\
\nonumber
&
&
\coprod
\left(
\sB (\sO(1) \ltimes \ZZ)
\underset{ \sB \GL_2(\ZZ) }
\times
\bigl(
(\CC\PP^\infty)^2\times \GL_2(\RR) 
\bigr)_{/\GL_2(\ZZ)}
\right)^{\amalg \NN}
\\
\label{q7}
&
\underset{\rm base-change}
{~\simeq~}
&
\left(
(\CC\PP^\infty)^2\times \GL_2(\RR) 
\bigr)_{/\GL_2(\ZZ)}
\right)
\coprod
\left(
(\CC\PP^\infty)^2\times \GL_2(\RR) 
\bigr)_{/\sO(1) \ltimes \ZZ}
\right)^{\amalg \NN}
\\
\label{q8}
&
\underset{\rm Lem~\ref{t66}}
{~\simeq~}
&
\left(
{
(\CC\PP^\infty)^2
}_{/\Omega \bigl(\GL_2(\RR)_{/\GL_2(\ZZ)} \bigr)}
\right)
\coprod
\left(
{
(\CC\PP^\infty)^2
}_{/\Omega \bigl(\GL_2(\RR)_{/\sO(1)\ltimes \ZZ} \bigr)}
\right)^{\amalg \NN}
\\
\label{q9}
&
\underset{\rm explicit~identifications}
{~\simeq~}
&
\left(
{
(\CC\PP^\infty)^2
}_{/\Braid}
\right)
\coprod
\left(
{
(\CC\PP^\infty)^2
}_{/\ZZ}
\times 
\sB \ZZ
\right)^{\amalg \NN}
~.
\end{eqnarray}
The first identification follows from Observation~\ref{t4}.
The bottom horizontal map in Observation~\ref{t4} reveals that the action $ \TT^2 \rtimes \GL_2(\ZZ) \lacts \ZZ^2 \times \GL_2(\RR)$ can be identified as the diagonal action of the action
\[
\TT^2 \rtimes \GL_2(\ZZ) \xra{~\pr~} \GL_2(\ZZ) \xra{~\rm include~} \underset{\rm left~mult}\lacts \GL_2(\RR)
\]
together with the action
\[
\TT^2 \rtimes \GL_2(\ZZ) \xra{~\pr~} \GL_2(\ZZ) \xra{~\rm include~} \underset{\rm standard}\lacts \ZZ^2
~.
\]
The equivalence~(\ref{q2}) identifies the $\TT^2 \rtimes \GL_2(\ZZ)$-quotient as the $\TT^2$-quotient followed by the $\GL_2(\ZZ)$-quotient.
The equivalence~(\ref{q3})  is a consequence of the $\TT^2$-action being trivial on both factors.
The equivalence~(\ref{q4}) is an instance of the general base-change identity $(X\times Y)_{/G} \simeq (X_{/G}) \underset{\sB G}\times (Y_{/G})$.  
The equivalence~(\ref{q5}) is the orbit-stabilizer theorem, as we now explain.
By Observation~\ref{q11}, two elements $\mbox{\scriptsize$\begin{bmatrix} u \\ v \end{bmatrix} $}, \mbox{\scriptsize$\begin{bmatrix} s \\ t \end{bmatrix} $}\in \ZZ^2$ are in the same $\GL_2(\ZZ)$-orbit if and only if their greatest common divisors ${\sf gcd}(u,v) = {\sf gcd}(s,t)\in \ZZ_{\geq 0}$ agree.
In particular, there is a bijection between the set of orbits and the subset
\[
\ZZ_{\geq 0}
~\cong~
\left\{
\mbox{\scriptsize$\begin{bmatrix} g \\ 0 \end{bmatrix}$} \right\}
~\subset~
\ZZ^2
~.
\]
Furthermore, the stabilizer of $g \in \ZZ_{\geq 0}$ is
\[
{\sf Stab}_{\GL_2(\ZZ)}\left(\mbox{\scriptsize$ \begin{bmatrix} g \\ 0 \end{bmatrix}$} \right)
~=~
\begin{cases}
\GL_2(\ZZ)
&
~,
\text{ if }g = 0 
\\
\left\{
\mbox{\scriptsize$\begin{bmatrix}
1 & b
\\
0 & d
\end{bmatrix}$}
\right\}
\cong
\sO(1) \ltimes \ZZ
&
~,
\text{ if }g \neq 0 
\end{cases}
~.
\]
Therefore, the quotient
\[
{
\ZZ^2
}_{/\GL_2(\ZZ)}
~\simeq~
\underset{g\in \ZZ_{\geq 0}}
\coprod
\sB {\sf Stab}_{\GL_2(\ZZ)}\left( \mbox{\scriptsize$\begin{bmatrix} g \\ 0 \end{bmatrix}$} \right)
~\simeq~
\sB \GL_2(\ZZ)
\coprod
\sB
\left(
\sO(1) \ltimes \ZZ
\right)^{ \amalg \NN}
~.
\]
The equivalence~(\ref{q6}) is the distribution of $\times$ over $\coprod$.  
The equivalence~(\ref{q7}) is an instance of the general base-change identity $X_{/H}\simeq \sB H \underset{\sB G} \times X_{/G}$.
The equivalence~(\ref{q8}) is an instance of Lemma~\ref{t66}.
The equivalence~(\ref{q9}) is a direct application of Proposition~\ref{t32} for the $0$-cofactor, 
and for each other cofactor it is an application of Proposition~\ref{t32} then a consequence of the diagram~(\ref{q10}) of pullbacks among spaces.



\end{proof}












For $\varphi\in \Fr(\TT^2)$ a framing of the torus, consider the orbit map of $\varphi$ for this continuous action of Lemma~\ref{t50}:
\[
{\sf Orbit}_\varphi
\colon
\Imm(\TT^2)
\xra{~( ~ {\sf constant}_{\varphi}~ , ~\id ~ )~}
\Fr(\TT^2)
\times 
\Imm(\TT^2)
\xra{~{\sf Act}~}
\Fr(\TT^2)
~,\qquad
f
\mapsto {\sf Act}(\varphi , f)
~.
\]







\begin{observation}
\label{t33}
After Observation~\ref{t4}, for each framing $\varphi \in \Fr(\TT^2)$, 
the orbit map for $\varphi$ fits into a solid diagram among topological spaces:
\begin{equation*}
\xymatrix@C=1em{
\Diff(\TT^2) 
\ar[rr]
\ar@{-->}[dr]^-{\sH_1}
&&
\Imm(\TT^2)
\ar[rrrr]^-{{\sf Orbit}_\varphi}
\ar@{-->}[dr]^-{\sH_1}
&&
&&
\Fr(\TT^2)
\ar[d]^-{{\rm Cor}~\ref{t25}}_-{\simeq}
\\
&
\GL_2(\ZZ) 
\ar[rr] 
&&
\EZ
\ar[rr]^-{ A \mapsto (A^T \vec{\varphi} , A) }
&&
\ZZ^2 \times \EZ
\ar[r]^-{\id \times \RR\underset{\ZZ}\ot }
&
\ZZ^2 \times \GL_2(\RR)
\\
\TT^2 \rtimes \GL_2(\ZZ)
\ar[uu]^-{\Aff}
\ar[rr]
\ar[ur]_-{\pr}
&&
\TT^2 \rtimes \EZ
\ar[uu]^(.35){\Aff} | \hole
\ar[ur]_-{\pr}
&&
,
}
\end{equation*}
where $\vec{\varphi} \in \ZZ^2$ is as in Theorem~\ref{Theorem A}(1).
The existence of the fillers follows from Observation~\ref{t99}.



\end{observation}






















\begin{remark}
\label{r5}
The point-set fiber of ${\sf Orbit}_{\varphi}$ over $\varphi$, which is the point-set stabilizer of the action $\Fr(\TT^2) \racts \Imm(\TT^2)$ of Lemma~\ref{t50}, 
consists of those local-diffeomorphisims $f$ for which the diagram among vector bundles, 
\[
\xymatrix{
\tau_{\TT^2}
\ar[rr]^-{\varphi}
\ar[d]_-{\sD f}
&&
\epsilon^2_{\TT^2}
\\
f^\ast 
\tau_{\TT^2}
\ar[rr]^-{f^\ast \varphi}
&&
f^\ast
\epsilon^2_{\TT^2}
\ar[u]_-=
,
}
\]
commutes.
For a generic framing $\varphi$, a local-diffeomorphism $f$ satisfies this rigid condition if and only if $f = \id_{\TT^2}$ is the identity diffeomorphism.
In the special case of the standard framing $\varphi_0$, a local-diffeomorphism $f$ satisfies this rigid condition if and only if $f = {\sf trans}_{f(0)} \circ {\sf quot}$ is translation in the group $\TT^2$ after a group-theoretic quotient $\TT^2 \xra{\rm quotient} \TT^2$. 
In particular, the point-set fiber of $\bigl( {\sf Orbit}_{\varphi_0} \bigr)_{|\Diff(\TT^2)}$ over $\varphi_0$ is $\TT^2$, and the homomorphism $\TT^2 \hookrightarrow \Diff(\TT^2)$ witnesses the inclusion of those diffeomorphisms that \emph{strictly} fix $\varphi_0$.  


On the other hand, the \emph{homotopy-}fiber of~${\sf Orbit}_{\varphi_0}$ over $\varphi_0$ is more flexible: 
it consists of pairs $(f, \gamma)$ in which $f$ is a local-diffeomorphism and $\gamma$ is a homotopy 
\[
\varphi_0
~ \underset{\gamma}\sim ~
\Act(\varphi_0,f)
~.
\]
As we will see, every orientation-preserving local-diffeomorphism $f$ admits a lift to this homotopy-fiber.  
In particular, small perturbations of such $f$, such as multiplication by bump functions in neighborhoods of $\TT^2$, can be lifted to this homotopy-fiber.
\end{remark}




\begin{definition}\label{d1}
Let $\varphi\in \Fr(\TT^2)$ be a framing of the torus.  
The space of \bit{framed local-diffeomorphisms}, and the space of \bit{framed diffeomorphisms}, of the framed smooth manifold $(\TT^2,\varphi)$ are respectively the pullbacks in the $\infty$-category $\Spaces$:
\begin{equation*}
\xymatrix{
\Imm^{\sf fr}(\TT^2,\varphi)
\ar[rr]
\ar[d]
&&
\Imm(\TT^2)
\ar[d]^-{{\sf Orbit}_\varphi}
&&
\Diff^{\sf fr}(\TT^2,\varphi)
\ar[rr]
\ar[d]
&&
\Diff(\TT^2)
\ar[d]^-{{\sf Orbit}_\varphi}
\\
\ast
\ar[rr]^-{\lag \varphi \rag}
&&
\Fr(\TT^2)
&
\text{ and }
&
\ast
\ar[rr]^-{\lag \varphi \rag}
&&
\Fr(\TT^2)
~.
}
\end{equation*}
In the case that the framing $\varphi = \varphi_0$ is the standard framing, we simply denote
\[
\Imm^{\sf fr}(\TT^2)
~:=~
\Imm^{\sf fr}(\TT^2,\varphi_0)
\qquad
\text{ and }
\qquad
\Diff^{\sf fr}(\TT^2)
~:=~
\Diff^{\sf fr}(\TT^2,\varphi_0)
~.
\]
\end{definition}





The following result follows directly from Lemma~\ref{t2} of Appendix \ref{sec.A}.
\begin{cor}\label{t3}
Let $\varphi \in \Fr(\TT^2)$ be a framing.
The space $\Diff^{\fr}(\TT^2 , \varphi)$ is canonically endowed with the structure of a continuous group over $\Diff(\TT^2)$.
With respect to this structure, 
there is a canonical identification between continuous groups:
\[
\Diff^{\fr}(\TT^2, \varphi)
~\simeq~
\Omega_{[\varphi]}  \bcM_1^{\fr} 
~.
\]


\end{cor}






\begin{observation}
\label{t43}
Let $\varphi \in \Fr(\TT^2)$ be a framing.
The kernel of $\Phi$ acts by rotating the framing, which is to say 
there is a canonically commutative diagram among continuous groups:
\[
\xymatrix{
\ZZ
\ar[d]^-{\cong}_-{\bigl\lag (\tau_1\tau_2)^6 \bigr\rag}
\ar[rr]^-{\simeq}
&&
\Omega_{\uno} \GL_2(\RR)
\ar[rr]^-{\Omega \bigl( A\mapsto A \cdot \varphi \bigr)}
&&
\Omega_{\varphi} \Fr(\TT^2)
\ar[d]
\\
\Ker(\Phi)
\ar[rr]
&&
\Braid
\ar[rr]^-{\Aff^{\fr}}
&&
\Diff^{\fr}(\TT^2, \varphi)
.
}
\]
Indeed, there is a canonically commutative diagram among spaces, in which each row is an $\Omega$-Puppe sequence:
\[
\xymatrix{
\Ker(\Phi)
\ar[rr]
\ar[d]
&&
\Braid
\ar[rr]^-{\Phi}
\ar[d]^-{\Aff^{\fr}}
&&
\GL_2(\ZZ)
\ar[rr]^-{\RR\underset{\ZZ}\ot}
\ar[d]^-{\Aff}
&&
\GL_2(\RR)
\ar[d]^-{\rm Rotate~the~framing~\varphi}
\\
\Omega_{\varphi} \Fr(\TT^2)
\ar[rr]
&&
\Diff^{\fr}(\TT^2,\varphi)
\ar[rr]
&&
\Diff(\TT^2)
\ar[rr]^-{{\sf Orbit}_{\varphi}}
&&
\Fr(\TT^2)
.
}
\]

\end{observation}

























































































\section{Proof of Theorem~\ref{Theorem A} and Corollary~\ref{t40}} \label{sec.proofs}
Theorem~\ref{Theorem A} consists of three statements. 
Theorem~\ref{Theorem A}(1) is implied by Proposition~\ref{t26}.
Theorem~\ref{Theorem A}(2a) is implied by Corollary~\ref{t3}.
Theorem~\ref{Theorem A}(2b) (as well as Theorem~\ref{Theorem A}(2a)) is implied by Lemma~\ref{t34} below.









Recall Notation~\ref{d6}, especially as it appears in Theorem~\ref{Theorem A}(1).
\begin{lemma}
\label{t34}
Let $\varphi\in \Fr(\TT^2)$ be a framing of the torus.
Consider the element $\vec{\varphi} \in \ZZ^2$ as in Theorem~\ref{Theorem A}(1).
\begin{enumerate}

\item
If $\vec{\varphi} = \vec{0}$, then there are canonical equivalences in the diagrams among continuous monoids:
\begin{equation}
\label{e54}
\xymatrix@C=1em{
\TT^2 \rtimes \Ebraid
\ar@{-->}[rr]^-{\simeq}_{\Aff^{\fr}}
\ar[d]_-{\id \rtimes \Psi}
&&
\Imm^{\fr}(\TT^2,\varphi) 
\ar[d]^-{\rm forget}
&&
\TT^2 \rtimes \Braid 
\ar@{-->}[rr]^-{\simeq}_{\Aff^{\fr}}
\ar[d]_-{\id \rtimes \Phi}
&&
\Diff^{\fr}(\TT^2,\varphi) 
\ar[d]^-{\rm forget}
\\
\TT^2 \rtimes \EZ
\ar[rr]^-{\simeq}_-\Aff
&&
\Imm(\TT^2)
&
\text{ and }
&
\TT^2 \rtimes  \GL_2(\ZZ)
\ar[rr]^-{\simeq}_-\Aff
&&
\Diff(\TT^2)
.
}
\end{equation}



\item
If $\vec{\varphi} \neq \vec{0}$, then there are canonical equivalences in the diagrams among continuous monoids:
\begin{equation}
%\label{e54}
\xymatrix@C=.7em{
\left( \TT^2 \underset{D_{\vec{\varphi}} , A_{\vec{\varphi}}} \rtimes (\NN^\times \ltimes \ZZ ) \right) \times \ZZ
\ar@{-->}[rr]^-{\simeq}_{\Aff^{\fr}}
\ar[d]^-{\id \rtimes \bigl( (d,b,k)\mapsto D_{\vec{\varphi},d} A_{\vec{\varphi}}^b \bigr)}
&&
\Imm^{\fr}(\TT^2,\varphi) 
\ar[d]^-{\rm forget}
&&
\TT^2 \underset{A_{\vec{\varphi}}} \rtimes \ZZ
\ar@{-->}[rr]^-{\simeq}_{\Aff^{\fr}}
\ar[d]^-{\id \rtimes \bigl( (b,k)\mapsto A_{\vec{\varphi}}^b \bigr)}
&&
\Diff^{\fr}(\TT^2,\varphi) 
\ar[d]^-{\rm forget}
\\
\TT^2 \rtimes \EZ
\ar[rr]^-{\simeq}_-\Aff
&&
\Imm(\TT^2)
&
\text{ and }
&
\TT^2 \rtimes  \GL_2(\ZZ)
\ar[rr]^-{\simeq}_-\Aff
&&
\Diff(\TT^2)
.
}
\end{equation}




\end{enumerate}


\end{lemma}



\begin{proof}
Using Observation~\ref{t21}, the canonical equivalences in the commutative diagrams on the right follow from those on the left.  


Consider the diagram in the $\infty$-category $\Spaces$.
\begin{enumerate}
\item
For $\vec{\varphi} = \vec{0}$:
\[
\xymatrix{
\TT^2 \rtimes\Ebraid
\ar[d]_-{\id \rtimes \Psi}
\ar[rr]^-{\pr}
&&
\Ebraid
\ar[d]_-{\Psi}
\ar[rrrr]^-{!}
&&
&&
\ast
\ar[d]^-{\lag (\vec{\varphi} , \varphi_{|0} \circ (\varphi_0)_{|0}^{-1})\rag}
\\
\TT^2 \rtimes \EZ 
\ar[d]_-{\Aff}^{\simeq}
\ar[rr]^-{\pr}
&&
\EZ
\ar[rr]^-{ A \mapsto (A^T \vec{\varphi} , A) }
&&
\ZZ^2 \times \EZ
\ar[rr]^-{\id \times \RR\underset{\ZZ}\ot }
&&
\ZZ^2 \times \GL_2(\RR)
\\
\Imm(\TT^2)
\ar[rrrrrr]^-{{\sf Orbit}_\varphi}
&&
&&
&&
\Fr(\TT^2)
\ar[u]_-{\rm Cor~\ref{t25}}^-{\simeq}
.
}
\]


\item
For $\vec{\varphi} \neq \vec{0}$:
\[
\xymatrix@C=.65em{
\left( \TT^2 \underset{D_{\vec{\varphi}} , A_{\vec{\varphi}}} \rtimes (\NN^\times \ltimes \ZZ ) \right) \times \ZZ
\ar[d]^-{\id \rtimes \bigl((d,b,k)\mapsto D_{\vec{\varphi},d} A_{\vec{\varphi}}^b\bigr)}
\ar[rr]^-{\pr}
&&
(\NN^\times \ltimes \ZZ ) \times \ZZ
\ar[d]^-{(d,b,k)\mapsto D_{\vec{\varphi},d} A_{\vec{\varphi}}^b}
\ar[rrrr]^-{!}
&&
&&
\ast
\ar[d]^-{\lag (\vec{\varphi} , \varphi_{|0} \circ (\varphi_0)_{|0}^{-1})\rag}
\\
\TT^2 \rtimes \EZ 
\ar[d]_-{\Aff}^{\simeq}
\ar[rr]^-{\pr}
&&
\EZ
\ar[rr]^-{ A \mapsto (A^T \vec{\varphi} , A) }
&&
\ZZ^2 \times \EZ
\ar[rr]^-{\id \times \RR\underset{\ZZ}\ot }
&&
\ZZ^2 \times \GL_2(\RR)
\\
\Imm(\TT^2)
\ar[rrrrrr]^-{{\sf Orbit}_\varphi}
&&
&&
&&
\Fr(\TT^2)
\ar[u]_-{\rm Cor~\ref{t25}}^-{\simeq}
.
}
\]

\end{enumerate}
Observation~\ref{t33} implies that each bottom rectangle canonically commutes.
Lemma~\ref{t1} and Corollary~\ref{t25} together imply each of these bottom rectangles witnesses a pullback.
Each of the top left squares is clearly a pullback.
Corollary~\ref{t31} states that each of the top middle squares is a pullback.
We conclude that each of the outer squares witnesses a pullback.  
The result follows by Definition~\ref{d1} of $\Imm^{\fr}(\TT^2,\varphi)$.



\end{proof}



















By applying the product-preserving functor $\Spaces \xra{\pi_0} {\sf Sets}$, Lemma~\ref{t34} implies the following.
\begin{cor}\label{r2}
There is a canonical isomorphism in the diagram of groups:
\[
\xymatrix{
\Braid
\ar@{-->}[rr]^-{\cong}
\ar[d]_-{\Phi}
&&
{\sf MCG}^{\sf fr}(\TT^2)
\ar[d]^-{\rm forget}
\\
\GL_2(\ZZ)
\ar[rr]^-{\cong}
&&
{\sf MCG}(\TT^2)
.
}
\]


\end{cor}



\begin{remark}\label{r1}
Proposition~\ref{t32} and Corollary~\ref{r2} grant a central extension among groups:
\[
1
\longrightarrow
\ZZ
\longrightarrow
{\sf MCG}^{\fr}(\TT^2)
\longrightarrow
{\sf MCG}^{\sf or}(\TT^2)
\longrightarrow
1
~.
\\
\]

\end{remark}







\begin{proof}[Proof of Corollary~\ref{t40}]
By construction, the diagram among spaces,
\[
\xymatrix{
\TT^2 \rtimes \EZ
\ar[rr]^-{\simeq}_-{\rm Cor~\ref{t22}}
\ar[dr]_-{\pr}
&&
\Imm(\TT^2)
\ar[dl]^-{\ev_0}
\\
&
\TT^2
&
,
}
\]
canonically commutes, in which the left vertical map is projection, and the right vertical map evaluates at the origin $0\in \TT^2$.  
Therefore, upon taking fibers over $0\in \TT^2$, the (left) commutative diagram~(\ref{e54}) among continuous monoids determines the commutative diagram among commutative monoids:
\begin{equation*}
\xymatrix{
\Ebraid
\ar[rrrr]^-{\simeq}
\ar[d]
&&
&&
\Imm^{\fr}(\TT^2 ~{\sf rel } ~0)
\ar[d]
\\
\EZ
\ar[rrrr]^-{\simeq}_-{\rm Cor~\ref{t22}}
\ar[drr]_-{\RR \underset{\ZZ}\ot}
&&
&&
\Imm(\TT^2 ~{\sf rel } ~0)
\ar[dll]^-{\sD_0}
\\
&&
\GL_2(\RR)
&&
,
}
\end{equation*}
in which the map $\RR \underset{\ZZ}\ot $ is the standard inclusion, and $\sD_0$ takes the derivative at the origin $0\in \TT^2$.
To finish, Corollary~\ref{t31} supplies the left pullback square in the following diagram among continuous groups, while the right pullback square is definitional:
\[
\xymatrix{
\Braid
\ar[rr]
\ar[d]
&&
\ast
\ar[d]
&&
\Diff(\TT^2 \smallsetminus \BB^2 ~{\sf rel}~\partial )
\ar[ll]
\ar[d]
\\
\GL_2(\ZZ)
\ar[rr]^-{\RR \underset{\ZZ}\ot}
&&
\GL_2(\RR)
&&
\Diff(\TT^2~{\sf rel}~0)
\ar[ll]_-{\sD_0}
.
}
\]
The result follows. 

\end{proof}




\section{Comparison with Sheering}
We use Theorem~\ref{Theorem A}(2) to show that the $\Diff^{\fr}(\TT^2)$ is generated by sheering.  
We quickly tour through some notions and results, which are routine after the above material.  

\begin{notation}
It will be convenient to define the projection $\TT^{2} \xra{\pr_{i}} \TT$ to be projection \emph{off} of the $i^{\rm th}$ coordinate. 
So for $\TT^{2} \ni p = (x_{p}, y_{p}),$ we have $\pr_{1}(p) = y_{p}$ and $\pr_{2}(p) = x_{p}.$
\end{notation}

Let $i\in \{1,2\}$.
Consider the topological subgroup and topological submonoid,
\[
\Diff(\TT^2 \xra{\pr_i} \TT)
~\subset~
\Diff(\TT^2)
\qquad
\text{ and }
\qquad
\Imm(\TT^2 \xra{\pr_i} \TT)
~\subset~
\Imm(\TT^2)
~,
\]
consisting of those (local-)diffeomorphisms $\TT^2\xra{f} \TT^2$ that lie over some (local-)diffeomorphism $\TT\xra{\ov{f}} \TT$:
\begin{equation}
\label{e87}
\xymatrix{
\TT^2
\ar[rr]^-f
\ar[d]_-{\pr_i}
&&
\TT^2 \ar[d]^-{\pr_i}
\\
\TT
\ar[rr]^-{\ov{f}}
&&
\TT
~.
}
\end{equation}
The topological space of \bit{framings} of $\TT^2 \xra{\pr_i} \TT$ is the subspace
\[
\Fr(\TT^2 \xra{\pr_i} \TT)
~\subset~
\Fr(\TT^2)
\]
consisting of those framings $\tau_{\TT^2}\xra{\varphi} \epsilon^2_{\TT^2}$ that lie over a framing $\tau_{\TT} \xra{\ov{\varphi}} \epsilon^1_{\TT}$:
\begin{equation}
\label{e70}
\xymatrix{
\tau_{\TT^2}
\ar[rr]^-{\varphi}_-{\cong}
\ar[d]_-{\sD \pr_i}
&&
\epsilon^2_{\TT^2}
\ar[d]^-{\pr_i \times \pr_i}
\\
\tau_{\TT}
\ar[rr]^-{\ov{\varphi}}_-{\cong}
&&
\epsilon^1_{\TT}
.
}
\end{equation}
Because $\pr_i$ is surjective, for a given $\varphi$, there is a unique $\ov{\varphi}$ as in~(\ref{e70}) if any.  
Better, $\varphi\mapsto \ov{\varphi}$ defines a continuous map:
\begin{equation}
\label{e82}
\Fr(\TT^2 \xra{\pr_i} \TT)
\longrightarrow
\Fr(\TT)
~,\qquad
\varphi
\mapsto 
\ov{\varphi}
~.
\end{equation}
Notice that the continuous right-action $\Act$ of Lemma~\ref{t50} evidently restricts as a continuous right-action:
\[
\Fr(\TT^2 \xra{\pr_i} \TT)
~\racts~
\Imm(\TT^2 \xra{\pr_i} \TT)
~.
\]
Furthermore, the map~(\ref{e82}) is evidently equivariant with respect to the morphism between topological monoids $\Imm(\TT^2 \xra{\pr_i} \TT) \xra{\rm forget} \Imm(\TT)$:
\[
\Bigl(
~
\Fr(\TT^2 \xra{\pr_i} \TT)
~\racts~
\Imm(\TT^2 \xra{\pr_i} \TT)
~\Bigr)
~
\xra{~\rm forget~}
~
\Bigl(
~
\Fr(\TT)
~\racts~
\Imm(\TT)
~\Bigr)
~,\qquad
\varphi
\mapsto 
\ov{\varphi}
~.
\]






Now let $\varphi \in \Fr(\TT^2 \xra{\pr_i} \TT)$ be a framing of the projection.
The orbit of $\varphi$ by this action is the map
\[
{\sf Orbit}_\varphi
\colon 
\Imm(\TT^2 \xra{\pr_i}\TT)
\longrightarrow
\Fr(\TT^2 \xra{\pr_i} \TT)
~,\qquad
f\mapsto \Act(\varphi,f)
~.
\]
The space of \bit{framed local-diffeomorphisms}, and the space of \bit{framed diffeomorphisms}, of $(\TT^2 \xra{\pr_i} \TT , \varphi)$ are respectively the homtopy-pullbacks among spaces:
\[
\xymatrix@C=1em{
\Imm^{\sf fr}(\TT^2\xra{\pr_i} \TT,\varphi)
\ar[r]
\ar[d]
&
\Imm(\TT^2\xra{\pr_i} \TT)
\ar[d]^-{{\sf Orbit}_\varphi}
&&
\Diff^{\sf fr}(\TT^2 \xra{\pr_i} \TT,\varphi)
\ar[r]
\ar[d]
&
\Diff(\TT^2\xra{\pr_i} \TT)
\ar[d]^-{{\sf Orbit}_\varphi}
\\
\ast
\ar[r]^-{\lag \varphi \rag}
&
\Fr(\TT^2\xra{\pr_i} \TT)
&
\text{ and }
&
\ast
\ar[r]^-{\lag \varphi \rag}
&
\Fr(\TT^2\xra{\pr_i} \TT)
~.
}
\]


As in Observation~\ref{t20},
the topological space $\Fr(\TT^2 \xra{\pr_i} \TT)$ is a torsor for the topological group $\Map\bigl( \TT^2 , \GL_{\{i\}\subset 2}(\RR) \bigr)$ of smooth maps from $\TT^2$ to the subgroup 
\[
\GL_{\{i\}\subset 2}(\RR)
~:=~
\Bigl\{
A \mid
Ae_i \in {\sf Span}\{e_i\}
\Bigr\}
~\subset~
\GL_2(\RR)
\]
consisting of those $2 \times 2$ matrices that carry the $i^{\rm th}$-coordinate line to itself. 
For each $i=1,2$, denote the intersections in $\GL_2(\RR)$:
\[
\xymatrix{
\SL_2(\ZZ)
\ar[rr]
\ar[d]
&&
\GL_2(\ZZ)
\ar[d]
&
&
\SL_{\{i\}\subset 2}(\ZZ)
\ar[rr]
\ar[d]
&&
\GL_{\{i\}\subset 2}(\ZZ)
\ar[d]
\\
\EpZ
\ar[rr]
&&
\EZ
&
\overset{- \cap \GL_{\{i\}\subset 2}(\RR)}\longmapsto
&
\sE^+_{\{i\}\subset 2}(\ZZ)
\ar[rr]
&&
\sE_{\{i\}\subset 2}(\ZZ)
}
\]


\begin{lemma}
\label{t62} 
For each $i=1,2$, the homotopy-equivalences between continuous monoids of Lemma~\ref{t1} and Corollary~\ref{t22} restrict as homotopy-equivalences between continuous monoids:
\[
\xymatrix{
\TT^2 \rtimes \GL_{\{i\}\subset 2}(\ZZ)
\ar[d]_-{\rm inclusion}
\ar@{-->}[r]^-{\Aff_i}_-{\simeq}
&
\Diff(\TT^2 \xra{\pr_i} \TT)
\ar[d]^-{\rm inclusion}
&&
\TT^2 \rtimes \sE_{\{i\}\subset 2}(\ZZ) 
\ar[d]_-{\rm inclusion}
\ar@{-->}[r]^-{\Aff_i}_-{\simeq}
&
\Imm(\TT^2 \xra{\pr_i} \TT)
\ar[d]^-{\rm inclusion}
\\
\TT^2 \rtimes \GL(\ZZ)
\ar[r]^-{\Aff}_-{\simeq}
&
\Diff(\TT^2)
&
\text{ and }
&
\TT^2 \rtimes \EZ 
\ar[r]^-{\Aff}_-{\simeq}
&
\Imm(\TT^2)
.
}
\]

\end{lemma}


\begin{proof}
Via the involution $\Sigma_2 \lacts \TT^2$ that swaps coordinates, the case in which $i=1$ implies the case in which $i=2$.
So we only consider the case in which $i=1$.


The left homotopy-equivalence is obtained from the right homotopy-equivalence by restricting to 
maximal continuous subgroups.
So we are reduced to establishing the right homotopy-equivalence.
Direct inspection reveals the indicated factorization $\Aff_1$ of the restriction of $\Aff$ to $\TT^2 \rtimes \sE_{\{1\}\subset 2}(\ZZ) \subset  \TT^2 \rtimes \EZ $.
So we are left to show that $\Aff_1$ is a homotopy-equivalence.  

Now, projection onto the $(1,1)$-entry defines a morphism between monoids, with kernel $\sK := \Bigl\{ \mbox{\scriptsize$\begin{bmatrix} 1 & b \\ 0 & d \end{bmatrix}$}\in \sE_{\{1\}\subset 2}(\ZZ) \Bigr\}$, which fits into a split short exact sequence of monoids:
\[
\xymatrix{
1
\ar[r] 
&
\sK
\ar[rr]
&&
\sE_{\{1\}\subset 2}(\ZZ) 
\ar[rr]_-{ (1,1)-{\rm entry}}
&&
(\ZZ \smallsetminus \{0\})^\times
\ar[r]
\ar@{-->}@(u,-)[ll]_-{ \mbox{\scriptsize$\begin{bmatrix} a & 0 \\ 0 & 1 \end{bmatrix} $} \mapsfrom ~a}
&
1
~.
}
\]

Now, because $\pr_1$ is surjective, for a given $f\in \Diff(\TT^2 \xra{\pr_i} \TT)$, there is a unique $\ov{f}\in \Diff(\TT)$ as in~(\ref{e87}).
Better, $\Diff(\TT^2 \xra{\pr_i}\TT) \ni f\mapsto \ov{f} \in \Diff(\TT)$ defines a forgetful morphism between topological monoids, whose kernel can be identified as the topological monoid of smooth maps from $\TT$ to $\Imm(\TT)$ with value-wise monoid-structure.
This is to say
there is a bottom short exact sequence of topological monoids, which splits as indicated:
\begin{equation}
\label{e68}
\xymatrix{
1
\ar[r] 
&
\TT \rtimes \sK
\ar@{..>}[d]
\ar[rr]^-{ (\id , \lag 0 \rag) \rtimes {\rm inclusion}}
&&
\TT^2 \rtimes \sE_{\{1\}\subset 2}(\ZZ)
\ar[d]_-{\Aff_1}
\ar[rr]_-{\pr_1 \rtimes (1,1)-{\rm entry}}
&&
\TT \rtimes (\ZZ \smallsetminus \{0\})^\times
\ar@{..>}[d]
\ar[r]
\ar@{-->}@(u,-)[ll]_-{ \left((0, z)  , { \mbox{\scriptsize$\begin{bmatrix} a & 0 \\ 0 & 1 \end{bmatrix}$}}\right) \mapsfrom (z, a) }
&
1
\\
1
\ar[r] 
&
\Map\bigl( \TT , \Imm(\TT) \bigr) 
\ar[rr]
&&
\Imm(\TT^2 \xra{\pr_1} \TT)
\ar[rr]_-{ f \mapsto \ov{f} }
&&
\Imm(\TT)
\ar[r]
\ar@{-->}@(u,-)[ll]_-{ \id_{\TT}\times f \mapsfrom f }
&
1
~
.
}
\end{equation}
Direct inspection of the definition of $\Aff$ reveals the downward factorizations making the commutative diagram~(\ref{e68}) among topological monoids. 
By the isotopy-extension theorem, the bottom short exact sequence among topological monoids forgets as a short exact sequence among continuous monoids. 
Using Lemma~\ref{t67},
the proof is complete upon showing that the left and right downward maps are equivalences between spaces.
It is routine to verify that the map $\Imm(\TT) \xra{\bigl(\ev_0 , \sH_1(-)\bigr)} \TT \rtimes (\ZZ \smallsetminus \{0\})^\times$ is a homotopy-inverse to the right downward map in~(\ref{e68}).


Now observe that the left downward morphism in~(\ref{e68}) fits into a diagram between short exact sequences of continuous monoids:
\begin{equation*}
\xymatrix@C=1em{
1
\ar[r] 
&
\ZZ
\ar@{..>}[d]
\ar[r]^-{b \mapsto \lag 0 \rag \rtimes  { \mbox{\scriptsize$\begin{bmatrix} 1 & b \\ 0 & 1 \end{bmatrix}$}}}
&
\TT \rtimes \sK 
\ar[d]
\ar[rr]_-{\id \rtimes (2,2)-{\rm entry}}
&&
\TT \rtimes (\ZZ \smallsetminus \{0\})^\times
\ar@{..>}[d]
\ar[r]
\ar@{-->}@(u,-)[ll]_-{\left(z, { \mbox{\scriptsize$\begin{bmatrix} 1 & 0 \\ 0 & d \end{bmatrix}$}} \right) \mapsfrom (z, d)  }
&
1
\\
1
\ar[r] 
&
\Map\bigl(
(0\in \TT)
,
(\id \in \Imm(\TT) )
\bigr)
\ar[r]_-{\rm forget}
&
\Map\bigl( \TT , \Imm(\TT) \bigr) 
\ar[rr]_-{ \ev_0 }
&&
\Imm(\TT)
\ar[r]
\ar@{-->}@(u,-)[ll]_-{{\rm constant}_{f} \mapsfrom f}
&
1
~
.
}
\end{equation*}
The right downward map here is a homotopy-equivalence, in the same way the right downward map in~(\ref{e68}) is a homotopy-equivalence.
Through this right downward identification of $\Imm(\TT)$, the left downward map is a homotopy-equivalence, with inverse given by taking $\pi_1$.  
Using Lemma~\ref{t67}, we conclude that the middle downward map is a homotopy-equivalence, as desired.  



\end{proof}


The Gram-Schmidt algorithm witnesses a deformation-retraction onto the inclusion from the intersection in $\GL_2(\RR)$:
\[
\sO(1)^2 = \sO(1)\times \sO(1) = \sO(2) \cap \GL_{\{i\}\subset 2}(\RR)
~\overset{\simeq}\hookrightarrow~ 
\GL_{\{i\}\subset 2}(\RR)
~.
\]

\begin{observation}
\label{prfr}
For each $i=1,2$, the sequence of homotopy-equivalences among topological spaces of Corollary~\ref{t25}, determined by a framing $\varphi \in \Fr(\TT^2 \xra{\pr_i} \TT)$, restricts as a sequence of homotopy-equivalences among topological spaces:
\begin{eqnarray*}
\Fr(\TT^2 \xra{\pr_i} \TT)
&
~
\xla{\cong}
~
&
\Map\bigl( \TT^2 , \GL_{\{i\}\subset 2}(\RR) \bigr)
\\
\nonumber
&
~
\xra{\simeq}
~
&
\Map \Bigl( \bigl( 0 \in \TT^2 \bigr) , \bigl( \uno \in \GL_{\{i\}\subset 2}(\RR) \bigr) \Bigr) \times \GL_{\{i\}\subset 2}(\RR)
\\
\nonumber
&
~
\xla{\simeq}
~
&
\Map \Bigl( \bigl( 0 \in \TT^2 \bigr) , \bigl( (+1 \in \sO(1) \bigr)^2 \Bigr) \times \sO(1)^2
\\
\nonumber
&
~\simeq~
&
\sO(1)^2
~.
\end{eqnarray*}

\end{observation}



\begin{observation}
\label{t63}
For each $i=1,2$, and each framing $\varphi \in \Fr(\TT^2 \xra{\pr_i} \TT)$, the diagram among topological spaces commutes:
\[
\xymatrix{
\TT^{2} \rtimes  \sE_{\{i\}\subset 2}(\ZZ) \ar[rr]^-{ {\sf Aff}_{i}} 
\ar[d]_{  \bigl(~{\rm sign~of~}(1,1){\text{-entry}} ~,~ {\rm sign~of~}(2,2){\text{-entry}} ~ \bigr) \circ {\sf proj}} 
&&
\Imm(\TT^2 \xra{\pr_i} \TT) \ar[d]^-{{\sf Orbit}_{\varphi}} 
\\
{\sf O}(1)^2
&&
\Fr(\TT^2 \xra{\pr_i} \TT) \ar[ll]_-{\rm Obs~\ref{prfr}}
.
}
\]
\end{observation} 



For each $i=1,2$, the action
$
\ZZ
\xra{ \lag U_i \rag }
\sE_{\{i\}\subset 2}(\ZZ)
\lacts 
\TT^2
$
as a topological group
defines the topological submonoid 
\[
\TT^2 \underset{U_i} \rtimes \ZZ
~ \subset~ 
\TT^2 \rtimes \sE_{\{i\}\subset 2}(\ZZ)
~.
\]
After Lemma~\ref{t62} and Observation~\ref{prfr}, Observation~\ref{t63} implies the following.
\begin{cor}
\label{t64}
For each $i=1,2$, 
and each framing $\varphi \in \Fr(\TT^2 \xra{\pr_i} \TT)$, 
there are canonical identifications among continuous monoids over the identification $\Aff_i$:
\[
\xymatrix{
\TT^2 \underset{U_{i}}\rtimes  \ZZ 
\ar[r]_-{\simeq}^-{\Aff_i^{\sf fr}}
\ar[d]_-{\id \rtimes \lag 
\tau_{i}
\rag   }
&
\Diff^{\fr}(\TT^2 \xra{\pr_i} \TT , \varphi) 
\ar[d]^-{\rm forget}
&&
\TT^{2} \rtimes \sE_{\{i\}\subset 2}(\ZZ)  
\ar[r]_-{\simeq}^-{\Aff_i^{\sf fr}}
\ar[d]_-{\id \rtimes \lag 
\w{\rm inclusion}
\rag   }
&
\Imm^{\sf fr}(\TT^2 \xra{\pr_i} \TT , \varphi)
\ar[d]^-{\rm forget}
\\
\TT^2 \rtimes \Braid  
\ar[r]^-{\simeq}_-{\rm Lem~\ref{t34}}
&
\Diff^{\fr}(\TT^2 , \varphi )
&
\text{ and }
&
\TT^2 \rtimes \Ebraid 
\ar[r]^-{\simeq}_-{\rm Lem~\ref{t34}}
&
\Imm^{\fr}(\TT^2, \varphi)
.
}
\]
\end{cor}



We now explain how the presentation~(\ref{e67}) of $\Braid$ gives a presentation of the continuous group $\Diff^{\fr}(\TT^2)$.
Observe the canonically commutative diagram among continuous groups:
\[
\xymatrix{
\TT^2 \ar[r]
\ar[d]
&
\Diff^{\fr}(\TT^{2} \xra{\pr_1} \TT)
\ar[d]
\\
\Diff^{\fr}(\TT^{2} \xra{\pr_2} \TT)
\ar[r]
&
\Diff^{\fr}(\TT^2),
}
\]
which results in a morphism from the pushout, $\Diff^{\fr}(\TT^{2} \xra{\pr_1} \TT)
\underset{\TT^2}
\coprod
\Diff^{\fr}(\TT^{2} \xra{\pr_2} \TT)
\longrightarrow
\Diff^{\fr}(\TT^2).
$
Recall the element $R \in \GL_2(\ZZ)$ from~(\ref{e64}).
The two homomorphisms \begin{tikzcd}
\ZZ \arrow[rr, yshift=0.7ex, "\lag \tau_{1} \tau_{2} \tau_{1}\rag"] \arrow[swap, rr, yshift=-0.7ex, "\lag \tau_{2} \tau_{1} \tau_{2}\rag"]
& 
& \ZZ \amalg \ZZ 
\end{tikzcd} determine two morphisms among continuous groups under $\TT^2$:



\begin{equation} \label{e66}
\small{
\begin{tikzcd}
\TT^2  \underset{R}\rtimes \ZZ \arrow[rr, yshift=0.7ex, "\id \rtimes \lag \tau_{1} \tau_{2} \tau_{1}\rag"] \arrow[swap, rr, yshift=-0.7ex, "\id \rtimes \lag \tau_{2} \tau_{1} \tau_{2}\rag"]
&
&
\TT^2 \underset{U_1,U_2} \rtimes (\ZZ \amalg \ZZ)
\arrow[r, "\simeq"]
\arrow[swap, r, "\rm Cor~\ref{t64}"]
&
\Diff^{\fr}(\TT^{2} \xra{\pr_1} \TT)
\underset{\TT^2}
\coprod
\Diff^{\fr}(\TT^{2} \xra{\pr_2} \TT)
\arrow[r]
&
\Diff^{\fr}(\TT^2)
~.
\end{tikzcd}
}
\end{equation}




\begin{cor}
\label{t49}
The diagram~(\ref{e66}) among continuous groups under $\TT^2$
witnesses a coequalizer.
In particular, for each $\infty$-category $\cX$, there is diagram among $\infty$-categories in which the outer square is a pullback:
\[
\xymatrix@C=1em{
&
\Mod_{\Diff^{\fr}(\TT^2)}(\cX)
\ar[r]
\ar[d]
&
\Mod_{\TT^2\underset{U_1} \rtimes  \ZZ }(\cX)
\underset{\Mod_{\TT^2}(\cX)}\times
\Mod_{\TT^2 \underset{U_2} \rtimes \ZZ}(\cX)
\ar[d]_-{\bigl( \id \rtimes \lag \tau_{1} \tau_{2} \tau_{1}\rag \bigr)^\ast \times \bigl( \id \rtimes \lag \tau_{2} \tau_{1} \tau_{2}\rag \bigr)^\ast}
&
\Mod_{\TT^2}(\cX)^{\lag U_1 , U_2 \rag}
\ar[d]
\ar[l]_-{\simeq}^-{\ref{t65}}
\\
\Mod_{\TT^2}(\cX)^{\lag R \rag}
\ar[r]^-{\simeq}_-{\ref{t65}}
&
\Mod_{\TT^2 \underset{R} \rtimes \ZZ }(\cX)
\ar[r]_-{\rm diagonal}
&
\Mod_{\TT^2 \underset{R} \rtimes \ZZ}(\cX)
\underset{\Mod_{\TT^2}(\cX)}\times
\Mod_{\TT^2 \underset{R} \rtimes \ZZ}(\cX)
&
\Mod_{\TT^2}(\cX)^{\lag R , R \rag}
\ar[l]_-{\simeq}^-{\ref{t65}}
.
}
\]
In particular, for $X\in \cX$ an object, an action $\Diff^{\fr}(\TT^2) \lacts X$ is 
\begin{enumerate}
\item
an action $\TT^2 \underset{\alpha}\lacts X$~,

\item
an identification $\alpha \circ R \underset{\gamma_R}\simeq \alpha$ of this action $\alpha$ with the action $\TT^2 \xra{R} \TT^2 \underset{\alpha}\lacts X$~,

\item
for $i=1,2$, extensions of this identification $\gamma_R$ to identifications $\alpha \circ U_i \underset{\gamma_{U_i}}\simeq \alpha$~.  


\end{enumerate}


\end{cor}



















