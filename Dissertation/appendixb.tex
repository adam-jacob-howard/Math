\chapter{Some Facts About Continuous Monoids} \label{sec.A}
We record some simple formal results concerning continuous monoids.

\begin{lemma}\label{t2}
Let $G \lacts X$ be an action of a continuous group on a space.
Let $\ast \xra{\lag x \rag } X$ be a point in this space.
Consider the stabilizer of $x$, which is the fiber of the orbit map of $x$:
\begin{equation}
\label{e52}
\xymatrix{
{\sf Stab}_G(x)
\ar[rrrr]
\ar[d]
&&
&&
\ast
\ar[d]^-{\lag x \rag}
\\
G \simeq G\times \ast 
\ar[rr]^-{ \id \times \lag x \rag} 
\ar@(u,u)[rrrr]_-{{\sf Orbit}_x}
&&
G \times X 
\ar[rr]^-{\sf act} 
&&
X
.
}
\end{equation}
There is a canonical identification in $\Spaces$ between this stabilizer and the based-loops at $[x]\colon \ast \xra{\lag x \rag} X \xra{\rm quotient} X_{/G}$ of the $G$-coinvariants,
\[
{\sf Stab}_G(x)
~\simeq~
\Omega_{[x]} ( X_{/G} )
~,
\]
through which the resulting composite morphism $\Omega_{[x]} ( X_{/G} )
\simeq {\sf Stab}_G(x)
\to 
G
$
canonically lifts to one between continuous groups.

\end{lemma}



\begin{proof}
By definition of a $G$-action, the orbit map $G\xra{{\sf Orbit}_x} X$ is canonically $G$-equivariant.
Taking $G$-coinvariants supplies an extension of the commutative diagram~(\ref{e52}) in $\Spaces$:
\[
\xymatrix{
{\sf Stab}_G(x) 
\ar[rr] \ar[d]
&&
G 
\ar[d]^-{{\sf Orbit}_x} \ar[rr]^-{\rm quotient}
&&
G_{/G} 
\simeq 
\ast
\ar[d]^-{({\sf Orbit}_x)_{/G}}
\\
\ast \ar[rr]^-{\lag x\rag}
&&
X
\ar[rr]^-{\rm quotient}
&&
X_{/G}
.
}
\]
Through the identification $G_{/G} \simeq \ast$, the right vertical map is identified as $\ast\xra{\lag [x] \rag} X_{/G}$.
Using that groupoids in $\Spaces$ are effective, the right square is a pullback.  
Because the lefthand square is defined as a pullback, it follows that the outer square is a pullback.
The identification ${\sf Stab}_G(x) \simeq \Omega_{[x]} ( X_{/G} ) $ follows.  
In particular, the space ${\sf Stab}_G(x)$ has the canonical structure of a continuous group.


Now, this continuous group ${\sf Stab}_G(x)$ is evidently functorial in the argument $G \lacts X \ni x$.  
In particular, the unique $G$-equivariant morphism $X\xra{!} \ast$ determines a morphism between continuous groups:
\[
{\sf Stab}_x(X)
\longrightarrow
{\sf Stab}_{\ast}(\ast)
~\simeq~
G
~.
\]


\end{proof}



\begin{lemma}
\label{t66}
Let $H \to G$ be a morphism between continuous groups.
Let $H\lacts X$ be an action on a space.  
There is a canonical map between spaces over $G_{/H}$, 
\[
X_{/\Omega(G_{/H})}
\longrightarrow 
(X \times G)_{/H}
~,
\]
from the coinvariants with respect to the action 
$
\Omega (G_{/H})
\xra{\Omega\text{-\rm Puppe}}
H
\lacts
X
$.
Furthermore, if the induced map $\pi_0(H)\to \pi_0(G)$ between sets of path-components is surjective, then this map is an equivalence.

\end{lemma}

\begin{proof}
The construction of the $\Omega$-Puppe sequence is so that the morphism 
$
\Omega (G_{/H})
\to 
H
$
witnesses the stabilizer of $\ast \xra{\rm unit} G$ with respect to the action $H \to G \underset{\rm left~trans}\lacts G$:
\[
\xymatrix{
\Omega (G_{/H})
\ar[rr]
\ar[d]
&&
H
\ar[d]
\\
\ast
\ar[rr]^-{\rm unit}
&&
G
.
}
\]
In particular, there is a canonical $\Omega (G_{/H})$-equivariant map
\[
X
~\simeq~
X\times \ast
\xra{ \id \times {\rm unit}}
X \times G
~.
\]
Taking coinvariants lends a canonically commutative diagram among spaces:
\begin{equation}
\label{f1ap}
\xymatrix{
X_{ \Omega (G_{/H}) }
\ar[d]
\ar[r]
&
(X \times  G)_{/H}
\ar[r]
\ar[d]
&
X_{/H}
\ar[d]
\\
\sB \Omega (G_{/H})
\ar[r]
&
G_{/H}
\ar[r]
&
\sB H
.
}
\end{equation}
This proves the first assertion.

We now prove the second assertion.
Because groupoid-objects are effective in the $\infty$-category $\Spaces$,
the $H$-coinvariants functor,
\[
\Fun ( \sB H , \Spaces )
\longrightarrow
\Spaces_{/\sB H}
~,\qquad
(H \lacts X)
\mapsto 
( X_{/H} \to \sB H)
~,
\]
is an equivalence between $\infty$-categories.
In particular, it preserves products.
It follows that the right square in~(\ref{f1ap}) witnesses a pullback.  
By definition of coinvariants of the restricted action $\Omega (G_{/H}) \to H \lacts X$, the outer square is a pullback.  
The connectivity assumption on the morphism $H\to G$ implies the left bottom horizontal map is an equivalence.  
We conclude that the left top horizontal map is also an equivalence, as desired.  

\end{proof}





Let $\fB N \xra{\lag N \lacts M \rag } {\sf Monoids}$ be an action of a continuous monoid on a continuous monoid.
This action can be codified as unstraightening of the composite functor $\fB N \to {\sf Monoids} \xra{ \fB }\Cat^{\ast/}_{(\infty,1)}$ is a pointed coCartesian fibration
$
(\fB M)_{/^{\sf l.lax} N}
\to
\fB N
$
.
The \bit{semi-direct product (of $N$ by $M$)} is the continuous monoid
\[
M \rtimes N := \End_{(\fB M)_{/^{\sf l.lax} N}} ( \ast )
~,
\]
which is endomorphisms of the point.\footnote{The underlying space of this continuous monoid is canonically identified as $M \times N$;
the 2-ary monoidal structure $\mu_{M \rtimes N}$ is canonically identified as the composite map between spaces:
\begin{eqnarray*}
\mu_{M \rtimes N }
\colon
(M \times N) \times (M \times N)
=
M \times (N \times M) \times M
&
\xra{ \id_M \times {\rm swap} \times \id_N} M \times ( M \times N ) \times N
\\
\xra{ \id_M \times \bigl( {\sf proj}_M , {\rm action} \bigr) \times \id_N }
M \times ( M \times N ) \times N =
&
( M \times M ) \times ( N \times N )
\xra{ \mu_M \times \mu_N }  M \times N 
~.
\end{eqnarray*}
}
Note the canonical morphism between monoids $M \rtimes N \to N$ whose kernel is $M$.


Dually, let $\fB N^{\op} \xra{\lag M \racts N \rag} {\sf Monoids}$ a \emph{right} be a action.
Consider the unstraightening of the composite functor $\fB N^{\op} \to {\sf Monoids} \xra{ \fB }\Cat^{\ast/}_{(\infty,1)}$ is a pointed Cartesian fibration
$
(\fB M)_{/^{\sf r.lax} N^{\op}}
\to
\fB N
$
.
The \bit{semi-direct product (of $N$ by $M$)} is the continuous monoid
\[
N \ltimes M := \End_{(\fB M)_{/^{\sf r.lax} N^{\op}}} ( \ast )
~,
\]
which is endomorphisms of the point.  
Note the canonical morphism between monoids $M \rtimes N \to N$ whose kernel is $M$.


\begin{observation}
\label{f5}
Let $N \lacts M$ be an action of a continuous monoid on a continuous monoid.
There is a canonical identification between continuous monoids under $M^{\op}$ and over $N^{\op}$:
\[
\left(
M \rtimes N
\right)^{\op}
~\simeq~
\left(
N^{\op}
\ltimes 
M^{\op}
\right)
~.
\]

\end{observation}




The next result is a characterization of semi-direct products.  
\begin{lemma}
\label{t67}
Let $A \overset{i}{\underset{r} \leftrightarrows} N$ be a retraction between continuous monoids (so $r \circ i \simeq \id_N$).
\begin{itemize}

\item
If the canonical map between spaces
\begin{equation}
\label{f2ap}
\Ker(r) 
\times 
N
\xra{ {\rm inclusion} \times i }
A \times A
\xra{\mu_A}
A
\end{equation}
is an equivalence,\footnote{Note that this condition is always satisfied if $N$ is a continuous group.}
then there is a canonical action $N \underset{\lambda}\lacts \Ker(r)$ \footnote{
The action map associated to $\lambda$ can be written as the composition
\[
N
\times
\Ker(r)
\xra{ i \times {\rm inclusion}}
A \times A
\xra{~\mu_A~}
A
\underset{(\ref{f2ap})}{\xla{~\simeq~}}
\Ker(r)\times N
\xra{~\sf proj~}
\Ker(r)
~.
\]
}
for which there is a canonical equivalence between monoids:
\[
\Ker(r) \underset{\lambda} \rtimes N
~\simeq~
A
~.
\]


\item
If the canonical map between spaces
\[
N
\times 
\Ker(r) 
\xra{ \sigma \times {\rm inclusion} }
A \times A
\xra{\mu_A}
A
\]
is an equivalence,\footnote{Note that this condition is always satisfied if $N$ is a continuous group.}
then there is a canonical action $\Ker(r) \underset{\rho}\racts N $ for which there is a canonical equivalence between monoids:
\[
\Ker(r) \underset{\rho} \rtimes N
~\simeq~
A
~.
\]


\end{itemize}



\end{lemma}

\begin{proof}
By way of Observation~\ref{f5}, the two assertions imply one another by taking Cartesian/coCartesian duals of coCartesian/Cartesian fibrations.
So we are reduced to proving the first assertion.

Consider the retraction $\fB A \overset{\fB i}{\underset{\fB r} \leftrightarrows} \fB N$ among pointed $\infty$-categories.
Note that $\fB i$ is essentially surjective.  
Note that $\Ker(r)$ is the fiber of $\fB r$ over $\ast \to \fB N$.

Let $c_1 \xra{\lag n \rag} \fB N$ be a morphism.
Consider the commutative diagram among $\infty$-categories:
\[
\xymatrix{
c_0
\ar[rr]^-{\lag \ast \rag}
\ar[d]_-{s}
&&
\fB A
\ar[d]^-{\fB r}
\\
c_1
\ar[rr]^-{\lag n \rag}
\ar@{-->}[urr]^-{\lag i(n) \rag}
&&
\fB N
.
}
\]
The assumption on the retraction implies the diagonal filler is initial among all such fillers.
This is to say that the morphism $i(n)$ in $\fB A$ is coCartesian over $\fB r$. 
Because $\fB i$ is essentially surjective, this shows that $\fB r$ is a coCartesian fibration.
The result now follows from the definition of the semi-direct product $\Ker(r) \underset{\lambda} \rtimes N$.    

\end{proof}







\begin{prop}
\label{t65}
Let $\cX$ be an $\infty$-category.
Let $\fB N \xra{\lag  N \lacts M \rag} {\sf Monoids}$ be an action of a continuous monoid $N$ on a continuous monoid $M$.
Consider the pre-composition-action:
\[
\fB N^{\op}
\xra{~ \lag N \lacts M \rag^{\op}~ } 
{\sf Monoids}^{\op}
\xra{~\Mod_{-}(\cX)~}
\Cat_{(\infty,1)}
~.
\]
There is a canonical identification over $\Mod_{M^{\op}}(\cX)$ from the $\infty$-category of $(M \rtimes N)^{\op}$-modules in $\cX$ to that of $M^{\op}$-modules in $\cX$ with the structure of being left-laxly invariant with respect to this precomposition $N^{\op}$-action:
\[
\Mod_{
\left(
M \rtimes N
\right)^{\op}
}(\cX)
~\simeq~
\Mod_{M^{\op}}(\cX)^{{\sf l.lax} N^{\op}}
~.
\]
In particular, there is a canonical fully faithful functor from the (strict) $N$-invariants,
\[
\Mod_{M^{\op}}(\cX)^{N}
~\hookrightarrow~
\Mod_{\left( M \rtimes N\right)^{\op}}(\cX)
~.
\]
which is an equivalence if the continuous monoid $N$ is a continuous group.


\end{prop}





\begin{proof}
The second assertion follows immediately from the first.  
The first assertion is proved upon justifying the sequence of equivalences among $\infty$-categories, each which is evidently over $\Mod_M(\cX)$:
\begin{eqnarray*}
\Mod_{\left( M \rtimes N \right)^{\op} }(\cX)
&
\underset{\rm (a)}\simeq
&
\Fun\bigl( 
~
\fB (M \rtimes N )^{\op}
~,~ 
\cX 
~
\bigr)
\\
&
\underset{\rm (b)}\simeq
&
\Fun\bigl( 
~
\fB ( N^{\op} \ltimes M^{\op} )
~,~ 
\cX 
~
\bigr)
\\
&
\underset{\rm (c)}\simeq
&
\Fun_{/\fB N^{\op}}
\Bigl(
~
\fB N^{\op}
~,~
\Fun^{\sf rel}_{\fB N^{\op}}
\bigl( 
\fB ( N^{\op} \ltimes M^{\op} )
,
\cX \times \fB N^{\op}
\bigr)
~
\Bigr)
\\
&
\underset{\rm (d)}\simeq
&
\Fun_{/\fB N^{\op}}\Bigl( 
~
\fB N^{\op}
~,~ 
\Fun^{\sf rel}_{\fB N^{\op}}
\bigl( 
(\fB M^{\op})_{/^{\sf r.lax} N}
,
\cX \times \fB N^{\op}
\bigr)
~
\Bigr)
\\
&
\underset{\rm (e)}\simeq
&
\Fun_{/\fB N^{\op}}\Bigl( 
~
\fB N^{\op}
~,~ 
\Fun(
\fB M^{\op}
,
\cX)_{/^{\sf l.lax} N^{\op}}
~
\Bigr)
\\
&
\underset{\rm (f)}\simeq
&
\Fun_{/\fB N^{\op}}\Bigl( 
~
\fB N^{\op}
~,~
\Mod_{M^{\op}}(\cX)_{/^{\sf l.lax} N^{\op}}
~
\Bigr)
\\
&
\underset{\rm (g)}\simeq
&
\Mod_{M^{\op}}(\cX)^{{\sf l.lax}N^{\op}}
~.
\end{eqnarray*}
The identifications~(a) and~(f) are both the definition of $\infty$-categories of modules for continuous monoids in $\cX$.
The identification~(b) is Observation~\ref{f5}.
By definition of semi-direct product monoids, 
the Cartesian unstraightening the composite functor 
$
\fB N
\xra{\lag N \lacts M^{\op}\rag } 
{\sf Monoids}
\xra{\fB}
\Cat_{(\infty,1)}
$
is the Cartesian fibration:
\[
\fB (N^{\op} \ltimes M^{\op}) 
\longrightarrow 
\fB N^{\op}
~.
\]
Being a Cartesian fibration ensures the existence of the \bit{relative functor $\infty$-category} (see~\cite{fib}).
The identification~(c) is direct from the definition of relative functor $\infty$-categories.  
Furthermore, there is a definitional identification of the \bit{right-lax coinvariants} $\fB (N^{\op} \ltimes M^{\op})  \simeq (\fB M^{\op})_{/^{\sf r.lax}N}$ over $\fB N^{\op}$ (see Appendix~A of~\cite{non-com-geom}), which determines the identification~(d).
The identification~(e) follows from the codification of the $N^{\op}$-action on $\Fun(\fB M^{\op} , \cX)$ in the statement of the proposition.
The identification~(g) is the definition of \bit{left-lax invariants} (see Appendix~A of~\cite{non-com-geom}).




\end{proof}










The commutativity of the topological group $\TT^2$ determines a canonical identification $\TT^2 \cong (\TT^2)^{\op}$ between topological groups, and therefore between continuous groups.
Together with Observation~\ref{t60}, we have the following consequence of Proposition~\ref{t65}.
\begin{cor}
\label{f6}
For $\cX$ an $\infty$-category, there is a canonical identification between $\infty$-categories over $\Mod_{\TT^2}(\cX)$:
\[
\Mod_{\left(\TT^2 \rtimes \Ebraid \right)^{\op}}(\cX)
~\simeq~
\Mod_{\TT^2}(\cX)^{{\sf l.lax} \Ebraid}
~.
\]

\end{cor}










