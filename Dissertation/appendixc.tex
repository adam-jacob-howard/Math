
\chapter{Some Facts About the Braid Group and Braid Monoid}
\label{sec.B}
Here we collect some facts about the braid group on 3 strands, and the braid monoid on 3 strands.






\section{Ambidexterity of $\Ebraid$} \label{ambidex}




\begin{observation}
\label{t60}
Taking transposes of matrices identifies the nested sequence among monoids with the nested sequence of their opposites:
\[
\Bigl(
\SL_2(\ZZ)
\subset
\EpZ
\subset
\GL_2^+(\RR)
\Bigr)
~\overset{{}^{~{}~}(-)^T}\cong~
\Bigl(
\SL_2(\ZZ)^{\op}
\subset
\EpZ^{\op}
\subset
\GL_2^+(\RR)^{\op}
\Bigr)
~.
\]
By covering space theory, these identifications canonically lift as identifications between nested sequences among monoids and their opposites:
\[
\Bigl(
\Braid
\subset
\Ebraid
\subset
\w{\GL}_2^+(\RR)
\Bigr)
~\overset{{}^{~{}~}(-)^T}\cong~
\Bigl(
\Braid^{\op}
\subset
\Ebraid^{\op}
\subset
\w{\GL}_2^+(\RR)^{\op}
\Bigr)
~.
\]
\end{observation}


\begin{cor}
\label{t61}
For each $\infty$-category $\cX$, there are canonical identifications
\[
\Mod_{\Braid}(\cX)
~\simeq~
\Mod_{\Braid^{\op}}(\cX)
\qquad
\text{ and }
\qquad
\Mod_{\Ebraid}(\cX)
~\simeq~
\Mod_{\Ebraid^{\op}}(\cX)
\]
between the $\infty$-category of (left-)modules in $\cX$ and that of right-modules in $\cX$.

\end{cor}






\begin{remark}
The composite isomorphism
$\Braid \underset{\cong}{\xra{(-)^{T}}} \Braid^{\op}  \underset{\cong}{\xra{(-)^{-1}}} \Braid$ 
is the involution of $\Braid$ given in terms of the presentation~(\ref{e67}) by exchanging $\tau_1$ and $\tau_2$. 
Similarly, the involution
$\SL_2(\ZZ) \underset{\cong}{\xra{(-)^{T}}} \SL_2(\ZZ)^{\op}  \underset{\cong}{\xra{(-)^{-1}}} \SL_2(\ZZ)$ exchanges $U_1$ and $U_2$.  

\end{remark}







\section{Comments About $\Braid$ and $\Ebraid$}





\begin{observation}
In $\Braid$, there is an identity of the generator of $\Ker(\Phi)$:
\[
(\tau_{1} \tau_{2} \tau_{1})^4 
=
( \tau_{1} \tau_{2})^6
=
(\tau_{2} \tau_{1} \tau_{2})^4
\in {\sf Ker}(\Phi)
~.
\]
For that matter, since the matrix
\begin{equation}
\label{e64}
R~:=~ U_1 U_2 U_3 ~=~ 
\begin{bmatrix}
0 & 1 
\\
-1 & 0
\end{bmatrix}
~=~
U_2 U_1 U_2
~\in~\GL_2(\ZZ)
\end{equation}
implements rotation by $-\frac{\pi}{2}$, 
then $R^4 = \uno$ in $\GL_2(\ZZ)$.

\end{observation}



The following result is an immediate consequence of how $\Ebraid$ is defined in equation (\ref{e77}), using that the continuous group $\GL^+_2(\RR)$ is a path-connected 1-type.
\begin{cor}
\label{t31}
There are pullbacks among continuous monoids:
\begin{equation*}
\xymatrix{
\Braid
\ar[rr]
\ar[d]_-{\Phi}
&&
\Ebraid
\ar[d]_-{\Psi} \ar[rr]
&&
\ast \ar[d]^-{\lag \uno \rag}
\\
\GL_2(\ZZ)
\ar[rr]
&&
\EZ
\ar[rr]^-{\RR \underset{\ZZ}\ot}
&&
\GL_2(\RR)
.
}
\end{equation*}
In particular, there is a canonical identification between continuous groups over $\GL_2(\ZZ)$:
\[
\Braid
~\simeq~
\Omega\bigl(
\GL_2(\RR)_{/\GL_2(\ZZ)}
\bigr)
\qquad
\bigl({\rm ~over~}
\GL_2(\ZZ)
~\bigr)
~.
\]

\end{cor}







\begin{observation}
\label{t39}
The inclusion $\SL_2(\ZZ)\subset \EpZ$ between submonoids of $\GL_2^+(\RR)$ determines an inclusion between topological monoids:
\begin{equation}
\label{e4}
\TT^2 \rtimes \Braid
\longrightarrow
\TT^2 \rtimes \Ebraid
~.
\end{equation}
After Observation~\ref{t21}, this inclusion~(\ref{e4}) witnesses the maximal subgroup, both as topological monoids and as monoid-objects in the $\infty$-category $\Spaces$.

\end{observation}







\begin{remark}
We give an explicit description of $\Ebraid$.
In~\cite{rawn}, the author gives an explicit description for the universal cover of $\SP_{2}(\RR) = \SL_{2}(\RR)$ (and goes on to establish the pullback square of Proposition~\ref{t32}).
Following those methods, consider the maps
\[
\phi
\colon
\GL_2(\RR)
\longrightarrow
\SS^1;
\qquad
 A
\mapsto \frac{(a + d) + i(b - c)}{|(a + d) + i(b - c)|}
\]
writing
$
A =
\begin{bmatrix}
a & b 
\\
c & d
\end{bmatrix}.
$
As in \cite{rawn}, consider a map
\[
\eta \colon \GL_2(\RR) \times \GL_2(\RR) \rightarrow \RR
\]
for which
\[
e^{i\eta(A, B)} = \frac{1 - \alpha_{A}\overline{\alpha_{B^{-1}}}}{|1 - \alpha_{A}\overline{\alpha_{B^{-1}}}|}
\qquad
\text{ where }
\qquad
\alpha_{A} = \frac{a^{2} + c^{2} - b^{2} - d^{2} - 2i(ad + bc)}{(a + d)^{2} + (b - c)^{2}}.
\]



In these terms, the monoid $\Ebraid$ can be identified as the subset
\[
\Ebraid
~:=~
\bigl\{
(A, s)  \mid  \phi(A) = e^{is}
\bigr\} 
~\subset~
\EpZ \times \RR
~,~
\text{with monoid-law }
\]
\[
(A, s) \cdot (B, t) := \bigl(AB, s + t + \eta(A, B) \bigr)
~.
\]
\end{remark}





\section{Group-Completion of $\Ebraid$}



The continuous group $\GL_2^+(\RR)$ is path-connected with $\pi_1\bigl( \GL_2^+(\RR) , \uno \bigr) \cong \ZZ$.
Consequently, there is a central extension
\begin{equation}
\label{e84}
1
\longrightarrow
\ZZ
\longrightarrow
\w{\GL}_2^+(\RR)
\xra{\rm universal~cover}
\GL_2^+(\RR)
\longrightarrow
1
~.
\end{equation}
Consider the inclusion as scalars $\RR_{>0}^\times \underset{\rm scalars} \hookrightarrow \GL_2^+(\RR)$.
Contractibility of the topological group $\RR_{>0}^\times$ implies base-change of this central extension~(\ref{e84}) along this inclusion as scalars splits.
In particular, 
for $\RR\underset{\QQ}\ot \colon \GL_2^+(\QQ)\subset \GL_2^+(\RR)$ the subgroup with rational coefficients, 
there are lifts among continuous groups in which the squares are pullbacks:

\[
\xymatrix@C=1em{
\NN^\times
\ar[rr]
\ar[drrrrrr]_-{\rm scalars}
\ar@{-->}@(u,-)[rrrrrr]^-{~}
&&
\QQ_{>0}^{\times}
\ar[rr]
\ar[drrrrrr]
\ar@{-->}@(u,u)[rrrrrr]^-{\rm \w{scalars}}
&&
\RR_{>0}^{\times}
\ar[drrrrrr]
\ar@{-->}@(-,u)[rrrrrr]^-{~}
&&
\Ebraid
\ar[d]
\ar[rr]_-{\w{\QQ\underset{\ZZ}\ot}}
&&
\w{\GL}_2^+(\QQ)
\ar[d]
\ar[rr]_-{\w{\RR\underset{\QQ}\ot}}
&&
\w{\GL}_2^+(\RR)
\ar[d]^-{\rm cover}_-{\rm universal}
\\
&&&
&&&
\EpZ
\ar[rr]_-{\QQ\underset{\ZZ}\ot}
&&
\GL_2^+(\QQ)
\ar[rr]_-{\RR\underset{\QQ}\ot}
&&
\GL_2^+(\RR)
.
}
\]




\begin{prop}
\label{t59}
Each of the diagrams among continuous monoids
\[
\xymatrix{
\NN^\times
\ar[rr]^-{\rm scalars}
\ar[d]_-{\rm inclusion}
&&
\EZ
\ar[d]^-{\QQ\underset{\ZZ}\ot}
&&
\NN^\times
\ar[rr]^-{\rm \w{scalars}}
\ar[d]_-{\rm inclusion}
&&
\Ebraid
\ar[d]^-{\w{\QQ\underset{\ZZ}\ot}}
\\
\QQ_{>0}^{\times}
\ar[rr]^-{\rm scalars}
&&
\GL_2(\QQ)
&
\text{ and }
&
\QQ_{>0}^{\times}
\ar[rr]^-{\rm \w{scalars}}
&&
\w{\GL}_2^+(\QQ)
}
\]
witnesses a pushout.
In particular, because $\NN^\times \xra{\rm inclusion}\QQ^\times_{>0}$ witnesses group-completion among continuous monoids, then each of the right downward morphisms witnesses group-completion among continuous monoids.  



\end{prop}

\begin{proof}
We explain the following commutative diagram among spaces:
\[
\xymatrix{
\EZ
\ar[rrrr]^-{\RR\underset{\ZZ}\ot }
\ar@{-->}[dr]_-{\rm (a)}
\ar[drrr]
&
&&
&
\GL_2(\QQ)
\ar@{-->}[dlll]_-{\rm (b)}
\\
&
\underset{\NN^{\sf div}}\colim
~\EZ
\ar@{-->}[rr]_-{\rm (c)}
&&
\EZ[ (\NN^\times)^{-1} ]
\ar[ur]_-{\ov{\RR\underset{\ZZ}\ot}}
&
}
\]
The top horizontal arrow is the standard inclusion.
Here, scalar matrices embed the multiplicative monoid of natural numbers $\NN^\times \underset{\rm scalars}\subset \EZ$.
The bottom right term, equipped with the diagonal arrow to it, is the indicated localization (among continuous monoids). 
The up-rightward arrow is the unique morphism between continuous monoids under $\EZ$, which exists because the continuous monoid $\GL_2(\QQ)$ is a continuous group.
The solid diagram of spaces is thusly forgotten from a diagram among continuous monoids.  


Next, the poset $\NN^{\sf div}$ is the natural numbers with partial order given by divisibility: $r\leq s$ means $r$ divides $s$.   
Consider the functor 
\[
F_{\EZ} \colon \NN^{\sf div} \longrightarrow \Sets
\hookrightarrow
\Spaces
~,\qquad
r\mapsto \EZ
~{}~
\text{ and }
~{}~
(r\leq s)
\mapsto 
\bigl(
~
\EZ \xra{ \frac{s}{r} \cdot -} \EZ
~
\bigr)
~.
\]
The colimit term in the above diagram is 
$
\colim\bigl( 
F_{\EZ}
\bigr)
$,
which can be identified as the classifying space of the poset
\[
{\sf Un}\bigl( F_{\EZ} \bigr)
~=~
\]
\[
\Bigl(
~
\NN\times \EZ
\text{ , with partial order }(r,A) \leq (s,B) \text{ meaning }
r\leq s \text{ in }\NN^{\sf div}
\text{ and }
\frac{s}{r} \cdot A=B
~
\Bigr)
~.
\]
\begin{itemize}
\item
The dashed arrow~$\rm (a)$ is the canonical map from the $1$-cofactor of the colimit.  

\item
The dashed arrow~$\rm (b)$ is implemented by the map $\w{\rm (b)}\colon \GL_2(\QQ)\xra{A\mapsto (r_A,r_A\cdot A)} \NN\times \EZ$ where $r_A\in \NN$ is the smallest natural number for which the matrix $r_A\cdot  A\in \EZ$ has integer coefficients.
The triangle with sides $\rm (a)$ and $\rm (b)$ evidently commutes.   

\item
The dashed arrow~$\rm (c)$ is implemented by the map \[
\w{\rm (c)}\colon
{\sf Un}\bigl( F_{\EZ} \bigr)
 \xra{(r,A)\mapsto r^{-1}A} \EZ[ (\NN^\times)^{-1} ].
 \]The triangle with sides~$\rm (a)$ and $\rm (c)$ evidently commutes.
We now argue that the map~$\rm (c)$ is an equivalence between spaces.  


Observe the identification between continuous monoids 
\[
\underset{p~{\rm prime}} \bigoplus (\ZZ_{\geq 0},+)
\xra{~\cong~}
\NN^\times
~,\qquad
\bigl(
\{ p~{\rm prime}\}
\xra{\eta}
\ZZ_{\geq 0}
\bigr)
\mapsto 
\underset{p~\rm prime}\prod p^{\eta(p)}
~,
\]
as a direct sum, indexed by the set of prime numbers, of free monoids each on a single generator.
For $S$ a set of prime numbers, denote by $\lag S \rag^{\times} \subset \NN^\times$ the submonoid generated by $S$.  
For $S$ a set of primes, and for $p\in S$, 
the above identification as a direct sum of monoids restricts as an identification $(\ZZ_{\geq 0},+ ) \times \lag S \smallsetminus \{p\} \rag^{\times} \cong  \lag \{p\} \rag^{\times}  \times \lag S \smallsetminus \{p\} \rag^{\times}   \cong \lag S \rag^{\times}$.

Next, observe an identification of the poset $\NN^{\sf div} \simeq (\fB \NN^\times)^{\ast/}$ as the undercategory of the deloop.  
Through this identification, and the above identification supplies an identification between posets from the direct sum (based at initial objects) indexed by the set of prime numbers:
\[
\underset{p~\rm prime} \bigoplus (\ZZ_{\geq 0},\leq)
\xra{~\cong~}
\NN^{\sf div}
~,\qquad
\bigl(
\{ p~{\rm prime}\}
\xra{\chi}
\ZZ_{\geq 0}
\bigr)
\mapsto 
\underset{p~\rm prime}\prod p^{\chi(p)}
~.
\]
For $S$ a set of prime numbers, denote by $\lag S \rag^{\sf div} \subset \NN^{\sf div}$ the full subposet generated by $S$.  
For $S$ a set of primes, and for $p\in S$, 
the above identification as a direct sum of posets restricts as an identification $(\ZZ_{\geq 0},\leq ) \times \lag S \smallsetminus \{p\} \rag^{\sf div} \cong  \lag \{p\} \rag^{\sf div}  \times \lag S \smallsetminus \{p\} \rag^{\sf div}   \cong \lag S \rag^{\sf div}$.
In particular, 
the standard linear order on the set of prime natural numbers determines the sequence of functors
\begin{equation}
\label{e85}
\NN^{\sf div}
\xra{{\sf loc}_2}
\lag p>2 \rag^{\sf div}
\xra{{\sf loc}_3}
\lag p > 3 \rag^{\sf div}
\xra{{\sf loc}_5}
\lag p > 5 \rag^{\sf div}
\xra{{\sf loc}_7}
\dots
~,
\end{equation}
each which is isomorphic with projection off of $(\ZZ_{\geq 0},\leq)$.  
In particular, each projection is a coCartesian fibration, so left Kan extension along each functor is computed as a sequential colimit.  
Because $\NN^\times \underset{\rm scalars}\subset \EZ$ is (strictly) central, 
so too is $(\ZZ_{\geq 0},+)\cong \lag \{p\}\rag^{\times}\subset \EZ$.  
The following claim follows from these observations, using induction on the standardly ordered set of primes. 
\begin{itemize}
\item[{\bf Claim.}]
For each prime $q$, 
left Kan extension of $F_{\EZ}$ along the composite functor $\NN^{\sf div}\xra{{\sf loc}_q^1} \lag p>q \rag^{\sf div}$ is the functor
\[
F_{\EZ\bigl[ (\lag p' \leq q \rag^{\times})^{-1}\bigr]}
\colon \lag p > q \rag^{\sf div} 
\xra{~({\sf loc}_q^1)_!(\EZ)~} 
\Spaces
~,
\]
\[
r\mapsto \EZ\bigl[ (\lag p' \leq q \rag^{\times})^{-1}\bigr]
\qquad
\text{ and }
\qquad
\]
\[
(r\leq s)
\mapsto 
\bigl(
\EZ\bigl[ (\lag p' \leq q \rag^{\times})^{-1}\bigr] \xra{ \frac{s}{r} \cdot -} \EZ\bigl[ (\lag p' \leq q \rag^{\times})^{-1}\bigr]
\bigr)
~,
\]
that evaluates on each $r$ as the localization $\EZ\bigl[ (\lag p' \leq q \rag^{\times})^{-1} \bigr]$, and on each relation $r \leq s$ as scaling by $\frac{s}{r}$.
\end{itemize}
Next, the colimit of this sequence~(\ref{e85}) is $\underset{q ~ \rm prime}\bigcap \lag p > q \rag^{\sf div} \simeq \ast$ terminal.  
Consequently, there is a canonical identification 
\begin{eqnarray}
\nonumber
\colim( F_{\EZ} )
&
~\simeq ~
&
\underset{q \in \{2<3<5\cdots\}}\colim 
\Bigl(
({\sf loc}^1_q)_! \bigl( F_{\EZ\bigl[ (\lag p' \leq q \rag^{\times})^{-1}\bigr]} \bigr)
\Bigr)
\\
\nonumber
&
~\simeq~
&
\underset{q \in \{2<3<5\cdots\}}\colim 
\Bigl(
F_{\EZ\bigl[ (\lag p' \leq q \rag^{\times})^{-1}\bigr]}
\Bigr)
\\
\nonumber
&
~\simeq~
&
\EZ\Bigl[ 
\bigl(
\underset{q \in \{2<3<5<\cdots\}} \bigcup \lag p' \leq q \rag^{\times} 
\bigr)^{-1}
\Bigr]
~=~
\EZ[ (\NN^\times)^{-1} ]
~.
\end{eqnarray}



\item
By inspection, the resulting self-map of $\GL_2(\QQ)$ is the identity.  
The natural transformation
\[
\xymatrix{
{\sf Un}\bigl( F_{\EZ} \bigr)
\ar@(u,u)[rr]^-{\id}
\ar[d]_-{\w{\rm (c)}}
&
\Uparrow
&
{\sf Un}\bigl( F_{\EZ} \bigr)
\\
\EZ[(\NN^\times)^{-1}]
\ar[rr]^{\ov{\RR\underset{\ZZ}\ot}}
&&
\GL_2(\QQ)
\ar[u]_-{\w{\rm (b)}}
,
}
\]
given by, for each $(s,B)\in {\sf Un}\bigl( F_{\EZ} \bigr)$, the relation $\bigl( r_{s^{-1}\cdot B} , r_{s^{-1}\cdot B} \cdot (s^{-1}\cdot B) \bigr)  \leq (s,B)$, witnesses an identification of the resulting self-map of $\underset{\NN^{\sf div}}\colim \EZ$ with the identity.  
\end{itemize}
We conclude that the map $\EZ[(\NN^\times)^{-1}]\xra{\ov{\RR\underset{\ZZ}\ot}} \GL_2(\QQ)$ is an equivalence.
It follows that the left square in the statement of the proposition is a pushout because the morphism $\NN^\times \xra{\rm inclusion}\QQ_{>0}^{\times}$ witnesses a group-completion (among continuous monoids).

The same argument also implies the square
\[
\xymatrix{
\NN^\times
\ar[rr]^-{\rm scalars}
\ar[d]_-{\rm inclusion}
&&
\EpZ 
\ar[d]^-{\QQ\underset{\ZZ}\ot}
\\
\QQ_{>0}^\times
\ar[rr]^-{\rm scalars}
&&
\GL_2^+(\QQ)
}
\]
also witnesses a pushout among continuous monoids.
Base-change along the central extension~(\ref{e84}) among continuous groups
reveals that the right square is also a pushout among continuous groups.

\end{proof}





\section{Relationship with the Finite Orbit Category of $\TT^2$}
Recall the $\infty$-category ${\sf Orbit}_{\TT^2}^{\sf fin}$ of transitive $\TT^2$-spaces with finite isotropy, and $\TT^2$-equivariant maps between them. 
Recall that the action $\Ebraid \to \EZ \lacts \TT^2$ on the topological group determines an action 
\begin{equation}
\label{e92}
\Ebraid \underset{\rm Obs~\ref{t60}}\simeq \Ebraid^{\op} \lacts {\sf Orbit}^{\sf fin}_{\TT^2}
~.
\end{equation}



\begin{prop}
\label{t68}
There is a canonical identification of the $\infty$-category of coinvariants with respect to the action~(\ref{e92}): 
\[
\Bigl(
{\sf Orbit}^{\sf fin}_{\TT^2}
\Bigr)_{/\Ebraid}
\xra{~\simeq~}
\fB \bigl(
\TT^2 \rtimes \Ebraid
\bigr)
~.
\]

\end{prop}

\begin{proof}
Recall that $\Ebraid \subset \w{\GL}^+_2(\RR)$ is defined as a submonoid of a group.
As a result, the left-multiplication action by its maximal subgroup,
$\w{\GL}_2^+(\ZZ) \lacts \Ebraid$, is free.  
Consequently, the space of objects $\Obj\bigl( (\fB \Ebraid)^{\ast/} \bigr) \simeq \Ebraid_{/\w{\GL}^+_2(\ZZ)}\xra{\cong} \EpZ_{/\GL_2^+(\ZZ)}$ is simply the quotient set of $\Ebraid$ by its maximal subgroup acting via left-multiplication, which is bijective with the quotient of $\EpZ$ by its maximal subgroup via the canonical projection $\Ebraid \to \EpZ$.  
The space of morphisms between objects represented by $A, B\in \EpZ$,
\[
\Hom_{(\fB \Ebraid)^{\ast/}}\bigl( [A] , [B] \bigr)\simeq  \{X \in \EpZ \mid XA=B\}\subset \EpZ
\]
is simply the set of factorizations in $\EpZ$ of $B$ by $A$.
In particular, the $\infty$-category $(\fB \Ebraid)^{\ast/}$ is a poset.  
We now identify this poset essentially through Pontrjagin duality.  



Consider the poset $\sP^{\sf fin}_{\TT^2}$ of finite subgroups of $\TT^2$ ordered by inclusion.  
We now construct mutually inverse functors between posets:
\begin{equation}
\label{e90}
(\fB \Ebraid)^{\ast/}
\xra{~[A]
\mapsto 
\Ker\bigl( \TT^2 \xra{A} \TT^2 \bigr)
~}
\sP^{\sf fin}_{\TT^2}
\qquad
\text{ and }
\qquad
\sP^{\sf fin}_{\TT^2}
\xra{~C
\mapsto 
\bigl[ \ZZ^2 \xra{A_C} \ZZ^2 \bigr]
~}
(\fB \Ebraid)^{\ast/}
\end{equation}
The first functor assigns to $[A]$ the kernel of the endomorphism of $\TT^2$ induced by a representative $A \in \EpZ \lacts \TT^2$.   
The second functor assigns to $C$ the endomorphism $(\ZZ^2 \xra{A_C}\ZZ^2)\in \EpZ$ defined as follows.
The preimage $\ZZ^2 \subset \quot^{-1}(C) \subset \RR^2 \xra{\quot} \RR^2_{/\ZZ^2} =:\TT^2$ by the quotient is a lattice in $\RR^2$ that contains the standard lattice cofinitely.
There is a unique pair of non-negative-quadrant vectors $(u_1,u_2)\in (\RR_{\geq 0})^2\times (\RR_\geq 0)^2$ that generate this lattice $\quot^{-1}(C)$ and agree with the standard orientation of $\RR^2$.
Then $A_C \in \EpZ$ is the unique matrix for which $A_C u_i = e_i$ for $i=1,2$.  
It is straight-forward to verify that these two assignments in~(\ref{e90}) indeed respect partial orders, and are mutually inverse to one another.
Observe that the action~(\ref{e92}) descends as an action $\Ebraid^{\op}\lacts \sP^{\sf fin}_{\TT^2}$, with respect to which the equivalences~(\ref{e90}) are $\Ebraid^{\op}$-equivariant.





Next, reporting the stabilizer of a transitive $\TT^2$-space defines a functor
\[
{\sf Orbit}^{\sf fin}_{\TT^2} \xra{ (\TT^2 \lacts T)\mapsto {\sf Stab}_{\TT^2}(t)} \sP^{\sf fin}_{\TT^2}.
\]
Evidently, this functor is conservative.
Notice also that this functor is a left fibration; its straightening is the composite functor
\begin{equation}
\label{e93}
\sP^{\sf fin}_{\TT^2}
\xra{~C\mapsto \frac{\TT}{C}~}
{\sf Groups}
\xra{~\sB~}
\Spaces
~.
\end{equation}
Observe that the action~(\ref{e92}) descends as an action $\Ebraid^{\op}\lacts \sP^{\sf fin}_{\TT^2}$.  



The result follows upon constructing a canonical filler in the diagram among $\infty$-categories witnessing a pullback:
\[
\xymatrix{
{\sf Orbit}_{\TT^2}^{\sf fin}
\ar@{-->}[rrrr]
\ar[d]
&&
&&
\Ar\bigl( \fB (\TT^2 \rtimes \Ebraid ) \bigr)
\ar[d]^-{\Ar(\fB {\sf proj}) }
\\
\sP^{\sf fin}_{\TT^2}
\ar[rr]^-{\simeq}_-{(\ref{e90})}
&&
(\fB \Ebraid)^{\ast/}
\ar[rr]^-{\rm forget}
&&
\Ar\bigl( \fB \Ebraid \bigr)
.
}
\]
By definition of semi-direct products,
the canonical functor $\fB (\TT^2 \rtimes \Ebraid) \xra{\fB \sf proj} \fB \Ebraid$ is a coCartesian fibration.
Because the $\infty$-category $\fB \TT^2 =\sB \TT^2$ is an $\infty$-groupoid, this coCartesian fibration is conservative, and therefore a left fibration.
Consequently, the functor 
\[
\Ar\bigl( \fB (\TT^2 \rtimes \Ebraid) \bigr)
\to
\Ar( \fB \Ebraid )
\]
is also a left fibration.
Therefore, the base-change of this left fibration along $
(\fB \Ebraid)^{\ast/}
\xra{\rm forget}
\Ar\bigl( \fB \Ebraid \bigr)
$ is again a left fibration:
\begin{equation}
\label{e91}
\Ar\bigl( \fB (\TT^2 \rtimes \Ebraid) \bigr)^{|\sB \TT^2}
\longrightarrow
(\fB \Ebraid)^{\ast/}
~\underset{(\ref{e90})}\simeq~
\sP^{\sf fin}_{\TT^2}
~.
\end{equation}
Direct inspection identifies the straightening of this left fibration~(\ref{e91}) as~(\ref{e93}).


\end{proof}


