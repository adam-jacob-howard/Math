\documentclass{amsart}

\headheight=8pt
\topmargin=0pt
\textheight=624pt
\textwidth=432pt
\oddsidemargin=18pt
\evensidemargin=18pt

\usepackage{amsmath}
\usepackage{amsfonts}
\usepackage{amssymb}
\usepackage{amsthm}
\usepackage{comment}
\usepackage{epsfig}
\usepackage{psfrag}
\usepackage{mathrsfs}
\usepackage{amscd}
\usepackage[all]{xy}
\usepackage{rotating}
\usepackage{lscape}
\usepackage{amsbsy}
\usepackage{verbatim}
\usepackage{moreverb}
\usepackage{color}
\usepackage{bbm}
\usepackage{eucal}
\usepackage{stackrel}

\usepackage{tikz-cd}
\usetikzlibrary{patterns,shapes.geometric,arrows,decorations.markings}
\usepackage{tikz-3dplot}

\usepackage{caption}
\usepackage{subcaption}

\colorlet{lightgray}{black!15}

\tikzset{->-/.style={decoration={
  markings,
  mark=at position .5 with {\arrow{>}}},postaction={decorate}}}
\tikzset{midarrow/.style={decoration={
    markings,
    mark=at position {#1} with {\arrow{>}}},postaction={decorate}} }



\pagestyle{plain}

\newtheorem{maintheorem}{Theorem}[section]
\renewcommand*{\themaintheorem}{\Alph{maintheorem}}

\newtheorem{theorem}{Theorem}[section]
\newtheorem{innercustomthm}{Theorem}
\newenvironment{mythm}[1]
  {\renewcommand\theinnercustomthm{#1}\innercustomthm}
  {\endinnercustomthm}


%\newtheorem{theorem}{Theorem}[section]
\newtheorem{prop}[theorem]{Proposition}
\newtheorem{lemma}[theorem]{Lemma}
\newtheorem{cor}[theorem]{Corollary}
\newtheorem{conj}{Conjecture}




\theoremstyle{definition}
\newtheorem{definition}[theorem]{Definition}
\newtheorem{summary}[theorem]{Summary}
\newtheorem{note}[theorem]{Note}
\newtheorem{ack}[theorem]{Acknowledgments}
\newtheorem{observation}[theorem]{Observation}
\newtheorem{construction}[theorem]{Construction}
\newtheorem{terminology}[theorem]{Terminology}
\newtheorem{remark}[theorem]{Remark}
\newtheorem{problem}[theorem]{Problem}
\newtheorem{example}[theorem]{Example}
\newtheorem{q}[theorem]{Question}
\newtheorem{notation}[theorem]{Notation}
\newtheorem{criterion}[theorem]{Criterion}
\newtheorem*{conventions*}{Convention}
\newtheorem{convention}{Convention}



\theoremstyle{remark}


\definecolor{orange}{rgb}{.95,0.5,0}
\definecolor{light-gray}{gray}{0.75}
\definecolor{brown}{cmyk}{0, 0.8, 1, 0.6}
\definecolor{plum}{rgb}{.5,0,1}


\DeclareMathOperator{\Link}{\sf Link}
\DeclareMathOperator{\Fin}{\sf Fin}
\DeclareMathOperator{\vect}{\sf Vect}
\DeclareMathOperator{\Vect}{\cV{\sf ect}}
\DeclareMathOperator{\Sfr}{S-{\sf fr}}
\DeclareMathOperator{\nfr}{\mathit{n}-{\sf fr}}

\DeclareMathOperator{\pr}{\mathsf{pr}}
\DeclareMathOperator{\ev}{\mathsf{ev}}


\DeclareMathOperator{\bBar}{\sf Bar}
\DeclareMathOperator{\Alg}{\sf Alg}
\DeclareMathOperator{\man}{\sf Man}
\DeclareMathOperator{\Man}{\cM{\sf an}}
\DeclareMathOperator{\Mod}{\sf Mod}
\DeclareMathOperator{\unzip}{\sf Unzip}
\DeclareMathOperator{\Snglr}{\cS{\sf nglr}}
\DeclareMathOperator{\TwAr}{\sf TwAr}
\DeclareMathOperator{\cSpan}{\sf cSpan}
\DeclareMathOperator{\Kan}{\sf Kan}
\DeclareMathOperator{\Psh}{\sf PShv}
\DeclareMathOperator{\LFib}{\sf LFib}
\DeclareMathOperator{\CAlg}{\sf CAlg}


\DeclareMathOperator{\cpt}{\sf cpt}
\DeclareMathOperator{\Aut}{\sf Aut}
\DeclareMathOperator{\colim}{{\sf colim}}
\DeclareMathOperator{\relcolim}{{\sf rel.\!colim}}
\DeclareMathOperator{\limit}{{\sf lim}}
\DeclareMathOperator{\cone}{\sf cone}
\DeclareMathOperator{\Der}{\sf Der}
\DeclareMathOperator{\Ext}{\sf Ext}
\DeclareMathOperator{\hocolim}{\sf hocolim}
\DeclareMathOperator{\holim}{\sf holim}
\DeclareMathOperator{\Hom}{\sf Hom}
\DeclareMathOperator{\End}{\sf End}
\DeclareMathOperator{\ulhom}{\underline{\Hom}}
\DeclareMathOperator{\fun}{\sf Fun}
\DeclareMathOperator{\Fun}{{\sf Fun}}
\DeclareMathOperator{\Iso}{\sf Iso}
\DeclareMathOperator{\map}{\sf Map}
\DeclareMathOperator{\Map}{{\sf Map}}
\DeclareMathOperator{\Mapc}{{\sf Map}_{\sf c}}
\DeclareMathOperator{\Gammac}{{\Gamma}_{\!\sf c}}
\DeclareMathOperator{\Tot}{\sf Tot}
\DeclareMathOperator{\Spec}{\sf Spec}
\DeclareMathOperator{\Spf}{\sf Spf}
\DeclareMathOperator{\Def}{\sf Def}
\DeclareMathOperator{\stab}{\sf Stab}
\DeclareMathOperator{\costab}{\sf Costab}
\DeclareMathOperator{\ind}{\sf Ind}
\DeclareMathOperator{\coind}{\sf Coind}
\DeclareMathOperator{\res}{\sf Res}
\DeclareMathOperator{\Ker}{\sf Ker}
\DeclareMathOperator{\coker}{\sf Coker}
\DeclareMathOperator{\pt}{\sf pt}
\DeclareMathOperator{\Sym}{\sf Sym}

\DeclareMathOperator{\str}{\sf str}

\DeclareMathOperator{\exit}{\sf Exit}
\DeclareMathOperator{\Exit}{\bcE{\sf xit}}

\DeclareMathOperator{\cylr}{{\sf Cylr}}

\DeclareMathOperator{\shift}{\sf shift}

\newcommand{\bit}[1]{\textbf{\textit{#1}}}
\newcommand{\racts}{\curvearrowleft}
\newcommand{\lacts}{\curvearrowright}




\DeclareMathOperator{\Cat}{{\sf Cat}}
\DeclareMathOperator{\fCat}{{\sf fCat}}
\DeclareMathOperator{\cat}{\fC{\sf at}}
\DeclareMathOperator{\Gcat}{{\sf GCat}_{\oo}}
\DeclareMathOperator{\gcat}{{\sf GCat}}
\DeclareMathOperator{\Dcat}{{\sf GCat}}
\DeclareMathOperator{\dcat}{\fG\fC{\sf at}}
\DeclareMathOperator{\Mcat}{\cM{\sf Cat}}
\DeclareMathOperator{\mcat}{\fD{\sf Cat}}






\DeclareMathOperator{\Ar}{{\sf Ar}}
\DeclareMathOperator{\twar}{{\sf TwAr}}


\DeclareMathOperator{\diskcat}{\sf {\cD}isk_{\mathit n}^\tau-Cat_\infty}
\DeclareMathOperator{\mfdcat}{\sf {\cM}fd_{\mathit n}^\tau-Cat_\infty}
\DeclareMathOperator{\diskone}{\sf {\cD}isk_{1}^{\vfr}-Cat_\infty}

\DeclareMathOperator{\symcat}{\sf Sym-Cat_\infty}
\DeclareMathOperator{\encat}{\cE_{\mathit n}-\sf Cat}
\DeclareMathOperator{\moncat}{\sf Mon-Cat_\infty}

\DeclareMathOperator{\inrshv}{\sf inr-shv}
\DeclareMathOperator{\clsshv}{\sf cls-shv}



\DeclareMathOperator{\qc}{\sf QC}
\DeclareMathOperator{\m}{\sf Mod}
\DeclareMathOperator{\bi}{\sf Bimod}
\DeclareMathOperator{\perf}{\sf Perf}
\DeclareMathOperator{\shv}{\sf Shv}
\DeclareMathOperator{\Shv}{\sf Shv}



\DeclareMathOperator{\psh}{\sf PShv}
\DeclareMathOperator{\gshv}{\sf GShv}
\DeclareMathOperator{\csh}{\sf Coshv}
\DeclareMathOperator{\comod}{\sf Comod}
\DeclareMathOperator{\M}{\mathsf{-Mod}}
\DeclareMathOperator{\coalg}{\mathsf{-coalg}}
\DeclareMathOperator{\ring}{\mathsf{-rings}}
\DeclareMathOperator{\alg}{\mathsf{Alg}}
\DeclareMathOperator{\artin}{{\sf Artin}}%{\disk_{\mathit n}\alg^{\sf Art}_{\mathit k}}
\DeclareMathOperator{\art}{\mathsf{Art}}
\DeclareMathOperator{\triv}{\mathsf{Triv}}
\DeclareMathOperator{\cobar}{\mathsf{cBar}}
\DeclareMathOperator{\ba}{\mathsf{Bar}}

\DeclareMathOperator{\shvp}{\sf Shv_{\sf p}^{\sf cbl}}

\DeclareMathOperator{\lkan}{{\sf LKan}}
\DeclareMathOperator{\rkan}{{\sf RKan}}

\DeclareMathOperator{\Diff}{{\sf Diff}}
\DeclareMathOperator{\sh}{\sf shv}



\DeclareMathOperator{\calg}{\mathsf{CAlg}}
\DeclareMathOperator{\op}{\mathsf{op}}
\DeclareMathOperator{\relop}{\mathsf{rel.op}}
\DeclareMathOperator{\com}{\mathsf{Com}}
\DeclareMathOperator{\bu}{\cB\mathsf{un}}
\DeclareMathOperator{\bun}{\sf Bun}

\DeclareMathOperator{\pbun}{\sf PBun}




\DeclareMathOperator{\cMfld}{{\sf c}\cM\mathsf{fld}}
\DeclareMathOperator{\cBun}{{\sf c}\cB\mathsf{un}}
\DeclareMathOperator{\Bun}{\cB\mathsf{un}}

\DeclareMathOperator{\dbu}{\mathsf{DBun}}

\DeclareMathOperator{\dbun}{\mathsf{DBun}}

\DeclareMathOperator{\bsc}{\mathsf{Bsc}}
\DeclareMathOperator{\snglr}{\sf Snglr}

\DeclareMathOperator{\Bsc}{\cB\mathsf{sc}}


\DeclareMathOperator{\arbr}{\mathsf{Arbr}}
\DeclareMathOperator{\Arbr}{\cA\mathsf{rbr}}
\DeclareMathOperator{\Rf}{\cR\mathsf{ef}}
\DeclareMathOperator{\drf}{\mathsf{Ref}}


\DeclareMathOperator{\st}{\mathsf{st}}
\DeclareMathOperator{\sk}{\mathsf{sk}}
\DeclareMathOperator{\Ex}{\mathsf{Ex}}

\DeclareMathOperator{\sd}{\mathsf{sd}}

\DeclareMathOperator{\inr}{\mathsf{inr}}

\DeclareMathOperator{\cls}{\mathsf{cls}}
\DeclareMathOperator{\act}{\mathsf{act}}
\DeclareMathOperator{\rf}{\mathsf{ref}}
\DeclareMathOperator{\pcls}{\mathsf{pcls}}
\DeclareMathOperator{\opn}{\mathsf{open}}
\DeclareMathOperator{\emb}{\mathsf{emb}}
\DeclareMathOperator{\Cylo}{\mathsf{Cylo}}
\DeclareMathOperator{\Cylr}{\mathsf{Cylr}}


\DeclareMathOperator{\cbl}{\mathsf{cbl}}

\DeclareMathOperator{\pcbl}{\mathsf{p.cbl}}


\DeclareMathOperator{\gl}{\mathsf{GL}_1}

\DeclareMathOperator{\Top}{\mathsf{Top}}
\DeclareMathOperator{\Mfd}{{\cM}\mathsf{fd}}
\DeclareMathOperator{\cMfd}{{\sf c}{\cM}\mathsf{fd}}
\DeclareMathOperator{\Mfld}{{\sf Mfld}}
\DeclareMathOperator{\mfd}{\mathsf{Mfd}}
\DeclareMathOperator{\Emb}{\mathsf{Emb}}
\DeclareMathOperator{\enr}{\fE\mathsf{nr}}
\DeclareMathOperator{\LEnr}{\mathsf{LEnr}}
\DeclareMathOperator{\diff}{\mathsf{Diff}}
\DeclareMathOperator{\conf}{\mathsf{Conf}}

\DeclareMathOperator{\MC}{\mathsf{MC}}
\DeclareMathOperator{\strat}{\mathsf{Strat}}
\DeclareMathOperator{\Strat}{\cS\mathsf{trat}}
\DeclareMathOperator{\kan}{\mathsf{Kan}}

\DeclareMathOperator{\dd}{{\cD}\mathsf{isk}}

\DeclareMathOperator{\loc}{\mathsf{Loc}}



\DeclareMathOperator{\poset}{\mathsf{Poset}}



\DeclareMathOperator{\spaces}{\cS\mathsf{paces}}
\DeclareMathOperator{\Spaces}{\cS\mathsf{paces}}

\DeclareMathOperator{\Space}{{\cS}\sf paces}
\DeclareMathOperator{\spectra}{\cS\mathsf{pectra}}
\DeclareMathOperator{\Spectra}{\cS\mathsf{pectra}}
\DeclareMathOperator{\mfld}{\mathsf{Mfld}}
\DeclareMathOperator{\Disk}{{\sf Disk}}
\DeclareMathOperator{\cdisk}{{\sf c}\cD{\mathsf{isk}}}
\DeclareMathOperator{\cDisk}{{\sf c}\cD{\mathsf{isk}}}
\DeclareMathOperator{\sing}{\mathsf{Sing}}
\DeclareMathOperator{\set}{{\mathsf{Sets}}}
\DeclareMathOperator{\Aux}{\cA{\mathsf{ux}}}
\DeclareMathOperator{\Adj}{\mathsf{Adj}}


\DeclareMathOperator{\Dtn}{\cD{\mathsf{isk}^\tau_{\mathit n}}}


\DeclareMathOperator{\sm}{\mathsf{sm}}
\DeclareMathOperator{\vfr}{\sf vfr}
\DeclareMathOperator{\fr}{\sf fr}
\DeclareMathOperator{\sfr}{\sf sfr}


\DeclareMathOperator{\bord}{\mathsf{Bord}}
\DeclareMathOperator{\Bord}{{\sf Bord}_1^{\fr}}
\DeclareMathOperator{\Bordk}{\cB{\sf ord}_1^{\fr}(\RR^k)}

\DeclareMathOperator{\Corr}{{\sf Corr}}
\DeclareMathOperator{\corr}{{\sf Corr}}

\DeclareMathOperator{\fcorr}{{\sf FCorr}}
\DeclareMathOperator{\pcorr}{{\sf PCorr}}



\DeclareMathOperator{\Sing}{\mathsf{Sing}}


\DeclareMathOperator{\BTop}{\sf BTop}
\DeclareMathOperator{\BO}{{\mathsf BO}}


\DeclareMathOperator{\Lie}{\sf Lie}



\def\ot{\otimes}

\DeclareMathOperator{\fin}{\sf Fin}

\DeclareMathOperator{\oo}{\infty}


\DeclareMathOperator{\hh}{\sf HC}

\DeclareMathOperator{\free}{\sf Free}
\DeclareMathOperator{\fpres}{\sf FPres}


\DeclareMathOperator{\fact}{\sf Fact}
\DeclareMathOperator{\ran}{\sf Ran}

\DeclareMathOperator{\disk}{\sf Disk}

\DeclareMathOperator{\ccart}{\sf cCart}
\DeclareMathOperator{\cart}{\sf Cart}
\DeclareMathOperator{\rfib}{\sf RFib}
\DeclareMathOperator{\lfib}{\sf LFib}


\DeclareMathOperator{\tr}{\triangleright}
\DeclareMathOperator{\tl}{\triangleleft}


\newcommand{\lag}{\langle}
\newcommand{\rag}{\rangle}


\newcommand{\w}{\widetilde}
\newcommand{\un}{\underline}
\newcommand{\ov}{\overline}
\newcommand{\nn}{\nonumber}
\newcommand{\nid}{\noindent}
\newcommand{\ra}{\rightarrow}
\newcommand{\la}{\leftarrow}
\newcommand{\xra}{\xrightarrow}
\newcommand{\xla}{\xleftarrow}

\newcommand{\weq}{\xrightarrow{\sim}}
\newcommand{\cofib}{\hookrightarrow}
\newcommand{\fib}{\twoheadrightarrow}

\def\llarrow{   \hspace{.05cm}\mbox{\,\put(0,-2){$\leftarrow$}\put(0,2){$\leftarrow$}\hspace{.45cm}}}
\def\rrarrow{   \hspace{.05cm}\mbox{\,\put(0,-2){$\rightarrow$}\put(0,2){$\rightarrow$}\hspace{.45cm}}}
\def\lllarrow{  \hspace{.05cm}\mbox{\,\put(0,-3){$\leftarrow$}\put(0,1){$\leftarrow$}\put(0,5){$\leftarrow$}\hspace{.45cm}}}
\def\rrrarrow{  \hspace{.05cm}\mbox{\,\put(0,-3){$\rightarrow$}\put(0,1){$\rightarrow$}\put(0,5){$\rightarrow$}\hspace{.45cm}}}

\def\cA{\mathcal A}\def\cB{\mathcal B}\def\cC{\mathcal C}\def\cD{\mathcal D}
\def\cE{\mathcal E}\def\cF{\mathcal F}\def\cG{\mathcal G}\def\cH{\mathcal H}
\def\cI{\mathcal I}\def\cJ{\mathcal J}\def\cK{\mathcal K}\def\cL{\mathcal L}
\def\cM{\mathcal M}\def\cN{\mathcal N}\def\cO{\mathcal O}\def\cP{\mathcal P}
\def\cQ{\mathcal Q}\def\cR{\mathcal R}\def\cS{\mathcal S}\def\cT{\mathcal T}
\def\cU{\mathcal U}\def\cV{\mathcal V}\def\cW{\mathcal W}\def\cX{\mathcal X}
\def\cY{\mathcal Y}\def\cZ{\mathcal Z}

\def\AA{\mathbb A}\def\BB{\mathbb B}\def\CC{\mathbb C}\def\DD{\mathbb D}
\def\EE{\mathbb E}\def\FF{\mathbb F}\def\GG{\mathbb G}\def\HH{\mathbb H}
\def\II{\mathbb I}\def\JJ{\mathbb J}\def\KK{\mathbb K}\def\LL{\mathbb L}
\def\MM{\mathbb M}\def\NN{\mathbb N}\def\OO{\mathbb O}\def\PP{\mathbb P}
\def\QQ{\mathbb Q}\def\RR{\mathbb R}\def\SS{\mathbb S}\def\TT{\mathbb T}
\def\UU{\mathbb U}\def\VV{\mathbb V}\def\WW{\mathbb W}\def\XX{\mathbb X}
\def\YY{\mathbb Y}\def\ZZ{\mathbb Z}

\def\sA{\mathsf A}\def\sB{\mathsf B}\def\sC{\mathsf C}\def\sD{\mathsf D}
\def\sE{\mathsf E}\def\sF{\mathsf F}\def\sG{\mathsf G}\def\sH{\mathsf H}
\def\sI{\mathsf I}\def\sJ{\mathsf J}\def\sK{\mathsf K}\def\sL{\mathsf L}
\def\sM{\mathsf M}\def\sN{\mathsf N}\def\sO{\mathsf O}\def\sP{\mathsf P}
\def\sQ{\mathsf Q}\def\sR{\mathsf R}\def\sS{\mathsf S}\def\sT{\mathsf T}
\def\sU{\mathsf U}\def\sV{\mathsf V}\def\sW{\mathsf W}\def\sX{\mathsf X}
\def\sY{\mathsf Y}\def\sZ{\mathsf Z}

\def\bA{\mathbf A}\def\bB{\mathbf B}\def\bC{\mathbf C}\def\bD{\mathbf D}
\def\bE{\mathbf E}\def\bF{\mathbf F}\def\bG{\mathbf G}\def\bH{\mathbf H}
\def\bI{\mathbf I}\def\bJ{\mathbf J}\def\bK{\mathbf K}\def\bL{\mathbf L}
\def\bM{\mathbf M}\def\bN{\mathbf N}\def\bO{\mathbf O}\def\bP{\mathbf P}
\def\bQ{\mathbf Q}\def\bR{\mathbf R}\def\bS{\mathbf S}\def\bT{\mathbf T}
\def\bU{\mathbf U}\def\bV{\mathbf V}\def\bW{\mathbf W}\def\bX{\mathbf X}
\def\bY{\mathbf Y}\def\bZ{\mathbf Z}
\def\bdelta{\mathbf\Delta}
\def\bDelta{\mathbf\Delta}
\def\blambda{\mathbf\Lambda}


\def\fA{\frak A}\def\fB{\frak B}\def\fC{\frak C}\def\fD{\frak D}
\def\fE{\frak E}\def\fF{\frak F}\def\fG{\frak G}\def\fH{\frak H}
\def\fI{\frak I}\def\fJ{\frak J}\def\fK{\frak K}\def\fL{\frak L}
\def\fM{\frak M}\def\fN{\frak N}\def\fO{\frak O}\def\fP{\frak P}
\def\fQ{\frak Q}\def\fR{\frak R}\def\fS{\frak S}\def\fT{\frak T}
\def\fU{\frak U}\def\fV{\frak V}\def\fW{\frak W}\def\fX{\frak X}
\def\fY{\frak Y}\def\fZ{\frak Z}

\def\bcA{\boldsymbol{\mathcal A}}\def\bcB{\boldsymbol{\mathcal B}}\def\bcC{\boldsymbol{\mathcal C}}
\def\bcD{\boldsymbol{\mathcal D}}\def\bcE{\boldsymbol{\mathcal E}}\def\bcF{\boldsymbol{\mathcal F}}
\def\bcG{\boldsymbol{\mathcal G}}\def\bcH{\boldsymbol{\mathcal H}}\def\bcI{\boldsymbol{\mathcal I}}
\def\bcJ{\boldsymbol{\mathcal J}}\def\bcK{\boldsymbol{\mathcal K}}\def\bcL{\boldsymbol{\mathcal L}}
\def\bcM{\boldsymbol{\mathcal M}}\def\bcN{\boldsymbol{\mathcal N}}\def\bcO{\boldsymbol{\mathcal O}}\def\bcP{\boldsymbol{\mathcal P}}\def\bcQ{\boldsymbol{\mathcal Q}}\def\bcR{\boldsymbol{\mathcal R}}
\def\bcS{\boldsymbol{\mathcal S}}\def\bcT{\boldsymbol{\mathcal T}}\def\bcU{\boldsymbol{\mathcal U}}
\def\bcV{\boldsymbol{\mathcal V}}\def\bcW{\boldsymbol{\mathcal W}}\def\bcX{\boldsymbol{\mathcal X}}
\def\bcY{\boldsymbol{\mathcal Y}}\def\bcZ{\boldsymbol{\mathcal Z}}

\def\ccD{{\sf c}\boldsymbol{\mathcal D}}
\def\bcM{\boldsymbol{\mathcal M}}

\DeclareMathOperator{\Stri}{\boldsymbol{\cS}{\sf tri}}
\DeclareMathOperator{\btheta}{\boldsymbol{\Theta}}
\DeclareMathOperator{\adj}{{\sf adj}}


\DeclareMathOperator{\uno}{\mathbbm{1}}





\DeclareMathOperator{\Braid}{{\sf Braid}_3}
\DeclareMathOperator{\Ebraid}{\w{\sf E}_{2}^{+}(\ZZ)}
\DeclareMathOperator{\GL}{\sf GL}
\DeclareMathOperator{\SP}{\sf SP}
\DeclareMathOperator{\SL}{\sf SL}
\DeclareMathOperator{\quot}{\sf quot}
\DeclareMathOperator{\Fr}{\sf Fr}
\DeclareMathOperator{\id}{\sf id}
\DeclareMathOperator{\Act}{\sf Act}
\DeclareMathOperator{\trans}{\sf trans}
\DeclareMathOperator{\Bdl}{\sf Bdl}
\DeclareMathOperator{\sHH}{\sf HH}
\DeclareMathOperator{\HHt}{{\sf HH}^{(2)}}
\DeclareMathOperator{\Obj}{\sf Obj}
\DeclareMathOperator{\Perf}{\sf Perf}
\DeclareMathOperator{\Imm}{\sf Imm}
\DeclareMathOperator{\Aff}{\sf Aff}
\DeclareMathOperator{\EZ}{\sE_2(\ZZ)}
\DeclareMathOperator{\EpZ}{\sE^+_2(\ZZ)}
\DeclareMathOperator{\PShv}{\sf PShv}




\begin{document}


\title{Natural symmetries of secondary Hochschild homology}


\author{David Ayala and Adam Howard}





\address{Department of Mathematics\\Montana State University\\Bozeman, MT 59717}
\email{david.ayala@montana.edu}
\address{Department of Mathematics\\Montana State University\\Bozeman, MT 59717}
\email{adam.howard1@montana.edu}
\thanks{DA was supported by the National Science Foundation under awards 1812055 and 1945639.  This material is based upon work supported by the National Science Foundation under Grant No. DMS-1440140, while DA was in residence at the Mathematical Sciences Research Institute in Berkeley, California, during the Spring 2020 semester.}






\begin{abstract}
We identify the group of framed diffeomorphisms of the torus as a semi-direct product of the torus with the braid group on 3 strands; we also identify the topological monoid of framed local-diffeomorphisms of the torus in similar terms.
It follows that the framed mapping class group is this braid group.
We show that the group of framed diffeomorphisms of the torus acts on twice-iterated Hochschild homology, and explain how this recovers a host of familiar symmetries of such.  
In the case of a Cartesian symmetric monoidal structure, we show that this action extends to the monoid of framed local-diffeomorphisms of the torus.
Based on this, we propose a definition of an unstable secondary cyclotomic structure, and show that iterated Hochschild homology possesses such.  

\end{abstract}



\keywords{??.}

\subjclass[2010]{Primary ??. Secondary ??.}

\maketitle


\tableofcontents



%{\color{magenta}
%
%
%For David to do.
%\begin{enumerate}
%
%\item
%Address all magenta comments in the body.
%
%\i
%
%
%\end{enumerate}
%
%
%}





{\color{red}




Yo Adam -- do these things to the extent that you're able.
\begin{enumerate}

%\item
%Change all instances of $\tau_{1,2}$ to $\tau_1$, and all instances of $\tau_{2,3}$ to $\tau_2$. 
%\textcolor{blue}{1st pass done. Done done.} 
%
%\item
%Label the three main results (Theorem A, Theorem B1, Theorem B2), in those terms, not by the random numbering given by LaTex. \textcolor{blue}{done}

\item
Address all red comments in the body.

\item
On second thought, I think the notation $\Ebraid$ isn't good.
Perhaps better would be $\w{\sE}^+_2(\ZZ)$, or something like that.

\item
Reverse the direction of each and every semi-direct product.
For instance, $\Braid \ltimes \TT^2$ instead of $\TT^2 \rtimes \Braid$.

%\item
%Invent a term for $\square$, and change it's notation accordingly. \textcolor{blue}{I went with $\Ebraid$, other possibilities we sort of discussed where, $\ebraid$ and $\Eraid$.}
%
%\item
%Replace all instance of coinvariants $X_G$ by $X_{/G}$.  \textcolor{blue}{done}

\item
Make sure internally-referenced labels are the correct ones. \textcolor{blue}{1st pass done}
Unlabel any equations that are not internally-referenced somewhere else in the paper. \textcolor{blue}{done, except magenta}

\item
Fill in missing internal-references, to the extent that it's clear what's intended. \textcolor{blue}{1st pass done, mostly missing external references}

\item
Fix multiply-defined internal-references (look at the latex compile output for this).

\item
Make consistent: labeling arrows by their name above versus below the arrow, especially when there's a $\simeq$ below/above the arrow. 


\item
Make sure all notation is defined.
If not, introduce a definition (as efficiently as reasonable).




\end{enumerate}



}


\section*{Introduction}




This paper has two main results.
The first main result identifies the continuous group of \bit{framed diffeomorphisms}, and the continuous monoid of \bit{framed local-diffeomorphisms}, of a framed torus.
The second main result constructs actions of these monoids on iterated Hochschild homology.
We abstract such action as \bit{unstable secondary cyclotomic structure}.  
We state these results right away, as Theorem~\ref{Theorem A} and Theorem~\ref{t36}, and direct a reader to the body of the paper for definitions and proofs.  





\begin{conventions*}
\begin{itemize}
\item[~]

\item
Out of convenience, we work in the $\infty$-category $\Spaces$ of spaces, or $\infty$-groupoids, an object in which is a \bit{space}.  
This $\infty$-category can be presented as the $\infty$-categorical localization of the ordinary category of compactly-generated Hausdorff topological spaces that are homotopy equivalent with a CW complex, localized on the weak homotopy equivalences.  
So we present some objects in $\Spaces$ by naming a topological space.  

\item
By a pullback square among spaces we mean a pullback square in the $\infty$-category $\Spaces$.
Should the square be presented by a homotopy-commutative square among topological spaces, then the canonical map from the initial term in the square to the homotopy-pullback is a weak homotopy-equivalence.  

\item
By a \bit{continuous group} (resp. \bit{continuous monoid}) we mean a group-object (resp. monoid-object) in $\Spaces$.

\item
For $X \in \cX$ an object in an $\infty$-category, and for $M$ a continuous monoid, 
an \bit{action of $M$ on $X$}, denoted $M \lacts X$, is an extension
$
\lag X \rag 
\colon 
\ast 
\to 
\fB M 
\xra{ \lag M \lacts X \rag}
\cX
$.
%\[
%\xymatrix{
%\ast 
%\ar[rr]^-{\lag X \rag}
%\ar[d]
%&&
%\cX
%\\
%\fB M
%\ar@{-->}[urr]_-{\lag M \lacts X \rag}
%&&
%.
%}
%\]
The $\infty$-category of \bit{$M$-modules in $\cX$} is
\[
\Mod_M(\cX)
~:=~
\Fun( \fB M , \cX)
~.
\]

\item
For $G\lacts X$ an action of a continuous group on a space, the space of \bit{coinvariants} is the colimit
\[
X_{/G}
~:=~
\colim\bigl(
\sB G
\xra{~\lag G \lacts X \rag~}
\Spaces
\bigr)
~\in~
\Spaces
~.
\]
Should the action $G\lacts X$ be presented by a continuous action of a topological group on a topological space, then this space of coinvariants can be presented by the homotopy-coinvariants.  

\item
We work with $\infty$-operads, as developed in~\cite{HA}.
As so, they are implicitly symmetric.
Some $\infty$-operads are presented as discrete operads, such as ${\sf Assoc}$, while some are presented as topological operads, such as the little 2-disks opera $\cE_2$.




\end{itemize}



\end{conventions*}





\subsection{Moduli and isogeny of framed tori}
Here we state our first result, which identifies the entire symmetries of a framed torus.  



The \bit{braid group on 3 strands} can be presented as
\begin{equation}
\label{e67}
%\begin{equation}\label{e15}
\Braid 
~\cong~
\Bigl \lag~ \tau_{1}~,~ \tau_{2}~ \mid ~ \tau_{1}\tau_{2}\tau_{1} ~=~ \tau_{2} \tau_{1} \tau_{2} ~\Bigr\rag
~.
%\end{equation}
\end{equation}
Through this presentation, there is a standard representation
\begin{equation}
\label{e63}
%\begin{equation}
%\label{e1}
\Phi
\colon
\Braid
\xra{\bigl\lag~\tau_{1} ~\mapsto ~U_1~,~\tau_{2} ~ \mapsto ~U_2  \bigr\rag}
\GL_2(\ZZ)
~,
\qquad
\text{ where }
U_1
~:=~
\begin{bmatrix} 
1 & 1
\\
0 & 1
\end{bmatrix}
\qquad
\text{ and }
\qquad
U_2~:=~
\begin{bmatrix} 
1 & 0
\\
-1 & 1
\end{bmatrix}
~.
\end{equation}
%As recalled in~\S\ref{sec.braids}, there is a standard representation of the braid group on 3 strands,
%\[
%\Phi
%\colon 
%\Braid
%\longrightarrow
%\GL_2(\ZZ)
%~,
%\]
%which 
The homomorphism $\Phi$ defines an action $\Braid \xra{\Phi} \GL_2(\ZZ) \lacts \TT^2$
% and $\Braid \lacts (\CC\PP^\infty)^2$ 
as a topological group.  
This action defines a topological group:
\[
\Braid \ltimes \TT^2
~.
\]


%%% DO THIS <<<<<
The following result, which is essentially due to Milnor, is the starting point of this paper.  
\begin{prop}
[{\color{red} \cite{milnor ... (find the citation ... should be in Rawnsley)}}]
\label{t32}
The image of $\Phi$ is the subgroup $\SL_2(\ZZ)$; the kernel of $\Phi$ is central, and is freely generated by the element $(\tau_{1}\tau_{2})^6  \in \Braid$.
In other words, $\Phi$ fits into a central extension among groups:
\begin{equation}
\label{e46}
%\begin{equation}\label{e16}
1
\longrightarrow
\ZZ
\xra{~\bigl\lag (\tau_{1}\tau_{2})^6 \bigr\rag~}
\Braid
\xra{~\Phi~}
\SL_2(\ZZ)
\longrightarrow  
1
~.~
%\end{equation}
\end{equation}
%
%\end{prop}
%
%
%
%\begin{prop}
%[\cite{??}]
%\label{t30}
Furthermore, this central extension~(\ref{e46}) is classified by the element
\[
\Bigl[{\sf BSL}_2(\ZZ) \xra{\sB~\rm standard} {\sf BSL}_2(\RR) \simeq \sB^2 \ZZ
\Bigr] 
~\in~ 
\sH^2\bigl( \SL_2(\ZZ) ; \ZZ \bigr)
~,
\]
which is to say there is a canonical top horizontal homomorphism making a pullback among groups:
\[
\xymatrix{
\Braid
\ar@{-->}[rr]
\ar[d]_-{\Phi}
&&
\w{\SL}_2(\RR)  \ar[d]^-{\rm universal~cover}
\\
\SL_2(\ZZ)
\ar[rr]_-{\rm standard}^-{\RR\underset{\ZZ}\ot}
&&
\SL_2(\RR)
.
}
\]
%in which the right vertical homomorphism witnesses a universal cover.  
%{\color{red}
%which can be presented as the 2-cocycle
%\[
%\rho\colon
%\SL_2(\ZZ)^{\times 2}
%\longrightarrow
%\ZZ
%~,\qquad
%??
%~.
%\]
%}



\end{prop}






Consider the subgroup $\GL^+_2(\RR)\subset \GL_2(\RR)$ consisting of those $2\times 2$ matrices with positive determinant -- it is the connected component of the identity matrix.  
Consider the submonoid
\[
\RR\underset{\ZZ}\ot
\colon
\EpZ
~\subset~
\GL^+_2(\RR)
\]
consisting of those $2\times 2$ matrices with positive determinant whose entries are integers.
Consider the pullback among monoids:
\begin{equation}
\label{e77}
\xymatrix{
\Ebraid
\ar[d]_-{\Psi} \ar[rr]
&&
\w{\GL}^+_2(\RR)  \ar[d]^-{\rm universal~cover}
\\
\EpZ
\ar[rr]^-{\RR \underset{\ZZ}\ot}
&&
\GL^+_2(\RR)
.
}
\end{equation}
%As we are pulling back along a projection and an inclusion, we see that $\Ebraid = (\EpZ  \times \ZZ) \cap \w{\GL^+_2(\RR)}$ and by an abuse of notation the map $\Psi$ is the natural projection map. \newline \newline
This morphism $\Psi$ supplies a canonical action $\Ebraid \xra{\Psi} \EpZ \lacts \TT^2$ as a topological group.
This action defines a topological monoid
\[
\Ebraid \ltimes \TT^2 
~.
\]
%This is explained more thoroughly in section \S\ref{sec.monoid}.


%\textcolor{blue}{
%\begin{remark}
%In \cite{rawn} the author introduces the circle function $\phi: \SP_{2}(\RR) \rightarrow \SS^{1}$ and an associated function $\eta: (\SP_{2}(\RR))^{\times 2} \rightarrow \RR$ for which the covering group $\w{\SP_{2}(\RR)}$ can be expressed as $\{(g, c) \in \SP_{2}(\RR) \times \RR : \phi(g) = e^{ic}\}$ with the group multiplication given by $(A, i) \cdot (B, j) = (AB, i + j + \eta(A, B)).$ \newline \newline
%We can then extend these methods to give a description of the universal covering group $\w{\GL_{2}^{+}(\RR)}$ by projecting onto $\SP_{2}(\RR),$ take 
%\[
%{\sf proj}: \GL_{2}^{+}(\RR) \rightarrow \SL_{2}(\RR) = \SP_{2}(\RR); \hspace{15pt} A \mapsto \frac{1}{\sqrt{det A}}A.
%\]
%Then we can take $\phi' = \phi \circ {\sf proj}$ and $\eta' = \eta \circ {\sf proj}^{\times 2}$ and define 
%\[
%\w{\GL_{2}^{+}(\RR)} := \{ (g, c) \in \GL_{2}^{+}(\RR) \times \RR : \phi'(g) = e^{ic} \}
%\]
%with group multiplication again given by $(A, i) \cdot (B, j) = (AB, i + j + \eta'(A, B))$ We then define $\Ebraid$ to be the strict pullback 
%\[
%\xymatrix{
%\Ebraid
%\ar[d]_-{\Psi} \ar[rr]
%&&
%\w{\GL^+_2(\RR)}  \ar[d]^-{\rm universal~cover}
%\\
%\EpZ
%\ar[rr]^-{\RR \underset{\ZZ}\ot}
%&&
%\GL^+_2(\RR)
%}
%\]
%where $\EpZ \subset \End_{\sf Groups}(\ZZ^2)$ is the submonoid consisting of those integral $2\times 2$ matrices with positive determinant. As we are pulling back along a projection and an inclusion, we see that $\Ebraid = (\EpZ  \times \ZZ) \cap \w{\GL^+_2(\RR)}$ and by an abuse of notation the map $\Psi$ is the natural projection map. \newline \newline
%The projection supplies a canonical action $\Ebraid \xra{\Psi} \EpZ \lacts \TT^2$ as a topological group.
%This defines a topological monoid
%\[
%\TT^2 \rtimes \Ebraid
%~.
%\]
%The reason for calling this pullback $\Ebraid$ is that it naturally contains $\Braid$ in the sense that pulling back along the inclusion of $\SL_{2}(\ZZ) \hookrightarrow \EpZ$ and $\Psi$ yields $\Braid.$ This is explained more thoroughly in section \S\ref{sec.monoid}.
%\end{remark}
%}


%In~\S\ref{sec.monoid} we introduce the monoid 
%\[
%\Ebraid  %%%% Introduction of name, and preliminary definition DO THIS <<<<<<<<<<<<<<
%~=~
%\EpZ
%\times
%\ZZ
%~,\qquad
%\text{ with }
%\qquad
%(A,i)\cdot (B,j)
%~:=~
%(AB , \rho(AB)+i+j)
%~,
%\]
%central extension of monoids
%\[
%1
%\longrightarrow
%\ZZ
%\longrightarrow
%\Ebraid
%\longrightarrow
%\EpZ
%\longrightarrow
%1
%~,
%\]
%where $\EpZ\subset \End_{\sf Groups}(\ZZ^2)$ is the submonoid consisting of those integral $2\times 2$ matrices with positive determinant, and where 
%\[
%\rho(A,B)
%~:=~
%\begin{cases}
%1~,
%&
%??
%\\
%0~,
%&
%??
%\end{cases}
%~.
%\]
%The projection supplies a canonical action $\Ebraid \xra{\Psi} \EpZ \lacts \TT^2$ as a topological group.
%This defines a topological monoid
%\[
%\TT^2 \rtimes \Ebraid
%~.
%\]





For $\varphi \colon  \tau_{\TT^2} \cong \epsilon^2_{\TT^2}$ a framing of the torus, as Definition~\ref{d1}, we introduce the continuous group $\Diff^{\fr}(\TT^2,\varphi)$ of \bit{framed diffeomorphisms}, and the continuous monoid $\Imm^{\fr}(\TT^2, \varphi)$ of \bit{framed local-diffeomorphisms} of the torus.
%This topological monoid $\Imm^{\fr}(\TT^2, \varphi)$ determines the pointed $\infty$-category $\fB \Imm^{\fr}(\TT^2, \varphi)$, which is its deloop.
%The $\infty$-undercategory $\bigl( \fB \Imm^{\fr}(\TT^2, \varphi) \bigr)^{\ast/}$ can be interpreted as the \emph{divisibility $\infty$-category} of the monoid $\Imm^{\fr}(\TT^2, \varphi)$.  
 







\begin{mythm}{A}%[Theorem~A]
\label{Theorem A}

\begin{enumerate}

\item[~]

\item
The set of framed-diffeomorphism-types of framed tori is canonically a torsor for $\ZZ^2$.  



\item
Let $\varphi\colon \epsilon_{\TT^2} \cong \tau_{\TT^2}$ be a framing of the torus.  

\begin{enumerate}

\item
There is a canonical identification of the continuous group of framed diffeomorphisms of $(\TT^2,\varphi)$ with a semi-direct product of the torus with the braid group on 3 strands:
\[
\Diff^{\fr}(\TT^2,\varphi)
~\simeq~
\Braid \ltimes \TT^2
~.
\]







\item
There is a canonical identification of the continuous monoid of framed local-diffeomorphisms of $(\TT^2,\varphi)$ as a semi-direct product of the torus with the monoid $\Ebraid$:
\[
\Imm^{\fr}(\TT^2,\varphi)
~\simeq~
\Ebraid \ltimes \TT^2
~.
\]


\end{enumerate}




%\item
%There is a canonical identification between $\infty$-categories:
%\[
%{\sf Orbit}_{\TT^2}^{\sf fin}
%~\simeq~
%\bigl( \fB \Imm^{\fr}(\TT^2, \varphi) \bigr)^{\ast/}
%~.
%\]


\end{enumerate}

\end{mythm}


Taking path-components, Theorem~\ref{Theorem A}(2a) has the following immediate consequence.
\begin{cor}
\label{t38}
Let $\varphi$ be a framing of the torus.
There is a canonical identification of the framed mapping class group of $(\TT^2,\varphi)$ as the braid group on 3 strands:
\[
{\sf MCG}^{\fr}(\TT^2,\varphi)
~\cong~
\Braid
~.
\]
\end{cor}


\begin{remark}
Consider the moduli space $\bcM^{\fr}_1$ of framed tori.  
Theorem~\ref{Theorem A}(1)\&(2a) can be phrased as the assertion that $\bcM^{\fr}_1$ has $\ZZ^2$-many path-components, each of which is the space of homotopy coinvariants ${(\CC\PP^\infty)^2}_{{\sf h} \Braid}$ with respect to the action $\Braid \xra{\Phi} \GL_2(\ZZ) \lacts \sB^2 \ZZ^2 \simeq (\CC\PP^\infty)^{\times 2}$.
A neat result of~{\color{red}\cite{??}} gives an isomorphism between groups: 
\[
%\begin{equation}
%\label{e38}
\Braid
~\cong~ 
\pi_1 (\SS^3 \smallsetminus {\sf Trefoil} )
~.
%\end{equation}
\]
%As the space $\SS^3 \smallsetminus {\sf Trefoil}$ is a path-connected 1-type, there results an equivalence between spaces:
%\[
%\sB \Braid
%~\simeq~
%\SS^3 \smallsetminus {\sf Trefoil}
%~.
%\]
Using that $\SS^3 \smallsetminus {\sf Trefoil}$ is a path-connected 1-type, this isomorphism reveals that each path-component $(\bcM^{\fr}_1)_{[\varphi]} \subset \bcM^{\fr}_1$ fits into a fiber sequence:
\[
(\CC\PP^\infty)^2
\longrightarrow
(\bcM^{\fr}_1)_{[\varphi]}
\longrightarrow
\bigl( \SS^3 \smallsetminus {\sf Trefoil} \bigr)
~.
\]


\end{remark}





\begin{remark}
Theorem~\ref{Theorem A}(2a) implies the equivalence-type of the continuous group $\Diff^{\fr}(\TT^2,\varphi)$ is independent of the framing $\varphi$ of the torus.  

\end{remark}


%{\color{red}
%Do you want to forgo mention of framings from here on?
%If so, this `notation' environment should be filled out.
%If not, this notation environment could be omitted.
%\begin{notation}
%...
%We might therefore denote this common continuous group $\Diff^{\fr}(\TT^2)$, with the framing omitted from the notation.  
%
%\end{notation}
%
%}


Theorem~?? of~\cite{??} identifies the oriented mapping class group of a punctured torus with parametrized boundary as the braid group on 3 strands, as it is equipped with a homomorphism to the oriented mapping class group of the torus.  
Together with Corollary~\ref{t38}, this results in an identification between mapping class groups.  
The next result lifts this identification to continuous groups; it is proved in~\S\ref{sec??}.  

%The next result is proved in~\S\ref{sec.??}.


%%%%%%%%%%%
%{\color{red}
%...
%for instance:
%\[
%\Braid = \{ ( A \in  \SL_2(\ZZ) ,  \text{ null homotopies of }
%\ast \xra{\lag A \rag} \SL_2(\ZZ) \to \SL_2(\RR) \simeq {\sf SO}(2) \simeq \SS^1 ) \} .
%\]
%there are $\ZZ \simeq \Omega \SS^1$-many null-homotopies. 
%So $\Ker(\Phi) = \{ \text{ null-htpies of } \lag 1 \rag \colon \ast \xra{\lag \uno \rag} \SL_2(\ZZ) \to \SS^1\} = \Omega_1 \SS^1 \simeq \ZZ$.

%Let $\rho:= \tau_1 \tau_2 \tau_1 \in \Braid$.  
%$\Phi(\rho) = R = {\sf Rot}_{-\pi/2} \in \SL_2(\ZZ) \subset \Diff(\TT^2)$.
%As a \bit{framed}-diffeo of $\TT^2$, $\rho = (R \in \Diff(\TT^2) , \varphi_0 \underset{\gamma} \sim R^\ast \varphi_0\colon \tau \xra{\sD R = R} R^\ast \tau \xra{R^\ast \varphi_0} R^\ast \epsilon^2 = \epsilon^2)$
%where $\gamma$ is the path in $\Fr(\TT^2)$ from $\varphi_0$ to $R^\ast \varphi_0$ which is a path in $\Fr(\TT^2) = \ZZ^2 \times \sO(2)$ from $(0,0,\uno)$ to $(0,0,R)$.
%So this is a path in $\SS^1$ from $1\in \SS^1$ to $-i\in \SS^1$.
%This $\rho = (R,\gamma_{shortest})$.
%So $\rho^4 = (\uno = R^4 , \gamma_{short} \star R\gamma_{short} \star R^2\gamma_{short} \star R^3\gamma_{short})$.
%This path is the generating loop in $\SS^1$ ~! 
%So $\rho^4 \in \Ker(\Phi)$ (since $R^4 = \uno$), and is `given by' the generator of $\ZZ\simeq \Omega \sO(2) \simeq \Omega \Fr(\TT^2)$.
%In as much as $\Fr(\TT^2) = \Map(\TT^2 , \sO(2) )$, is there a local (in $\TT^2$) description of this loop?
%In other words, for $\alpha\in \Omega \Fr(\TT^2)= \Omega \Map(\TT^2 , \sO(2)) = \Omega \Map(\TT^2 , \SS^1)$ to be this loop, can it be described as $\alpha_t\colon \TT^2 \to \SS^1$ is constant outside of a neighborhood of $0\in \TT^2$ for all time $t$?
%Well, I think $\alpha$ is most naturally described as $\alpha_t \colon \TT^2 \xra{!} \ast \xra{{\sf Rot}_{2\pi t}} \sO(2)$.  
%\[
%\xymatrix{
%\epsilon^2 \ar[rr]^-L
%\ar[d]_-{\varphi_0}
%&&
%R^\ast \epsilon^2  = \epsilon^2
%\ar[d]^-{R^\ast \varphi_0}
%\\
%\tau
%\ar[rr]^-{R = \sD R}
%&&
%R^\ast \tau
%}
%\]
%${\sf Rot}_{\pi/2} R^\ast \tau_{\TT^2} \xra{\varphi_0} R^\ast \epsilon^2_{\TT^2}$
%}
%%%%%%%%%%%%%%%%%%%%%%%%%%%


\begin{cor}
\label{t40}
Let $\varphi$ be a framing of the torus.
Fix a smooth framed embedding from the closed 2-disk $\DD^2 \hookrightarrow \TT^2$ extending the inclusion $\{0\} \hookrightarrow \TT^2$ of the origin.
%This determines embeddings $\BB^2 \subset \DD^2 \hookrightarrow \TT^2$ from the open unit ball, and the closed unit disk, resulting in a smooth manifold $\TT^2 \smallsetminus \BB^2$ with boundary.
There are canonical identifications among continuous groups over $\Diff(\TT^2)$:
\[
\Diff^{\fr}(\TT^2~{\sf rel}~0, \varphi)
~
\simeq
~
\Braid
~\simeq~
\Diff(\TT^2 ~{\sf rel}~ \DD^2)
~.
\]
In particular, there are canonical isomorphisms among groups over ${\sf MCG}(\TT^2)$:
\[
{\sf MCG}^{\fr}(\TT^2,\varphi)
~\cong~
\Braid
~\cong~
{\sf MCG}(\TT^2 \smallsetminus \BB^2 ~{\sf rel}~ \partial)
~,
\]
where $\BB^2 \subset \DD^2$ is the open 2-ball. 


\end{cor}



























































\subsection{Natural symmetries of iterated Hochschild homology}
\label{sec.second}


Here we state our second main result, which identifies natural symmetries of \bit{secondary Hochschild homology}, 
\[
\HHt(A)~:=~\sHH\bigl( \sHH(A) \bigr)
~, 
\]
which is simply twice-iterated Hochschild homology, 
of a \bit{2-algebra} 
\[
A~\in~ \Alg_2(\cV)~:=~ \Alg_{\sf Assoc}\bigl( \Alg_{\sf Assoc}(\cV) \bigr)~,\footnote{
Dunn's additivity~\cite{??} (see also Theorem~?? of~\cite{HA}), gives a canonical equivalence between $\infty$-categories $\Alg_{\cE_2}(\cV) \xra{\simeq}\Alg_2(\cV)$ from that of algebras over the operad of little 2-disks, to that of 2-algebras.
In particular, there is a canonical forgetful functor from commutative algebras $\CAlg(\cV) \to \Alg_2(\cV)$, so each commutative algebra canonically determines a 2-algebra.
}
\]
which is simply an (associative) algebra among (associative) algebras in $\cV$.


\begin{notation}
Here, and throughout this Section~\S\ref{sec.second}, we fix $\cV$ to be a $\ot$-presentable symmetric monoidal $\infty$-category.  

\end{notation}












Theorem~\ref{Theorem A}(2a) has the following consequence, proved in~\S\ref{sec.fact.hmlgy} using factorization homology.
\begin{mythm}{B.1} %[Theorem~B1]
\label{t36}
Let $A \in \Alg_2(\cV) $ be a $2$-algebra in a $\ot$-presentable symmetric monoidal $\infty$-category $\cV$.
%
%\begin{enumerate}
%\item
There is a canonical action of the continuous group $\Braid \ltimes \TT^2$ on iterated Hochschild homology:
\begin{equation}
\label{e49}
\Braid \ltimes \TT^2
~\lacts~
\HHt(A)
~.
\end{equation}
%which extends the following familiar symmetries.
%\begin{enumerate}
%\item
%The action $\sB \ZZ^2 \simeq \TT^2 \hookrightarrow \TT^2 \rtimes \Braid
%\underset{(\ref{e49})}{~\lacts~}
%\HHt(A)$
%witnesses the pair of Connes' cyclic operators.
%
%\item
%The action $\ZZ \xla{\lag \tau_{1,2}\rag } \Braid \hookrightarrow \TT^2 \rtimes \Braid
%\underset{(\ref{e49})}{~\lacts~}
%\HHt(A)$
%witnesses the ??
%
%
%\item
%The action $\ZZ \xla{\lag \tau_{2,3}\rag } \Braid \hookrightarrow \TT^2 \rtimes \Braid
%\underset{(\ref{e49})}{~\lacts~}
%\HHt(A) \simeq \HHt(A)$
%witnesses the ??
%
%
%\item
%The action $\ZZ\underset{\rm ??~\ref{??}}{~\cong~} {\sf Ker}(\Phi)  \hookrightarrow \Braid \hookrightarrow \TT^2 \rtimes \Braid
%\underset{(\ref{e49})}{~\lacts~}
%\HHt(A) \simeq \int_{\TT^2} A \simeq \int_{\{0\}\in \TT^2} (A \lacts A)$
%witnesses the ??
%
%
%
%
%\end{enumerate}



%\item
%In the case that the monoidal structure of $\cV$ be Cartesian\footnote{This is to say that the symmetric monoidal structure $\ot = \times$ is given by products.}, there is a further canonical extension
%\begin{equation}
%\label{e50}
%\TT^2 \rtimes \Ebraid
%~\lacts~
%\HHt(A)
%~.
%\end{equation}
%Furthermore, through the fully faithful functor of Corollary~\ref{t30}, the isogeny-action~(\ref{e50}) defines an unstable 2-cyclotomic structure on the iterated Hochschild homology of $A$:
%\[
%\Bigl(
%\TT^2 \rtimes \Ebraid
%~\lacts~
%\HHt(A)
%\Bigr)
%~\in~
%{\sf Cyc}^{{\sf un} (2)}(\cX)
%~.
%\]
%
%
%\end{enumerate}


\end{mythm}






%In fact, the isogeny-action~(\ref{e50}) satisfies a self-similarity condition.  

%\begin{remark}
%Theorem~\ref{Theorem A}(4) can be interpreted as the action $\Ebraid \to \EpZ \lacts {\sf Orbit}^{\sf fin}_{\TT^2}$ witnessing the free lax $\Ebraid$-module generated by $\ast$.  
%
%\end{remark}










%\begin{theorem}
%\label{t11}
%\begin{enumerate}
%\item[~]





%
%
%\item
%Taking image of induced maps on degree-1 homology identifies the quotient
%\[
%{\sf Image}(\sH_1)
%\colon
%\Imm^{\fr}(\TT^2)_{/\Diff^{\fr}(\TT^2)}
%\xra{~\simeq~}
%\Bigl\{
%\Lambda
%\underset{\rm cofin}
%\subset
%\ZZ^2
%\Bigr\}
%~,
%\]
%as the set of cofinite subgroups of $\ZZ^2$.  
%
%
%\item
%Applying degree-1 homology defines a morphism between topological monoids,
%\[
%\sH_1 \colon 
%\Imm^{\fr}(\TT^2)
%\longrightarrow
%\EpZ
%~,
%\]
%to that of positive-determinant endomorphisms of the group $\ZZ^2$.
%\begin{enumerate}
%\item
%This morphism is surjective.
%
%\item
%The preimage of $\uno \in \EpZ$ is canonically identified as the topological group $\TT^2 \times \ZZ$.
%
%\item
%Each other preimage is a torsor with respect to the resulting canonical $\TT^2 \times \ZZ$-action.  
%
%
%\end{enumerate}
%








%There is a canonical identification of the topological monoid $\Imm^{\fr}(\TT^2)$ of framed local-diffeomorphisms of the torus as a pullback involving the monoid of non-zero-determinant endomorphisms of $\ZZ^2$:
%\[
%\xymatrix{
%\Imm^{\fr}(\TT^2)
%\ar[rrrr]
%\ar[d]
%&&
%&&
%\ast
%\ar[d]
%\\
%\TT^2 \rtimes \EZ
%\ar[rr]^-{\pr}
%&&
%\EZ
%\ar[rr]^-{- \underset{\ZZ}\otimes \RR}
%&&
%\GL_2(\RR)\simeq \sO(2)
%~.
%}
%\]

%
%
%\end{enumerate}
%
%
%\end{theorem}




%As discussed in~\S\ref{sec.cyc},
%Theorem~\ref{Theorem A}(3) supplies a coCartesian fibration between $\infty$-categories,
%\[
%\fB \Imm^{\fr}(\TT,\varphi) 
%\longrightarrow
%\fB \Ebraid
%~,
%\]
%classifying the action $\Ebraid \underset{\Psi}\lacts \sB \TT^2 \simeq (\CC\PP^\infty)^2$.  


















%\subsection{Recovering familiar symmetries of $\HHt(A)$}
We now explain how Theorem~\ref{t36} extends familiar, or at least expected, symmetries of $\HHt(A)$, and how the action $\Braid \ltimes \TT^2 \lacts \HHt(A)$ can be phrased in terms of these expected symmetries.  



%Connes' cyclic operator is an action $\TT \simeq \sB \ZZ \lacts \sHH(A)$, which is canonically functorial in the argument $A$.  
%This results in an action $\TT^2 \simeq \sB \ZZ^2 \lacts \HHt(A)$, which is canonically functorial in the $2$-algebra $A$.  
Let $\cW$ be an $\ot$-presentable symmetric monoidal $\infty$-category. 
Let $B$ be an associative algebra in $\cW$.
Taking Hochschild homology with coefficients defines a functor from the $\infty$-category of $(B,B)$-bimodules:
\[
%\Aut_{\Alg(\cV)}(A)
%\longrightarrow
{\sf BiMod}_{(B,B)}
\xra{~\sHH(B,-)~}
\cW
~.
\]
Each endomorphism $B\xra{\sigma}B$ of the associative algebra $B$ determines a $(B,B)$-bimodule structure $B_\sigma$ on the underlying object $B$, which is characterized by $B\xra{\id}B$ being equivariant with respect to $(B,B)\xra{(\id, \sigma )} (B,B)$.  
This assignment $\sigma\mapsto B_\sigma$ canonically assembles as a functor from the space of endomorphisms to the $\infty$-category of $(B,B)$-bimodules:
\[
\End_{\Alg(\cW)}(B)
\longrightarrow
{\sf BiMod}_{(B,B)}
~,\qquad
\sigma
\mapsto 
B_\sigma
~.
\]
This results in a composite functor
\[
\End_{\Alg(\cW)}(B)
\longrightarrow
{\sf BiMod}_{(B,B)}
\xra{~\sHH(B,-)~}
\cW
~,\qquad
\sigma
\mapsto 
\sHH(B,B_\sigma)
~.
\]
This functor restricts to automorphisms of $\id\mapsto \sHH(B,B_{\id}) = \sHH(B)$ as a morphism between continuous groups:
\begin{equation}
\label{e56}
\Omega_{\id} \Aut_{\Alg(\cW)}(B)
~=~
\Omega_{\id} \End_{\Alg(\cW)}(B)
\longrightarrow
\Aut_{\cW}\bigl( \sHH(B) \bigr)
~.
\end{equation}


Now, take $\cW = \Alg(\cV)$ to be the $\infty$-category of associative algebras in an $\ot$-presentable symmetric monoidal $\infty$-category $\cV$, and take $B = \sHH(A)$ to be the Hochschild homology of a $2$-algebra $A\in \Alg\bigl( \Alg(\cV) \bigr)$. 
\bit{Connes' cyclic operator} is a $\TT$-action $\TT\simeq \sB \ZZ \lacts \sHH(A)$, which is canonically functorial in the argument $A$.  
This results in a morphism between continuous groups
\begin{equation}
\label{e57}
\TT
\xra{~\bigl \lag \TT \lacts \sHH(A) \bigr \rag~}
\Aut_{\Alg(\cV)}\bigl( \sHH(A) \bigr)
~.
\end{equation}
We conclude a symmetry of Hochschild homology, which manifests as the composite morphisms between continuous groups:
\[
\ZZ
~\simeq~
\Omega_1 \TT
\xra{~\Omega (\ref{e57}) ~}
\Omega_{\id} \Aut_{\Alg(\cV)}\bigl( \sHH(A) \bigr)
\xra{~(\ref{e56})~}
\Aut_{\cV}\Bigl( \HHt(A) \Bigr)
~.
\]
This symmetry interacts with the two Connes' cyclic operators $\TT^2 \lacts \HHt(A)$ by assembling as a \bit{sheer-action},
\begin{equation}
\label{e60}
{\sf Sheer}_1
\colon
\ZZ
\underset{U_1}\ltimes
\TT^2
~\lacts~
\HHt(A) 
~,
\end{equation}
where this semi-direct product is defined by $\ZZ\xra{\lag  U_1 \rag} \GL_2(\ZZ)\simeq \Aut_{\sf Groups}(\TT^2)$ (see~(\ref{e63})).
Corollary~\ref{t45} states that swapping the two associative algebra structures on $A$ does not effect Hochschild homology.
There results another \bit{sheer-action}:
\begin{equation}
\label{e71}
{\sf Sheer}_2
\colon
\ZZ 
\underset{U_2}\ltimes
\TT^2
\underset{\cong}{\xra{(-1) \ltimes \id}}
\ZZ
\underset{U_2^{-1}}\ltimes
\TT^2
~\lacts~
\HHt(A) 
~,
\end{equation}
where this semi-direct product is defined by $\ZZ\xra{\lag  U_2 \rag} \GL_2(\ZZ)\simeq \Aut_{\sf Groups}(\TT^2)$ (see~(\ref{e63})).
%\begin{equation}
%\label{e59}
%\sB\ZZ^2 \rtimes \ZZ
%\underset{ (\ref{e60}) }\lacts
%\sHH\bigl( \sHH(A)_1 \bigr)_2
%\xra{~\simeq~}
%\int_{\TT^2} A
%\xla{~\simeq~}
%\sHH\bigl( \sHH(A)_2 \bigr)_1
%\underset{ (\ref{e60}) }\racts
%\sB\ZZ^2 \rtimes 
%~.
%\end{equation}


The following result is proved in~\S\ref{sec.??}.
\begin{cor}
\label{t55}
The action of Theorem~\ref{t36} is a canonical extension of these sheer-actions:
for $i=1,2$, 
\[
{\sf Sheer}_i
\colon
\ZZ
\underset{ U_i }\ltimes \TT^2 
\xra{\lag \tau_i \rag \ltimes \id}
\Braid \ltimes \TT^2 
~\underset{\rm Thm~\ref{t36}}\lacts~
\HHt(A)
~.
\]

\end{cor}



\begin{remark}
One phrasing of Theorem~\ref{t36} is that the two actions~(\ref{e60}) and~(\ref{e71}) are not independent.
For starters, these sheer-actions~(\ref{e60}) and~(\ref{e71}) define an action
\begin{equation}
\label{e59}
{\sf Sheers}
\colon
(\ZZ \amalg \ZZ) 
\underset{ U_1,U_2 }\ltimes \TT^2
~\lacts~
\HHt(A)
~.
~
\footnote{This pushout appearing here is in the category of groups, where it is often referred to as a \emph{free product}.}
\end{equation}
Next, the two symmetries $U_1,U_2 \in \GL_2(\ZZ)\simeq \Aut_{\sf Groups}(\TT^2)$ satisfy the relation
$R := U_1U_2U_2 = \begin{bmatrix} 0 & 1 \\ -1 & 0 \end{bmatrix} = U_2 U_1 U_2$. 
Denoting the generators $\lag \tau_{1} , \tau_{2} \rag = \ZZ \amalg \ZZ$, 
this results in two natural actions
\begin{equation}
\label{e72}
\ZZ \underset{R} \ltimes \TT^2
%\overset{\id \rtimes \lag \tau_{1}\tau_{2} \tau_{1} \rag}
%{\underset{\id \rtimes \lag \tau_{2}\tau_{1}\tau_{2} \rag} \rightrightarrows} %DONT LOVE THIS
%\stackrel[\id \rtimes \lag \tau_{1}\tau_{2} \tau_{1} \rag]{a}{\rightrightarrows}
\stackrel[\mathrm{\lag \tau_{2}\tau_{1}\tau_{2} \rag \ltimes \id }]{\mathrm{\lag \tau_{1}\tau_{2} \tau_{1} \rag} \ltimes \id}{\overrightarrow{\underrightarrow{\hspace{2cm}}}}
( \ZZ \amalg \ZZ) \underset{U_1,U_2} \ltimes \TT^2
~\underset{(\ref{e59})}\lacts~
\HHt(A)
~.
\end{equation}
%{\color{red}
%Make two labeled arrows here??
%}\textcolor{blue}{best solution I've found}
Via the standard presentation~(\ref{e67}) of $\Braid$, Theorem~\ref{t36} can be precisely rephrased as an identification of these two actions~(\ref{e72}).  
\end{remark}

%\[
%\xymatrix{
%\sB\Bigl( \sB \ZZ^2
% (\ZZ \overset{\rm Groups}\amalg \ZZ)
%\Bigr)
%\ar[rr]^{ \bigl \lag \sB \ZZ^2
%\rtimes (\ZZ \overset{\rm Groups}\amalg \ZZ)\underset{ (\ref{e59}) }\lacts~
%\HHt(A) \bigr\rag }
%&&
%\Aut_{\cV}\Bigl(
%\HHt(A)
%\Bigr)
%\\
%\sB\Bigl( \sB \ZZ^2
%\rtimes (\ZZ \overset{\rm Groups}\amalg \ZZ)
%\Bigr)
%\ar[rr]^{ \bigl \lag \sB \ZZ^2
%\rtimes (\ZZ \overset{\rm Groups}\amalg \ZZ)\underset{ (\ref{e59}) }\lacts~
%\HHt(A) \bigr\rag }
%&&
%.
%}
%\]


%
%\begin{equation}
%\label{e58}
%&&
%\sHH\bigl( \sHH(A)_1 \bigr)_2
%\ar[drr]^-{\simeq}
%&&
%\\
%\sB\ZZ^2
%\ar[urr]
%\ar[drr]
%&&
%&&
%\int_{\TT^2} A
%~.
%\\
%&&
%\sHH\bigl( \sHH(A)_2 \bigr)_1
%\ar[urr]_-{\simeq}
%&&
%\end{equation}





















Next, the short exact sequence~(\ref{e46}) of Proposition~\ref{t32} implies an identification between moduli spaces:
\[
\Bigl\{
\text{ extensions of $\Braid\lacts \HHt(A)$ along $\Phi$ to an action $\SL_2(\ZZ)\lacts \HHt(A)$ }
\Bigr\}
\]
\[
~\simeq~
\Bigl\{
\text{ trivializations of $\Ker(\Phi) \lacts \HHt(A)$ }
\Bigr\}
~.
\]
{\color{red}
LaTexing issue.
How to break the lines in the above braces into, like, three, and enlarge the braces, so that the above expression can fit into a single (horizontally aligned) display.
}

{\color{magenta}
\begin{remark}
In the case that $\cV = ( \Mod_{\Bbbk} , \underset{\Bbbk} \ot )$ for some commutative ring $\Bbbk$, 
then an extension of $\Braid \lacts \HHt(A)$ along $\Phi$ to $\SL_2(\ZZ) \lacts \HHt(A)$ is obstructed by the element
\[
\bigl[
\HHt(\beta_A)
\bigr]
~\in~
{\sf Ext}^1\bigl( \HHt(A) , \Bbbk[\epsilon] \bigr)
~.
\]

\end{remark}
}


We are interested in identifying the action $\Ker(\Phi)\lacts \HHt(A)$ in familiar, or at least expected, terms.
Corollary~\ref{t54} does just this, in terms of the familiar/expected symmetry of secondary Hochschild homology given by \bit{braiding-conjugation}, as we now explain.  
A starting point for this symmetry is given from the following result which was essentially due to Dunn.
Recall the topological operad $\cE_2$ of little 2-disks, which we regard as an $\infty$-operad in the standard manner.  
\begin{theorem}[{\color{red} \cite{dunn??}, Theorem~?? of~\cite{HA}}]
\label{t52}
There is a canonical equivalence from the $\infty$-category of $\cE_2$-algebras in $\cV$ to that of 2-algebras in $\cV$:
\[
\Alg_{\cE_2}(\cV)
\xra{~\simeq~}
\Alg_2(\cV)
~.
\]

\end{theorem}

After Theorem~\ref{t52}, the standard continuous action $\sO(2) \lacts \cE_2$ on the topological operad immediately implies the following.  
\begin{cor}
\label{t53}
There is a canonical action of the continuous group $\sO(2) \lacts \Alg_2(\cV)$.  
In particular, for each 2-algebra $A$ in $\cV$, 
the orbit map with respect to the action of Corollary~\ref{t53} lends to a canonical 
%there is the orbit-map for the action of Corollary~\ref{t53},
%\[
%{\sf Orbit}_A
%\colon
%\sO(2)
%\xra{~( \id , {\sf const}_A)~}
%\sO(2)
%\times
%\Alg_2(\cV)
%\xra{~\rm action~}
%\Alg_2(\cV)
%~,
%\]
%which determines a 
symmetry of $A$:
\[
\beta_A
\colon
\ZZ
~\simeq~
\Omega_{\uno} {\sf SO}(2)
~=~
\Omega_{\uno} \sO(2)
\xra{~\Omega {\sf Orbit}_{A}~}
\Aut_{\Alg_2(\cV)}(A)
~.
\]


\end{cor}



\begin{remark}
This symmetry $\beta_A$ on each 2-algebra $A$ is \bit{braiding-conjugation}.
For instance, this this symmetry $\beta_A$ is the identity on the underlying object (so, $\beta_A(1) = \id_A$), and for $\mu\in \cE_2(2)$ it supplies the commutativity of the diagram in $\cV$,
\[
\xymatrix{
A \ot A
\ar[rr]^-{ \id \ot \id}
\ar[d]_-{\mu_A}
&&
A \ot A
\ar[d]^-{\mu_A}
&&
\text{ given by the point }
\\
A
\ar[rr]^-{\id}
&&
A
,
&&
\beta_A(2)\colon \ast \xra{\lag 1 \rag} \ZZ \simeq \Omega_{\mu} \cE_2(2) \to \Omega_{\mu_A} \Hom_{\cV}(A\ot A , A)
~.
}
\]
\end{remark}
%
%\begin{cor}
%\label{t41}
%Let $\cV$ be an $\ot$-presentable symmetric monoidal $\infty$-category.
%Let $A \in \Alg_2(\cV)$ be a 2-algebra in $\cV$.


The following result is a direct consequence of Observation~\ref{t43}, 
and inspection of the action $\Braid\lacts \HHt(A)$ of Theorem~\ref{t36}, proved in~\S\ref{sec.??}. 
\begin{cor}
\label{t54}
Let $A$ be a 2-algebra in $\cV$.
Through the action of Theorem~\ref{t36}, the kernel of $\Phi$ acts $\HHt(\beta_A)$:
there is a canonically commutative diagram among continuous groups:
\[
\xymatrix{
\ZZ
\ar[d]^-{\cong}_-{\bigl\lag (\tau_1\tau_2)^6 \bigr\rag}
%\ar[rr]^-{\simeq}
\ar[rrrr]^-{\beta_A}
&&
%\Omega_{\uno} {\sf SO}(2)
&&
\Aut_{\Alg_2(\cV)}(A)
\ar[d]^-{\HHt}
\\
\Ker(\Phi)
\ar[rr]
&&
\Braid
\ar[rr]^-{\rm Thm~\ref{t36}}
&&
\Aut_{\cV}\bigl(
\HHt(A)
\bigr)
.
}
\]

\end{cor}


%
%The action
%\[
%\ZZ
%\underset{(\ref{e??}){\color{red} reference??}}\cong {\sf Ker}(\Phi)
%\longrightarrow
%\Braid
%\xra{~\rm Thm~\ref{Theorem A}~}
%\Diff^{\fr}(\TT^2)
%\underset{\rm Thm~\ref{t36}}{~\lacts~}
%\HHt(A)
%\]
%can be presented as follows.
%Proposition~\ref{t??} grants a canonical identification in $\cV$,
%\[
%\int_{\TT^2} A
%~\simeq~
%\HHt(A)
%~,
%\]
%in which the lefthand side is defined upon regarding $A$ as an $\cE_2$-algebra in $\cV$, via Dunn's additivity~(\cite{dunn}, see also Theorem~?? of~\cite{HA}).
%\[
%\ZZ \simeq \Omega \sO(2)
%\xra{\beta_A} 
%\Aut_{\Alg_{\cE_2}(\cV)}(A)
%\longrightarrow
%\Aut_{\Mod^{\cE_2}_A}(A)
%\xra{\int_{\{0\}\subset \TT^2} (A \lacts -)}
%\Aut_\cV\Bigl(
%\int_{\{0\}\subset \TT^2} (A \lacts A)
%\Bigr)
%~\simeq~
%\Aut_\cV\Bigl(
%\int_{\TT^2} A
%\Bigr)
%%~\simeq~
%%\Aut_\cV\Bigl(
%%\HHt(A)
%%\Bigr)
%~.
%\]
%
%
%\end{cor}
%

















In particular, there is the following immediate consequence of Proposition~\ref{t32}.
\begin{cor}
\label{t37}
Let $A$ be a 2-algebra in $\cV$.
An ${\sf SO}(2)$-fixed-point-structure on $A\in \Alg_2(\cV)$ 
determines a trivialization of the action $\Ker(\Phi) \lacts \HHt(A)$, and thereafter an extension along $\Phi$ of the actions $\Braid \to \Braid \ltimes \TT^2 \lacts \HHt(A)$ to 
actions
\[
\SL_2(\ZZ) \to \SL_2(\ZZ) \ltimes \TT^2 ~ \lacts~ \HHt(A)
~.  
\]

\end{cor}








\begin{remark}
It is not generally true that the action $\Braid \lacts \HHt(A)$ 
extends along $\Phi$ as an action $\SL_2(\ZZ) \lacts \HHt(A)$. 
%This is not generally the case.
%For starters, the forgetful functor
%$
%\Alg\bigl(
%\Alg(\cV) 
%\bigr)^{{\sf SO}(2)}
%\to
%\Alg\bigl(
%\Alg(\cV) 
%\bigr)
%$
%is not generally surjective.
Examples demonstrating this can be derived from~\cite{BZ.Jordan  INTEGRATING QUANTUM GROUPS OVER SURFACES}.

\end{remark}








































%\begin{remark}
%Through Corollary~\ref{t38}, there happens to be an isomorphism between the framed mapping class group and the oriented mapping class group of a punctured torus.  
%\[
%{\sf MCG}^{\fr}(\TT^2,\varphi)
%~\cong~
%\Braid
%~\cong~
%{\sf MCG}^{\sf or}(\TT^2 \smallsetminus 0)
%~,
%\]
%which in fact lies over ${\sf MCG}^{\sf or}(\TT^2)$, the oriented mapping class group of the torus.
%
%\end{remark}













\subsection{Isogenic symmetries of secondary Hochschild homology}

{\color{magenta}
Recall the $\infty$-category ${\sf Orbit}_{\TT^2}^{\sf fin}$ of transitive $\TT^2$-spaces with finite isotropy, and $\TT^2$-equivariant maps between them. 
For $\cX$ an $\infty$-category, the $\infty$-category of \bit{genuine finite $\TT^2$-objects in $\cX$} is
\[
\cX^{{\sf g}_{\sf fin}. \TT^2}
~:=~
\Fun\bigl(
({\sf Orbit}^{\sf fin}_{\TT^2})^{\op} , \cX
\bigr)
~.
\]
The action $\Ebraid \to \EpZ \lacts \TT^2$ as a topological group supplies an action:
\begin{equation}
\label{e48}
\Ebraid^{\op}~ \lacts~ \cX^{{\sf g}_{\sf fin}. \TT^2}
~.
\end{equation}
%which is manifest as the Cartesian fibration
%\begin{equation}
%\label{e48}
%\Bigl(
%\Ebraid^{\op} \lacts \cX^{{\sf g}_{\sf fin}. \TT^2}
%\Bigr)
%~=~
%\Bigl(
%~
%\Fun^{\sf rel}_{\fB \Ebraid}\bigl( \fB ( \TT^2 \rtimes \Ebraid )  ,  \un{\cX} \bigr)
%\longrightarrow
%\fB \Ebraid
%~
%\Bigr)
%~.
%\end{equation}


We propose the following.
\begin{definition}
\label{d2}
%Let $\cX$ be an $\infty$-category.
%An \bit{unstable 2-cyclotomic object in $\cX$} is a genuine-finite $\TT^2$-object in $\cX$ together with the structure of being $\Ebraid$-fixed. 
The $\infty$-category of \bit{unstable 2-cyclotomic objects} in an $\infty$-category $\cX$ is
\[
{\sf Cyc}^{{\sf un} (2)}(\cX)
~:=~
\bigl( \cX^{{\sf g}_{\sf fin}. \TT^2}\bigr)^{{\sf h}\Ebraid}
~.
\]


\end{definition}










%With respect to this action~(\ref{e48}), a (strict) fixed point is a genuine-finite $\TT^2$-object $(\TT^2 \overset{{\sf g}_{\sf fin}}\lacts X) \in \cX^{{\sf g}_{\sf fin}. \TT^2}$ together with a coherent system, indexed by $\w{A}\in \Ebraid$, of identifications between genuine-finite $\TT^2$-objects in $\cX$:
%\[
%X
%~\simeq~
%\w{A}^\ast X
%~.
%%~,\qquad
%%\text{  ~in~}\cX^{{\sf g}_{\sf fin}. \TT^2}
%~.
%\]

Theorem~\ref{Theorem A}(2a) has the following consequence, which is proved in~\S\ref{sec.cyc}.
\begin{cor}
\label{t30}
Let $\cX$ be an $\infty$-category.
Let $\varphi$ be a framing of the torus.
There is a canonical equivalence between the $\infty$-category of $\Imm^{\fr}(\TT^2,\varphi)$-modules in $\cX$ and the left??-lax invariants with respect to the action~(\ref{e48}):
\[
\Mod_{\Imm^{\fr}(\TT^2 , \varphi)} ( \cX )
%~:=~
%\Fun\bigl( \fB \Imm^{\fr}(\TT^2 , \varph) , \cX \bigr)
\xra{~\simeq~}
\bigl( \cX^{{\sf g}_{\sf fin}. \TT^2}\bigr)^{{\sf l.lax} \Ebraid}
~.
\]
In particular, there is a canonical fully faithful functor from the (strict) invariants:
\[
{\sf Cyc}^{{\sf un} (2)}(\cX)
~\hookrightarrow~
\Mod_{\Imm^{\fr}(\TT^2 , \varphi)} ( \cX )
~.
%\Fun\bigl( \fB \Imm^{\fr}(\TT^2 , \varph) , \cX \bigr)
\]


\end{cor}




\begin{remark}
The underpinning of Corollary~\ref{t30} is that the group $\Braid$ has no non-trivial finite subgroups, unlike the group $\SL_2(\ZZ)$ which has the single non-trivial finite subgroup: $\Bigl\lag \begin{bmatrix}-1 & 0 \\ 0 & -1 \end{bmatrix} \Bigr\rag \subset \SL_2(\ZZ)$.

\end{remark}

}




\begin{mythm}{B.2}
\label{t51}

Let $\cX$ be a presentable $\infty$-category in which finite products distribute over colimits separately in each variable.  
Regard $\cX$ as a symmetric monoidal $\infty$-category via the Cartesian symmetric monoidal structure.  
For each 2-algebra $A\in \Alg_2(\cX)$, 
the action~(\ref{e49}) of Theorem~\ref{t36} canonically extends
\begin{equation}
\label{e50}
\Ebraid \ltimes \TT^2
~\lacts~
\HHt(A)
~.
\end{equation}
Furthermore, through the fully faithful functor of Corollary~\ref{t30}, the isogeny-action~(\ref{e50}) defines an unstable 2-cyclotomic structure on the iterated Hochschild homology of $A$:
\[
\Bigl(
\Ebraid \ltimes \TT^2 
~\lacts~
\HHt(A)
\Bigr)
~\in~
{\sf Cyc}^{{\sf un} (2)}(\cX)
~.
\]


\end{mythm}




\begin{remark}
\label{r3}
We explain a relationship between an unstable secondary cyclotomic structure and an iterated unstable cyclotomic structure.
Notice the commutative diagram among groups, 
\[
\xymatrix{
\NN^\times \times \NN^\times
\ar@{-->}[rr]
\ar[d]
&&
\RR_{>0}^{\times} \times \RR_{>0}^{\times}
\ar[d]
\\
\EpZ 
\ar[rr]^-{\RR\underset{\ZZ}\ot}
&&
\GL_2^+(\RR)
,
}
\]
in which the downward homomorphisms are inclusions as diagonal matrices:
$
(r,s)\mapsto \begin{bmatrix} s & 0 \\ 0 & r \end{bmatrix}
~,
$
Contractibility of the continuous group $\RR_{>0}^\times$ thusly supplies a lift
there is a continuous section 
\begin{equation}
\label{e81}
\NN^\times \times \NN^\times
\longrightarrow
\Ebraid
~.
\end{equation}
With respect to this morphism between monoids~(\ref{e81}), the product isomorphism $\TT\times \TT \xra{\times} \TT^2$ is equivariant. 
%Next, taking products defines a functor between $\infty$-categories:
%\[
%{\sf Orbit}^{\sf fin}_{\TT}
%\times
%{\sf Orbit}^{\sf fin}_{\TT}
%\xra{~\times~}
%{\sf Orbit}^{\sf fin}_{\TT^2}
%~,\qquad
%\bigl(
%\TT_{/C}
%,
%\TT_{/D}
%\bigr)
%\mapsto 
%\TT^2_{/C\times D}
%~.
%\]
%This functor is evidently equivariant with respect to the morphism between monoids~(\ref{e81}).
For $\cX$ an $\infty$-category, this results in a forgetful functor from unstable secondary cyclotomic objects to iterated unstable cyclotomic objects:
\[
{\sf Cyc}^{{\sf un} (2)}(\cX)
\longrightarrow
{\sf Cyc}^{\sf un}\bigl( {\sf Cyc}^{\sf un}(\cX) \bigr)
~.
\]
This functor is generally not an equivalence.\footnote{
{\color{blue}
(This footnote could be commented out entirely.)
\\
Indeed, first note that it is evidently $\TT^2$-equivariant. 
Next, consider the commutative square among $\infty$-categories, and restriction functors among them:
\[
\xymatrix{
\Mod_{\w{\GL}_2^+(\QQ)}(\cX)
\ar[rr]
\ar[d]
&&
%\Mod_{\TT^2\rtimes (\NN^\times)^2 }(\cX)^{\TT^2} \simeq 
{\sf Cyc}^{(2)}(\cX)^{\TT^2}
\ar[d]
\\
\Mod_{ (\QQ_{>0}^{\times})^2 }(\cX)
\ar[rr]
&&
%\Mod_{\TT^2\rtimes \Ebraid}(\cX)^{\TT^2} \simeq 
{\sf Cyc}\bigl( {\sf Cyc}(\cX) \bigr) ^{\TT^2}
.
}
\]
Proposition~\ref{t59} implies the horizontal functors are fully faithful.  
To finish, simply note that the left vertical restriction functor is not generally fully faithful, as demonstrated by the homomorphism between group cohomologies with coefficients in a commutative ring $\Bbbk$, 
\[
\sH^\ast( \w{\GL}_2^+(\QQ) ;  \Bbbk )
\longrightarrow
\sH^\ast( ( \QQ_{>0}^\times)^2 ; \Bbbk )
~,
\]
not being injective. 
}
}
\end{remark}



















{\color{blue}
(This whole subsection might reasonably be removed.)

\subsection{Analogy with dimension 1}

Theorems~\ref{Theorem A} and~\ref{t36}, as well as Definition~\ref{d2}, are in precise analogy with the established 1-dimensional array of results and definitions.  
We explain.
\begin{enumerate}
\item
There are pullback squares among monoid-objects in the $\infty$-category $\Spaces$:
\[
\xymatrix{
\NN^\times \ar[rr]
\ar[d]
&&
\End^{{\sf det}\neq 0}(\ZZ)
\ar[d]
&
&
\Ebraid \ar[rr]
\ar[d]
&&
\EZ
\ar[d]
\\
\ast
\ar[rr]
&&
\End^{{\sf det}\neq 0}(\RR) \simeq \sO(1)
&
\text{ and }
&
\ast
\ar[rr]
&&
\End^{{\sf det}\neq 0}(\RR^2) \simeq \sO(2)
.
}
\]


\item
There are pullback squares among $\infty$-categories:
\[
\xymatrix{
{\sf Orbit}_{\TT}^{\sf fin} 
\ar[rr]
\ar[d]
&&
\fB (\NN^\times \ltimes \TT)
\ar[d]
&
&
{\sf Orbit}_{\TT^2}^{\sf fin}  \ar[rr]
\ar[d]
&&
\fB ( \Ebraid \ltimes \TT^2) 
\ar[d]
\\
( \fB  \NN^\times )^{\ast/}
\ar[rr]
&&
\fB  \NN^\times
&
\text{ and }
&
( \fB \Ebraid )^{\ast/}
\ar[rr]
&&
\fB \Ebraid
~,
}
\]
which in particular codify actions as $\infty$-categories:
\[
\NN^\times 
~\lacts~
{\sf Orbit}_{\TT}^{\sf fin} 
\qquad
\text{ and }
\qquad
\Ebraid
~\lacts~
{\sf Orbit}_{\TT^2}^{\sf fin} 
~.
\]


\item
For each $\infty$-category $\cX$, there are identifications between $\infty$-categories of modules and of left-lax fixed-points:
\[
\Mod_{\NN^\times \ltimes \TT}(\cX)
~\simeq~
\bigl( \cX^{{\sf g}_{\sf fin}. \TT}\bigr)^{{\sf l.lax}\NN^\times}
\qquad
\text{ and }
\qquad
\Mod_{\Ebraid \ltimes \TT^2 }(\cX)
~\simeq~
\bigl( \cX^{{\sf g}_{\sf fin}. \TT^2}\bigr)^{{\sf l.lax}\Ebraid }
~.
\]
These identifications yield fully faithful functors
\begin{equation}
\label{e51}
{\sf Cyc}^{\sf un}(\cX)
:=
\bigl( \cX^{{\sf g}_{\sf fin}. \TT}\bigr)^{{\sf h}\NN^\times}
~\hookrightarrow~
\Mod_{\NN^\times \ltimes \TT}(\cX)
\qquad
\text{ and }
\qquad
{\sf Cyc}^{{\sf un} (2)}(\cX)
:=
\bigl( \cX^{{\sf g}_{\sf fin}. \TT^2}\bigr)^{{\sf h}\Ebraid }
~\hookrightarrow~
\Mod_{\Ebraid  \ltimes \TT^2}(\cX)
~.
\end{equation}



\item
There are canonical identifications between monoid-objects in the $\infty$-category $\Spaces$:
\[
\NN^\times \ltimes \TT
\xra{~\simeq~}
\Imm^{\fr}(\TT)
\qquad
\text{ and }
\qquad
\Ebraid \ltimes \TT^2
\xra{~\simeq~}
\Imm^{\fr}(\TT^2)
~.
\]

\item
For $\cV$ a $\ot$-presentably symmetric monoidal $\infty$-category, there are canonical actions
\[
\TT
~\lacts~
\sHH(A)
\qquad
\text{ and }
\qquad
\Braid
\ltimes \TT^2 
~\lacts~
\sHH\bigl( \sHH(A_2) \bigr)
~.
\]
If the monoidal structure of $\cV$ is Cartesian, these actions canonically extend as actions:
\[
\NN^\times \ltimes \TT
~\lacts~
\sHH(A)
\qquad
\text{ and }
\qquad
\Ebraid \ltimes \TT^2 
~\lacts~
\sHH\bigl( \sHH(A_2) \bigr)
~.
\]
Through the fully faithful functors~(\ref{e51}), these extended actions define objects:
\[
\Bigl(
\NN^\times \ltimes \TT 
~\lacts~
\sHH(A)
\Bigr)
~\in~
{\sf Cyc}^{\sf un}(\cV)
\qquad
\text{ and }
\qquad
\Bigl(
 \Ebraid \ltimes \TT^2
~\lacts~
\sHH\bigl( \sHH(A_2) \bigr)
\Bigr)
~\in~
{\sf Cyc}^{{\sf un} (2)}(\cV)
~.
\]


\end{enumerate}
}











\subsection{Remarks on secondary cyclotomic trace}


We see the role of Corollary~\ref{t30} as informing an approach to secondary cyclotomic traces.  

Let $\Bbbk$ be a commutative ring spectrum.
Let $A\in \Alg_2(\Mod_{\Bbbk})$.
%\footnote{Recall that a commutative $\Bbbk$-algebra forgets as a associative algebra in associative $\Bbbk$-algebras.}
Recall the $\Bbbk$-linear Dennis trace map: $\sK(A) \xra{{\sf tr}} \sHH(A)$~(see~\cite{??}).
The cyclic trace map is a canonical factorization of this Dennis trace map through negative-cyclic homology
$
\sK(A)
\xra{{\sf tr}^{\TT}}
{\sHH}^-(A)
:=
\sHH(A)^{\TT}
$
~(see~\cite{??  look in ``cyclo'' for a citation}).
Iterating this cyclic trace map results in a map between $\Bbbk$-modules:
\[
\sK\bigl( \sK(A) \bigr)
\xra{~{\sf tr}^{\TT}( {\sf tr}^{\TT})~}
\sHH^-\bigl( \sHH^-(A)\bigr)
~.
\]

We expect the work of~\cite{reuben} on universal properties of secondary $\sK$-theory to yield a solution to the following.
\begin{conj}
\label{c1}
The iterated cyclic trace map factors through the $\Braid$-fixed points:
\[
\xymatrix{
&&
\sHH\bigl( \sHH( A ) \bigr)^{\Braid \ltimes \TT^2} \ar[d] 
\\
\sK\bigl(\sK(A) \bigr)
\ar[rr]^-{~{\sf tr}^{\TT}( {\sf tr}^{\TT})~}
\ar@{-->}[urr]
&&
\sHH^-\bigl( \sHH^-(A) \bigr)
.
}
\]




\end{conj}

%We expect that the construction of the Dennis trace map lends to a canonical factorization~(a) in the diagram below.
%The universal property of secondary $\sK$-theory, as articulated in~\cite{mazel.gee.ruben}, thereafter results in a factorization~(b):
%\begin{equation*}
%%\label{e36} 
%{\Small
%\xymatrix{
%\Bbbk \Bigl \lag \Obj\Bigl( \Perf\bigl( \Perf(A) \bigr) \Bigr) \Bigr \rag
%\ar@{-->}[rr]^-{\rm (a)} \ar[dd]
%&&
%\sHH\Bigl( \sHH\bigl( \Perf\bigl( \Perf(A) \bigr) \bigr) \Bigr)^{{\sf h} ( \TT^2 \rtimes \Braid) }  \ar[d]
%&&
%\sHH\bigl( \sHH( A ) \bigr)^{{\sf h} ( \TT^2 \rtimes \Braid) } \ar[d] \ar[ll]^-{\simeq}
%\\
%&&
%\sHH\Bigl( \sHH\bigl( \Perf\bigl( \Perf(A) \bigr)^{{\sf h} \TT}  \bigr) \Bigr)^{{\sf h} \TT}  \ar[d]
%&&
%\sHH\bigl( \sHH( A )^{{\sf h} \TT} \bigr)^{{\sf h} \TT} \ar[d] \ar[ll]^-{\simeq}
%\\
%\sK\bigl( \sK( A ) \bigr) \ar[rr]^-{{\sf tr}({\sf tr})}
%\ar[urr]^-{{\sf tr}^-({\sf tr}^-)}
%\ar@(-,u)@{-->}[uurr]^-{\rm (b)}
%&&
%\sHH\Bigl( \sHH\bigl( \Perf\bigl( \Perf(A) \bigr) \bigr) \Bigr)
%&&
%\sHH\bigl( \sHH( A ) \bigr) \ar[ll]^-{\simeq}_-{\rm Morita~invariance}
%.
%}
%}
%\end{equation*}
For the case in which $\Bbbk = \SS$ is the sphere spectrum, where standard notation is $\sHH = {\sf THH}$ and referred to as \emph{topological} Hochschild homology, 
the cyclic trace map factors further through the topological cyclotomic homology,
$\sK(A) \xra{{\sf tr}^{\sf Cyc}} {\sf TC}(A) := {\sf THH}(A)^{\sf Cyc}$, 
%\[
%\xymatrix{
%&&
%\sHH( A )^{\sf Cyc} \ar[d] 
%\\
%\sK(A)
%\ar[rr]^-{~ {\sf tr}^{\TT}~}
%\ar@{-->}[urr]^-{{\sf tr}^{\sf Cyc}}
%&&
%\sHH(A)^{\TT}
%,
%}
%\]
which is the \emph{cylotomic} fixed-points with respect to a canonical \emph{cyclotomic structure} on topological Hochschild homology.  
Iterating this cyclotomic trace map results in a map between spectra:
\[
\sK\bigl( \sK(A) \bigr)
\xra{~{\sf tr}^{\sf Cyc}( {\sf tr}^{\sf Cyc})~}
{\sf TC}\bigl( {\sf TC}(A)\bigr)
~.
\]
Following the developments in~\ref{cyclo}, we further expect Definition~\ref{d2} to lend to a definition of a (stable) secondary cyclotomic object, and that secondary topological Hochschild homology
canonically admits a secondary cyclotomic structure.  
For secondary topological cyclotomic homology to be the fixed-points with respect to this structure, ${\sf TC}^{(2)}(A):={\sf THH}\bigl( {\sf THH}(A) \bigr)^{\sf Cyc^{(2)}}$, we expect the work~\cite{reuben} on secondary $\sK$-theory to further lend a secondary cyclotomic trace map, which we state as the following.
\begin{conj}
\label{c2}
The iterated cyclotomic trace map factors through the secondary topological cyclotomic homology, compatibly with the factorization of Conjecture~\ref{c1}:
\[
\xymatrix{
\sK\bigl(\sK( A) \bigr)
\ar@(d,-)[ddrr]_-{\rm Conj~\ref{c1}}
\ar@{-->}[drr]^-{{\sf tr}^{\sf Cyc^{(2)}}}
\ar@(u,-)[drrrr]^-{~{\sf tr}^{\sf Cyc}( {\sf tr}^{\sf Cyc})~}
&&
&&
\\
&&
{\sf TC}^{(2)}(A)
\ar[rr]
\ar[d]
&&
{\sf TC}\bigl( {\sf TC}(A) \bigr)
\ar[d]
\\
&&
{\sf THH}\bigl( {\sf THH}(A) \bigr)^{ \Braid \ltimes \TT^2 }
\ar[rr]
&&
{\sf THH}^-\bigl( {\sf THH}^-(A) \bigr)
~.
}
\]
\end{conj}


\begin{remark}
One might be encouraged by Remark~\ref{r3} to expect that the secondary cyclotomic trace map ${\sf tr}^{{\sf Cyc}^{(2)}}$ of Conjecture~\ref{c2} is locally-constant (in the variable $A$), thereby correcting the failure for the iterated cyclotomic trace map ${\sf tr}^{\sf Cyc}( {\sf tr}^{\sf Cyc})$ to be locally-constant. 
However, we do not expect for this to be so.
Namely, the local-constancy of the cyclotomic trace map $\sK(A) \xra{{\sf tr}^{\sf Cyc}} {\sf TC}(A)$ relies in an essential way on calculations of~\cite{hesselholt}, which identify the fiber of the canonical map ${\sf TC}(A \ltimes V) \to {\sf TC}(A)$ associated to a square-zero extension of $A$.  
These calculations in turn rely on the simple observation that, for each $i \geq 0$, the canonical action $\TT \simeq \Diff^{\fr}(\TT)\lacts {\sf Conf}_i(\TT)_{\Sigma_i}$ on unordered configuration space canonically factors as a $\TT_{/\sC_i}$-torsor.  
Because the canonical action $\Braid \ltimes \TT^2 \simeq \Diff^{\fr}(\TT^2) \lacts {\sf Conf}_{i}(\TT^2)_{\Sigma_i}$ does not apparently have any such property, 
we do not expect for the secondary cyclotomic trace map of Conjecture~\ref{c1} to be locally constant.  

\end{remark}




%{\color{red}
%Dark Seid
%by Elon Musks wife
%}














\section{Moduli and isogeny of framed tori}




\subsection{Moduli and isogeny of tori}

Vector addition, as well as the standard vector norm, gives $\RR^2$ the structure of a topological abelian group.
Consider its closed subgroup $\ZZ^2\subset \RR^2$.  
The \bit{torus} is the quotient in the short exact sequence of topological abelian groups:
\[
0
\longrightarrow
\ZZ^2
\xra{\rm inclusion}
\RR^2 
\xra{~\quot~}
\TT^2
\longrightarrow
0
~.
\] 
Because $\RR^2$ is connected, and because $\ZZ^2$ acts cocompactly by translations on $\RR^2$, the torus $\TT^2$ is connected and compact.  
The quotient map $\RR^2 \xra{\quot} \TT^2$ endows the torus with the structure of a Lie group, and in particular a smooth manifold.
Consider the submonoid
\begin{equation*}
%\label{e47}
\EZ
~:=~
\Bigl\{
\ZZ^2 \xra{A} \ZZ^2 \mid {\sf det}(A) \neq 0
\Bigr\}
~\subset~
\End_{\sf Groups}(\ZZ^2)
~,
\end{equation*}
consisting of the cofinite endomorphisms of the group $\ZZ^2$.  
Using that the smooth map $\RR^2 \xra{\quot} \TT^2$ is a covering space and $\TT^2$ is connected, there is a canonical continuous action on the topological group:
\begin{equation}
\label{e2}
\EZ
~\lacts~
\TT^2
~,\qquad
A \cdot q
~:=~
\quot( A\w{q} )
\qquad
{\Small \bigl(\text{for any }\w{q} \in \quot^{-1}(q) \bigr) }
~.
\footnote{
Note that (\ref{e2}) indeed does not depend on $\w{q} \in \quot^{-1}(q)$.
}
\end{equation}
This homomorphism~(\ref{e2}) defines a semi-direct product topological monoid:
\[
%\begin{equation}
%\label{e6}
\EZ \ltimes \TT^2
~.
%\end{equation}
\]

%I'm sure this could be said better //// Where to place this... 
% \noindent Even more $\RR^2 \xra{\quot} \TT^2$ is a local diffeomorphism. So for every $\tilde{q} \in \RR^{2}$ the map $$D_{\tilde{q}}\quot: T_{\tilde{q}}\RR^{2} \rightarrow T_{q}\TT^{2}$$ is an isomorphism and we can define the inverse isomorphism $$D_{q}\quot^{-1}: T_{q}\TT^{2} \rightarrow T_{\tilde{q}}\RR^{2}$$ for $\tilde{q} = \bigl( \quot^{-1}(q) \cap [0, 1) \times [0, 1) \bigr).$

%Transition 

Consider the topological monoid of smooth local-diffeomorphisms of the torus:
\[
\Imm(\TT^2)
~\subset~
\Map(\TT^2 , \TT^2)
~,
\] 
which is endowed with the subspace topology of the $\sC^\infty$-topology on the set of smooth self-maps of the torus.

\begin{observation}
\label{t21}
\begin{enumerate}

\item[~]

\item
The standard inclusion $\GL_2(\ZZ) \hookrightarrow \EZ$ witnesses the maximal subgroup.
It follows that the standard inclusion $\GL_2(\ZZ) \ltimes \TT^2 \hookrightarrow \EZ \ltimes \TT^2$ witnesses the maximal subgroup, both as topological monoids and as monoid-objects in the $\infty$-category $\Spaces$.


\item
The standard monomorphism $\Diff(\TT^2) \hookrightarrow \Imm(\TT^2)$ witnesses the maximal subgroup, both as topological monoids and as monoid-objects in the $\infty$-category $\Spaces$.

\end{enumerate}
\end{observation}






Consider the morphism between topological monoids:
\begin{equation}
\label{e7}
\Aff: \EZ \ltimes \TT^2
\longrightarrow
\Imm(\TT^2)
~,\qquad
(A, p)
\mapsto
\Bigl(
q\mapsto 
Aq + p
\Bigr)
~.
\end{equation}
%Observation~\ref{t21} implies this morphism~(\ref{e7}) restricts as a morphism between topological groups:
%\begin{equation}
%\label{e7}
%\Aff: \TT^2 \rtimes \GL_2(\ZZ)
%\longrightarrow
%\Diff(\TT^2)
%~,\qquad
%(p,A)
%\mapsto
%\Bigl(
%q\mapsto 
%Aq + p
%\Bigr)
%~.
%\end{equation}









We record the following classical result.
\begin{lemma}\label{t1}
The restriction of the morphism~(\ref{e7}) to maximal subgroups is a homotopy-equivalence:
\[
\Aff \colon
 \GL_2(\ZZ) \ltimes \TT^2
\xra{~\simeq~}
\Diff(\TT^2)
~,\qquad
(A, p)
\mapsto
\Bigl(
q\mapsto 
Aq + p
\Bigr)
~.
\]
\end{lemma}


\begin{proof}
Let $G$ be a locally path-connected topological group, which we regard as a continuous group.
Denote by $G_{\uno} \subset G$ the path-component containing the identity element in $G$.
This subspace $G_{\uno}\subset G$ is a normal subgroup, and the sequence of continuous homomorphisms
\[
1
\longrightarrow
G_{\uno}
\xra{~\rm inclusion~}
G
\xra{~\rm quotient~}
\pi_0(G)
\longrightarrow
1
\]
is a fiber-sequence among continuous groups.
This fiber sequence is evidently functorial in the argument $G$.
In particular, there is a commutative diagram among topological groups 
\[
\xymatrix{
1 \ar[d]_-= \ar[r]
&
\TT^{2} = \bigl( \GL_2(\ZZ) \ltimes \TT^{2} \bigr)_{\uno}
\ar[d]_-{\Aff_{\uno}}
\ar[r]^-{\rm inc}
&
\GL_2(\ZZ) \ltimes \TT^{2} 
\ar[d]_-{\Aff} 
\ar[r]^-{\rm quot}
&
\pi_0 \bigl( \GL_2(\ZZ) \ltimes \TT^{2} \bigr)
=
\GL_2(\ZZ)
\ar[d]_-{\pi_0(\Aff)}
\ar[r]
&
1 \ar[d]_-=
\\
1
\ar[r]
&
\Diff(\TT^2)_{\uno}
\ar[r]^-{\rm inc}
&
\Diff(\TT^2) 
\ar[r]^-{\rm quot}
&
\pi_0 \bigl( \Diff(\TT^2) \bigr)
\ar[r]
&
1
.
}
\] 
Now, by the 5-lemma applied to homotopy groups, if both horizontal sequences are exact, if the vertical homomorphisms $\Aff_{\uno}$ and $\pi_0(\Aff)$ are homotopy equivalences, then the vertical homomorphism $\Aff$ is a homotopy equivalence as well.



Theorem 2.D.4 of~\cite{rolf}, along with the Kirby torus trick {\color{red} Improve citation here}, implies $\pi_0(\Aff)$ is an isomorphism.  



It remains to show $\Aff_{\uno}$ is a homotopy equivalence. 
With respect to the canonical continuous action $\Diff(\TT^2)_{\uno} \lacts \TT^2$, 
the orbit of the identity element $0\in \TT^2$ is the evaluation map 
\begin{equation}\label{e25}
\ev_0
\colon 
\Diff(\TT^2)_{\uno}
\longrightarrow
\TT^2
~.
\end{equation}
Note that the composition,
\[
\id
\colon
\TT^2
\xra{~\Aff_{\uno}~}
\Diff(\TT^2)_{\uno}
\xra{~(\ref{e25})~}
\TT^2
~,
\]
is the identity map.
Now, Theorem 1b of \cite{ee} gives that the subspace of those diffeomorphisms of the torus that fix $0\in \TT^2$ has contractible path-components.
This implies the stabilizer ${\sf Stab}_0\bigl(\Diff(\TT^2)_{\uno}\bigr)$ is contractible, which implies $\Aff_{\uno}$ is a homotopy equivalence.  



\end{proof}






\begin{remark}
\label{r10}
By the classification of compact surfaces, the moduli space $\bcM_1$ of smooth tori is path-connected, and as so is 
\[
\bcM_1
~\simeq~
{\sf BDiff}(\TT^2)
\underset{\rm Lem~\ref{t1}}{~\simeq~}
\sB\bigl(
\GL_2(\ZZ) \ltimes \TT^2
\bigr)
~.
\]
In particular, this path-connected moduli space fits into a fiber sequence
\[
(\CC\PP^\infty)^2
\longrightarrow
\bcM_1
\longrightarrow
{\sf BGL}_2(\ZZ)
~,
\]
which is classified by the standard action $\GL_2(\ZZ) \lacts \sB^2 \ZZ^2 \simeq (\CC\PP^\infty)^2$.

\end{remark}







%Recall from~(\ref{e47}) the submonoid $\EZ \subset \End_{\sf Groups}(\ZZ^2)$ consisting of those integral $2\times 2$ matrices with non-zero determinant.  
Consider the set of \bit{cofinite subgroups} of $\ZZ^2$:
\begin{equation*}
%\label{e61} 
\bcL(2) 
~:=~
\Bigl\{
\Lambda
\underset{\rm cofin}
\subset
\ZZ^2
\Bigr\}
~.
\end{equation*}
\begin{observation}
\label{t22}

\begin{enumerate}

\item[~]

\item
The orbit-stabilizer theorem immediately verifies the composite map 
$
\EZ \ltimes \TT^2
\xra{\pr}
\EZ
\xra{\rm Image} 
\bcL(2)
$
witnesses the quotient:
\[
\EZ_{/\GL_2(\ZZ) \ltimes \TT^2 } \ltimes \TT^2
\xra{~\cong~}
\EZ_{/\GL_2(\ZZ)}
\xra{~\cong~}
\bcL(2)
~.
\]


\item
Using that each finite-sheeted cover over $\TT^2$ is diffeomorphic with $\TT^2$, the classification of covering spaces implies the map given by taking the image of degree-1 homology
$
\Imm(\TT^2)
\xra{{\sf Image}(\sH_1)}
\bcL(2)
$
witnesses the quotient:
\[
\Imm(\TT^2)_{/\Diff(\TT^2)}
\xra{~\cong~}
\bcL(2)
~.
\]


\item
The diagram
\[
\xymatrix{
\EZ \ltimes \TT^2
\ar[rr]^-{\Aff}
\ar[d]_-{\pr}
&&
\Imm(\TT^2) 
\ar[d]^-{{\rm Image}(\sH_1)}
\ar[dll]_-{\sH_1}
\\
\EZ
\ar[rr]^-{\rm Image}
&&
\bcL(2)
}
\]
commutes.

\end{enumerate}


\end{observation}














%After Lemma~\ref{t1}, Observation~\ref{t22} implies the following classical result.
\begin{cor}
\label{t22}
The morphism (\ref{e7}) between topological monoids is a homotopy-equivalence:
\[
\Aff
\colon
\EZ \ltimes \TT^2
\xra{~\simeq~}
\Imm(\TT^2)
~.
\]

\end{cor}

\begin{proof}

Consider the morphism between fiber sequences in the $\infty$-category $\Spaces$:
\[
\xymatrix{
\EZ \ltimes \TT^2
\ar[rr]^-{\rm quotient}
\ar[d]_-{\Aff}
&&
\bigl(
\EZ \ltimes \TT^2
\bigr)_{\GL_2(\ZZ) \ltimes \TT^2}
\ar[rr]
\ar[d]^-{\Aff_{\Aff}}
&&
\sB\bigl( \GL_2(\ZZ) \ltimes \TT^2 \bigr)
\ar[d]^-{\sB \Aff}
\\
\Imm(\TT^2)
\ar[rr]^-{\rm quotient}
&&
\Imm(\TT^2)_{/\Diff(\TT^2)}
\ar[rr]
&&
\sB \Diff(\TT^2)
.
}
\]
Lemma~\ref{t1} implies the right vertical map is an equivalence.
Observation~\ref{t22} implies the middle vertical map is an equivalence.
It follows that the left vertical map is an equivalence, as desired.



\end{proof}



















\subsection{Framings}



A \bit{framing} of the torus is a trivialization of its tangent bundle: $\varphi\colon  \tau_{\TT^2} \cong  \epsilon^2_{\TT^2}$.
Consider the topological \bit{space of framings} of the torus:
\[
\Fr(\TT^2)
~:=~
\Iso_{\Bdl_{\TT^2}}\bigl( \tau_{\TT^2} , \epsilon^2_{\TT^2}  \bigr)
~\subset~
\Map(  \sT \TT^2  , \TT^2 \times \RR^2 )
~,
\]
which is endowed with the subspace topology of the $\sC^\infty$-topology on the set of smooth maps between total spaces.  
The quotient map $\RR^2\xra{\quot}\TT^2$ endows the smooth manifold $\TT^2$ with a \bit{standard framing} $\varphi_0$:
for 
\[
\trans\colon \TT^2 \times \TT^2 
\xra{~(p,q)\mapsto \trans_p(q) := p+q~} 
\TT^2
\]
the abelian multiplication rule of the Lie group $\TT^2$, 
\[
%\begin{equation}
%\label{e5}
(\varphi_0)^{-1}
\colon
\epsilon^2_{\TT^2}
\xra{~\cong~}
\tau_{\TT^2}
~,\qquad
\TT^2 \times \RR^2 \ni (p,v)
\mapsto
\bigl(p,\sD_{0}(\trans_p \circ \quot) (v) \bigr)
\in \sT \TT^2
~.
%\end{equation}
\]
%In this way we regard $\TT^2$ as a \bit{framed smooth 2-manifold}.



The next sequence of observations culminates as an identification of this space of framings.
\begin{observation}
\label{t20}


\begin{enumerate}
\item[~]


\item
Postcomposition gives the topological space $\Fr(\TT^2)$ the structure a left-torsor for the topological group $\Iso_{\Bdl_{\TT^2}}\bigl( \epsilon^2_{\TT^2} , \epsilon^2_{\TT^2} \bigr)$.
In particular, the orbit map of a of framing $\varphi \in \Fr(\TT^2)$ is a homeomorphism:
\begin{equation}
\label{e41}
\Iso_{\Bdl_{\TT^2}}\bigl( \epsilon^2_{\TT^2} , \epsilon^2_{\TT^2} \bigr)
\xra{~\cong~}
\Fr(\TT^2)
~,\qquad
\alpha \mapsto 
\alpha \circ \varphi 
~.
\end{equation}


\item
Consider the topological space $\Map \bigl( \TT^2 , \GL_2(\RR) \bigr)$ of smooth maps from the torus to the standard smooth structure on $\GL_2(\RR)$, which is endowed with the $\sC^\infty$-topology. 
The map
\begin{equation}
\label{e43}
\Map \bigl( \TT^2 , \GL_2(\RR) \bigr)
\xra{~\cong~}
\Iso_{\Bdl_{\TT^2}}\bigl( \epsilon^2_{\TT^2} , \epsilon^2_{\TT^2} \bigr)
~,\qquad
a
\mapsto 
\Bigl(
\TT^2\times \RR^2
\xra{\bigl( p,v \bigr) \mapsto \bigl( p,a_p(v) \bigr) }
\TT^2 \times \RR^2
\Bigr)
~,
\end{equation}
is a homeomorphism.

\item
The map to the product with based maps,
\begin{equation}
\label{e44}
\Map \bigl( \TT^2 , \GL_2(\RR) \bigr)
\xra{~\cong~}
\Map_\ast\Bigl( ( 0\in \TT^2) , ( \uno \in \GL_2(\RR) \bigr) \Bigr)
\times
\GL_2(\RR)
~,\qquad
a
\mapsto 
\bigl(
~
a(0)^{-1} a~ ,~ a(0)
~
\bigr)
~,
\end{equation}
is a homeomorphism.


\item
Because both of the spaces $\TT^2$ and $\GL_2(\RR)$ are 1-types with the former path-connected, 
the map,
\[
\pi_1
\colon
\Map_\ast\Bigl( ( 0\in \TT^2) , ( \uno \in \GL_2(\RR) \bigr) \Bigr)
\xra{~\simeq~}
{\sf Homo}\Bigl( \pi_1\bigl( 0 \in \TT^2 \bigr) , \pi_1\bigl( \uno \in \GL_2(\RR) \bigr) \Bigr) 
~,
\]
is a homotopy-equivalence.

\item
Evaluation on the standard basis for $\pi_1(0\in \TT^2) \xra{\cong} \pi_1(0\in \TT)^2 \cong \ZZ^2$ defines a homeomorphism:
\begin{equation}
\label{e45}
{\sf Homo}\Bigl( \pi_1\bigl( 0 \in \TT^2 \bigr) , \pi_1\bigl( \uno \in \GL_2(\RR) \bigr) \Bigr) 
\xra{~\cong~}
\pi_1\bigl( \uno \in \GL_2(\RR)^2 \bigr)
~\cong~
\ZZ^2
%~,\qquad
%g
%\mapsto 
%\Bigl(
%~
%g\bigl( [ \TT \times \{1\}] \bigr)
%~,~
%g\bigl( [ \{1\}\times  \TT ] \bigr)
%~
%\Bigr)
~.
\end{equation}

\end{enumerate}

\end{observation}




Observation~\ref{t20}, together with the Gram-Schmidt homotopy-equivalence $\sO(2) \xra{\simeq} \GL_2(\RR)$, yields the following.
\begin{cor}
\label{t25}
A choice of framing $\varphi \in \Fr(\TT^2)$ determines a composite homotopy-equivalence:
\begin{eqnarray*}
%\label{e42}
\Fr(\TT^2) 
&
\underset{\simeq}{
\xla{~(\ref{e43})\circ (\ref{e41})~}
}
&
\Map\bigl( \TT^2 , \GL_2(\RR) \bigr)
\\
\nonumber
&
\underset{\simeq}{
\xra{~(\ref{e44})~}
}
&
\Map_\ast \Bigl( \bigl( 0 \in \TT^2 \bigr) , \bigl( \uno \in \GL_2(\RR) \bigr) \Bigr) \times \GL_2(\RR) 
\\
\nonumber
&
\underset{\simeq}{
\xra{~\pi_1 \times \id ~}
}
&
{\sf Homo}\Bigl( \pi_1\bigl( 0 \in \TT^2 \bigr) , \pi_1\bigl( \uno \in \GL_2(\RR) \bigr) \Bigr) \times \GL_2(\RR) 
\\
\nonumber
&
\underset{\simeq}{
\xra{(\ref{e45}) \times \id }
}
&
\ZZ^2 \times \GL_2(\RR)
\\
\nonumber
&
\underset{\simeq}{
\xla{ \id \times {\sf GS}}
}
&
\ZZ^2 \times \sO(2)
~.
\end{eqnarray*}

\end{cor}




















\subsection{Moduli of framed tori}


Consider the map:
\[
\Act\colon
\Fr(\TT^2)
\times
\Imm(\TT^2)
\longrightarrow
\Fr(\TT^2)
~,
\]
\[
( \varphi , f )
\mapsto 
\Bigl(
~
\tau_{\TT^2} 
\underset{\cong}{\xra{\sD f}}
f^\ast \tau_{\TT^2}
\underset{\cong}{ \xra{f^\ast \varphi} }
f^\ast \epsilon^2_{\TT^2}
=
\epsilon^2_{\TT^2}
~
\Bigr)
~.
\]

\begin{lemma}
\label{t50}
The map $\Act$ is a continuous right action of the topological monoid $\Imm(\TT^2)$ on the topological space $\Fr(\TT^2)$.
In particular, there is a continuous right action of the topological group $\Diff(\TT^2)$ on the topological space $\Fr(\TT^2)$. 

\end{lemma}

\begin{proof}
Consider the topological subspace of the topological space of smooth maps between total spaces of tangent bundles, which is endowed with the $\sC^\infty$-topology,
\[
{\sf Bdl}^{\sf fw.iso}(\tau_{\TT^2} , \tau_{\TT^2})
~\subset~
\Map\bigl( \sT \TT^2 , \sT \TT^2 \bigr)
~,
\]
consisting of the smooth maps between tangent bundles that are fiberwise isomorphisms.
Notice the factorization
\[
\Act\colon
\Fr(\TT^2)
\times
\Imm(\TT^2)
\xra{\id \times \sD}
\Fr(\TT^2)
\times 
{\sf Bdl}^{\sf fw.iso}(\tau_{\TT^2} , \tau_{\TT^2})
\xra{\circ}
\Fr(\TT^2)
\]
as first taking the derivative, followed by composition of bundle morphisms.  
The definition of the $\sC^\infty$-topology is so that the first map in this factorization is continuous. 
The second map in this factorization is continuous because composition is continuous with respect to $\sC^\infty$-topologies.
We conclude that $\Act$ is continuous.  

We now show that $\Act$ is an action.
Clearly, for each $\varphi \in \Fr(\TT^2)$, there is an equality $\Act( \varphi  , \id) = \varphi$.
Next, let $g,f\in \Imm(\TT^2)$, and let $\varphi \in \Fr(\TT^2)$.
The chain rule, together with universal properties for pullbacks, gives that the diagram among smooth vector bundles
\[
\xymatrix{
\tau_{\TT^2}
\ar@(u,u)[rrrrrr]^-{\sD (g\circ f)}
\ar[rr]_-{\sD g}
%\ar[d]
&&
g^\ast 
\tau_{\TT^2}
\ar[rr]_-{g^\ast \sD f}
%\ar[d]
&&
f^\ast g^\ast \tau_{\TT^2}
\ar[d]^-{f^\ast g^\ast \varphi}
\ar[rr]_-=
&&
(g\circ f)^\ast \tau_{\TT^2}
\ar[d]^-{(g \circ f)^\ast \varphi}
\\
\epsilon^2_{\TT^2}
&&
g^\ast \epsilon^2_{\TT^2}
\ar[ll]_-=
&&
f^\ast g^\ast \epsilon^2_{\TT^2}
\ar[ll]_-=
&&
(g \circ f)^\ast \epsilon^2_{\TT^2}
\ar[ll]_-=
\ar@(d,d)[llllll]^=
}
\]
commutes.
Inspecting the definition of $\Act$, the commutativity of this diagram implies the euqlity $\Act\bigl( \Act(\varphi, g) , f \bigr) = \Act( \varphi , g\circ f)$, as desired.  



\end{proof}




\begin{remark}
\label{r8}
The space of homotopy-coinvariants with respect to this conjugation action $\Act$ can be interpreted as the \bit{moduli space of framed tori}:
\[
\bcM^{\fr}_1
~:=~
\Fr(\TT^2)_{/ \Diff(\TT^2)}
~.
%\footnote{Here, we use the subscript ${\sf h}$ just for emphasis: we care considering the homotopy-coinvariants for this continuous right action $\Diff(\TT^2) \lacts \Fr(\TT^2)$.  
%}
\]


\end{remark}




%\begin{remark}
%\label{r9}
%In diagrams, the value $\Act(\varphi,f)$ is the framing
%\begin{equation}\label{e299}
%\xymatrix{
%\TT^{2} \times \RR^{2} 
%\ar[d] \ar[rr]^-{f^{-1} \times \id} 
%&& 
%\TT^{2} \times \RR^{2} 
%\ar[rr]^-{\varphi} 
%\ar[d] 
%&& 
%{\sT}\TT^{2} 
%\ar[rr]^-{\sD f} 
%\ar[d] 
%&& 
%{\sT}\TT^{2} 
%\ar[d] 
%\\
%\TT^{2} 
%\ar[rr]^-{f^{-1}} 
%&& 
%\TT^{2}  
%\ar[rr]^-{\id} 
%&& 
%\TT^{2}
%\ar[rr]^-{f} 
%&& 
%\TT^{2}
%~.
%}
%\end{equation} 
%
%\end{remark}


\begin{observation}\label{t4}
Through Corollary~\ref{t25}, the action $\Act$ is compatible with familiar actions.
Specifically, a choice of framing $\varphi \in \Fr(\TT^2)$ fits $\Act$ into a commutative diagram among topological spaces:
\begin{equation*}%\label{e19}
\xymatrix{
\Fr(\TT^2)
\times
\Imm(\TT^2)
\ar[rrr]^-{\Act}
&
&&
\Fr(\TT^2)
\\
\Map\bigl( \TT^2 , \GL_2(\RR) \bigr)
\times
\bigl(
\EZ \ltimes \TT^2 
\bigr)
\ar[u]^-{ {\rm Cor}~\ref{t25} \times \Aff}_-{\simeq}
\ar[r]^-{ \id \times \pr }
\ar[d]_-{ {\rm Cor}~\ref{t25} \times \id}^-{\simeq}
&
\Map\bigl( \TT^2 , \GL_2(\RR) \bigr)
\times
\EZ 
\ar[rr]^-{\rm value-wise}_-{\rm multiply}
\ar[d]^-{\id \times {\rm Cor}~\ref{t25}}_-{\simeq}
&&
\Map\bigl( \TT^2 , \GL_2(\RR) \bigr)
\ar[d]^-{{\rm Cor}~\ref{t25}}_-{\simeq}
\ar[u]_-{{\rm Cor}~\ref{t25}}^-{\cong}
\\
\bigl( \ZZ^2 \times \GL_2(\RR) \bigr)
\times
\bigl(
\EZ \ltimes \TT^2 
\bigr)
\ar[r]^-{ \id \times \pr }
&
\bigl( \ZZ^2\times  \GL_2(\RR) \bigr)
\times
\EZ 
\ar[rr]^-{(a,b,A;B)\mapsto (a,b,AB)}
&&
%\underset{\rm swap}
%\cong
%\ZZ^2 \times \bigl( \GL_2(\RR)\times \GL_2(\RR) \bigr)
\ZZ^2 \times \GL_2(\RR) 
.
}
\end{equation*}


\end{observation}







\begin{cor}
\label{t26}
A choice of framing $\varphi \in \Fr(\TT^2)$ canonically determines an identification in the $\infty$-category $\Spaces$ of the homotopy-coinvariants for $\Act$,
\[
\Fr(\TT^2)_{/\Diff(\TT^2)} 
\xra{~\simeq~}
\ZZ^2
\times
\Bigl(
{(\CC\PP^\infty)^2}_{ \Omega
\bigl( \GL_2(\RR)_{/\GL_2(\ZZ)} \bigr)
}
\Bigr)
~,
\]
through which $\varphi$ selects the $(0,0)$-path-component. 
Each path-component in this expression fits into the fiber sequence,
\[
(\CC\PP^\infty)^2
\longrightarrow
{(\CC\PP^\infty)^2}_{/\Omega
\bigl( \GL_2(\RR)_{/\GL_2(\ZZ)} \bigr)
}
\longrightarrow
\bigl( \GL_2(\RR)_{/\GL_2(\ZZ)} \bigr)
~,
\]
which is classified by the action
\[
\Omega
\bigl( \GL_2(\RR)_{/\GL_2(\ZZ)} \bigr)
\xra{~\pi_0~}
\pi_1
\bigl( \GL_2(\RR)_{/\GL_2(\ZZ)} \bigr)
\longrightarrow
\GL_2(\ZZ)
~\lacts~
\sB^2 \ZZ^2 \simeq (\CC\PP^\infty)^2
~.
\]
Furthermore, the resulting projection is given by
\[
\Fr(\TT^2)_{/\Diff(\TT^2)}
\longrightarrow
\ZZ^2
~,\qquad
[\varphi']
\mapsto
\Bigl[
\TT \vee \TT = \sk_1(\TT^2)
\xra{ \varphi^{-1} \varphi' _{|\sk_1(\TT^2)} }
\GL_2(\RR) 
\Bigr]
\in \pi_1\bigl(\uno \in \GL_2(\RR)\bigr)^2 \cong \ZZ^2
~.
\]



\end{cor}


\begin{proof}
We explain the following sequences of identifications in the $\infty$-category $\Spaces$.
\begin{eqnarray}
\nonumber
\Fr(\TT^2)_{/\Diff(\TT^2)} 
&
\underset{\rm Obs~\ref{t4}}
{~\simeq~}
&
\Bigl(
\ZZ^2
\times
\GL_2(\RR) 
\Bigr)_{\GL_2(\ZZ) \ltimes \TT^2}
\\
\nonumber
&
\underset{\rm triv~on~\ZZ^2}
{~\simeq~}
&
\ZZ^2
\times
\Bigl(
\GL_2(\RR) 
\Bigr)_{ \GL_2(\ZZ) \ltimes \TT^2 }
\\
\nonumber
&
\underset{\rm iterate~coinvariants}
{~\simeq~}
&
\ZZ^2
\times
\Bigl(
\bigl(
\GL_2(\RR) 
\bigl)_{\TT^2}
\Bigr)_{ \GL_2(\ZZ)}
\\
\nonumber
&
\underset{\rm \TT^2-action~trivial}
{~\simeq~}
&
\ZZ^2
\times
\Bigl(
(\CC\PP^\infty)^2\times \GL_2(\RR) 
\Bigr)_{ \GL_2(\ZZ)}
\\
\nonumber
&
\underset{\rm \Omega-Puppe}
{~\simeq~}
&
\ZZ^2
\times
\Bigl(
(\CC\PP^\infty)^2
\Bigr)_{ \Omega \bigl( \GL_2(\RR)_{/\GL_2(\ZZ)} \bigr) }
~.
\end{eqnarray}
The first identification follows from Observation~\ref{t4}.
The bottom horizontal map in Observation~\ref{t4} reveals that the action $\GL_2(\ZZ) \ltimes \TT^2 \lacts \ZZ^2 \times \GL_2(\RR)$ is the diagonal action of the trivial action on $\ZZ^2$ and the action \[
\GL_2(\ZZ) \ltimes \TT^2 \xra{~\pr~} \GL_2(\ZZ) \xra{~\rm include~} \underset{\rm left~mult}\lacts \GL_2(\RR)
~.
\]
This supplies the second identification.  
The third identification is iteration of coinvariants.
The fourth identification follows because the restricted $\TT^2$-action is trivial.  
The final identification is an instance of the following general fact.

Let $H \to G$ be a morphism between continuous groups that is $\pi_0$-surjective.
Let $H\lacts X$ be an action on a space.  
The $\Omega$-Puppe sequence supplies a canonical morphism 
$
\Omega (G_{/H})
\to 
H
$
between continuous groups.
Specifically, this morphism witnesses the stabilizer of $\ast \xra{\rm unit} G$ with respect to the action $H \to G \underset{\rm left~trans}\lacts G$:
\[
\xymatrix{
\Omega (G_{/H})
\ar[rr]
\ar[d]
&&
H
\ar[d]
\\
\ast
\ar[rr]^-{\rm unit}
&&
G
.
}
\]
In particular, there is a canonical $\Omega (G_{/H})$-equivariant map
\[
X
~\simeq~
X\times \ast
\xra{ \id \times {\rm unit}}
X \times G
~.
\]
Taking coinvariants lends a canonically commutative diagram among spaces:
\[
\xymatrix{
X_{ \Omega (G_{/H}) }
\ar[d]
\ar[r]
&
(X \times  G)_{/H}
\ar[r]
\ar[d]
&
X_{/H}
\ar[d]
\\
\sB \Omega (G_{/H})
\ar[r]
&
G_{/H}
\ar[r]
&
\sB H
.
}
\]
Because groupoid-objects are effective in the $\infty$-category $\Spaces$,
the coinvariants functor,
\[
\Fun ( \sB H , \Spaces )
\longrightarrow
\Spaces_{/\sB H}
~,\qquad
(H \lacts X)
\mapsto 
( X_{/H} \to \sB H)
~,
\]
is an equivalence between $\infty$-categories.
In particular, it preserves products.
It follows that the right square witnesses a pullback.  
By definition of coinvariants of the restricted action $\Omega (G_{/H}) \to H \lacts X$, the outer square is a pullback.  
The connectivity assumption on the morphism $H\to G$ implies the left bottom horizontal map is an equivalence.  
We conclude that the left top horizontal map is also an equivalence, as desired.  



\end{proof}






























For $\varphi\in \Fr(\TT^2)$ a framing of the torus, consider the orbit map of $\varphi$ for this continuous action of Lemma~\ref{t50}:
\[
%\begin{equation}\label{e10}
{\sf Orbit}_\varphi
\colon
\Imm(\TT^2)
\xra{~( ~ {\sf constant}_{\varphi}~ , ~\id ~ )~}
\Fr(\TT^2)
\times 
\Imm(\TT^2)
\xra{~{\sf Act}~}
\Fr(\TT^2)
~,\qquad
f
\mapsto {\sf Act}(\varphi , f)
~.
%\end{equation}
\]

\begin{observation}
\label{t33}
After Observation~\ref{t4}, for each framing $\varphi \in \Fr(\TT^2)$, 
the orbit map for $\varphi$ fits into a commutative diagram among topological spaces:
\begin{equation*}%\label{e19}
\xymatrix{
\Diff(\TT^2) 
\ar[rr]
\ar@{-->}[dr]^-{\sH_1}
&&
\Imm(\TT^2)
\ar[rrrr]^-{{\sf Orbit}_\varphi}
\ar@{-->}[dr]^-{\sH_1}
&&
&&
\Fr(\TT^2)
\ar[d]^-{{\rm Cor}~\ref{t25}}_-{\simeq}
\\
&
\GL_2(\ZZ) 
\ar[rr] 
&&
\EZ
\ar[r]^-{\RR\underset{\ZZ}\ot }_-{\rm standard}
&
\GL_2(\RR)
\ar[rr]^-{ \lag (0,0) \rag \times \id}
&&
\ZZ^2 \times \GL_2(\RR)
\\
\GL_2(\ZZ) \ltimes \TT^2
\ar[uu]^-{\Aff}
\ar[rr]
\ar[ur]_-{\pr}
&&
\EZ \ltimes \TT^2
\ar[uu]^(.35){\Aff} | \hole
\ar[ur]_-{\pr}
&&
.
}
\end{equation*}
%{\color{red}
%LaTex issue:
%\\
%\textcolor{blue}{fixed} How to move the ``$\Aff$'', so it doesn't interfere with the horizontal arrow??
%}

The existence and commutativity of the fillers follows form Observation~\ref{t22}.

\end{observation}

\begin{remark}
\label{r5}
The point-set fiber of ${\sf Orbit}_{\varphi}$ over $\varphi$, which is the point-set stabilizer of the action $\Fr(\TT^2) \racts \Imm(\TT^2)$ of Lemma~\ref{t50}, 
consists of those local-diffeomorphisims $f$ for which the diagram among vector bundles, 
\[
\xymatrix{
\tau_{\TT^2}
\ar[rr]^-{\varphi}
\ar[d]_-{\sD f}
&&
\epsilon^2_{\TT^2}
\\
f^\ast 
\tau_{\TT^2}
\ar[rr]^-{f^\ast \varphi}
&&
f^\ast
\epsilon^2_{\TT^2}
\ar[u]_-=
,
}
\]
commutes.
In the case of the standard framing $\varphi_0$, a local-diffeomorphism $f$ satisfies this rigid condition if and only if $f = {\sf trans}_{f(0)} \circ {\sf quot}_C$ is translation in the group $\TT^2$ after a group-theoretic quotient $\TT^2 \xra{\rm quotient} \TT^2$. 
In particular, the point-set fiber of $\bigl( {\sf Orbit}_{\varphi_0} \bigr)_{|\Diff(|TT^2)}$ over $\varphi_0$ is $\TT^2$, and the homomorphism $\TT^2 \hookrightarrow \Diff(\TT^2)$ witnesses the inclusion of those diffeomorphisms that \emph{strictly} fix $\varphi_0$.  


On the other hand, the \emph{homotopy-}fiber of~${\sf Orbit}_{\varphi_0}$ over $\varphi_0$ is more flexible: 
it consists of pairs $(f, \gamma)$ in which $f$ is a local-diffeomorphism and $\gamma$ is a homotopy 
\[
\varphi_0
~ \underset{\gamma}\sim ~
\Act(\varphi_0,f)
~.
\]
As we will see, every orientation-preserving local-diffeomorphism $f$ admits a lift to this homotopy-fiber.  
In particular, small perturbations of such $f$, such as multiplication by bump functions in neighborhoods of $\TT^2$, can be lifted to this homotopy-fiber.
\end{remark}




\begin{definition}\label{d1}
Let $\varphi\in \Fr(\TT^2)$ be a framing of the torus.  
The space of \bit{framed local-diffeomorphisms}, and the space of \bit{framed diffeomorphisms}, of the framed smooth manifold $(\TT^2,\varphi)$ are respectively the pullbacks in the $\infty$-category $\Spaces$:
\begin{equation*}%\label{e11}
\xymatrix{
\Imm^{\sf fr}(\TT^2,\varphi)
\ar[rr]
\ar[d]
&&
\Imm(\TT^2)
\ar[d]^-{{\sf Orbit}_\varphi}
&&
\Diff^{\sf fr}(\TT^2,\varphi)
\ar[rr]
\ar[d]
&&
\Diff(\TT^2)
\ar[d]^-{{\sf Orbit}_\varphi}
\\
\ast
\ar[rr]^-{\lag \varphi \rag}
&&
\Fr(\TT^2)
&
\text{ and }
&
\ast
\ar[rr]^-{\lag \varphi \rag}
&&
\Fr(\TT^2)
~.
}
\end{equation*}
In the case that the framing $\varphi = \varphi_0$ is the standard framing, we simply denote
\[
\Imm^{\sf fr}(\TT^2)
~:=~
\Imm^{\sf fr}(\TT^2,\varphi_0)
\qquad
\text{ and }
\qquad
\Diff^{\sf fr}(\TT^2)
~:=~
\Diff^{\sf fr}(\TT^2,\varphi_0)
~.
\]
\end{definition}





%%%%%%%%%%%%%%%%%%%%%%%%%%%%%%%%%%%%%%%%%%%%%%%%%%



\begin{lemma}\label{t2}
Let $G \lacts X$ be an action of a continuous on a space $X$.
Let $\ast \xra{\lag x \rag } X$ be a point in $X$.
Consider stabilizer of $x$, which is the fiber of the orbit map of $x$:
\begin{equation}
\label{e52}
\xymatrix{
{\sf Stab}_G(x)
\ar[rrrr]
\ar[d]
&&
&&
\ast
\ar[d]^-{\lag x \rag}
\\
G \simeq G\times \ast 
\ar[rr]^-{ \id \times \lag x \rag} 
\ar@(d,d)[rrrr]_-{{\sf Orbit}_x}
&&
G \times X 
\ar[rr]^-{\sf act} 
&&
X
.
}
\end{equation}
There is a canonical identification in $\Spaces$ between this stabilizer and the based-loops at $[x]\colon \ast \xra{\lag x \rag} X \xra{\rm quotient} X_{/G}$ of the $G$-coinvariants,
\[
{\sf Stab}_G(x)
~\simeq~
\Omega_{[x]} X_{/G}
~,
\]
through which the resulting composite morphism $\Omega_{[x]} X_{/G}
\simeq {\sf Stab}_G(x)
\to 
G
$
canonically lifts to one between continuous groups.

\end{lemma}



\begin{proof}
By definition of a $G$-action, the orbit map $G\xra{{\sf Orbit}_x} X$ is canonically $G$-equivariant.
Taking $G$-coinvariants supplies an extension of the commutative diagram~(\ref{e52}) in $\Spaces$:
\[
%\begin{equation}\label{e18}
\xymatrix{
{\sf Stab}_G(x) 
\ar[rr] \ar[d]
&&
G 
\ar[d]^-{{\sf Orbit}_x} \ar[rr]^-{\rm quotient}
&&
G_{/G} 
\simeq 
\ast
\ar[d]^-{({\sf Orbit}_x)_{G}}
\\
\ast \ar[rr]^-{\lag x\rag}
&&
X
\ar[rr]^-{\rm quotient}
&&
X_{/G}
.
}
%\end{equation}
\]
Through the identification $G_{/G} \simeq \ast$, the right vertical map is identified as $\ast\xra{\lag [x] \rag} X_{/G}$.
Using that groupoids in $\Spaces$ are effective, the right square is a pullback.  
Because the lefthand square is defined as a pullback, it follows that the outer square is a pullback.
The identification ${\sf Stab}_G(x) \simeq \Omega_{[x]} X_{/G}$ follows.  
In particular, the space ${\sf Stab}_G(x)$ has the canonical structure of a continuous groups.


Now, this continuous group ${\sf Stab}_G(x)$ is evidently functorial in the argument $G \lacts X \ni x$.  
In particular, the unique $G$-equivariant morphism $X\xra{!} \ast$ determines a morphism between continuous groups
\[
{\sf Stab}_x(X)
\longrightarrow
{\sf Stab}_{\ast}(\ast)
~\simeq~
G
~.
\]


\end{proof}

\begin{cor}\label{t3}
Let $\varphi \in \Fr(\TT^2)$ be a framing.
There space $\Diff^{\fr}(\TT^2 , \varphi)$ is canonically endowed with the structure of a continuous group over $\Diff(\TT^2)$.
With respect to this structure, 
there is a canonical identification,
\[
\Diff^{\fr}(\TT^2, \varphi)
~\simeq~
\Omega_{[\varphi]} \bigl( \Fr(\TT^2)_{/\Diff(\TT^2)} \bigr)
~.
\]


\end{cor}






%\begin{proof}
%The commutative diagram among spaces~(\ref{e11}) extends as a commutative diagram in $\Spaces$:
%\begin{equation}\label{e18}
%\xymatrix{
%\Diff^{\fr}(\TT^2,\varphi) \ar[rr] \ar[d]
%&&
%\Diff(\TT^2) \ar[d]^-{{\sf Orbit}_\varphi} \ar[rr]^-{\rm quotient}
%&&
%\Diff(\TT^2)_{/\Diff(\TT^2)} \ar[d]^-{({\sf Orbit}_\varphi)_{/\Diff(\TT^2)}}
%\\
%\ast \ar[rr]^-{\lag \varphi\rag}
%&&
%\Fr(\TT^2)
%\ar[rr]^-{\rm quotient}
%&&
%\Fr(\TT^2)_{/\Diff(\TT^2)}
%,
%}
%\end{equation}
%in which the rightmost terms are homotopy-coinvariants.  
%Observe the canonical contractibility $\Diff(\TT^2)_{/\Diff(\TT^2)}\simeq\ast$, with respect to which the right vertical map selects $[\varphi] \in \Fr(\TT^2)_{/\Diff(\TT^2)}$.
%Through this contractibility, the right square is a homotopy pullback square.  
%Because the lefthand square is defined as a pullback square, it follows that the outer square is a  pullback square.
%The result follows.
%
%\end{proof}







\begin{observation}
\label{t43}
Let $\varphi \in \Fr(\TT^2)$ be a framing.
The kernel of $\Phi$ acts by rotating framing, which is to say 
there is a canonically commutative diagram among continuous groups:
\[
\xymatrix{
\ZZ
\ar[d]^-{\cong}_-{\bigl\lag (\tau_1\tau_2)^6 \bigr\rag}
\ar[rr]^-{\simeq}
&&
\Omega_{\uno} \GL_2(\RR)
\ar[rr]^-{\Omega \bigl( A\mapsto A \cdot \varphi \bigr)}
&&
\Omega_{\varphi} \Fr(\TT^2)
\ar[d]
\\
\Ker(\Phi)
\ar[rr]
&&
\Braid
\ar[rr]^-{\Aff^{\fr}}
&&
\Diff^{\fr}(\TT^2, \varphi)
.
}
\]
%
%\end{cor}
%
%\begin{proof}
%%[Proof of Corollary~\ref{t43}]
Indeed, there is a canonically commutative diagram among spaces, in which each row is an $\Omega$-Puppe sequence:
\[
\xymatrix{
\Ker(\Phi)
\ar[rr]
\ar[d]
&&
\Braid
\ar[rr]^-{\Phi}
\ar[d]^-{\Aff^{\fr}}
&&
\GL_2(\ZZ)
\ar[rr]^-{\RR\underset{\ZZ}\ot}
\ar[d]^-{\Aff}
&&
\GL_2(\RR)
\ar[d]^-{\rm Rotate~the~framing~\varphi}
\\
\Omega_{\varphi} \Fr(\TT^2)
\ar[rr]
&&
\Diff^{\fr}(\TT^2,\varphi)
\ar[rr]
&&
\Diff(\TT^2)
\ar[rr]^-{{\sf Orbit}_{\varphi}}
&&
\Fr(\TT^2)
.
}
\]
By definition of $\Diff^{\fr}(\TT^2,\varphi)$, both of the horizontal sequences are $\Omega$-Puppe sequences.
The result follows.  

\end{observation}











































\subsection{Braid group and braid monoid}
\label{sec.monoid}
Here we collect some facts about the braid group on 3 strands, and the braid monoid on 3 strands.





%Notice the commutative diagram among topological monoids
%\begin{equation*}
%%\label{e53}
%\xymatrix{
%\SL_2(\ZZ)
%\ar[d]_-{\rm inclusion}
%\ar[rr]^-{\RR \underset{\ZZ}\ot}
%&&
%\SL_2(\RR)
%\ar[d]^-{\rm inclusion}
%\\
%\EpZ
%\ar[rr]^-{\RR \underset{\ZZ}\ot}
%&&
%\End^{{\sf det} > 0}(\RR^2)
%,
%}
%\end{equation*}
%where the bottom right term is the connected-component of $\uno \in \GL_2(\RR)$.  
%
%
%
%
%\begin{definition}
%\label{d3}
%The \bit{braid group on 3 strands}, and respectively the monoid which we refer to as \bit{EndBraid}, are the pullbacks among monoids:
%\[
%\xymatrix{
%\Braid
%\ar[rr]
%\ar[d]_-{\Phi}
%&&
%\Ebraid
%\ar[d]_-{\Psi} \ar[rr]
%&&
%\w{\GL}^+_2(\RR)  \ar[d]^-{\rm universal~cover}
%\\
%\SL_2(\ZZ)
%\ar[rr]
%&&
%\EpZ
%\ar[rr]^-{\RR \underset{\ZZ}\ot}
%&&
%\GL^+_2(\RR)
%.
%}
%\]
%
%\end{definition}
%
%
%
%
%
%
%
%
%
%
%
%
%
%The homomorphisms~$\Phi$ and $\Psi$ determine actions on the topological group $\TT^2$: 
%\begin{equation*}
%%\label{e3}
%\Braid
%\xra{~\Phi~}
%\SL_2(\ZZ)
%\hookrightarrow
%\GL_2(\ZZ)
%\underset{~(\ref{e2})~}
%\lacts
%\TT^2
%\qquad
%\text{ and }
%\qquad
%\Ebraid
%\xra{~\Psi~}
%\EpZ
%\hookrightarrow
%\EZ
%\underset{~(\ref{e2})~}
%\lacts
%\TT^2
%~.
%\end{equation*}
%These actions define a semi-direct product topological group and a semi-direct product topological monoid, and Definition~\ref{d3} supplies an inclusion between them:
%\begin{equation}
%\label{e4}
%\TT^2 \rtimes \Braid
%\longrightarrow
%\TT^2 \rtimes \Ebraid
%~.
%\end{equation}



\begin{remark}
We give an explicit description of $\Ebraid$.
In~\cite{rawn}, the author gives an explicit description for the universal cover of $\SP_{2}(\RR) = \SL_{2}(\RR)$ (and goes on to establish the pullback square of Proposition~\ref{t32}).
Following those methods, consider function
\[
\phi
\colon
\GL_2(\RR)
\longrightarrow
\SS^1
~,\qquad
A \mapsto \frac{(a + d) + i(b - c)}{|(a + d) + i(b - c)|}
~,
\]
and the argument map
%\emph{circle function} $\phi: \SL_{2}(\RR) \rightarrow \SS^{1}$ and an associated function 
\[
\eta 
\colon
(\GL_{2}(\RR))^{\times 2} 
\longrightarrow 
\RR
~,\qquad
(A, B) \mapsto
\arg\Bigl(
1 - \alpha_{A}\overline{\alpha_{B^{-1}}}
\Bigr)
~
\]
where
\[
A =
\begin{bmatrix}
a & b 
\\
c & d
\end{bmatrix}
\text{ and }
\alpha_{A} = \frac{a^{2} + c^{2} - b^{2} - d^{2} - 2i(ad + bc)}{(a + d)^{2} + (b - c)^{2}}.
\]
Then the universal cover can be expressed as the subset 
\[
\w{\GL}^+_{2}(\RR)
~:=~
\bigl\{
(A, c)  \mid  \phi(A) = e^{ic}
\bigr\}
~\subset~
\GL_2(\RR)\times \RR
~,~
\text{with group-law }(A, c) \cdot (B, d) := \bigl(AB, c + d + \eta(A, B) \bigr)
~.
\]
%We can then extend these methods to give a description of the universal covering group $\w{\GL}_{2}^{+}(\RR)$ by first projecting onto $\SP_{2}(\RR),$ take 
%\[
%{\sf proj}: \GL_{2}^{+}(\RR) \rightarrow \SL_{2}(\RR) = \SP_{2}(\RR); \hspace{15pt} A \mapsto \frac{1}{\sqrt{det A}}A.
%\]
%Then we can take $\phi' = \phi \circ {\sf proj}$ and $\eta' = \eta \circ {\sf proj}^{\times 2}$ and define 
%\[
%\w{\GL}_{2}^{+}(\RR) := \{ (g, c) \in \GL_{2}^{+}(\RR) \times \RR : \phi'(g) = e^{ic} \}
%\]
%with group multiplication again given by $(A, i) \cdot (B, j) = (AB, i + j + \eta'(A, B)).$
So the monoid
\[
\Ebraid
~:=~
\bigl\{
(A, c)  \mid  \phi(A) = e^{ic}
\bigr\}
~\subset~
\EpZ \times \RR
~,~
\text{with monoid-law }(A, c) \cdot (B, d) := \bigl(AB, c + d + \eta(A, B) \bigr)
~.
\]
\end{remark}


\begin{observation}
\label{t39}
The inclusion $\SL_2(\ZZ)\subset \EpZ$ between submonoids of $\GL_2^+(\RR)$ determines an inclusion between topological monoids:
\begin{equation}
\label{e4}
\Braid \ltimes \TT^2
\longrightarrow
\Ebraid \ltimes \TT^2 
~.
\end{equation}
After Observation~\ref{t21}, this inclusion~(\ref{e4}) witnesses the maximal subgroup, both as topological monoids and as monoid-objects in the $\infty$-category $\Spaces$.

\end{observation}

































\begin{observation}
In $\Braid$, there is an identity of the generator of $\Ker(\Phi)$:
\[
(\tau_{1} \tau_{2} \tau_{1})^4 
=
( \tau_{1} \tau_{2})^6
=
(\tau_{2} \tau_{1} \tau_{2})^4
\in {\sf Ker}(\Phi)
~.
\]
For that matter, since the matrix
\begin{equation}
\label{e64}
R~:=~ U_1 U_2 U_3 ~=~ 
\begin{bmatrix}
0 & 1 
\\
-1 & 0
\end{bmatrix}
~=~
U_2 U_1 U_2
~\in~\GL_2(\ZZ)
\end{equation}
implements rotation by $-\frac{\pi}{2}$, 
then $R^4 = \uno$ in $\GL_2(\ZZ)$.

\end{observation}



For $\RR\underset{\QQ}\ot \colon \GL_2^+(\QQ)\subset \GL_2^+(\RR)$ the subgroup with rational coefficients, consider the pullback among groups:
\[
\w{\GL}_2^+(\QQ)
~:=~
\GL_2^+(\QQ)
\underset{
\GL_2^+(\RR)
}
\times
\w{\GL}_2(\RR)
~.
\]
\begin{prop}
\label{t59}
Each of the standard inclusions
\[
\QQ\underset{\ZZ}\ot
\colon
\EZ
\longrightarrow
\GL_2(\QQ)
\qquad
\text{ and }
\qquad
\Ebraid
\longrightarrow
\w{\GL}_2^+(\QQ)
\]
witnesses a group-completion between (a priori continuous) monoids.
In other words, the induced maps between pointed spaces are equivalences:
\[
{\sf BE}_2(\ZZ)
\xra{~\simeq~}
{\sf BGL}_2(\QQ)
\qquad
\text{ and }
\qquad
\sB
\Ebraid
\xra{~\simeq~}
{\sf B \w{GL}}_2^+(\QQ)
~.
\]

\end{prop}

\begin{proof}
First note that the (a priori continuous) monoid $\NN^\times\cong \underset{p~{\rm prime}} \bigoplus (\NN,+)$ is the free (a priori continuous) monoid on the set of prime natural numbers.  
Next note the monomorphism between monoids $\NN^\times \subset \EZ$ as scalar matrices, the image which is central.  
Recall the elementary formula for the inverse of a 2x2 matrix:
\[
\text{ for }A = \begin{bmatrix} a & b \\ c & d  \end{bmatrix}\in \EZ
\qquad
\text{ then }
\qquad
A^{-1} = \frac{1}{ad-bc}\begin{bmatrix} a & b \\ c & d  \end{bmatrix} \in \GL_2(\QQ)
~.
\]
Together, these facts imply the canonical inclusion $\QQ\underset{\ZZ}\ot \colon \EZ \to \GL_2(\QQ)$ witnesses the localization on $\NN^\times \subset \EZ$:
\[
\EZ[ (\NN^\times)^{-1} ]
~\cong~
\GL_2(\QQ)
~.
\]
In particular, this localization is a (a priori continuous) group.
The left identification follows from the universal property of group-completion.
This immediately implies $\QQ\underset{\ZZ}\ot \colon \EpZ \to \GL_2^+(\QQ)$ also witnesses a group-completion.  
The right result follows via base-change along the central extension among continuous groups
\[
1
\longrightarrow
\ZZ
\longrightarrow
\w{\GL}_2^+(\RR)
\longrightarrow
\GL_2^+(\RR)
\longrightarrow
1
~.
\]

\end{proof}



The following result is an immediate consequence of the Definition~\ref{d?? definition of EBraid}, using that the continuous group $\GL^+_2(\RR)$ is a path-connected 1-type.
\begin{cor}
\label{t31}
%The homomorphism~$\Phi$ fits into a pullback square among topological groups in which the right vertical homomorphism is the universal cover:
%\begin{equation}\label{e23}
%\xymatrix{
%\Braid \ar[d]_-{\Phi} \ar[rr]
%&&
%\w{\SL}_2(\RR)\ar[d]^-{\rm universal~cover}
%\\
%\SL_2(\ZZ)
%\ar[rr]^-{\RR \underset{\ZZ}\ot}_-{\rm standard}
%&&
%\SL_2(\RR)
%.
%}
%\end{equation}
There are pullbacks among continuous monoids:
\begin{equation*}%\label{e24}
\xymatrix{
\Braid
\ar[rr]
\ar[d]_-{\Phi}
&&
\Ebraid
\ar[d]_-{\Psi} \ar[rr]
&&
\ast \ar[d]^-{\lag \uno \rag}
\\
\GL_2(\ZZ)
\ar[rr]
&&
\EZ
\ar[rr]^-{\RR \underset{\ZZ}\ot}
&&
\GL_2(\RR)
.
}
\end{equation*}
In particular, there is a canonical identification between continuous groups over $\GL_2(\ZZ)$:
\[
\Braid
~\simeq~
\Omega\bigl(
\GL_2(\RR)_{/\GL_2(\ZZ)}
\bigr)
\qquad
\bigl({\rm ~over~}
\GL_2(\ZZ)
~\bigr)
~.
\]

\end{cor}









































\subsection{Proof of Theorem~\ref{Theorem A} and Corollaries~\ref{t40} }
Theorem~\ref{Theorem A} consists of three statements. 
Theorem~\ref{Theorem A}(1) is a consequence of Corollary~\ref{t26}.
Theorem~\ref{Theorem A}(2) follows immediately from the following.
%Theorem~\ref{Theorem A}(2b) . 
%{\color{red}
%Modify this blurb so it matches the way that theorem, and the logic in this paper, are presented.  
%}



\begin{lemma}
\label{t34}
There are canonical equivalences in the diagrams among continuous monoids:
\begin{equation}
\label{e54}
\xymatrix{
\Ebraid \ltimes \TT^2
\ar@{-->}[rr]^-{\simeq}_{\Aff^{\fr}}
\ar[d]_-{\Psi \ltimes \id}
&&
\Imm^{\fr}(\TT^2) 
\ar[d]^-{\rm forget}
&&
\Braid \ltimes \TT^2
\ar@{-->}[rr]^-{\simeq}_{\Aff^{\fr}}
\ar[d]_-{\Phi \ltimes \id}
&&
\Diff^{\fr}(\TT^2) 
\ar[d]^-{\rm forget}
\\
\TT^2 \ltimes \EZ
\ar[rr]^-{\simeq}_-\Aff
&&
\Imm(\TT^2)
&
\text{ and }
&
\GL_2(\ZZ) \ltimes \TT^2
\ar[rr]^-{\simeq}_-\Aff
&&
\Diff(\TT^2)
.
}
\end{equation}


\end{lemma}



\begin{proof}
Using Observation~\ref{t21}, the canonical equivalence in the commutative diagram on the right follows from that on the left.  


Consider the diagram in the $\infty$-category $\Spaces$:
\[
\xymatrix{
\Ebraid \ltimes \TT^2
\ar[d]_-{\Psi \ltimes \id}
\ar[rr]^-{\pr}
&&
\Ebraid
\ar[d]_-{\Psi}
\ar[rr]^-{!}
&&
\ast
\ar[rr]^-{!}
\ar[d]^-{\lag \uno \rag}
&&
\ast
\ar[d]^-{\lag (0,0,\uno)\rag}
\\
\EZ \ltimes \TT^2
\ar[d]_-{\Aff}^{\simeq}
\ar[rr]^-{\pr}
&&
\EZ
\ar[rr]^-{\RR\underset{\ZZ}\ot }_-{\rm standard}
&&
\GL_2(\RR)
\ar[rr]^-{\lag (0,0) \rag \times \id}
&&
\ZZ^2 \times \GL_2(\RR)
\\
\Imm(\TT^2)
\ar[rrrrrr]^-{{\sf Orbit}_\varphi}
&&
&&
&&
\Fr(\TT^2)
\ar[u]_-{\rm Cor~\ref{t25}}
.
}
\]
Observation~\ref{t33} implies that the bottom rectangle canonically commutes.
Lemma~\ref{t1} and Corollary~\ref{t25} together imply this bottom rectangle witnesses a pullback.
The top left square is clearly a pullback.
Because $\ZZ^2$ is discrete, the top right square is a pullback.  
Corollary~\ref{t31} states that the top middle square is a pullback.
We conclude that the outer square witnesses a pullback.  
The result follows by Definition~\ref{d1} of $\Imm^{\fr}(\TT^2,\varphi)$.



\end{proof}













By applying the product-preserving functor $\Spaces \xra{\pi_0} {\sf Sets}$, Lemma~\ref{t34} implies the following.
\begin{cor}\label{r2}
There is a canonical isomorphism in the diagram of groups:
\[
\xymatrix{
\Braid
\ar@{-->}[rr]^-{\cong}
\ar[d]_-{\Phi}
&&
{\sf MCG}^{\sf fr}(\TT^2)
\ar[d]^-{\rm forget}
\\
\GL_2(\ZZ)
\ar[rr]^-{\cong}
&&
{\sf MCG}(\TT^2)
.
}
\]


\end{cor}



\begin{remark}\label{r1}
Proposition~\ref{t32} and Corollary~\ref{r2} grant a central extension among groups:
\[
%\begin{equation}\label{e17}
1
\longrightarrow
\ZZ
\longrightarrow
{\sf MCG}^{\fr}(\TT^2)
\longrightarrow
{\sf MCG}^{\sf or}(\TT^2)
\longrightarrow
1
~.
\\
%\end{equation}
\]

\end{remark}























%
%\subsection{Comparison with known automorphisms}
%
%
%
%
%
%
%
%In this discussion, we leave framings as understood and implicit in notation.  
%In~\S\ref{sec.??} we introduce the continuous group and continuous monoids,
%\[
%\Diff^{\fr}(\TT^2 \xra{\pr}\TT)
%\qquad
%\text{ and }
%\qquad
%\Imm^{\fr}(\TT^2 \xra{\pr} \TT)
%~,
%\]
%of framed diffeomorphisms and framed local-diffeomorphisms of the projection $\TT^2 \xra{\pr} \TT$.  
%In~\S\ref{sec.??} we also introduce a pair of injective morphisms between monoids,
%\[
%\ZZ
%\longrightarrow
%\Braid
%\qquad
%\text{ and }
%\qquad
%\Ebraid'
%\longrightarrow
%\Ebraid
%~.
%\]
%
%This projection is framed in the sense that the diagram among vector bundles
%\[
%\xymatrix{
%\epsilon^2_{\TT^2} 
%\ar[rr]^{\varphi_0}
%\ar[d]_-{\pr_{\pr}}
%&&
%\tau_{\TT^2}
%\ar[d]^-{\sD \pr}
%\\
%\epsilon^1_{\TT}
%\ar[rr]^-{\varphi_1}
%&&
%\tau_{\TT}
%}
%\]
%commutes.  
%The following consequences of Theorem~\ref{Theorem A} are proved in~\S\ref{sec.compare.symmetries}.
%\begin{cor}
%\label{t42}
%Let $\varphi$ be a framing of the torus.
%There is a canonical commutative diagram among continuous monoids:
%\begin{equation}
%\label{e54}
%\xymatrix{
%\TT^2 \rtimes \ZZ
%\ar[rrrr]^-{\simeq}_-{\rm Lem~\ref{??}}
%\ar[drr]
%\ar[dd]_-{\lag \tau_{1,2} \rag}
%&&
%&&
%\Diff^{\fr}(\TT^2 \xra{\pr} \TT) \ar[drr]
%\ar[dd]
%&&
%\\
%&&
%\TT^2 \rtimes \Ebraid'
%\ar[rrrr]^-{\simeq}_-{\rm Lem~\ref{??}}
%\ar[dd]
%&&
%&&
%\Imm^{\fr}(\TT^2 \xra{\pr}\TT)
%\ar[dd]
%\\
%\TT^2 \rtimes \Braid
%\ar[rrrr]^-{\simeq}_-{\rm Thm~\ref{Theorem A}}
%\ar[drr]
%\ar[dd]
%&&
%&&
%\Diff^{\fr}(\TT^2) \ar[drr]
%\ar[dd]
%&&
%\\
%&&
%\TT^2 \rtimes \Ebraid
%\ar[rrrr]^-{\simeq}_-{\rm Thm~\ref{Theorem A}}
%\ar[dd]
%&&
%&&
%\Imm^{\fr}(\TT^2)
%\ar[dd]
%\\
%\TT^2 \rtimes \GL_2(\ZZ)
%\ar[rrrr]^-{\simeq}_-{\rm Lem~\ref{t??}}
%\ar[drr]
%&&
%&&
%\Diff(\TT^2)
%\ar[drr]
%&&
%\\
%&&
%\TT^2 \rtimes \EZ
%\ar[rrrr]^-{\simeq}_-{\rm Cor~\ref{t??}}
%&&
%&&
%\Imm(\TT^2)
%.
%}
%\end{equation}
%\end{cor}
%
%
%
%
%
%
%
%\begin{proof}[Proof of Corollary~\ref{t42}]
%Recall the diagram~(\ref{e54}) among continuous monoids.
%Commutativity of~(\ref{e??}) implies commutativity of the left square.   
%Forgetting framing establishes commutativity of the right square.  
%The construction~?? of the morphism $\TT^2\rtimes \Ebraid \to \Imm^{\fr}(\TT^2)$ lies over $\Aff$, thereby witnessing commutativity of the front square.  The commutativity of the cube follows.  
%
%\end{proof}


















\begin{proof}[Proof of Corollary~\ref{t40}]
By construction, the diagram among spaces,
\[
\xymatrix{
\EZ \ltimes \TT^2
\ar[rr]^-{\simeq}_-{\rm Cor~\ref{t22}}
\ar[dr]_-{\pr}
&&
\Imm(\TT^2)
\ar[dl]^-{\ev_0}
\\
&
\TT^2
&
,
}
\]
canonically commutes, in which the left vertical map is projection, and the right vertical map evaluates at the origin $0\in \TT^2$.  
Therefore, upon taking fibers over $0\in \TT^2$, the (left) commutative diagram~(\ref{e54}) among continuous monoids determines the commutative diagram among commutative monoids:
\begin{equation*}
%\label{e55}
\xymatrix{
%\Braid
%\ar[rrrr]^-{\simeq}
%\ar[drr]
%\ar[dd]
%&&
%&&
%\Diff^{\fr}(\TT^2 ~{\sf rel } ~0) \ar[drr]
%\ar[dd]
%&&
%\\
%&&
\Ebraid
\ar[rrrr]^-{\simeq}
\ar[d]
&&
&&
\Imm^{\fr}(\TT^2 ~{\sf rel } ~0)
\ar[d]
%\\
%\GL_2(\ZZ)
%\ar[rrrr]^-{\simeq}
%\ar[drr]
%&&
%&&
%\Diff(\TT^2 ~{\sf rel}~ 0)
%\ar[drr]
%&&
\\
%&&
\EZ
\ar[rrrr]^-{\simeq}_-{\rm Cor~\ref{t22}}
\ar[drr]_-{\RR \underset{\ZZ}\ot}
&&
&&
\Imm(\TT^2 ~{\sf rel } ~0)
\ar[dll]^-{\sD_0}
\\
%&&
&&
\GL_2(\RR)
&&
,
}
\end{equation*}
in which the map $\RR \underset{\ZZ}\ot $ is the standard inclusion, and $\sD_0$ takes the derivative at the origin $0\in \TT^2$.
To finish, Corollary~\ref{t31} supplies the left pullback square in the following diagram among continuous groups, while the right pullback square is definitional:
\[
\xymatrix{
\Braid
\ar[rr]
\ar[d]
&&
\ast
\ar[d]
&&
\Diff(\TT^2 \smallsetminus \BB^2 ~{\sf rel}~\partial )
\ar[ll]
\ar[d]
\\
\GL_2(\ZZ)
\ar[rr]^-{\RR \underset{\ZZ}\ot}
&&
\GL_2(\RR)
&&
\Diff(\TT^2~{\sf rel}~0)
\ar[ll]_-{\sD_0}
.
}
\]
The result follows. 

\end{proof}


























































































\subsection{Comparison with sheering}
We use Theorem~\ref{Theorem A}(2a) to show that the $\Diff^{\fr}(\TT^2)$ is generated by sheering.  
We quickly tour through some notions and results, which are routine after the above material.  

\begin{notation}
It will be convenient to define the projection $\TT^{2} \xra{\pr_{i}} \TT$ to be projection \textit{off} of the $i-$th coordinate. So for $\TT^{2} \ni p = (x_{p}, y_{p}),$ we have $\pr_{1}(p) = y_{p}$ and $\pr_{2}(p) = x_{p}.$
\end{notation}

Let $ i\in \{1,2\}$.
Consider the topological subgroup and topological submonoid,
\[
\Diff(\TT^2 \xra{\pr_i} \TT)
~\subset~
\Diff(\TT^2)
\qquad
\text{ and }
\qquad
\Imm(\TT^2 \xra{\pr_i} \TT)
~\subset~
\Imm(\TT^2)
~,
\]
consisting of those (local-)diffeomorphisms $\TT^2\xra{f} \TT^2$ that lie over some (local-)diffeomorphism $\TT\xra{\ov{f}} \TT$:
\[
\xymatrix{
\TT^2
\ar[rr]^-f
\ar[d]_-{\pr_i}
&&
\TT^2 \ar[d]^-{\pr_i}
\\
\TT
\ar[rr]^-{\ov{f}}
&&
\TT
~.
}
\]
Because $\pr_i$ is surjective, for a given $f$, there is a unique $\ov{f}$ if any.
Better, $f\mapsto \ov{f}$ defines a forgetful morphism between topological monoids, whose kernel is the topological monoid of smooth maps from $\TT$ to $\Imm(\TT)$ with value-wise monoid-structure, which is to say
there is a short exact sequence of topological monoids:
\begin{equation}
\label{e68}
\xymatrix{
1
\ar[r] 
&
\Map\bigl( \TT , \Imm(\TT) \bigr) 
\ar[rr]
&&
\Imm(\TT^2 \xra{\pr_i} \TT)
\ar[rr]_-{ \rm forget }
&&
\Imm(\TT)
\ar[r]
\ar@{-->}@(u,-)[ll]
&
1
.
}
\end{equation}
The isotopy-extension theorem implies this short exact sequence among topological groups forgets as a short exact sequence among continuous groups.  
Furthermore, this short exact sequence of continuous monoids~(\ref{e68}) splits, as indicated, by way of the precomposition action $\Map\bigl( \TT , \Imm(\TT) \bigr)  \racts \Imm(\TT)$:
there is a canonical equivalence between continuous monoids:
\begin{equation*}
%\label{e69}
\Imm(\TT^2 \xra{\pr_i} \TT)
~\simeq~
\Imm(\TT)  \ltimes  \Map\bigl( \TT , \Imm(\TT) \bigr)
~.
\end{equation*}





Now, the topological space of \bit{framings} of $\TT^2 \xra{\pr_i} \TT$ is the subspace
\[
\Fr(\TT^2 \xra{\pr_i} \TT)
~\subset~
\Fr(\TT^2)
\]
consisting of those framings $\tau_{\TT^2}\xra{\varphi} \epsilon^2_{\TT^2}$ that lie over a framing $\tau_{\TT} \xra{\ov{\varphi}} \epsilon^1_{\TT}$:
\begin{equation}
\label{e70}
\xymatrix{
\tau_{\TT^2}
\ar[rr]^-{\varphi}_-{\cong}
\ar[d]_-{\sD \pr_i}
&&
\epsilon^2_{\TT^2}
\ar[d]^-{\pr_i \times \pr_i}
\\
\tau_{\TT}
\ar[rr]^-{\ov{\varphi}}_-{\cong}
&&
\epsilon^1_{\TT}
.
}
\end{equation}
The continuous right action $\Act$ of Lemma~\ref{t50} evidently restricts as a continuous right action
\[
\Fr(\TT^2 \xra{\pr_i} \TT)
~\racts~
\Imm(\TT^2 \xra{\pr_i} \TT)
~.
\]
Now, because $\pr_i$ is surjective, for a given $\varphi$, there is a unique $\ov{\varphi}$ as in~(\ref{e70}) if any.  
Better, $\varphi\mapsto \ov{\varphi}$ defines a continuous map, which is evidently equivariant with respect to the morphism between topological monoids $\Imm(\TT^2 \xra{\pr_i} \TT) \xra{\rm forget} \Imm(\TT)$:
\[
\Bigl(
~
\Fr(\TT^2 \xra{\pr_i} \TT)
~\racts~
\Imm(\TT^2 \xra{\pr_i} \TT)
~\Bigr)
~
\xra{~\rm forget~}
~
\Bigl(
~
\Fr(\TT)
~\racts~
\Imm(\TT)
~\Bigr)
~,\qquad
\varphi
\mapsto 
\ov{\varphi}
~.
\]






Now let $\varphi \in \Fr(\TT^2 \xra{\pr_i} \TT)$ be a framing of the projection.
The orbit of $\varphi$ by this action is the map
\[
{\sf Orbit}_\varphi
\colon 
\Imm(\TT^2 \xra{\pr_i}\TT)
\longrightarrow
\Fr(\TT^2 \xra{\pr_i} \TT)
~,\qquad
f\mapsto \Act(\varphi,f)
~.
\]
The space of \bit{framed local-diffeomorphisms}, and the space of \bit{framed diffeomorphisms}, of $(\TT^2 \xra{\pr_i} \TT , \varphi)$ are respectively the homtopy-pullbacks among spaces:
\[
%\begin{equation}\label{e11}
\xymatrix{
\Imm^{\sf fr}(\TT^2\xra{\pr_i} \TT,\varphi)
\ar[r]
\ar[d]
&
\Imm(\TT^2\xra{\pr_i} \TT)
\ar[d]^-{{\sf Orbit}_\varphi}
&&
\Diff^{\sf fr}(\TT^2 \xra{\pr_i} \TT,\varphi)
\ar[r]
\ar[d]
&
\Diff(\TT^2\xra{\pr_i} \TT)
\ar[d]^-{{\sf Orbit}_\varphi}
\\
\ast
\ar[r]^-{\lag \varphi \rag}
&
\Fr(\TT^2\xra{\pr_i} \TT)
&
\text{ and }
&
\ast
\ar[r]^-{\lag \varphi \rag}
&
\Fr(\TT^2\xra{\pr_i} \TT)
~.
}
\]
%\end{equation}
%{\color{red}
%You can see I petered out around here.
%Consider taking this up: just fill in definitions/set-up/logic as efficiently/trimly as possible to get to the Observation, Observation, Corollary below.
%}

The topological space $\Fr(\TT^2 \xra{\pr_i} \TT)$ is a torsor for the topological group $\Map\bigl( \TT^2 , \GL_{\{i\}\subset 2}(\RR) \bigr)$ of smooth maps from $\TT^2$ to the subgroup 
\[
\GL_{\{i\}\subset 2}(\RR)
~:=~
\Bigl\{
A \mid
Ae_i \in {\sf Span}\{e_i\}
\Bigr\}
~\subset~
\GL_2(\RR)
\]
consisting of those $2 \times 2$ matrices that carry the $i^{\rm th}$-coordinate line to itself. 

\begin{lemma} \label{prfr}
The homotopy equivalence of lemma (\ref{t25}) restricts to the following homotopy equivalence of spaces:
\begin{equation} \label{e801}
\Fr(\TT^2 \xra{\pr_i} \TT)
~\simeq~
{\sf O}(1) \times {\sf O}(1). 
\end{equation}
\end{lemma}
\begin{proof}
Upon a choice of framing $\varphi \in \Fr(\TT^2 \xra{\pr_i} \TT),$ we have the homeomorphism 
\[
\Fr(\TT^2 \xra{\pr_i} \TT) \approx {\sf Map}\Big( (\TT^2 \xra{\pr_i} \TT), (\GL_{\{i\}\subset 2}(\RR) \rightarrow \GL_{1}(\RR)) \Big)
\]
where the latter space consists of those maps which lie over a map making the diagram below commute
\[ {\sf Map}\Big( (\TT^2 \xra{\pr_i} \TT), (\GL_{\{i\}\subset 2}(\RR) \rightarrow \GL_{1}(\RR)) \Big) :=
\left\{ \begin{tikzcd}
\TT^{2} \arrow[r] \arrow[d,"\pr_{i}"]
&
\GL_{\{i\}\subset 2}(\RR) \arrow[d, "e_{i}^{T}(-)e_{k}"]
\\
\TT \arrow[dashed, r]
&
\GL_{1}(\RR).
\end{tikzcd} \right\}
\]
where $k \in \{1, 2\} - i.$ This homeomorphism is $(\ref{e41}) \circ (\ref{e43})$ restricted to the appropriate subspaces.
Now, $\GL_{1}(\RR) \simeq \RR^{\times} \simeq {\sf O}(1)$ and $\GL_{\{i\}\subset 2}(\RR) \simeq \RR^{\times} \times \RR^{\times} \simeq {\sf O}(1) \times {\sf O}(1)$ as we can take a straight line homotopy to diagonal matrices. Then note we have the following inclusion which is a homotopy equivalence:
\[
\left\{ \begin{tikzcd}
\TT^{2} \arrow[r] \arrow[d,"\pr_{i}"]
&
\GL_{\{i\}\subset 2}(\RR) \arrow[d, ""]
\\
\TT \arrow[dashed, r]
&
\GL_{1}(\RR).
\end{tikzcd} \right\}
\hookleftarrow
\left\{ \begin{tikzcd}
\TT^{2} \arrow[r] \arrow[d,"\pr_{i}"]
&
{\sf O}(1) \times {\sf O}(1) \arrow[d, "\pr_{i}"]
\\
\TT \arrow[dashed, r]
&
{\sf O}(1).
\end{tikzcd} \right\} \simeq {\sf Map}(\TT^{2}, {\sf O}(1) \times {\sf O}(1))
\]
where the last equivalence is due to the diagram being fixed after choosing such a map. As $\TT^{2}$ is connected we have that ${\sf Map}(\TT^{2}, {\sf O}(1) \times {\sf O}(1)) \simeq {\sf O}(1) \times {\sf O}(1).$
\end{proof}



 

%For $i = 1,2$, 
%consider the projection onto the $i^{\rm th}$-coordinate
 %$\TT^2 \xra{\pr_i} \TT$ from the framed torus to the framed circle.  


%\textcolor{blue}{
%Recall the matrices $U_1,U_2 \in \GL_2(\ZZ)$ form~(\ref{e63}). These induce actions of $\ZZ$ onto $\TT^{2}$ defined by $k \cdot p = U_{i}^{k}p$ which define the semi-direct products $\ZZ \underset{U_{i}}\ltimes \TT^{2}$. There is also a natural action of ${\sf O}(1) \times {\sf O}(1)$ on $\TT^{2}$ defined by $(m, n) \cdot p = (x_{p}, y_{p}) = (mx_{p}, ny_{p}).$ We can then extend this action to $\ZZ \underset{U_{i}}\ltimes \TT^{2}$ where ${\sf O}(1) \times {\sf O}(1)$ acts trivially on $\ZZ,$ which defines $({\sf O}(1) \times {\sf O}(1)) \ltimes (\ZZ \underset{U_{i}}\ltimes \TT^{2}).$
%\textcolor{red}{or is the action conjugation? of $\SS^{1} \subset \CC^{2}$}
%}
 
\begin{lemma} %Maybe call this sucker a Lemma
There exist the following factorizations between topological groups: %{\sf E}_{2}(\ZZ)
\[
\begin{tikzcd}
{\sf E}_{\{1\} \subset 2}(\ZZ) \ltimes \TT^{2} \arrow[dashed, r, "{\sf Aff}_{1}"] \arrow[hookrightarrow]{d}& \Imm(\TT^2 \xra{\pr_1} \TT)  \arrow[hookrightarrow]{d} \\
{\sf E}_{2}(\ZZ) \ltimes \TT^{2} \arrow[r, "{\sf Aff}"] & \Imm(\TT^{2}),
\end{tikzcd}
\hspace{15pt}
\begin{tikzcd}
{\sf E}_{\{2\} \subset 2}(\ZZ) \ltimes \TT^{2} \arrow[dashed, r, "{\sf Aff}_{2}"] \arrow[hookrightarrow]{d}& \Imm(\TT^2 \xra{\pr_2} \TT)  \arrow[hookrightarrow]{d} \\
{\sf E}_{2}(\ZZ) \ltimes \TT^{2} \arrow[r, "{\sf Aff}"] & \Imm(\TT^{2})
\end{tikzcd}
\]
where ${\sf E}_{\{1\} \subset 2}(\ZZ)$ consists of the upper triangular matrices in ${\sf E}_{2}(\ZZ)$ and ${\sf E}_{\{2\} \subset 2}(\ZZ)$ consists of the lower triangular matrices in ${\sf E}_{2}(\ZZ).$ 
\end{lemma}
\begin{proof}
We need to show that the image of ${\sf Aff}_{1}$ lies in $\Imm(\TT^2 \xra{\pr_1} \TT).$ In other words we need to show that for $(A, p)$ in the domain, the map $$\Aff_{1}(A, p): \TT^{2} \rightarrow \TT^{2}; \hspace{15pt} q \mapsto Aq + p$$ lies over an immersion of $\TT.$ Note the following commutative square:
for $p = (x_{p}, y_{p}) \in \TT^{2}$ and
\[
A=
\left[ {\begin{array}{cc}
 a & b \\
 0 & d \\
\end{array} } \right] \in {\sf E}_{\{1\} \subset 2}(\ZZ), \hspace{ 20 pt}
\xymatrix{
\TT^{2} \ni q = (x_{q}, y_{q})
\ar[r]^-{\Aff_{1}(A, p)}
\ar[d]_-{\pr_{1}}
&
(ax_{q} + by_{q} + x_{p} , dy_{q} + y_{p}) \in \TT^{2} \ar[d]_-{\pr_{1}}
\\
\TT \ni y_{q}
\ar[r]^-{\trans_{y_{p}} \circ{~} {\sf scale}_{d}}
&
dy_{q} + y_{p} \in \TT
~.
}
\]
Both $a, d$ are nonzero integers and so we see that the bottom map is indeed an immersion of $\TT.$ A similar computation shows that ${\sf \Aff_{2}}(A, p)$ lies over an immersion of $\TT.$
\end{proof}


\begin{prop}
For $i = 1, 2$ the map ${\sf Aff}_{i}$ is a homotopy equivalence of spaces.
\end{prop}
\begin{proof}
Consider the following commutative diagram of topological spaces:
\begin{equation}\label{cd1}
\xymatrix{
1 \ar[d]_-= \ar[r]
&
{\sf E}_{\{1\} \subset 2}(\ZZ)_{ker} \ltimes \TT
\ar[d]_-{F}
\ar[r]^-{inc_{x}}
&
{\sf E}_{\{1\} \subset 2}(\ZZ) \ltimes \TT^{2} 
\ar[d]_-{\Aff_{i}} 
\ar[r]^-{d \ltimes \pr_{1}}
&
{\sf E}_{1}(\ZZ) \ltimes \TT
\ar[d]_-{{\Aff}_{\TT}}
\ar[r]
&
1 \ar[d]_-=
\\
1
\ar[r]
&
{\sf Map}(\TT, \Imm(\TT))
\ar[r]^-{}
&
\Imm(\TT^2 \xra{\pr_{1}} \TT) 
\ar[r]^-{}
&
\Imm(\TT)
\ar[r]
&
1
}
\end{equation}
where ${\sf E}_{\{1\} \subset 2}(\ZZ)_{ker}$ consists of matrices of the form
\[
\left\{
\left[ {\begin{array}{cc}
 a & b \\
 0 & 1 \\
\end{array} } \right] : a, b \in \ZZ \text{ and } a \neq 0 \right\}.
\] 
and $inc_{x}: {\sf E}_{\{1\} \subset 2}(\ZZ)_{ker} \ltimes \TT \longrightarrow {\sf E}_{\{1\} \subset 2}(\ZZ) \ltimes \TT^{2} $ is given by the inclusion $(A, \theta) \mapsto (A, (\theta, 0)).$ This makes the top sequence exact.
The map ${\Aff}_{\TT}: {\sf E}_{1}(\ZZ) \ltimes \TT \rightarrow \Imm(\TT)$ is defined by $(c, \theta) \mapsto (\gamma \mapsto c\gamma + \theta),$ which is well known to be a homotopy equivalence. Then the map $F$ is defined by 
\[
{\sf E}_{\{1\} \subset 2}(\ZZ)_{ker} \ltimes \TT \ni \left(A = \left[ {\begin{array}{cc}
 a & b \\
 0 & 1 \\
\end{array} } \right] , \theta \right) \mapsto (\psi \mapsto f_{a, b, \theta, \psi}) \in {\sf Map}(\TT, \Imm(\TT))
\]
where $f_{a, b, \theta, \psi}(\gamma) = a\gamma + b\psi + \theta.$ We will show that $F$ is also a homotopy equivalence. Consider the commutative diagram with exact rows:
\[
\xymatrix{
1 \ar[d]_-= \ar[r]
&
\ZZ 
\ar[r]^-{UR}
\ar[d]^-{G}
&
{\sf E}_{\{1\} \subset 2}(\ZZ)_{ker} \ltimes \TT
\ar[d]_-{F}
\ar[r]^-{a \ltimes \id}
&
{\sf E}_{1}(\ZZ) \ltimes \TT
\ar[r]^-{}
\ar[d]^{{\sf Aff}_{\TT}}
&
1 \ar[d]_-=
\\
1
\ar[r]
&
{\sf Map}_{\ast}(0 \in \TT, \id_{\TT} \in \Imm(\TT))
\ar[r]^-{{\sf inc}}
&
{\sf Map}(\TT, \Imm(\TT))
\ar[r]^-{ev_{0}}
&
\Imm(\TT)
\ar[r]^-{}
&
1
}
\] 
where
\[UR(b) =  \left(\left[ {\begin{array}{cc}
 1 & b \\
 0 & 1 \\
\end{array} } \right] , 0 \right)
\hspace{10pt} \text{and} \hspace{10pt} 
G(b) = (\psi \mapsto (g_{b, \psi}: \TT \rightarrow \TT)); \hspace{5 pt} g_{b, \psi}(\gamma) = \gamma + b\psi.
\]
Again, ${\sf Aff}_{\TT}$ is a homotopy equivalence and so 
\[{\sf Map}_{\ast}(0 \in \TT, \id_{\TT} \in \Imm(\TT)) \simeq {\sf Map}_{\ast}(\TT, {\sf E}_{1}(\ZZ) \ltimes \TT) =  {\sf Map}_{\ast}(\TT,\TT).
\]
Now consider the map $\w{G}: \ZZ \rightarrow {\sf Map}_{\ast}(\TT,\TT)$ which is just $G$ followed by the homotopy equivalence above. Then $\w{G}(b)$ is just the based map $\psi \mapsto b\psi$ which is the homotopy inverse of $\pi_{0}: {\sf Map}_{\ast}(\TT,\TT) \rightarrow \ZZ.$ Therefore $\w{G}$ is a homotopy equivalence which implies that $G$ must be a homotopy equivalence. Finally by the 5-lemma we have that the map $F$ is a homotopy equivalence.

Now as the rows of the diagram (\ref{cd1}) are exact and the outer columns, F and ${\sf Aff}_{\TT},$ are homomtopy equivalences the 5-lemma yields that ${\sf Aff}_{1}$ is also a homotopy equivalence. A similar argument shows that ${\sf Aff}_{2}$ is a homotopy equivalence as well.
\end{proof}


\begin{prop} \label{p45}
The following diagrams commute:
\[
\xymatrix{
{\sf E}_{\{1\} \subset 2}(\ZZ) \ltimes \TT^{2} \ar[r]^-{ {\sf Aff}_{1}}
\ar[d]^{\left(\frac{a}{|a|}, \frac{d}{|d|}\right)} 
&
\Imm(\TT^2 \xra{\pr_1} \TT) \ar[d]^-{{\sf Orbit}_{\varphi}} 
\\
{\sf O}(1) \times {\sf O}(1) 
&
\Fr(\TT^2 \xra{\pr_1} \TT) \ar@(-,-)[l]^-{\simeq},
}
\hspace{15pt}
\xymatrix{
{\sf E}_{\{2\} \subset 2}(\ZZ) \ltimes \TT^{2} \ar[r]^-{ {\sf Aff}_{2}}
\ar[d]^{\left(\frac{a}{|a|}, \frac{d}{|d|}\right)} 
&
\Imm(\TT^2 \xra{\pr_2} \TT) \ar[d]^-{{\sf Orbit}_{\varphi}} 
\\
{\sf O}(1) \times {\sf O}(1) 
&
\Fr(\TT^2 \xra{\pr_2} \TT) \ar@(-,-)[l]^-{\simeq}.
}
\]
\end{prop} 
\begin{proof}
Given some element $(A, p)$ of ${\sf E}_{\{i\} \subset 2}(\ZZ) \ltimes \TT^{2}$ we want to identify the associated element of ${\sf Orbit}_{\varphi} \circ {\sf Aff}_{i}(A, p)$ in ${\sf O}(1) \times {\sf O}(1).$ We know that the differential ${\sf D Aff}_{i}(A, p) = A$ and therefore
\[
{\sf Orbit}_{\varphi} \circ {\sf Aff}_{i}(A, p) = \tau_{\TT^2} 
\underset{\cong}{\xra{A}}
{\sf Aff}_{i}(A, p)^\ast \tau_{\TT^2}
\underset{\cong}{ \xra{{\sf Aff}_{i}(A, p)^\ast \varphi} }
{\sf Aff}_{i}(A, p)^\ast \epsilon^2_{\TT^2}
=
\epsilon^2_{\TT^2}.
\]
and then through the homotopy equivalence above in lemma \ref{prfr} we have that the associated element of ${\sf O}(1) \times {\sf O}(1)$ corresponds to reading off the signs of the diagonal of $A.$
\end{proof}


\begin{cor}
There exists a homotopy equivalence of spaces 
\[
{\sf E}_{\{i\} \subset 2}^{++}(\ZZ) \ltimes \TT^{2} \simeq \Imm^{\sf fr}(\TT^2 \xra{\pr_i} \TT) 
\]
where ${\sf E}_{\{i\} \subset 2}^{++}(\ZZ)$ are those elements of ${\sf E}_{\{i\} \subset 2}(\ZZ)$ whose diagonal elements are positive.
\end{cor}
 \begin{proof}
 We have that $\Imm^{\sf fr}(\TT^2 \xra{\pr_i} \TT)$ is defined to be the homotopy fiber
 \[
\Imm^{\sf fr}(\TT^2 \xra{\pr_i} \TT) := {\sf hofib}_{\varphi}(\Imm(\TT^2 \xra{\pr_i} \TT) \xra{{\sf Orbit}_{\varphi}} \Fr(\TT^2 \xra{\pr_i} \TT)).
\]
Then as ${\sf Aff}_{i}$ and $(\ref{e801})$ are homotopy equivalences of spaces, there exists a homotopy equivalence 
\[
{\sf hofib}_{\varphi}\big(\Imm(\TT^2 \xra{\pr_i} \TT) \xra{{\sf Orbit}_{\varphi}} \Fr(\TT^2 \xra{\pr_i} \TT)\big) 
\] 
\[
\simeq \hspace{5pt} {\sf hofib}_{(1,1)}\big({\sf E}_{\{i\} \subset 2}(\ZZ) \ltimes \TT^{2} \xra{{\sf Aff}_{i}} \Imm(\TT^2 \xra{\pr_i} \TT) \xra{{\sf Orbit}_{\varphi}} \Fr(\TT^2 \xra{\pr_i} \TT) \xra{(\ref{e801})} {\sf O}(1) \times {\sf O}(1)\big) 
\]
\[
\overset{\text{proposition }\ref{p45}}= {\sf hofib}_{(1,1)}\big({\sf E}_{\{i\} \subset 2}(\ZZ) \ltimes \TT^{2} \xra{\left(\frac{a}{|a|}, \frac{d}{|d|}\right)} {\sf O}(1) \times {\sf O}(1)\big).
\]
But as the last map is projection onto a discrete space, the homotopy fiber over $(1, 1)$ is equivalent to the standard fiber over $(1, 1)$ which is ${\sf E}_{\{i\} \subset 2}^{++}(\ZZ) \ltimes \TT^{2}.$
\end{proof}
 
 
 \begin{observation} \label{t14}
Note that in restricting the maps of proposition \ref{p45} to the maximal subgroup of ${\sf E}_{\{i\} \subset 2}(\ZZ)$ we have that $$\GL_{\{i\} \subset 2}(\ZZ) \ltimes \TT^{2} \xra{(\ref{e801}) \circ {\sf Orbit}_{\varphi} \circ {\sf Aff_{i}}} {\sf O}(1) \times {\sf O}(1)$$ is equal to projection off of the diagonal elements of $\GL_{\{i\} \subset 2}(\ZZ).$ Therefore the homotopy fiber of $(1, 1)$ of this map is just the fiber over $(1, 1)$ which we will denote as $\GL_{\{i\} \subset 2}^{++}(\ZZ) \ltimes \TT^{2},$ where $\GL_{\{i\} \subset 2}^{++}(\ZZ)$ consists of those matrices in $\GL_{\{i\} \subset 2}(\ZZ)$ with 1's on the diagonal. Then observe the isomorphisms $\ZZ \xra{\lag U_{i} \rag} \GL_{\{i\} \subset 2}^{++}(\ZZ)$ where $b \mapsto U_{i}^{b}.$ Therefore we have that 
\[
\ZZ \underset{U_{i}}\ltimes \TT^{2} \xra{\cong} \GL_{\{i\} \subset 2}^{++}(\ZZ) \ltimes \TT^{2} \simeq  
\Diff^{\fr}(\TT^2 \xra{\pr_i} \TT).
\]
where we have $\ZZ \ni k \lacts p \in \TT^{2}$ by $U_{i}^{k}p.$ 
 
%Therefore we have the homotopy equivalences $\ZZ \underset{U_{i}}\ltimes \TT^{2} \simeq \Diff^{\fr}(\TT^2 \xra{\pr_i} \TT).$
\end{observation}
 
%\begin{observation}
%\label{t14}
%The group of framed automorphisms of $\TT^2 \xra{\pr_i} \TT$ 
%fits into a split short-exact sequence among continuous groups,
%\begin{equation}
%\label{e31}
%\xymatrix{
%1
%\ar[r]
%\ar[d]_=
%&
%\TT \times \ZZ
%\ar[r]
%\ar[d]_-{\simeq}
%&
%\ZZ \underset{U_i} \ltimes \TT^2
%\ar[rr]
%\ar[d]_-{ \lag \tau_i \rag \ltimes \trans }
%&&
%\NN^\times
%\ltimes
%\TT
%\ar[d]_-{\trans}
%\ar@(u,-)[ll]^-{}
%\ar[r]
%&
%1
%\ar[d]^=
%\\
%1
%\ar[r]
%&
%\Map\bigl(\TT , \Imm^{\fr}(\TT) \bigr)
%\ar[r]
%&
%\Imm^{\fr}\bigl(
%\TT^2 \xra{\pr_i} \TT
%\bigr)
%\ar[rr]_-{\rm forget}
%&&
%\Imm^{\fr}(\TT)
%\ar@(d,-)[ll]^-{}
%\ar[r]
%&
%1
%~,
%}
%\end{equation}
%that is classified by the pre-composition action $\Diff^{\fr}(\TT) \lacts \Map\bigl(\TT , \Diff^{\fr}(\TT) \bigr)$.
%Recall the canonical identifications between continuous groups
%\[
%\TT \xra{~\simeq~} \Diff^{\fr}(\TT)
%\qquad
%\text{ and }
%\qquad
%\Map(\TT , \TT)
%~\simeq~
%\TT \times \ZZ
%~.
%\]
%Through these identifications, this split short-exact sequence~(\ref{e31}) is canonically identified as the split short-exact sequence
%\begin{equation*}
%\label{e32}
%\xymatrix{
%1
%\ar[rr]
%&&
%\TT \times \ZZ
%\ar[rr]
%&&
%\ZZ \underset{U_i}\ltimes \TT^2 
%\ar[rr]
%&&
%\TT
%\ar@(u,-)[ll]^-{}
%\ar[rr]
%&&
%1
%~,
%}
%\end{equation*}
%where the action $\ZZ \lacts \TT^2$ is via a sheering elements
%$
%U_i 
%\in \GL_2(\ZZ)
%\to
%\Aut_{\sf Groups}(\TT^2)
%$
%of~(\ref{e63}).


%\end{observation}




\begin{observation}
\label{t15}
Through observation ~\ref{t14}, there are canonically commutative diagrams among continuous groups:
\[
\xymatrix{
\ZZ \underset{U_{1}}\ltimes \TT^2 
\ar[r]^-{\simeq}_-{\rm Obs~\ref{t14}}
\ar[d]_-{\lag 
\tau_{1}
\rag \ltimes \id }
&
\Diff^{\fr}(\TT^2 \xra{\pr_1} \TT) 
\ar[d]^-{\rm forget}
&&
\ZZ \underset{U_{2}}\ltimes \TT^2
\ar[r]^-{\simeq}_-{\rm Obs~\ref{t14}}
\ar[d]_-{ \lag 
\tau_{2}
\rag \ltimes \id }
&
\Diff^{\fr}(\TT^2 \xra{\pr_2} \TT) 
\ar[d]^-{\rm forget}
\\
\Braid \ltimes \TT^2 
\ar[r]^-{\simeq}_-{\rm Lem~\ref{t34}}
&
\Diff^{\fr}(\TT^2)
&
\text{ and }
&
\Braid \ltimes \TT^2
\ar[r]^-{\simeq}_-{\rm Lem~\ref{t34}}
&
\Diff^{\fr}(\TT^2)
.
}
\\
\]

\end{observation}









We now explain how the presentation~(\ref{e67}) of $\Braid$ gives a presentation of the continuous group $\Diff^{\fr}(\TT^2)$.
Observe the canonically commutative diagram among continuous groups:
\[
\xymatrix{
\TT^2 \ar[rr]
\ar[d]
&&
\Diff^{\fr}(\TT^2 \xra{\pr_1} \TT)
\ar[d]
\\
\Diff^{\fr}(\TT^2 \xra{\pr_2} \TT)
\ar[rr]
&&
\Diff^{\fr}(\TT^2)
.
}
\]
It results in a morphism from the pushout in continuous groups:
\begin{equation}
\label{e65}
\Diff^{\fr}(\TT^2\xra{\pr_1} \TT)
\underset{\TT^2}
\coprod
\Diff^{\fr}(\TT^2 \xra{\pr_2} \TT)
\longrightarrow
\Diff^{\fr}(\TT^2)
~.
\end{equation}
Recall the element $R \in \GL_2(\ZZ)$ from~(\ref{e64}).
The two homomorphisms
\[
\begin{tikzcd}
\ZZ \arrow[rr, yshift=0.7ex, "\lag \tau_{1} \tau_{2} \tau_{1}\rag"] \arrow[swap, rr, yshift=-0.7ex, "\lag \tau_{2} \tau_{1} \tau_{2}\rag"]
& 
& \ZZ \amalg \ZZ % \arrow[d, xshift=0.7ex] \arrow[d, xshift=-0.7ex] 
\end{tikzcd}
\]
%$$
%\ZZ \stackrel[\mathrm{discharge}]{\mathrm{charge}}{\overrightarrow{\underrightarrow{\hspace{5cm}}}}  \ZZ \amalg \ZZ
%$$
%\[
%\xymatrix{
%\ZZ
%\ar@(u,u)[rr]^-{??}
%\ar@(d,d)[rr]_-{??}
%&&
%\ZZ
%\amalg 
%\ZZ
%}
%\]
%{\color{red}
%Improve the LaTexing here, so that it doesn't take up so much darn room.
%For instance, just two flat horizontal arrows would suffice.
%}
determine two morphisms among continuous groups under $\TT^2$:
%\begin{equation}
%\label{e66}
%\xymatrix{
%\TT^2 \underset{R}\rtimes \ZZ
%\ar@(u,u)[rr]^-{\id \rtimes ??}
%\ar@(d,d)[rr]_-{\id \rtimes ??}
%&&
%\TT^2 \underset{U_1,U_2} \rtimes (\ZZ \amalg \ZZ)
%\ar[r]_-{\rm Obs~\ref{t15}}^-{\simeq}
%&
%\Diff^{\fr}(\TT^2\xra{\pr_1} \TT)
%\underset{\TT^2}
%\coprod
%\Diff^{\fr}(\TT^2 \xra{\pr_2} \TT)
%\ar[r]^-{(\ref{e65})}
%&
%\Diff^{\fr}(\TT^2)
%}
%\end{equation}

\begin{equation} \label{e66}
\begin{tikzcd}
\ZZ \underset{R}\ltimes \TT^2 \arrow[rr, yshift=0.7ex, "\lag \tau_{1} \tau_{2} \tau_{1}\rag \ltimes \id"] \arrow[swap, rr, yshift=-0.7ex, "\lag \tau_{2} \tau_{1} \tau_{2}\rag \ltimes \id "]
&
&
(\ZZ \amalg \ZZ) \underset{U_1,U_2} \ltimes \TT^2
\arrow[r, "\simeq"]
\arrow[swap, r, "\rm Obs~\ref{t15}"]
%\arrow[r]_-{\rm Obs~\ref{t15}}^-{\simeq}
&
\Diff^{\fr}(\TT^2\xra{\pr_1} \TT)
\underset{\TT^2}
\coprod
\Diff^{\fr}(\TT^2 \xra{\pr_2} \TT)
\arrow[r, "(\ref{e65})"]
&
\Diff^{\fr}(\TT^2)
\end{tikzcd}
\end{equation}



\begin{cor}
\label{t49}
The diagram among continuous groups under $\TT^2$,~(\ref{e66}), 
witnesses a coequalizer.
%In other words, the canonical morphism between continuous groups
%\[
%\xymatrix{
%\TT^2 \rtimes
%\Bigl(
%\ZZ
%\ar@(u,u)[rr]^-{ \id \rtimes \lag \tau_{1,2} \tau_{2,3} \tau_{1,2}\rag}
%\ar@(d,d)[rr]_-{ \id \rtimes \lag \tau_{2,3} \tau_{1,2} \tau_{2,3}\rag}
%&&
%\ZZ \amalg \ZZ
%\Bigr)
%\ar[rr]^-{\simeq}
%&&
%\Diff^{\fr}(\TT^2)
%~,
%}
%\]
%is an equivalence.
In particular, for each $\infty$-category $\cX$, there is a pullback diagram among $\infty$-categories of modules in $\cX$:
\[
\xymatrix{
\Mod_{\Diff^{\fr}(\TT^2)}(\cX)
\ar[rr]
\ar[d]
&&
\Mod_{\ZZ \underset{U_1} \ltimes \TT^2}(\cX)
\underset{\Mod_{\TT^2}(\cX)}\times
\Mod_{\ZZ  \underset{U_2} \ltimes \TT^2}(\cX)
\ar[d]^-{\bigl(\lag \tau_{1} \tau_{2} \tau_{1}\rag \ltimes \id \bigr)^\ast \times \bigl(\lag \tau_{2} \tau_{1} \tau_{2}\rag \ltimes \id \bigr)^\ast}
\\
\Mod_{\ZZ \underset{R} \ltimes \TT^2}(\cX)
\ar[rr]^-{\rm diagonal}
&&
\Mod_{ \ZZ \underset{R} \ltimes \TT^2}(\cX)
\underset{\Mod_{\TT^2}(\cX)}\times
\Mod_{\ZZ \underset{R} \ltimes \TT^2}(\cX)
.
}
\]
In particular, for $X\in \cX$ an object, an action $\Diff^{\fr}(\TT^2) \lacts X$ is 
\begin{enumerate}
\item
an action $\TT^2\lacts X$~,

\item
extensions of this action to actions $\ZZ \underset{U_1}\ltimes \TT^2 \lacts X$ and $\ZZ \underset{U_2} \ltimes \TT^2  \lacts X$~,

\item
an identification under $\TT^2 \lacts X$ of the two restricted actions
%\[
%\xymatrix{
%\TT^2 \underset{R}\rtimes \ZZ 
%\ar@(u,u)[rr]^-{ \id \rtimes \lag \tau_{1} \tau_{2} \tau_{1}\rag  } 
%\ar@(d,d)[rr]_-{ \id \rtimes \lag \tau_{2} \tau_{1} \tau_{2}\rag  } 
%&&
%\TT^2 \underset{U_1, U_2}\rtimes (\ZZ \amalg \ZZ) \lacts X
%~.
%}
%\]
\[
\begin{tikzcd}
\ZZ \underset{R}\ltimes \TT^2 \arrow[rr, yshift=0.7ex, "\lag \tau_{1} \tau_{2} \tau_{1}\rag \ltimes \id "] \arrow[swap, rr, yshift=-0.7ex, "\lag \tau_{2} \tau_{1} \tau_{2}\rag \ltimes \id "]
&
&
(\ZZ \amalg \ZZ) \underset{U_1, U_2}\ltimes \TT^2  \lacts X
\end{tikzcd}
\]


\end{enumerate}


\end{cor}



































%
%\subsection{Isogeny of framed tori}
%
%
%
%
%
%
%\begin{lemma}
%\label{t44}
%There is a canonical equivalence in the diagram among monoid-objects in $\Spaces$:
%\[
%\xymatrix{
%\TT^2 \rtimes \Ebraid
%\ar@{-->}[rr]^-{\simeq}_{\Aff^{\fr}}
%\ar[d]_-{\id\rtimes \Psi}
%&&
%\Imm^{\fr}(\TT^2) 
%\ar[d]^-{\rm forget}
%\\
%\TT^2 \rtimes \EZ
%\ar[rr]^-{\simeq}_-\Aff
%&&
%\Imm(\TT^2)
%.
%}
%\]
%
%\end{lemma}
%
%\begin{proof}
%{\color{magenta}
%Gotta construct the map first, somehow.
%}
%
%
%
%Consider the fiber sequence of spaces
%\[
%\xymatrix{
%\TT^2 \rtimes \Ebraid
%\ar[rr]^-{\rm quotient}
%\ar[d]_-{\Aff^{\fr}}
%&&
%(\TT^2 \rtimes \Ebraid)_{\TT^2 \rtimes \Braid} 
%\ar[d]^-{\Aff^{\fr}_{\Aff^{\fr}}}
%\ar[rr]
%&&
%\sB \bigl( 
%\TT^2 \rtimes \Braid
%\bigr)
%\ar[d]^-{\sB \Aff^{\fr}}
%\\
%\Imm^{\fr}(\TT^2)
%\ar[rr]^-{\rm quotient}
%&&
%\Imm^{\fr}(\TT^2)_{/\Diff^{\fr}(\TT^2)}
%\ar[rr]
%&&
%\sB \Diff(\TT^2)
%.
%}
%\]
%We seek to show that the left vertical map is an equivalence.
%Lemma~\ref{t34} implies the right vertical map is an equivalence.  
%We are therefore reduced to showing the middle vertical map $\Aff^{\fr}_{\Aff^{\fr}}$ is an equivalence.
%This map fits into the canonically commutative diagram among spaces:
%\[
%\xymatrix{
%\w{\GL_2}^{{\sf det}>0}(\RR)_{\Braid}
%\ar[d]^-{\rm (a)}
%&
%\Ebraid_{\Braid}
%\ar[d]^-{\rm (b)}
%\ar[l]_-{\rm inclusion}
%&
%(\TT^2 \rtimes \Ebraid)_{\TT^2 \rtimes \Braid} 
%\ar[r]^-{\Aff^{\fr}_{\Aff^{\fr}}}
%\ar[d]
%\ar[l]_-{\pr_{\pr}}
%&
%\Imm^{\fr}(\TT^2)_{/\Diff^{\fr}(\TT^2)}
%\ar[d]
%\\
%\GL_2(\RR)_{\GL_2(\ZZ)}
%&
%\EZ_{\GL_2(\ZZ)}
%\ar[l]_-{\rm inclusion}
%&
%\bigl( \TT^2 \rtimes \EZ\bigr)_{\TT^2 \rtimes \GL_2(\ZZ)}
%\ar[r]^-{\Aff_{\Aff}}
%\ar[l]_-{\pr_{\pr}}
%&
%\Imm(\TT^2)_{/\Diff(\TT^2)}
%.
%}
%\]
%Observation~\ref{t22} implies the bottom rightward map $\Aff_{\Aff}$ is an equivalence.  
%The leftward maps $\pr_{\pr}$ are evidently equivalences.
%The fiber of the right vertical map over a finite-sheeted covering space $E\xra{\pi} \TT^2$ is contractible: it is the space of framings $\w{\varphi}$ on $E$ together with an identification $\w{\varphi}\simeq \pi^\ast \varphi$ with the pullback framing.  
%Therefore, the right vertical map is an equivalence.  
%It remains to verify that the vertical map $\rm (b)$ is an equivalence.
%The definition of the monoid $\Ebraid$ as a pullback implies the left square is a pullback. 
%The third isomorphism theorem of group theory gives that the map $\rm (a)$ is an equivalence.
%It follows that the map $\rm (b)$ is an equivalence, as desired.
%
%%Note that the maps $\rm (a)$ and $\rm (b)$ factor
%%\[
%%{\rm (b)}
%%\colon
%%\Ebraid_{\Braid}
%%\ar[rr]
%%\ar[d]
%%&&
%%\EpZ_{\SL_2^(\ZZ)}
%%\ar[rr]
%%\ar[d]
%%&&
%%\EZ_{\GL_2(\ZZ)}
%%\\
%%{\rm (a)}
%%\colon
%%\w{\GL_2}^{{\sf det}>0}(\RR)_{\Braid}
%%\ar[rr]
%%&&
%%\GL_2^{{\sf det}>0}(\RR)_{\SL_2(\ZZ)}
%%\ar[rr]
%%&&
%%\GL_2(\RR)_{\GL_2(\ZZ)}
%%~.
%%\]
%%It is routine to verify that the second maps in each of these factorizations is a bijection.
%%The third isomorphism of group theory implies the map $\rm (a)$ is a bijection.  
%%Because the map $\Ebraid \to \EpZ$ is surjective, the first map in this factorization is surjective.
%%It is routine to verify that the second map in this factorization is a bijection.
%%It follows that the map $\rm (b)$ is surjective. 
%%It remains to show that the map $\rm (b)$ is injective.  
%%For this, observe that 
%
%
%
%\end{proof}































































































{\color{magenta}


\subsection{The finite orbit category of $\TT^2$}




\begin{equation}
\label{t75}
{\sf Orbit}^{\sf fin}_{\TT^2}
\longrightarrow
\Ar\bigl( \fB \Imm^{\fr}(\TT^2) \bigr)
~,\qquad
\Bigl(
{\TT^2}_{/C}
\mapsto 
\TT^2 \to {\TT^2}_{/C}
\Bigr)
~.
\end{equation}













\begin{remark}
$\bigl(\fB \Imm^{\fr}(\TT^2,\varphi) \bigr)^{\ast/}$ is a poset!
Say what it is.
\end{remark}





}
























\section{Natural symmetries of secondary Hochschild homology}

Let $A$ be an associative algebra in $\cV$.  
Consider the \bit{Hochschild homology} of $A$ in $\cV$:
\[
\sHH(A)
~:=~
\sHH_\cV(A) 
~:= ~
A \underset{ A^{\op}\ot A} \ot A
~\simeq~
\bigl|
\bBar_\bullet^{\sf cyc}(A)
\bigr|
~\in~ \cV
~.
\]
In~\cite{connes??}, Connes constructs a canonical $\TT$-action on $\sHH(A)$, which is functorial in the algebra $A$.
When working over the sphere spectrum (which is to say $\cV = ({\sf Spectra},\wedge)$) so that $\sHH_{\sf Spectra}(A) = {\sf THH}(A)$ is \bit{topological Hochschild homology}, in~\cite{??} Bokstedt-Hsaing-Madsen extend this $\TT$-action as a \bit{cyclotomic structure} on ${\sf THH}(A)$.  
In~\cite{cyclo} it is demonstrated how this cyclotomic structure on ${\sf THH}(A)$ is derived from an action of the continuous monoid $\NN^\times \ltimes \TT$ on the unstable version $\sHH_{\Spaces}(A)$.
Here, we prove Theorem~\ref{B1}, which constructs a canonical $\Braid \ltimes \TT^2$-action on $\sHH^{(2)}(A)$, which is functorial in the 2-algebra $A$.  
We then prove Theorem~\ref{B2}, which, in the case that $\cV = (\Spaces,\times)$, extends this action to one by the continuous monoid $\Ebraid \ltimes \TT^2$.  





\begin{conventions*}
We fix a symmetric monoidal $\infty$-category $\cV$, and assume it is $\ot$-presentable (meaning the underlying $\infty$-category $\cV$ is presentable, and $\ot$ distributes over colimits separately in each variable).

\end{conventions*}

\begin{example}
For $\Bbbk$ a commutative ring, take $(\cV,\ot) = \bigl({\sf Ch}_\Bbbk[\{{\sf quasi\text{-}isos}\}^{-1}],\underset{\Bbbk}\otimes^{\LL}\bigr)$ to be the $\infty$-categorical localization of chain complexes over $\Bbbk$ on quasi-isomorphisms, with derived tensor product over $\Bbbk$ presenting the symmetric monoidal structure.
More generally, for $R$ a commutative ring spectrum, take $(\cV,\ot) := (\Mod_R,\underset{R}\wedge)$ to be the $\infty$-category of $R$-module spectra and smash product over $R$ as the symmetric monoidal structure.


\end{example}







\subsection{Secondary Hochschild homology of 2-algebras}

In order for the Hochschild homology construction to be twice-iterated, we endow the entity $A\in \cV$ with an algebra structure among algebras.
\begin{definition}
\label{d4}
The $\infty$-category of \bit{2-algebras (in $\cV$)} is
\[
\Alg_2(\cV)
~:=~
\Alg_{\sf Assoc}\bigl(
\Alg_{\sf Assoc}(\cV)
\bigr)
~.
\]

\end{definition}


\begin{example}
A commutative algebra $A = (A,\mu)$ in $\cV$, 
determines the 2-algebra $(A,\mu,\mu)$ in $\cV$.
This association assembles as a functor
\[
\CAlg(\cV)
\longrightarrow
\Alg_2(\cV)
~,
\]
thusly supplying a host of examples of 2-algebras.

\end{example}



\begin{observation}
\label{t46}
Using that the tensor product of operads is defined by a ``hom-tensor'' adjunction, there is a canonical equivalence between $\infty$-categories:
\[
\Alg_{{\sf Assoc}\ot {\sf Assoc}}(\cV)
~\simeq~
\Alg_2(\cV)
~.
\]
In particular, swapping the two tensor-factors supplies an involution 
\[
\Sigma_2
~\lacts~ 
\Alg_2(\cV)
~.
\]

\end{observation}

\begin{remark}
\label{r12}
After Observation~\ref{t46}, a $2$-algebra in $\cV$ is an object $A\in \cV$ together with two associative algebra structures $\mu_1$ and $\mu_2$ on $A$, and compatibility between them which can be stated as either of the two equivalent structures:
\begin{itemize} 

\item
A lift of the morphism $A\ot A \xra{\mu_2} A$ in $\cV$ to a morphism $(A,\mu_1)\otimes (A,\mu_1) \xra{\mu_2} (A,\mu_1)$ in $\Alg_{\sf Assoc}(\cV)$.

\item
A lift of the morphism $A\ot A \xra{\mu_1} A$ in $\cV$ to a morphism $(A,\mu_2)\otimes (A,\mu_2) \xra{\mu_1} (A,\mu_2)$ in $\Alg_{\sf Assoc}(\cV)$.

\end{itemize}

\end{remark}



\begin{example}
\label{r13}
Consider the operad $\cE_2$ of little 2-disks.  
There is a standard morphism between operads ${\sf Assoc}\ot{\sf Assoc} \to \cE_2$.
Through Observation~\ref{t46}, restriction along this morphism defines a functor between $\infty$-categories
\begin{equation}
\label{e62}
\Alg_{\cE_2}(\cV)
\longrightarrow
\Alg_2(\cV)
~,
\end{equation}
thusly supplying some rich examples of 2-algebras (in fact, Dunn's additivity (Theorem~\ref{t48}) states that the functor~(\ref{e62}) is an equivalence). 
For instance, for $\Bbbk$ a commutative ring, a braided-monoidal $\Bbbk$-linear category $\sR$ is a 2-algebra in the $(2,1)$-category of $\Bbbk$-linear categories.  
{\color{magenta}
For instance, for $G$ a semi-simple Lie group, and for $k$ a level, ...??
}
 



\end{example}













\begin{theorem}[Dunn's additivity~\cite{dunn} (see also Theorem~?? of~\cite{HA})]
\label{t48}
The functor~(\ref{e62})
is an equivalence between $\infty$-categories.

\end{theorem}




\begin{remark}
The action $\sO(2)
\lacts
\Alg_2(\cV)$ of Corollary~\ref{t??}, afforded by Theorem~\ref{t48}, 
extending the evident $\Sigma_2 \wr \sO(1)$-action which swaps the two associative algebra structures (as the $\Sigma_2$-factor) and takes opposites of the two associative algebra structures (as the two $\sO(1)$-factors).
\end{remark}






\begin{definition}
\label{d5}
\bit{Secondary Hochschild homology} is the composite functor, given by twice-iterating Hochschild homology:
\[
\HHt
\colon
\Alg_2(\cV)
~:=~
\Alg_{\sf Assoc}\bigl(
\Alg_{\sf Assoc}(\cV)
\bigr)
\xra{~\sHH~}
\Alg_{\sf Assoc}(\cV)
\xra{~\sHH~}
\cV
~,
\]
\[
(A,\mu_1,\mu_1)
\mapsto
\bigl( \sHH(A,\mu_1) , \sHH(\mu_2) \bigr)
\mapsto
\sHH\bigl( 
\sHH(A,\mu_1)
,
\sHH(\mu_2)
\bigr)
~.
\]

\end{definition}









The following assertion is proved in~\S\ref{sec.fact.hmlgy}.
\begin{cor}
\label{t45}
For $A = (A,\mu_1,\mu_2)$ a $2$-algebra in $\cV$, swapping the roles of algebra structures does not change secondary Hochschild homology:
\[
\sHH\bigl( 
\sHH(A,\mu_1)
,
\sHH(\mu_2)
\bigr)
~\simeq~
\sHH\bigl( 
\sHH(A,\mu_2)
,
\sHH(\mu_1)
\bigr)
~.
\]
More precisely, with respect to the involution $\Sigma_2 \lacts \Alg_2(\cV)$ of Observation~\ref{t46}, secondary Hochschild homology is $\Sigma_2$-invariant:
\[
\HHt\colon \Alg_2(\cV)_{\Sigma_2}
\longrightarrow
\cV
~.
\]

\end{cor}












































\subsection{Proof of Theorem~\ref{t36} and Corollaries~\ref{t55},~\ref{t54},~\ref{t53}}
\label{sec.fact.hmlgy}
This subsection proves Theorem~\ref{t36} and Corollary~\ref{t55}, then Corollary~\ref{t54}, and then Corollary~\ref{t53}.  
The key idea is to use \bit{factorization homology}, as established in~\cite{old.fact}.  

Recall from~\cite{old.fact} the symmetric monoidal $\infty$-category $\Mfld^{\fr}_2$ whose objects are framed 2-manifolds, whose spaces of morphisms are spaces of framed embeddings between them, and whose symmetric monoidal structure is given by disjoint union.
Consider the full $\infty$-subcategories,
\begin{equation}
\label{t73}
\Disk^{\fr}_2 
~\hookrightarrow ~
\Mfld^{\fr}_2 
~\hookleftarrow ~
{\sf BDiff}^{\fr}(\TT^2)
~,
\end{equation}
respectively spanned by those framed 2-manifolds each of whose connected components is equivalent with $\RR^2$, and by those framed 2-manifolds that are equivalent with $\TT^2$.
The left full $\infty$-subcategory is closed with respect to the symmetric monoidal structure.
Restriction along these full $\infty$-subcategories determines the solid diagram among $\infty$-categories:
\begin{equation}
\label{e75}
\xymatrix{
\Alg_{\cE_2}(\cV)
\simeq
\Fun^{\ot}(\Disk_2^{\fr} , \cV )
\ar@{-->}@(u,u)[rr]^-{\int}
&&
\Fun^{\ot}(\Mfld_2^{\fr} , \cV)
\ar[r]^-{\rm restrict}
\ar[ll]_-{\rm restrict}
&
\Fun\bigl({\sf BDiff}^{\fr}(\TT^2) , \cV \bigr)
\simeq
\Mod_{\Diff^{\fr}(\TT^2)}(\cV)
~.
}
\end{equation}
\bit{Factorization homology} is defined as the left adjoint to the leftward restriction functor; 
factorization homology over the torus, as it is endowed with a canonical $\Diff^{\fr}(\TT^2)$-action, is the rightward composite functor:
\begin{equation}
\label{e74}
\int_{\TT^2}
\colon
\Alg_{\cE_2}(\cV)
\longrightarrow
\Mod_{\Diff^{\fr}(\TT^2)}(\cV)
~.
\end{equation}





Next, we explain the following diagram among $\infty$-categories: 
\begin{equation}
\label{e73}
\xymatrix{
\Alg_2(\cV)
\ar[d]_-{\HHt}
&
&
\Alg_{\cE_2}(\cV)
\ar[ll]^-{\simeq}_-{{\sf fgt}_1}
\ar[rr]^-{{\sf fgt}_2}_-{\simeq}
\ar[d]^-{\int_{\TT^2}}
\ar[dr]^-{\int_{\pr_2}}
\ar[dl]_-{\int_{\pr_1}}
&
&
\Alg_2(\cV)
\ar[d]^-{\HHt}
\\
\Mod_{\ZZ \underset{U_1} \ltimes \TT^2}(\cV)
&
\Mod_{\Diff^{\fr}(\pr_1)}(\cV)
\ar[l]^-{\simeq}_-{(\Aff_1^{\fr})^\ast}
\ar[dr]_-{\rm forget}
&
\Mod_{\Diff^{\fr}(\TT^2)}(\cV)
\ar[d]^-{\rm forget}
\ar[r]^-{\rm forget}
\ar[l]_-{\rm forget}
&
\Mod_{\Diff^{\fr}(\pr_2)}(\cV)
\ar[r]_-{\simeq}^-{(\Aff_2^{\fr})^\ast}
\ar[dl]^-{\rm forget}
&
\Mod_{\ZZ \underset{U_2} \ltimes \TT^2}(\cV)
\\
&
&
\Mod_{\TT^2}(\cV)
&
&
.
}
\end{equation}
\begin{itemize}
\item
The functors labeled by ``$\rm forget$'' are restriction along evident morphisms between continuous groups.  
The lower triangles canonically commute.

\item
The functor $\int_{\TT^2}$ is~(\ref{e74}).

\item
For $i=1,2$, the functor $\int_{\pr_i}$ is factorization homology over the circle $\TT$ of the \bit{pushforward} along the projection $\TT^2 \xra{\pr_i} \TT$ onto the $i^{\rm th}$-coordinate, as it is endowed with its canonical $\Diff^{\fr}(\pr_i)$-action.
{\color{red}
(Everywhere, change the notation $\Diff^{\fr}(\TT^2\xra{\pr_i} \TT)$ simply to $\Diff^{\fr}(\pr_i)$.)
}
The \bit{pushforward formula} (Prop~{pushforward formula??} of~\cite{oldfact}),
$
\int_{\pr_i}
\simeq
\int_{\TT}
\int_{\TT}
,
$
supplies commutativity of the upper triangles.  


\item
The functor $\Alg_{\cE_2} \xra{{\sf fgt}_1} \Alg_2(\cV)$ is restriction along the standard morphism between operads ${\sf Assoc}\ot {\sf Assoc} \xra{\rm standard} \cE_2$.
The functor $\Alg_{\cE_2} \xra{{\sf fgt}_2} \Alg_2(\cV)$ is restriction along the morphism between operads ${\sf Assoc}\ot {\sf Assoc} \xra{\rm swap} {\sf Assoc}\ot {\sf Assoc} \xra{\rm standard} \cE_2$.
Theorem~\ref{t??} implies that each of these functors are equivalences.  

\item
For $i=1,2$, the outer vertical functors are $\HHt$, as it is endowed with its canonical action
${\sf Sheer}_i\colon \ZZ \underset{U_i}\ltimes \TT^2 \underset{(\ref{e71})}\lacts \HHt(A)$ from~\S\ref{sec.??}, which is evidently functorial in $A\in \Alg_2(\cV)$.  
%Recall from~\S\ref{sec.??} the action 
%This action is evidently functorial in $A\in \Alg_2(\cV)$:
%\[
%\xymatrix{
%&&
%\Mod_{\TT^2 \underset{U_1} \rtimes \ZZ}(\cV) 
%\ar[d]^-{\rm foget}
%\\
%\Alg_2(\cV)
%\ar[rr]^-{\HHt}
%\ar[urr]^-{\bigl \lag  \TT^2 \underset{U_1} \rtimes \ZZ \underset{{\sf Sheer}_1}\lacts \HHt  \bigr \rag}
%&&
%\cV
%.
%}
%\]
%These lifts supply the left and right vertical functors.



\item
For each $i=1,2$, the functor $(\Aff_i^{\fr})^\ast$ is restriction along the functor $\Aff_i^{\fr}$ of~(\ref{e??}).
Theorem~\ref{t??} implies each of these functors is an equivalence.


\item
Theorem~?? of~\cite{old.fact} implies a canonical identification $\sHH \simeq \int_{\TT}$ between functors $\Alg_{\sf Assoc}(\cV) \to \Mod_{\TT}(\cV)$.
Commutativity of the upper tilted squares follows.  






\end{itemize}
In particular, for each 2-algebra $A\in \Alg_2(\cV)$, there is a canonical action $\Diff^{\fr}(\TT^2) \lacts \HHt(A)$.
Through Theorem~\ref{Theorem A}(2a), this is an action $\Braid \ltimes \TT^2 \lacts \HHt(A)$, which establishes the statement of Theorem~\ref{t36}.
Furthermore, commutativity of the diagram~(\ref{e73}) implies Corollary~\ref{t55}.  
Corollary~\ref{t54} follows from this commutative diagram, upon noticing that the top horizontal composite functor $\Alg_2(\cV) \to \Alg_2(\cV)$ is the involution of Observation~\ref{t46}.









%
%
%The following result is an immediate consequence of \bit{pushforward} for factorization homology.
%\begin{prop}
%\label{t47}
%The diagram among $\infty$-categories
%\[
%\xymatrix{
%\Alg_2(\cV)
%\ar[dd]_-{\rm involution~(Obs~\ref{t46})}^-{\simeq}
%\ar@(-,u)[drrrr]^-{\HHt}
%&&
%&&
%\\
%&&
%\Alg_{\cE_2}(\cV)
%\ar[dll]^-{\simeq}_-{\rm Thm~\ref{t48}}
%\ar[ull]_-{\simeq}^-{\rm Thm~\ref{t48}}
%\ar[rr]^-{\int_{\TT^2}}
%&&
%\cV
%\\
%\Alg_2(\cV)
%\ar[uu]
%\ar@(-,d)[urrrr]_-{\HHt}
%&&
%&&
%}
%\]
%canonically commutes.
%In particular, 
%the secondary Hochschild homology of a 2-algebra $A$ 
%is canonically identified as the factorization homology over the framed 2-torus of the unique $\cE_2$-algebra structure on $A$ lifting its 2-algebra structure:
%\begin{equation}
%\label{e34}
%\HHt(A)
%~\simeq~
%\int_{\TT^2} A
%~.
%\end{equation}
%
%\end{prop}
%
%
%
%\begin{proof}
%%
%%\begin{eqnarray}
%%\nonumber
%%\HHt(-)
%%&
%%~:=~
%%&
%%\sHH\bigl( 
%%\sHH(-)
%%\bigr)
%%\\
%%\nonumber
%%&
%%\int_{\TT}
%%\int_{\TT}
%%(
%%
%%)
%%&
%%
%%\\
%%
%%
%%
%%
%%\]
%%
%%(A,\mu_1,\mu_1)
%%\mapsto
%%\bigl( \sHH(A,\mu_1) , \sHH(\mu_2) \bigr)
%%\mapsto
%%\sHH\bigl( 
%%\sHH(A,\mu_1)
%%,
%%\sHH(\mu_2)
%%\bigr)
%%
%%
%%
%%\HHt(A)
%%
%%:=
%
%
%
%
%\end{proof}
%
%
%
%
%
%
%\begin{proof}[Proof of Theorem~\ref{t??}(1)]
%The development of factorization homology in~\cite{oldfact} is such that there is a canonical action on the functor $\Diff^{\fr}(\TT^2)\lacts \int_{\TT^2}$ given by factorization homology over the torus.
%This can be depicted as the upper lift in the diagram among $\infty$-categories:
%\[
%\xymatrix{
%&&
%&&
%&&
%\Mod_{\Diff^{\fr}(\TT^2)}(\cV)
%\ar[drr]_-{\simeq}^-{\rm Thm~\ref{Theorem A}(2a)}
%\ar[d]^-{\rm forget}
%&&
%\\
%\Alg_{\cE_2}(\cV)
%\ar@{-->}[urrrrrr]^-{\bigl \lag \Diff^{\fr}(\TT^2) \lacts \int_{\TT^2} \bigr \rag}
%\ar[drr]_-{\simeq}^-{\rm Thm~\ref{t48}}
%\ar[rrrrrr]_-{\int_{\TT^2}}
%&&
%&&
%&&
%\cV
%\ar[drr]^-=
%&&
%\Mod_{\TT^2 \rtimes \Braid}(\cV)
%\ar[d]^-{\rm forget}
%\\
%&&
%\Alg_2(\cV)
%\ar@{-->}[urrrrrr]^-{\bigl \lag \TT^2 \rtimes \Braid \lacts \HHt \bigr \rag}
%\ar[rrrrrr]_-{\HHt}
%&&
%&&
%&&
%\cV
%.
%}
%\]
%The lower lift follows by commutativity of the solid diagram, together with the down-rightward functors being equivalences, as indicated.  
%
%
%
%\end{proof}
%
%
%
%
%
%
%
%
%
%
%
%
%
%
%
%
%
%
%
%
%
%
%
%
%
%
%
%
%
%
%
%
%
%
%
%
%
%
%
%
%
%
%
%
%
%
%
%\subsection{Compatibility with familiar symmetries}
%
%
%







Next, observe that both of the full $\infty$-subcategories~(\ref{e73}) are closed with respect to the $\sO(2)\simeq \GL_2(\RR)$-action on $\Mfld_2^{\fr}$ given by change-of-framing.
It follows that the left adjoint, $\int$, is $\sO(2)$-equivariant.  
This implies the composite functor
\[
\int_{\TT^2}
\colon 
\Alg_{\cE_2}(\cV)
\longrightarrow
\Mod_{\Diff^{\fr}(\TT^2)}(\cV)
\]
is canonically $\sO(2)$-equivariant.  
Given this, Corollary~\ref{t53} follows from Observation~\ref{t43}. 






















\subsection{Proof of Theorem~\ref{t51}}
Here, we fix a presentable $\infty$-category $\cX$ in which products distribute over colimits separately in each variable.  
We regard $\cX$ as a $\ot$-presentable symmetric monoidal $\infty$-category, via its Cartesian monoidal structure.


We prove Theorem~\ref{t51} by extending factorization homology, via the developments of~\cite{afr2}.
Namely, recall from~\cite{afr2} the $\infty$-category $\Mfd_2^{\sfr}$ of \bit{solidly 2-framed stratified spaces}.  
Consider the full $\infty$-subcategory $\bcM_{=2}^{\sfr}\subset \Mfd_2^{\sfr}$ consisting of those solidly 2-framed stratified spaces each of whose strata is 2-dimensional.
\begin{observation}
\label{t56}
Inspection of the definition of $\Mfd_2^{\sfr}$ reveals the following.
\begin{enumerate}

\item
The the moduli space of objects 
\[
\Obj( \bcM_{=2}^{\sfr})
~\simeq~
\underset{[\Sigma,\varphi]} \coprod {\sf BDiff}^{\fr}(\Sigma,\varphi)
\]
is that of a framed 2-manifolds.  
In other words, there is a canonical bijection between framed-diffeomorphism-types of framed 2-manifolds and equivalence-classes of objects in $\bcM_{=2}^{\sfr}$, and for $(\Sigma,\varphi)$ a framed 2-manifold, there is a canonical identification between continuous groups:
\[
\Diff^{\fr}(\Sigma,\varphi)
~\simeq~
\Aut_{\bcM_{=2}^{\sfr}}(\Sigma,\varphi)
~.
\]

\item
Let $(\Sigma,\varphi)$ and $(\Sigma',\varphi')$ be framed 2-manifolds.
The space of morphisms from $(\Sigma,\varphi)$ to $(\Sigma',\varphi')$ in $\bcM_{=2}^{\sfr}$,
\[
\Hom_{\bcM_{=2}^{\sfr}}\bigl( (\Sigma,\varphi) , (\Sigma',\varphi') \bigr)
~\simeq~
\underset{[\w{\Sigma}\xra{\pi}\Sigma]} \coprod \Emb^{\fr}\bigl( (\w{\Sigma},\pi^\ast \varphi) , (\Sigma',\varphi') \bigr)_{/\Diff_{/\Sigma}(\w{\Sigma})}
~,
\]
is the moduli space of finite-sheeted covers over $\Sigma$ together with a framed-embedding from its total space to $(\Sigma',\varphi')$.

\item
Composition in $\bcM_{=2}^{\sfr}$ is given by base-change of framed embeddings along finite-sheeted covers, followed by composition of framed-embeddings:
\[
\Hom_{\bcM_{=2}^{\sfr}}\bigl( (\Sigma,\varphi) , (\Sigma',\varphi') \bigr)
\times
\Hom_{\bcM_{=2}^{\sfr}}\bigl( (\Sigma',\varphi') , (\Sigma'',\varphi'') \bigr)
\longrightarrow
\Hom_{\bcM_{=2}^{\sfr}}\bigl( (\Sigma,\varphi) , (\Sigma'',\varphi'') \bigr)
~,
\]
\[
\Bigl(
~
(\Sigma,\varphi) \xla{\pi} (\w{\Sigma},\pi^\ast \varphi) \xra{f} (\Sigma',\varphi')
~,~
(\Sigma',\varphi') \xla{\pi'} (\w{\Sigma}',{\pi'}^\ast \varphi') \xra{g} (\Sigma'',\varphi'')
~
\Bigr)
\]
\[
~\longmapsto~
\Bigl(
(\Sigma,\varphi) \xla{\pi \circ \pr_1} (\w{\Sigma} _{\pi}\underset{\Sigma'} \times_f \w{\Sigma}',(\pr_1\circ \pi)^\ast \varphi) \xra{f \circ \pr_2} (\Sigma',\varphi')
\Bigr)
~.
\]


\item
Evidently, framed embeddings form the left factor in a factorization system on $\bcM_{=2}^{\sfr}$, whose right factor is (the opposite of) framed finite-sheeted covers.  


\item
Finite products exist in $\bcM_{=2}^{\sfr}$, and are implemented by disjoin union of framed 2-manifolds.


\item
For each framing $\varphi$ of the 2-torus $\TT^2$, there is a canonical identification between continuous monoids:
\[
\Imm^{\fr}( \TT^2,\varphi)^{\op}
~\simeq~
\End_{\bcM_{=2}^{\sfr}}(\TT^2,\varphi)
~.
\]


\end{enumerate}

\end{observation}
Denote the full $\infty$-subcategory
\[
\iota
\colon
\bcD_{=2}^{\sfr}
~\subset~
\bcM_{=2}^{\sfr}
\]
consisting of those framed 2-manifolds each of that are equivalent with a finite disjoint union of framed Euclidean spaces.
Regard both $\bcD_{=2}^{\sfr}$ and $\bcM_{=2}^{\sfr}$ as symmetric monoidal $\infty$-categories, via their Cartesian monoidal structures.\footnote{It is straight-forward to observe that the full $\infty$-subcategory $\bcD_{=2}^{\sfr}\subset \bcM_{=2}^{\sfr}$ is closed under finite products.}
Notice the evident monomorphisms be symmetric monoidal $\infty$-categories,
\[
\rho
\colon 
\Disk_2^{\fr}
~\hookrightarrow~
\bcD_{=2}^{\sfr}
\qquad
\text{ and }
\qquad
\rho
\colon 
\Mfld_2^{\fr}
~\hookrightarrow~
\bcM_{=2}^{\sfr}
~,
\]
each of whose images consists of all objects yet only those morphisms $\bigl( (\Sigma,\varphi) \xla{\pi} (\w{\Sigma},\pi^\ast \varphi) \xra{f} (\Sigma',\varphi')\bigr)$ in which $\pi$ is a diffeomorphism.\footnote{In other words, $\rho$ is the inclusion of the left factor in the factorization system of Observation~\ref{t56}(4).}


Consider the full $\infty$-subcategory
\[
\Fun^{\times}\bigl(
\bigl( \bcD_{2}^{\sfr}, \cX \bigr)
\bigr)
~\subset~
\Fun\bigl(
\bigl( \bcD_{2}^{\sfr}, \cX \bigr)
\bigr)
\]
consisting of those functors that preserve finite products.  
\begin{prop}
\label{t58}
Restriction along $\rho$ defines an equivalence between $\infty$-categories:
\[
\rho^\ast
\colon
\Fun^{\times}\bigl(
\bigl( \bcD_{2}^{\sfr}, \cX \bigr)
\bigr)
\longrightarrow
\Fun^{\ot}\bigl( \Disk_2^{\fr} , \cX \bigr)
~\simeq~
\Alg_{\cE_2}(\cX)
~.
\]

\end{prop}

\begin{proof}
Let $A\in \Alg_{\cE_2}(\cX)\simeq \Fun^{\ot}(\Disk_2^{\fr} , \cX)$.
We seek to construct an extension:
\begin{equation}
\label{e80}
\xymatrix{
\Disk_2^{\fr}
\ar[rr]^-{A}
\ar[d]_-{\rho}
&&
\cX
\\
\bcD_{2}^{\sfr}
\ar@{-->}[urr]_-{{\rho^\ast}^{-1}(A)}
&&
}
\end{equation}
in which $\w{A}$ preserves finite products.
{\color{magenta}

DO IT

}
The extension $\w{A}$ is evidently functorial in $A\in \Alg_{\cE_2}(\cX)$.  



\end{proof}


The inverse of restriction along $\rho$ followed by left Kan extension along $\iota$ defines a composite functor
\[
\w{\int}
\colon
\Alg_{\cE_2}(\cX)
~\simeq~
\Fun^{\ot}( \Disk_2^{\fr} , \cX )
\xra{~{\rho^\ast}^{-1}~}
\Fun^{\times}( \bcD_{=2}^{\sfr} , \cX )
\xra{~\iota_!~}
\Fun^{\times}( \bcM_{=2}^{\sfr} , \cX )
~.
\]


\begin{prop}
\label{t57}
For each framed 2-manifold $(\Sigma,\varphi)$, the diagram among $\infty$-categories canonically commutes:
\[
\xymatrix{
\Alg_{\cE_2}(\cX)
\ar[rr]^-{\w{\int}}
\ar[d]_-{\int}
&&
\Fun^{\times}( \bcM_{=2}^{\sfr} , \cX )
\ar[rr]^-{\rm restriction}
&&
\Fun\bigl( {\sf BAut}_{\bcM_{=2}^{\sfr}}(\Sigma,\varphi) , \cX \bigr)
\ar[d]_-{\simeq}^-{\rm Obs~\ref{t56}(1)}
%&
%\underset{\rm Obs~\ref{t56}(5)}\simeq
%&
%\Fun\bigl( \fB \End_{\bcM_{=2}^{\sfr}}(\Sigma,\varphi) , \cX \bigr)
%\ar[d]^-{\rm forget}
\\
\Fun^{\ot}( \Mfld_2^{\fr}, \cX )
\ar[rr]^-{\rm restriction}
&&
\Fun\bigl( {\sf BAut}_{\Mfld_2^{\fr}}(\Sigma,\varphi) , \cX \bigr)
\ar[rr]^-{\simeq}
&&
\Mod_{\Diff^{\fr}(\Sigma,\varphi)}(\cX)
.
}
\]

\end{prop}


\begin{proof}
Let $A\in \Alg_{\cE_2}(\cX)\simeq \Fun^{\ot}(\Disk_2^{\fr} , \cX)$.
Let $(\Sigma,\varphi)$ be a framed 2-manifold.
Using Proposition~\ref{t??}, the monomorphism $\rho$ determines a canonical morphism between colimits in $\cX$,
\begin{eqnarray}
\label{e78}
\int_{\TT^2} A
&
~\simeq~
&
\colim
\Bigl(
\Disk_{2/(\TT^2,\varphi)}^{\fr}:=
\Disk_2^{\fr} \underset{\Mfld_2^{\fr}}\times \Mfld_{2/(\TT^2,\varphi)}^{\fr}
\xra{\pr}
\Disk_2^{\fr}
\xra{A}
\cX
\Bigr)
%\\
%\nonumber
%&
%~
%\underset{(\ref{e80})}
%\simeq
%~
%&
%\colim
%\Bigl(
%\Disk_{2/(\TT^2,\varphi)}^{\fr}:=
%\Disk_2^{\fr} \underset{\Mfld_2^{\fr}}\times \Mfld_{2/(\TT^2,\varphi)}^{\fr}
%\xra{\pr}
%\Disk_2^{\fr}
%\xra{\rho^\ast {\rho^\ast}^{-1}(A)}
%\cX
%\Bigr)
\\
\nonumber
&
\xra{~\rho~}
&
\colim
\Bigl(
\bcD_{=2/(\TT^2,\varphi)}^{\sfr}:=
\bcD_{=2}^{\sfr} \underset{\bcM_{=2}^{\sfr}}\times \bcM_{=2/(\TT^2,\varphi)}^{\sfr}
\xra{\pr}
\bcD_{=2}^{\sfr}
\xra{{\rho^\ast}^{-1}(A)}
\cX
\Bigr)
\\
\nonumber
&
~\simeq~
&
\w{\int}_{\TT^2} A
~.
\end{eqnarray}
This morphism is manifestly $\Diff^{\fr}(\Sigma,\varphi)$-equivariant and functorial in $A\in \Alg_{\cE_2}(\cX)$ as so.
So the proposition is proved upon showing this morphism~(\ref{e78}) is an equivalence.
The morphism~(\ref{e78}) is an equivalence provided the canonical functor
\begin{equation}
\label{e79}
\Disk_{2/(\TT^2,\varphi)}^{\fr}
\longrightarrow
\bcD_{=2/(\TT^2,\varphi)}^{\sfr}
\end{equation}
is final.
But the factorization system of Observation~\ref{t56}(4) reveals that this functor~(\ref{e79}) is a right adjoint: its left adjoint given by 
\[
\bcD_{=2/(\TT^2,\varphi)}^{\sfr}
\longrightarrow
\Disk_{2/(\TT^2,\varphi)}^{\fr}
~,\qquad
\bigl(D \xla{\pi} \w{D} \xra{f} (\TT^2,\varphi) \bigr)
\mapsto
\bigl(
\w{D} \xra{f} (\TT^2,\varphi)
\bigr)
~.
\]
Finality of the functor~(\ref{e79}) follows, which completes this proof.




\end{proof}





Now let $\varphi$ be a framing of the 2-torus $\TT^2$.
Proposition~\ref{t57}, together with Observation~\ref{t56}(6), immediately supply a filler in the commutative diagram among $\infty$-categories:
\begin{equation}
\label{e76}
\xymatrix{
&&
\Fun\bigl( \fB \End_{\bcM_{=2}^{\sfr}}(\TT^2,\varphi) , \cX \bigr)
\ar[rr]^-{\simeq}_-{\rm Obs~\ref{t56}(6)}
&&
\Mod_{\Imm^{\fr}(\TT^2,\varphi)^{\op}}(\cX)
\ar[d]^-{\rm forget}
\ar[rr]^-{\simeq}_{\rm Thm~\ref{Theorem A}(2b)}
&&
\Mod_{\square \ltimes \TT^2}(\cX)
\ar[d]^-{\rm forget}
\\
\Alg_{\cE_2}(\cX)
\ar[rrrr]^-{\bigl \lag \Diff^{\fr}(\TT^2,\varphi) \lacts \int_{\TT^2} \bigr \rag}_-{(\ref{e75})}
\ar@(u,-)@{-->}[urr]^-{\bigl \lag \Imm^{\fr}(\TT^2,\varphi)^{\op} \lacts  \w{\int}_{\TT^2} \bigr \rag}
&&
&&
\Mod_{\Diff^{\fr}(\TT^2,\varphi)}(\cX)
\ar[rr]^-{\simeq}_{\rm Thm~\ref{Theorem A}(2a)}
&&
\Mod_{\Braid \ltimes \TT^2}(\cX)
.
}
\end{equation}
The first statement of Theorem~\ref{t51} follows from this commutative diagram~(\ref{e76})
after the commutative diagram~(\ref{e73}). 


{\color{magenta}

We now prove the second statement of Theorem~\ref{t51}, that this action $\Imm^{\fr}(\TT^2,\varphi)\lacts \HHt(A)$ defines a secondary cyclotomic structure on $\HHt(A)$.  

}

%\end{cor}



















%{\color{magenta}
%\subsection{Compatibility with iterated unstable cyclotomic structure}
%We explain a relationship between an unstable secondary cyclotomic structure and a twice-iterated unstable cyclotomic structure.
%
%
%Taking products defines a functor between $\infty$-categories:
%\[
%{\sf Orbit}^{\sf fin}_{\TT}
%\times
%{\sf Orbit}^{\sf fin}_{\TT}
%\xra{~\times~}
%{\sf Orbit}^{\sf fin}_{\TT^2}
%~,\qquad
%\bigl(
%\TT_{/C}
%,
%\TT_{/D}
%\bigr)
%\mapsto 
%\TT^2_{/C\times D}
%~.
%\]
%This functor is evidently equivariant with respect to the morphism between monoids 
%\[
%\NN^\times \times \NN^\times
%\longrightarrow
%\Ebraid
%~,\qquad
%(s,r)\mapsto \begin{bmatrix} s & 0 \\ 0 & r \end{bmatrix}
%~.
%\]
%For $\cX$ an $\infty$-category, this results in a forgetful functor from unstable 2-cyclotomic objects to iterated unstable 1-cyclotomic objects:
%\[
%{\sf Cyc}^{{\sf un} (2)}(\cX)
%\longrightarrow
%{\sf Cyc}^{\sf un}\bigl( {\sf Cyc}^{\sf un}(\cX) \bigr)
%~.
%\]
%
%
%
%}






































%
%Let $A \in \Alg\Bigl( \Alg(\cV) \bigr)$ an associative algebra among associative algebras in $\cV$.
%Consider its iterated Hochschild homology
%\[
%\sHH\bigl( \sHH( A ) \bigr)
%~\in~
%\cV
%~.
%\]
%There is a continuous canonical action of 
%\begin{equation}
%\label{e33}
%\Diff^{\fr}(\TT^2 \xra{\pr}\TT) \underset{\rm Obs~\ref{t14}}\simeq \TT^2 \rtimes \ZZ \lacts \sHH\bigl( \sHH( A ) \bigr)
%~,
%\end{equation}
%which intertwines the two Connes' operators.  
%Through Dunn's additivity~(\cite{dunn??}), $A$ is canonically identified as an $\cE_2$-algebra in $(\cV,\ot)$.
%Thereafter, pushforward for factorization homology (Theorem~?? of~\cite{old.fact}) identifies the iterated Hochschild homology as factorization homology (in the sense of~\cite{old.fact}) over the framed torus:
%\begin{equation}
%\label{e34}
%\int_{\TT^2} A
%~\simeq~
%\sHH\bigl( \sHH( A ) \bigr)
%~\in~
%\cV
%~.
%\end{equation}
%Through this identification~(\ref{e34}), the continuous action~(\ref{e33}) extends via Observation~\ref{t15} to a continuous action
%\begin{equation}
%\label{e35}
%\TT^2 \rtimes \Braid
%\underset{\rm Thm~\ref{Theorem A}}\simeq 
%\Diff^{\fr}(\TT^2) 
%\lacts 
%\int_{\TT^2} A
%~
%\underset{(\ref{e34})}\simeq
%~
%\sHH\bigl( \sHH( A ) \bigr)
%~.
%\end{equation}


%
%
%Recall the Dennis trace map: $\sK(A) \xra{{\sf tr}} \sHH(A)$~(see~\cite{??}).
%There is a canonical factorization of this Dennis trace map through negative-cyclic homology
%$
%\sK(A)
%\xra{\sf tr^-}
%\sHH^-(A)
%:=
%\sHH(A)^{{\sf h}\TT}
%$
%~(see~\cite{??}).
%Iterating this trace map results in a map
%\[
%\sK\bigl( \sK(A) \bigr)
%\xra{~\sf tr^-( \sf tr^-)~}
%\sHH\bigl( \sHH(A)^{{\sf h}\TT}\bigr)^{{\sf h}\TT}
%~.
%\]
%Through the universal property of iterated $\sK$-theory, as articulated in~\cite{mazel.gee.ruben}, the construction of this iterated trace map immediately yields a canonical factorization:
%\begin{equation}
%\label{e36}
%{\Small
%\xymatrix{
%\Obj\Bigl( \Perf\bigl( \Perf(A) \bigr) \Bigr) \ar[rr] \ar[dd]
%&&
%\sHH\Bigl( \sHH\bigl( \Perf\bigl( \Perf(A) \bigr) \bigr) \Bigr)^{{\sf h} ( \TT^2 \rtimes \Braid) }  \ar[d]
%&&
%\sHH\bigl( \sHH( A ) \bigr)^{{\sf h} ( \TT^2 \rtimes \Braid) } \ar[d] \ar[ll]^-{\simeq}
%\\
%&&
%\sHH\Bigl( \sHH\bigl( \Perf\bigl( \Perf(A) \bigr)^{{\sf h} \TT}  \bigr) \Bigr)^{{\sf h} \TT}  \ar[d]
%&&
%\sHH\bigl( \sHH( A )^{{\sf h} \TT} \bigr)^{{\sf h} \TT} \ar[d] \ar[ll]^-{\simeq}
%\\
%\sK\bigl( \sK( A ) \bigr) \ar[rr]^-{{\sf tr}({\sf tr})}
%\ar[urr]^-{{\sf tr}^-({\sf tr}^-)}
%\ar@(-,u)@{-->}[uurr]
%&&
%\sHH\Bigl( \sHH\bigl( \Perf\bigl( \Perf(A) \bigr) \bigr) \Bigr)
%&&
%\sHH\bigl( \sHH( A ) \bigr) \ar[ll]^-{\simeq}_-{\rm Morita~invariance}
%.
%}
%}
%\end{equation}
%
%
%
%
%
%For the case $(\cV,\ot) = (\Spectra,\wedge)$, it is known~(\cite{??}) that the iterated cyclotomic trace map
%\[
%\sK\bigl( \sK( A ) \bigr) 
%\longrightarrow
%{\sf TC}\bigl( {\sf TC}( A ) )
%\]
%is not locally constant (in the argument $A$).  
%We hope the identification of Theorem~\ref{t11} is a step toward correcting this.  
%For instance, the work~\cite{cyclo} explains how the cyclotomic structure on $\sHH(A)$ is derived from canonical actions of the topological monoid $\TT \rtimes \NN^\times \simeq \Emb^{\fr}(\TT)  \lacts  \int_{\TT} A' \simeq \sHH(A)$ for $A'$ a suspension spectrum of an associative algebra in $\Spaces$.  
%By analogy, the lift~(\ref{e36}) suggests that the 2-dimensional analogue of a cyclotomic structure would be in terms of the topological monoid $??$, and that this structure on $\HHt(A)$ would be derived through the action~(\ref{e37}) from the case $(\cV,\ot) = (\Spaces,\times)$.
%To that end, we pose the following problem of improving the action~(\ref{e35}), which is merely homotopy-coherent, to a genuine action with respect to finite subgroups of $\TT^2 \rtimes \Braid$.
%\begin{problem}
%\label{c1}
%Let $\cV$ be a symmetric monoidal $\infty$-category that is $\ot$-presentable.
%Suppose that the monoidal structure of $\cV$ is the Cartesian one.
%Let $A \in \Alg\Bigl( \Alg(\cV) \bigr)$.
%Lift the action~(\ref{e35}) to a \bit{genuine finite} action. 
%Specifically, construct a canonical extension of the action~(\ref{e35}) as a functor:
%\begin{equation}
%\label{e37}
%\xymatrix{
%\TT^2 \underset{(\ref{e3})} \rtimes \Braid
%\ar[rr]^-{\rm Thm~\ref{Theorem A}}_-{\simeq}
%\ar[d]
%&&
%{\sf BDiff}^{\fr}(\TT^2)
%\ar[d]
%\ar[rrrr]^-{\Diff^{\fr}(\TT^2) 
%\underset{(\ref{e35})}\lacts
%\HHt(A)
%}
%&&
%&&
%\cV
%\\
%{\sf Orbit}^{\sf fin}_{\TT^2}
%\ar[rr]^-{\rm Thm~\ref{t11}}_-{\simeq}
%&&
%\Ar\bigl( \fB \Imm^{\fr}(\TT^2) \bigr)_{|{\sf BDiff}^{\fr}(\TT^2)}
%\ar@{-->}[urrrr]
%&&
%&&
%.
%}
%\end{equation}
%
%\end{problem}
%
%
%
%{\color{blue}
%\begin{example}
%$A = \Omega^2 Z$.
%
%\end{example}
%
%}












































































\begin{thebibliography}{9}
%\bibitem{alexander} J. W. Alexander, On the deformation of an n-cell, Proc. Nat. Acad. Sei. U.S.A. vol. 9 (1923) pp. 406-407.
\bibitem{ee} Earle, C. J.; Eells, J. The diffeomorphism group of a compact Riemann surface. Bull. Amer. Math. Soc. 73 (1967), no. 4, 557--559. https://projecteuclid.org/euclid.bams/1183528956
\bibitem{mil} J. Milnor, Introduction to Algebraic K-theory, Annals of Math. Study 72,
Princeton Univ. Press, 1971.
\bibitem{rawn} Rawnsley, J. On the Universal Covering Group of the Real Symplectic Group. Journal of Geometry and Physics. 2012
\bibitem{rolf} Rolfsen, D.; Knots and Links. Berkeley, Publish or Perish 1976
\end{thebibliography}



\end{document}