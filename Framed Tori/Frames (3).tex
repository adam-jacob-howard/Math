\documentclass{amsart}

\headheight=8pt
\topmargin=0pt
\textheight=624pt
\textwidth=432pt
\oddsidemargin=18pt
\evensidemargin=18pt

\usepackage{amsmath}
\usepackage{amsfonts}
\usepackage{amssymb}
\usepackage{amsthm}
\usepackage{comment}
\usepackage{epsfig}
\usepackage{psfrag}
\usepackage{mathrsfs}
\usepackage{amscd}
\usepackage[all]{xy}
\usepackage{rotating}
\usepackage{lscape}
\usepackage{amsbsy}
\usepackage{verbatim}
\usepackage{moreverb}
\usepackage{color}
\usepackage{bbm}
\usepackage{eucal}

\usepackage{tikz-cd}
\usetikzlibrary{patterns,shapes.geometric,arrows,decorations.markings}
\usepackage{tikz-3dplot}

\usepackage{caption}
\usepackage{subcaption}

\colorlet{lightgray}{black!15}

\tikzset{->-/.style={decoration={
  markings,
  mark=at position .5 with {\arrow{>}}},postaction={decorate}}}
\tikzset{midarrow/.style={decoration={
    markings,
    mark=at position {#1} with {\arrow{>}}},postaction={decorate}}}



\pagestyle{plain}

\newtheorem{theorem}{Theorem}[section]
\newtheorem{prop}[theorem]{Proposition}
\newtheorem{lemma}[theorem]{Lemma}
\newtheorem{cor}[theorem]{Corollary}
\newtheorem{conj}[theorem]{Conjecture}




\theoremstyle{definition}
\newtheorem{definition}[theorem]{Definition}
\newtheorem{summary}[theorem]{Summary}
\newtheorem{note}[theorem]{Note}
\newtheorem{ack}[theorem]{Acknowledgments}
\newtheorem{observation}[theorem]{Observation}
\newtheorem{construction}[theorem]{Construction}
\newtheorem{terminology}[theorem]{Terminology}
\newtheorem{remark}[theorem]{Remark}
\newtheorem{example}[theorem]{Example}
\newtheorem{q}[theorem]{Question}
\newtheorem{notation}[theorem]{Notation}
\newtheorem{criterion}[theorem]{Criterion}
\newtheorem{convention}[theorem]{Convention}




\theoremstyle{remark}


\definecolor{orange}{rgb}{.95,0.5,0}
\definecolor{light-gray}{gray}{0.75}
\definecolor{brown}{cmyk}{0, 0.8, 1, 0.6}
\definecolor{plum}{rgb}{.5,0,1}


\DeclareMathOperator{\Link}{\sf Link}
\DeclareMathOperator{\Fin}{\sf Fin}
\DeclareMathOperator{\vect}{\sf Vect}
\DeclareMathOperator{\Vect}{\cV{\sf ect}}
\DeclareMathOperator{\Sfr}{S-{\sf fr}}
\DeclareMathOperator{\nfr}{\mathit{n}-{\sf fr}}

\DeclareMathOperator{\pr}{\mathsf{pr}}
\DeclareMathOperator{\ev}{\mathsf{ev}}


\DeclareMathOperator{\bBar}{\sf Bar}
\DeclareMathOperator{\Alg}{\sf Alg}
\DeclareMathOperator{\man}{\sf Man}
\DeclareMathOperator{\Man}{\cM{\sf an}}
\DeclareMathOperator{\Mod}{\sf Mod}
\DeclareMathOperator{\unzip}{\sf Unzip}
\DeclareMathOperator{\Snglr}{\cS{\sf nglr}}
\DeclareMathOperator{\TwAr}{\sf TwAr}
\DeclareMathOperator{\cSpan}{\sf cSpan}
\DeclareMathOperator{\Kan}{\sf Kan}
\DeclareMathOperator{\Psh}{\sf PShv}
\DeclareMathOperator{\LFib}{\sf LFib}
\DeclareMathOperator{\CAlg}{\sf CAlg}


\DeclareMathOperator{\cpt}{\sf cpt}
\DeclareMathOperator{\Aut}{\sf Aut}
\DeclareMathOperator{\colim}{{\sf colim}}
\DeclareMathOperator{\relcolim}{{\sf rel.\!colim}}
\DeclareMathOperator{\limit}{{\sf lim}}
\DeclareMathOperator{\cone}{\sf cone}
\DeclareMathOperator{\Der}{\sf Der}
\DeclareMathOperator{\Ext}{\sf Ext}
\DeclareMathOperator{\hocolim}{\sf hocolim}
\DeclareMathOperator{\holim}{\sf holim}
\DeclareMathOperator{\Hom}{\sf Hom}
\DeclareMathOperator{\End}{\sf End}
\DeclareMathOperator{\ulhom}{\underline{\Hom}}
\DeclareMathOperator{\fun}{\sf Fun}
\DeclareMathOperator{\Fun}{{\sf Fun}}
\DeclareMathOperator{\Iso}{\sf Iso}
\DeclareMathOperator{\map}{\sf Map}
\DeclareMathOperator{\Map}{{\sf Map}}
\DeclareMathOperator{\Mapc}{{\sf Map}_{\sf c}}
\DeclareMathOperator{\Gammac}{{\Gamma}_{\!\sf c}}
\DeclareMathOperator{\Tot}{\sf Tot}
\DeclareMathOperator{\Spec}{\sf Spec}
\DeclareMathOperator{\Spf}{\sf Spf}
\DeclareMathOperator{\Def}{\sf Def}
\DeclareMathOperator{\stab}{\sf Stab}
\DeclareMathOperator{\costab}{\sf Costab}
\DeclareMathOperator{\ind}{\sf Ind}
\DeclareMathOperator{\coind}{\sf Coind}
\DeclareMathOperator{\res}{\sf Res}
\DeclareMathOperator{\Ker}{\sf Ker}
\DeclareMathOperator{\coker}{\sf Coker}
\DeclareMathOperator{\pt}{\sf pt}
\DeclareMathOperator{\Sym}{\sf Sym}

\DeclareMathOperator{\str}{\sf str}

\DeclareMathOperator{\exit}{\sf Exit}
\DeclareMathOperator{\Exit}{\bcE{\sf xit}}

\DeclareMathOperator{\cylr}{{\sf Cylr}}

\DeclareMathOperator{\shift}{\sf shift}




\DeclareMathOperator{\Cat}{{\sf Cat}}
\DeclareMathOperator{\fCat}{{\sf fCat}}
\DeclareMathOperator{\cat}{\fC{\sf at}}
\DeclareMathOperator{\Gcat}{{\sf GCat}_{\oo}}
\DeclareMathOperator{\gcat}{{\sf GCat}}
\DeclareMathOperator{\Dcat}{{\sf GCat}}
\DeclareMathOperator{\dcat}{\fG\fC{\sf at}}
\DeclareMathOperator{\Mcat}{\cM{\sf Cat}}
\DeclareMathOperator{\mcat}{\fD{\sf Cat}}






\DeclareMathOperator{\Ar}{{\sf Ar}}
\DeclareMathOperator{\twar}{{\sf TwAr}}


\DeclareMathOperator{\diskcat}{\sf {\cD}isk_{\mathit n}^\tau-Cat_\infty}
\DeclareMathOperator{\mfdcat}{\sf {\cM}fd_{\mathit n}^\tau-Cat_\infty}
\DeclareMathOperator{\diskone}{\sf {\cD}isk_{1}^{\vfr}-Cat_\infty}

\DeclareMathOperator{\symcat}{\sf Sym-Cat_\infty}
\DeclareMathOperator{\encat}{\cE_{\mathit n}-\sf Cat}
\DeclareMathOperator{\moncat}{\sf Mon-Cat_\infty}

\DeclareMathOperator{\inrshv}{\sf inr-shv}
\DeclareMathOperator{\clsshv}{\sf cls-shv}



\DeclareMathOperator{\qc}{\sf QC}
\DeclareMathOperator{\m}{\sf Mod}
\DeclareMathOperator{\bi}{\sf Bimod}
\DeclareMathOperator{\perf}{\sf Perf}
\DeclareMathOperator{\shv}{\sf Shv}
\DeclareMathOperator{\Shv}{\sf Shv}



\DeclareMathOperator{\psh}{\sf PShv}
\DeclareMathOperator{\gshv}{\sf GShv}
\DeclareMathOperator{\csh}{\sf Coshv}
\DeclareMathOperator{\comod}{\sf Comod}
\DeclareMathOperator{\M}{\mathsf{-Mod}}
\DeclareMathOperator{\coalg}{\mathsf{-coalg}}
\DeclareMathOperator{\ring}{\mathsf{-rings}}
\DeclareMathOperator{\alg}{\mathsf{Alg}}
\DeclareMathOperator{\artin}{{\sf Artin}}%{\disk_{\mathit n}\alg^{\sf Art}_{\mathit k}}
\DeclareMathOperator{\art}{\mathsf{Art}}
\DeclareMathOperator{\triv}{\mathsf{Triv}}
\DeclareMathOperator{\cobar}{\mathsf{cBar}}
\DeclareMathOperator{\ba}{\mathsf{Bar}}

\DeclareMathOperator{\shvp}{\sf Shv_{\sf p}^{\sf cbl}}

\DeclareMathOperator{\lkan}{{\sf LKan}}
\DeclareMathOperator{\rkan}{{\sf RKan}}

\DeclareMathOperator{\Diff}{{\sf Diff}}
\DeclareMathOperator{\sh}{\sf shv}



\DeclareMathOperator{\calg}{\mathsf{CAlg}}
\DeclareMathOperator{\op}{\mathsf{op}}
\DeclareMathOperator{\relop}{\mathsf{rel.op}}
\DeclareMathOperator{\com}{\mathsf{Com}}
\DeclareMathOperator{\bu}{\cB\mathsf{un}}
\DeclareMathOperator{\bun}{\sf Bun}

\DeclareMathOperator{\pbun}{\sf PBun}




\DeclareMathOperator{\cMfld}{{\sf c}\cM\mathsf{fld}}
\DeclareMathOperator{\cBun}{{\sf c}\cB\mathsf{un}}
\DeclareMathOperator{\Bun}{\cB\mathsf{un}}

\DeclareMathOperator{\dbu}{\mathsf{DBun}}

\DeclareMathOperator{\dbun}{\mathsf{DBun}}

\DeclareMathOperator{\bsc}{\mathsf{Bsc}}
\DeclareMathOperator{\snglr}{\sf Snglr}

\DeclareMathOperator{\Bsc}{\cB\mathsf{sc}}


\DeclareMathOperator{\arbr}{\mathsf{Arbr}}
\DeclareMathOperator{\Arbr}{\cA\mathsf{rbr}}
\DeclareMathOperator{\Rf}{\cR\mathsf{ef}}
\DeclareMathOperator{\drf}{\mathsf{Ref}}


\DeclareMathOperator{\st}{\mathsf{st}}
\DeclareMathOperator{\sk}{\mathsf{sk}}
\DeclareMathOperator{\Ex}{\mathsf{Ex}}

\DeclareMathOperator{\sd}{\mathsf{sd}}

\DeclareMathOperator{\inr}{\mathsf{inr}}

\DeclareMathOperator{\cls}{\mathsf{cls}}
\DeclareMathOperator{\act}{\mathsf{act}}
\DeclareMathOperator{\rf}{\mathsf{ref}}
\DeclareMathOperator{\pcls}{\mathsf{pcls}}
\DeclareMathOperator{\opn}{\mathsf{open}}
\DeclareMathOperator{\emb}{\mathsf{emb}}
\DeclareMathOperator{\Cylo}{\mathsf{Cylo}}
\DeclareMathOperator{\Cylr}{\mathsf{Cylr}}


\DeclareMathOperator{\cbl}{\mathsf{cbl}}

\DeclareMathOperator{\pcbl}{\mathsf{p.cbl}}


\DeclareMathOperator{\gl}{\mathsf{GL}_1}

\DeclareMathOperator{\Top}{\mathsf{Top}}
\DeclareMathOperator{\Mfd}{{\cM}\mathsf{fd}}
\DeclareMathOperator{\cMfd}{{\sf c}{\cM}\mathsf{fd}}
\DeclareMathOperator{\Mfld}{{\cM}\mathsf{fld}}
\DeclareMathOperator{\mfd}{\mathsf{Mfd}}
\DeclareMathOperator{\Emb}{\mathsf{Emb}}
\DeclareMathOperator{\enr}{\fE\mathsf{nr}}
\DeclareMathOperator{\LEnr}{\mathsf{LEnr}}
\DeclareMathOperator{\diff}{\mathsf{Diff}}
\DeclareMathOperator{\conf}{\mathsf{Conf}}

\DeclareMathOperator{\MC}{\mathsf{MC}}
\DeclareMathOperator{\strat}{\mathsf{Strat}}
\DeclareMathOperator{\Strat}{\cS\mathsf{trat}}
\DeclareMathOperator{\kan}{\mathsf{Kan}}

\DeclareMathOperator{\dd}{{\cD}\mathsf{isk}}

\DeclareMathOperator{\loc}{\mathsf{Loc}}



\DeclareMathOperator{\poset}{\mathsf{Poset}}



\DeclareMathOperator{\spaces}{\cS\mathsf{paces}}
\DeclareMathOperator{\Spaces}{\cS\mathsf{paces}}

\DeclareMathOperator{\Space}{{\cS}\sf paces}
\DeclareMathOperator{\spectra}{\cS\mathsf{pectra}}
\DeclareMathOperator{\Spectra}{\cS\mathsf{pectra}}
\DeclareMathOperator{\mfld}{\mathsf{Mfld}}
\DeclareMathOperator{\Disk}{\cD{\mathsf{isk}}}
\DeclareMathOperator{\cdisk}{{\sf c}\cD{\mathsf{isk}}}
\DeclareMathOperator{\cDisk}{{\sf c}\cD{\mathsf{isk}}}
\DeclareMathOperator{\sing}{\mathsf{Sing}}
\DeclareMathOperator{\set}{{\mathsf{Sets}}}
\DeclareMathOperator{\Aux}{\cA{\mathsf{ux}}}
\DeclareMathOperator{\Adj}{\mathsf{Adj}}


\DeclareMathOperator{\Dtn}{\cD{\mathsf{isk}^\tau_{\mathit n}}}


\DeclareMathOperator{\sm}{\mathsf{sm}}
\DeclareMathOperator{\vfr}{\sf vfr}
\DeclareMathOperator{\fr}{\sf fr}
\DeclareMathOperator{\sfr}{\sf sfr}


\DeclareMathOperator{\bord}{\mathsf{Bord}}
\DeclareMathOperator{\Bord}{{\sf Bord}_1^{\fr}}
\DeclareMathOperator{\Bordk}{\cB{\sf ord}_1^{\fr}(\RR^k)}

\DeclareMathOperator{\Corr}{{\sf Corr}}
\DeclareMathOperator{\corr}{{\sf Corr}}

\DeclareMathOperator{\fcorr}{{\sf FCorr}}
\DeclareMathOperator{\pcorr}{{\sf PCorr}}



\DeclareMathOperator{\Sing}{\mathsf{Sing}}


\DeclareMathOperator{\BTop}{\sf BTop}
\DeclareMathOperator{\BO}{{\mathsf BO}}


\DeclareMathOperator{\Lie}{\sf Lie}



\def\ot{\otimes}

\DeclareMathOperator{\fin}{\sf Fin}

\DeclareMathOperator{\oo}{\infty}


\DeclareMathOperator{\hh}{\sf HC}

\DeclareMathOperator{\free}{\sf Free}
\DeclareMathOperator{\fpres}{\sf FPres}


\DeclareMathOperator{\fact}{\sf Fact}
\DeclareMathOperator{\ran}{\sf Ran}

\DeclareMathOperator{\disk}{\sf Disk}

\DeclareMathOperator{\ccart}{\sf cCart}
\DeclareMathOperator{\cart}{\sf Cart}
\DeclareMathOperator{\rfib}{\sf RFib}
\DeclareMathOperator{\lfib}{\sf LFib}


\DeclareMathOperator{\tr}{\triangleright}
\DeclareMathOperator{\tl}{\triangleleft}


\newcommand{\lag}{\langle}
\newcommand{\rag}{\rangle}


\newcommand{\w}{\widetilde}
\newcommand{\un}{\underline}
\newcommand{\ov}{\overline}
\newcommand{\nn}{\nonumber}
\newcommand{\nid}{\noindent}
\newcommand{\ra}{\rightarrow}
\newcommand{\la}{\leftarrow}
\newcommand{\xra}{\xrightarrow}
\newcommand{\xla}{\xleftarrow}

\newcommand{\weq}{\xrightarrow{\sim}}
\newcommand{\cofib}{\hookrightarrow}
\newcommand{\fib}{\twoheadrightarrow}

\def\llarrow{   \hspace{.05cm}\mbox{\,\put(0,-2){$\leftarrow$}\put(0,2){$\leftarrow$}\hspace{.45cm}}}
\def\rrarrow{   \hspace{.05cm}\mbox{\,\put(0,-2){$\rightarrow$}\put(0,2){$\rightarrow$}\hspace{.45cm}}}
\def\lllarrow{  \hspace{.05cm}\mbox{\,\put(0,-3){$\leftarrow$}\put(0,1){$\leftarrow$}\put(0,5){$\leftarrow$}\hspace{.45cm}}}
\def\rrrarrow{  \hspace{.05cm}\mbox{\,\put(0,-3){$\rightarrow$}\put(0,1){$\rightarrow$}\put(0,5){$\rightarrow$}\hspace{.45cm}}}

\def\cA{\mathcal A}\def\cB{\mathcal B}\def\cC{\mathcal C}\def\cD{\mathcal D}
\def\cE{\mathcal E}\def\cF{\mathcal F}\def\cG{\mathcal G}\def\cH{\mathcal H}
\def\cI{\mathcal I}\def\cJ{\mathcal J}\def\cK{\mathcal K}\def\cL{\mathcal L}
\def\cM{\mathcal M}\def\cN{\mathcal N}\def\cO{\mathcal O}\def\cP{\mathcal P}
\def\cQ{\mathcal Q}\def\cR{\mathcal R}\def\cS{\mathcal S}\def\cT{\mathcal T}
\def\cU{\mathcal U}\def\cV{\mathcal V}\def\cW{\mathcal W}\def\cX{\mathcal X}
\def\cY{\mathcal Y}\def\cZ{\mathcal Z}

\def\AA{\mathbb A}\def\BB{\mathbb B}\def\CC{\mathbb C}\def\DD{\mathbb D}
\def\EE{\mathbb E}\def\FF{\mathbb F}\def\GG{\mathbb G}\def\HH{\mathbb H}
\def\II{\mathbb I}\def\JJ{\mathbb J}\def\KK{\mathbb K}\def\LL{\mathbb L}
\def\MM{\mathbb M}\def\NN{\mathbb N}\def\OO{\mathbb O}\def\PP{\mathbb P}
\def\QQ{\mathbb Q}\def\RR{\mathbb R}\def\SS{\mathbb S}\def\TT{\mathbb T}
\def\UU{\mathbb U}\def\VV{\mathbb V}\def\WW{\mathbb W}\def\XX{\mathbb X}
\def\YY{\mathbb Y}\def\ZZ{\mathbb Z}

\def\sA{\mathsf A}\def\sB{\mathsf B}\def\sC{\mathsf C}\def\sD{\mathsf D}
\def\sE{\mathsf E}\def\sF{\mathsf F}\def\sG{\mathsf G}\def\sH{\mathsf H}
\def\sI{\mathsf I}\def\sJ{\mathsf J}\def\sK{\mathsf K}\def\sL{\mathsf L}
\def\sM{\mathsf M}\def\sN{\mathsf N}\def\sO{\mathsf O}\def\sP{\mathsf P}
\def\sQ{\mathsf Q}\def\sR{\mathsf R}\def\sS{\mathsf S}\def\sT{\mathsf T}
\def\sU{\mathsf U}\def\sV{\mathsf V}\def\sW{\mathsf W}\def\sX{\mathsf X}
\def\sY{\mathsf Y}\def\sZ{\mathsf Z}

\def\bA{\mathbf A}\def\bB{\mathbf B}\def\bC{\mathbf C}\def\bD{\mathbf D}
\def\bE{\mathbf E}\def\bF{\mathbf F}\def\bG{\mathbf G}\def\bH{\mathbf H}
\def\bI{\mathbf I}\def\bJ{\mathbf J}\def\bK{\mathbf K}\def\bL{\mathbf L}
\def\bM{\mathbf M}\def\bN{\mathbf N}\def\bO{\mathbf O}\def\bP{\mathbf P}
\def\bQ{\mathbf Q}\def\bR{\mathbf R}\def\bS{\mathbf S}\def\bT{\mathbf T}
\def\bU{\mathbf U}\def\bV{\mathbf V}\def\bW{\mathbf W}\def\bX{\mathbf X}
\def\bY{\mathbf Y}\def\bZ{\mathbf Z}
\def\bdelta{\mathbf\Delta}
\def\bDelta{\mathbf\Delta}
\def\blambda{\mathbf\Lambda}


\def\fA{\frak A}\def\fB{\frak B}\def\fC{\frak C}\def\fD{\frak D}
\def\fE{\frak E}\def\fF{\frak F}\def\fG{\frak G}\def\fH{\frak H}
\def\fI{\frak I}\def\fJ{\frak J}\def\fK{\frak K}\def\fL{\frak L}
\def\fM{\frak M}\def\fN{\frak N}\def\fO{\frak O}\def\fP{\frak P}
\def\fQ{\frak Q}\def\fR{\frak R}\def\fS{\frak S}\def\fT{\frak T}
\def\fU{\frak U}\def\fV{\frak V}\def\fW{\frak W}\def\fX{\frak X}
\def\fY{\frak Y}\def\fZ{\frak Z}

\def\bcA{\boldsymbol{\mathcal A}}\def\bcB{\boldsymbol{\mathcal B}}\def\bcC{\boldsymbol{\mathcal C}}
\def\bcD{\boldsymbol{\mathcal D}}\def\bcE{\boldsymbol{\mathcal E}}\def\bcF{\boldsymbol{\mathcal F}}
\def\bcG{\boldsymbol{\mathcal G}}\def\bcH{\boldsymbol{\mathcal H}}\def\bcI{\boldsymbol{\mathcal I}}
\def\bcJ{\boldsymbol{\mathcal J}}\def\bcK{\boldsymbol{\mathcal K}}\def\bcL{\boldsymbol{\mathcal L}}
\def\bcM{\boldsymbol{\mathcal M}}\def\bcN{\boldsymbol{\mathcal N}}\def\bcO{\boldsymbol{\mathcal O}}\def\bcP{\boldsymbol{\mathcal P}}\def\bcQ{\boldsymbol{\mathcal Q}}\def\bcR{\boldsymbol{\mathcal R}}
\def\bcS{\boldsymbol{\mathcal S}}\def\bcT{\boldsymbol{\mathcal T}}\def\bcU{\boldsymbol{\mathcal U}}
\def\bcV{\boldsymbol{\mathcal V}}\def\bcW{\boldsymbol{\mathcal W}}\def\bcX{\boldsymbol{\mathcal X}}
\def\bcY{\boldsymbol{\mathcal Y}}\def\bcZ{\boldsymbol{\mathcal Z}}

\def\ccD{{\sf c}\boldsymbol{\mathcal D}}
\def\bcM{\boldsymbol{\mathcal M}}

\DeclareMathOperator{\Stri}{\boldsymbol{\cS}{\sf tri}}
\DeclareMathOperator{\btheta}{\boldsymbol{\Theta}}
\DeclareMathOperator{\adj}{{\sf adj}}


\DeclareMathOperator{\uno}{\mathbbm{1}}





\DeclareMathOperator{\Braid}{\sf Braid}
\DeclareMathOperator{\GL}{\sf GL}
\DeclareMathOperator{\SL}{\sf SL}
\DeclareMathOperator{\quot}{\sf quot}
\DeclareMathOperator{\Fr}{\sf Fr}
\DeclareMathOperator{\id}{\sf id}
\DeclareMathOperator{\Act}{\sf Act}
\DeclareMathOperator{\trans}{\sf trans}
\DeclareMathOperator{\Bdl}{\sf Bdl}
\DeclareMathOperator{\sHH}{\sf HH}
\DeclareMathOperator{\Obj}{\sf Obj}
\DeclareMathOperator{\Perf}{\sf Perf}


\begin{document}


\title{Framed diffeomorphisms of a torus}


\author{Adam Howard}




\address{Department of Mathematics\\Montana State University\\Bozeman, MT 59717}
\email{david.ayala@montana.edu}
\thanks{The author was supported by ...}






\begin{abstract}
We identify the topological group of framed diffeomorphisms of the standardly framed torus as a semi-direct product of the torus and a braid group on 3 strands.  
It follows that the framed mapping class group is the braid group on 3 strands.  
\end{abstract}



\keywords{??.}

\subjclass[2010]{Primary ??. Secondary ??.}

\maketitle


\tableofcontents


\section*{Introduction}

Vector addition, as well as the standard vector norm, gives $\RR^2$ the structure of a topological abelian group.
Consider its closed subgroup $\ZZ^2\subset \RR^2$.  
The \textit{\textbf{torus}} is the quotient in the short exact sequence of topological abelian groups:
\[
0
\longrightarrow
\ZZ^2
\xra{\rm inclusion}
\RR^2 
\xra{~\quot~}
\TT^2
\longrightarrow
0
~.
\] 
Because $\RR^2$ is connected, and because $\ZZ^2$ acts cocompactly by translations on $\RR^2$, the torus $\TT^2$ is connected and compact.  
The quotient map $\RR^2 \xra{\quot} \TT^2$ endows the torus with the structure of a Lie group, and in particular a smooth manifold.
Using that the smooth map $\RR^2 \xra{\quot} \TT^2$ is a covering space and $\TT^2$ is connected, the canonical homomorphism between groups
\begin{equation}
\label{e2}
\GL_2(\ZZ)
~:=~
\Aut_{\sf Groups}(\ZZ^2)
\xla{~\cong~}
\Aut_{\sf LieGroups}\bigl( \ZZ^2 \hookrightarrow \RR^2)
\xra{~\cong~}
\Aut_{\sf LieGroups}(\TT^2)
~,
\end{equation}
\[
A
\mapsto 
\Bigl(
q
\mapsto 
Aq
:=
\quot( A\w{q} )
\Bigr)
\qquad
{\Small (\text{for any }\w{q} \in \quot^{-1}(q)) }
~,
\]
is an isomorphism. Note that (\ref{e2}) does not depend on $\w{q} \in \quot^{-1}(q)$ and this homomorphism defines a semi-direct product topological group:
\begin{equation}
\label{e6}
\TT^2 \underset{(\ref{e2})}\rtimes \GL_2(\ZZ)
~.
\end{equation}

%I'm sure this could be said better //// Where to place this... 
% \noindent Even more $\RR^2 \xra{\quot} \TT^2$ is a local diffeomorphism. So for every $\tilde{q} \in \RR^{2}$ the map $$D_{\tilde{q}}\quot: T_{\tilde{q}}\RR^{2} \rightarrow T_{q}\TT^{2}$$ is an isomorphism and we can define the inverse isomorphism $$D_{q}\quot^{-1}: T_{q}\TT^{2} \rightarrow T_{\tilde{q}}\RR^{2}$$ for $\tilde{q} = \bigl( \quot^{-1}(q) \cap [0, 1) \times [0, 1) \bigr).$

%Transition 

Consider the space of smooth diffeomorphisms of the torus:
\[
\Diff(\TT^2)
~\subset~
\Map(\TT^2 , \TT^2)
~,
\] 
which is endowed with the subspace topology of the $\sC^\infty$ topology on the set of smooth self-maps of the torus.
Consider the following map.
\begin{equation}
\label{e7}
{\sf Aff}: \TT^2 \underset{(\ref{e2})}\rtimes \GL_2(\ZZ)
\longrightarrow
\Diff(\TT^2)
~,\qquad
(p,A)
\mapsto
\Bigl(
q\mapsto 
Aq + p
\Bigr)
~.
\end{equation}
The following lemmas show that ${\sf Aff}$ is a continuous homomorphism and a homotopy equivalence.
\begin{lemma} \label{t98}
The map defined in ~(\ref{e7}) $${\sf Aff}: \TT^2 \underset{(\ref{e2})}\rtimes \GL_2(\ZZ)
\longrightarrow
\Diff(\TT^2), 
 \qquad {\sf Aff}(p, A) = (q \mapsto Aq + p)$$
is a continuous homomorphism.
\end{lemma}
\begin{proof}
Since the maps $\GL_{2}(\ZZ) \xrightarrow{A \mapsto (q \mapsto{Aq})} \Diff(\TT^{2})$ and $\TT^{2} \xrightarrow{p \mapsto (q \mapsto q + p)} \Diff(\TT^{2})$ are both continuous, we have that ${\sf Aff}$ is also continuous. Now, for $(p_{1}, A_{1}), (p_{2}, A_{2}) \in  \TT^2 \underset{(\ref{e2})}\rtimes \GL_2(\ZZ)$ we have that $$(p_{1}, A_{1}) \cdot (p_{2}, A_{2}) = (A_{1}p_{2} + p_{1}, A_{1}A_{2}).$$ So the image of this map is $${\sf Aff}((p_{1}, A_{1}) \cdot (p_{2}, A_{2})) = {\sf Aff}((A_{1}p_{2} + p_{1}, A_{1}A_{2})) = (q \mapsto  A_{1}A_{2}q + A_{1}p_{2} + p_{1}).$$ The product structure on $\Diff(\TT^2)$ is composition, so $${\sf Aff}(p_{1}, A_{1}) \cdot {\sf Aff}(p_{2}, A_{2}) = (q\mapsto A_{1}q + p_{2}) \circ (q\mapsto A_{2}q + p_{2}) = (q \mapsto A_{2}q + p_{2} \mapsto A_{1}(A_{2}q + p_{2}) + p_{1})$$ $$= (q \mapsto A_{1}A_{2}q + A_{1}p_{2} + p_{1}).$$ Therefore we have that ${\sf Aff}((p_{1}, A_{1}) \cdot (p_{2}, A_{2})) = {\sf Aff}(p_{1}, A_{1}) \cdot {\sf Aff}(p_{2}, A_{2}),$ showing that ${\sf Aff}$ is indeed a group homomorphism. 
\end{proof}


\begin{lemma}\label{t1}
The continuous homomorphism 
$
{\sf Aff}: \TT^2 \underset{(\ref{e2})}\rtimes \GL_2(\ZZ)
\xra{~(\ref{e7})~}
\Diff(\TT^2)
$
is a homotopy equivalence.
\end{lemma}
\begin{proof}
For $G$ a locally-connected topological group, denote by $G_e \subset G$ the path-component containing the identity element in $G$.
This subspace $G_e\subset G$ is a normal subgroup, and the sequence of continuous homomorphisms
\[
1
\longrightarrow
G_e
\xra{~\rm inclusion~}
G
\xra{~\rm quotient~}
\pi_0(G)
\longrightarrow
1
\]
is a short-exact sequence.
This short-exact sequence is evidently functorial in the argument $G$.
In particular, there is a commutative diagram among topological groups 
\[
\xymatrix{
1 \ar[d]_-= \ar[r]
&
\TT^{2} = \bigl( \TT^{2} \underset{(\ref{e2})}\rtimes \GL_2(\ZZ) \bigr)_e
\ar[d]_-{{\sf Aff}_e}
\ar[r]^-{\rm inc}
&
\TT^{2} \underset{(\ref{e2})}\rtimes \GL_2(\ZZ) 
\ar[d]_-{{\sf Aff}} 
\ar[r]^-{\rm quot}
&
\pi_0 \bigl( \TT^{2} \underset{(\ref{e2})}\rtimes \GL_2(\ZZ)  \bigr)
=
\GL_2(\ZZ)
\ar[d]_-{\pi_0({\sf Aff})}
\ar[r]
&
1 \ar[d]_-=
\\
1
\ar[r]
&
\Diff(\TT^2)_e
\ar[r]^-{\rm inc}
&
\Diff(\TT^2) 
\ar[r]^-{\rm quot}
&
\pi_0 \bigl( \Diff(\TT^2) \bigr)
\ar[r]
&
1
.
}
\] 
Theorem 2.D.4. of~\cite{rolf} along with the Kirby torus trick gives us that the vertical homomorphism $\pi_0({\sf Aff})$ is an isomorphism.  %Get a better citation here.
Because both horizontal sequences are exact, if the vertical homomorphism ${\sf Aff}_{e}$ is a homotopy equivalence, then the vertical homomorphism ${\sf Aff}$ will aslo be a homotopy equivalence by the five lemma. We will now show that ${\sf Aff}_{e}$ is a homotopy equivalence. 
With respect to the canonical action of $\Diff(\TT^2)_e$ on $\TT^2$, the orbit of the identity element $0\in \TT^2$ is the evaluation map 
\begin{equation}\label{e25}
\ev_0
\colon 
\Diff(\TT^2)_e
\longrightarrow
\TT^2
~.
\end{equation}
Note that the composition,
\[
\id_{\TT^2}
\colon
\TT^2
\xra{~({\sf Aff})_e~}
\Diff(\TT^2)_e
\xra{~(\ref{e25})~}
\TT^2
~,
\]
is the identity map.
Therefore, to show that ${\sf Aff}_e$ is a homotopy equivalence, it is sufficient to show that the stabilizer, ${\sf Stab}_0\bigl(\Diff(\TT^2)_e\bigr) \subset \Diff(\TT^2)$ of $0\in \TT^2$ (elements $f \in \Diff(\TT^2)_e$ for which $f(0) = 0$) is contractible. Theorem 1b of \cite{ee} shows that the space diffeomorphisms of the torus which leave a specified point fixed is contractible, and therefore ${\sf Stab}_0\bigl(\Diff(\TT^2)_e\bigr)$ is contractible. %This might work quicker
%{\color{red}
%By the isotopy-extension theorem, the canonical map %Clear up some of this language/notation
%\[
%{\sf Stab}_0\bigl(\Diff(\TT^2)_e\bigr)
%\longrightarrow
%\Emb_0(\TT^1 \vee \TT^1 , \TT^2)_e %Specifically state what this thing is
%\]
%is a Serre fibration. 
%By ??, the base of this Serre fibration is contractible.  
%The sought contractibility of ${\sf Stab}_0\bigl(\Diff(\TT^2)_e\bigr)$ therefore follows from contractibility of
%$\Diff(\TT^2~\rm rel~ \TT^1 \vee \TT^1)$.   
%The quotient map $\RR^2 \xra{\quot} \TT^2$ determines an isomorphism between topological groups
%\[
%\Diff(\TT^2~\rm rel~ \TT^1 \vee \TT^1)
%\cong
%\Diff(\II^2~\rm rel~ \partial \II^2 )
%~.
%\]
%Finally, the Alexander Lemma ~\cite{alexander}, grants that $\Diff(\II^2~\rm rel~ \partial \II^2 )$ is contractible.  
%}
\end{proof}

%%MAYBE A BETTER TRANSITION HERE

The quotient map $\RR^2\xra{\quot}\TT^2$ endows the smooth manifold $\TT^2$ with a standard framing (ie, a trivialization of its tangent bundle): 
for 
\[
\trans\colon \TT^2 \times \TT^2 
\xra{~(p,q)\mapsto \trans_p(q) := p+q~} 
\TT^2
\]
the abelian multiplication rule of the Lie group $\TT^2$, 
\begin{equation}
\label{e5}
\varphi_0
\colon
\epsilon^2_{\TT^2}
\xra{~\cong~}
\tau_{\TT^2}
~,\qquad
\TT^2 \times \RR^2 \ni (p,v)
\mapsto
\bigl(p,\sD_{0}(\trans_p \circ \quot) (v) \bigr)
\in \sT \TT^2
~.
\end{equation}
In this way we regard $\TT^2$ as a \textit{\textbf{framed smooth 2-manifold}}.
Consider the space of framings of the torus:
\[
\Fr(\TT^2)
~:=~
\Iso_{\Bdl_{\TT^2}}\bigl( \epsilon^2_{\TT^2} , \tau_{\TT^2} \bigr)
~\subset~
\Map(\TT^2 \times \RR^2 , \sT \TT^2 )
~,
\]
which is endowed with the subspace topology of the $\sC^\infty$ topology on the set of smooth maps between total spaces.  
Note the evident continuous conjugation action of $\Diff(\TT^2)$ on $\Fr(\TT^2)$:
\begin{eqnarray}
\nonumber
\Act\colon
\Diff(\TT^2)\times \Fr(\TT^2)
&
\xra{~(f,\varphi)\mapsto (Df , \varphi , (f^{-1} , \id_{\RR^2}))~}
&
\Aut_{\Bdl_{\TT^2}}(\tau_{\TT^2})
\times 
\Iso_{\Bdl_{\TT^2}}\bigl( \epsilon^2_{\TT^2} , \tau_{\TT^2} \bigr)
\times
\Aut_{\Bdl_{\TT^2}}(\epsilon^2_{\TT^2})^{\op}
\\
\label{e9}
&
\xra{~(a,\varphi , b )\mapsto a \circ \varphi \circ b~}
&
\Fr(\TT^2)
~,
\end{eqnarray}
\[
(f,\varphi)
\mapsto 
\Bigl(
\TT^2 \times \RR^2 \xra{(p,v)\mapsto \sD f \bigl( \varphi(f^{-1}(p),v) \bigr)  }
\sT \TT^2
\Bigr)
~.
\]

Consider the orbit map of $\varphi_0$ for this continuous action:
\begin{equation}\label{e10}
\Diff(\TT^2)
\xra{~(~\id_{\Diff(\TT^2)}~ ,~ {\sf constant}_{\varphi_0}~)~}
\Diff(\TT^2)\times \Fr(\TT^2)
\xra{~(\ref{e9})~}
\Fr(\TT^2)
~.
\end{equation}
The fiber of~(\ref{e10}) is the stabilizer of $\varphi_0$ under the action~(\ref{e9}), which consists of those diffeomorphisms that `strictly' preserve the standard framing $\varphi_0$ of $\TT^2$.
The homotopy fiber of~(\ref{e10}) is the homotopy stabilizer of $\varphi_0$ under the action of~(\ref{e9}), which consists of those diffeomorphisms that `homotopy coherently' preserve the standard framing $\varphi_0$ of $\TT^2$.
\begin{definition}\label{d1}
The topological space of \textit{\textbf{framed diffeomorphisms}} of the framed smooth manifold $(\TT^2,\varphi_0)$ is
\begin{equation}\label{e11}
\Diff^{\sf fr}(\TT^2)
~:=~
\Diff^{\sf fr}(\TT^2,\varphi_0)
~:=~
{\sf hofib}_{\varphi_0}\Bigl( 
\Diff(\TT^2)
\xra{~(\ref{e10})~}
\Fr(\TT^2)
\Bigr)
~.
\end{equation}
\end{definition}
In other words, there is a homotopy-pullback diagram among topological spaces:
\begin{equation}\label{e13}
\xymatrix{
\Diff^{\fr}(\TT^2) \ar[rr] \ar[d]
&&
\Diff(\TT^2) \ar[d]^-{(\ref{e10})}
\\
\ast \ar[rr]^-{\lag \varphi_0\rag}
&&
\Fr(\TT^2)
.
}
\end{equation}


\begin{remark}\label{r5} 
The image of $f$ under (\ref{e10}) is the framing
\begin{equation}\label{e299}
\xymatrix{
\TT^{2} \times \RR^{2} \ar[d] \ar[rr]^-{f^{-1} \times id} && \TT^{2} \times \RR^{2} \ar[rr]^-{\varphi_{0}} 
\ar[d] && {\sT}\TT^{2} \ar[rr]^-{\sD f} \ar[d] && {\sT}\TT^{2} \ar[d] \\
\TT^{2} \ar[rr]^-{f^{-1}} && \TT^{2}  \ar[rr]^-{id} && \TT^{2} \ar[rr]^-{f} && \TT^{2},}
\end{equation} 
so the strict fiber of (\ref{e10}) over $\varphi_{0}$, will be those diffeomorphisims $f$ such that diagram (\ref{e299}) is precisely $\varphi_{0}.$ The only diffeomorphisms of $\TT^{2}$ which satisfy this rigid condition are translations. In considering the homotopy fiber of (\ref{e10}), we get all diffeomorphisms $f$ such that (\ref{e299}) is homotopic to $\varphi_{0}.$ This results in $\Diff^{\sf fr}(\TT^2)$ being a much larger class of diffeomorphisms including small perturbations such as multiplication by bump functions in neighborhoods of $\TT^{2}.$
\end{remark}

%%%%%%%%%%%%%%%%%%%%%%%%%%%%%%%%%%%%%%%%%%%%%%%%%%

\begin{lemma}\label{t3}
There is a canonical equivalence
\[
\Diff^{\fr}(\TT^2)
\xra{~\simeq~}
\Omega \bigl( \Fr(\TT^2)_{{\sf h} \Diff(\TT^2)} \bigr)
\]
over $\Diff(\TT^2) \xra{\simeq}  \Omega \bigl( \ast_{{\sf h} \Diff(\TT^2)} \bigr)$.
In particular, the underlying space of $\Diff^{\fr}(\TT^2)$ can be endowed with the structure of a group-object in the $\infty$-category $\Spaces$, with respect to which the canonical map $\Diff^{\fr}(\TT^2) \to \Diff(\TT^2)$ is a morphism between group-objects in $\Spaces$,.
\end{lemma}


\begin{proof}
The commutative diagram among topological spaces~(\ref{e13}) extends as a commutative diagram among topological spaces:
\begin{equation}\label{e18}
\xymatrix{
\Diff^{\fr}(\TT^2) \ar[rr] \ar[d]
&&
\Diff(\TT^2) \ar[d]^-{(\ref{e10})} \ar[rr]^-{\rm quotient}
&&
\Diff(\TT^2)_{{\sf h}\Diff(\TT^2)} \ar[d]^-{(\ref{e10})_{{\sf h}\Diff(\TT^2)}}
\\
\ast \ar[rr]^-{\lag \varphi_0\rag}
&&
\Fr(\TT^2)
\ar[rr]^-{\rm quotient}
&&
\Fr(\TT^2)_{{\sf h}\Diff(\TT^2)}
,
}
\end{equation}
in which the rightmost terms are homotopy coinvariants.  
Observe the canonical contractibility $\Diff(\TT^2)_{{\sf h}\Diff(\TT^2)}\simeq\ast$.
Through this contractibility, notice that the right square is a homotopy pullback square.  
Because the lefthand square is defined as a homotopy pullback square, it follows that the outer square is a homotopy pullback square, from which the result follows.
\end{proof}



Consider the composable sequence of group homomorphism 
The \textit{\textbf{braid group on 3 strands}} has a standard presentation, generated by adjacent twists:
\begin{equation}\label{e15}
\Braid_3 
~\cong~
\Bigl \lag~ \tau_{1,2}~,~ \tau_{2,3}~ \mid ~ \tau_{1,2}\tau_{2,3}\tau_{1,2} ~=~ \tau_{2,3} \tau_{1,2} \tau_{2,3} ~\Bigr\rag
~.
\end{equation}
Lemma~\ref{t95} below verifies that the assignments
\begin{equation}
\label{e1}
\Braid_3
\xra{~\tau_{1,2} \mapsto \begin{bmatrix} 1 & 1 \\ 0 & 1 \end{bmatrix}~,~\tau_{2,3}\mapsto \begin{bmatrix} 1 & 0 \\ -1 & 1 \end{bmatrix} ~}
\GL_2(\ZZ)
\end{equation}
define a homomorphism between groups.

\begin{remark}\label{r1}
The homomorphism $\Braid_3\xra{~(\ref{e1})~} \GL_2(\ZZ)$ factors through the subgroup $\SL_2(\ZZ)\subset \GL_2(\ZZ)$.  
This factorization is surjective, with central kernel freely generated by the element $(\tau_{1,2}\tau_{2,3})^6 \in \Braid_3$.
In other words, there is a central extension among groups:
\begin{equation}\label{e16}
1
\longrightarrow
\ZZ
\xra{~\bigl\lag (\tau_{1,2}\tau_{2,3})^6 \bigr\rag~}
\Braid_3
\xra{~(\ref{e1})~}
\SL_2(\ZZ)
\longrightarrow
1
~.
\end{equation}

\end{remark}
The composite homomorphism 
\begin{equation}
\label{e3}
\Braid_3
\xra{~(\ref{e1})~}
\GL_2(\ZZ)
\xra{~(\ref{e2})~}
\Aut_{\sf Groups}(\TT^2)
\end{equation}
defines the semi-direct product topological group:
\begin{equation}
\label{e4}
\TT^2 \underset{(\ref{e3})}\rtimes \Braid_3
~.
\end{equation}
There results a composite continuous group homomorphism
\begin{equation}\label{e8}
\TT^2 \underset{(\ref{e3})}\rtimes \Braid_3
\xra{~\id_{\TT^2} \rtimes (\ref{e1})~}
\TT^2 \underset{(\ref{e3})}\rtimes \GL_2(\ZZ)
\xra{~(\ref{e7})~}
\Diff(\TT^2)
~.
\end{equation}
We now state our main result; its proof occupies Section~\ref{s1}.  
\begin{theorem}\label{main.theorem}

There is a canonical homotopy-commutative diagram among topological spaces:
\begin{equation}\label{e14}
\xymatrix{
\TT^2 \underset{(\ref{e3})}\rtimes \Braid_3  \ar[rr]^-{(\ref{e8})} \ar[d]
&&
\Diff(\TT^2) \ar[d]^-{(\ref{e10})}
\\
\ast \ar[rr]^-{\lag \varphi_0\rag}
&&
\Fr(\TT^2)
.
}
\end{equation}
From the universal property of homotopy-pullbacks, this induces a continuous map 
\begin{equation}\label{e12}
\TT^2 \underset{(\ref{e3})}\rtimes \Braid_3
\longrightarrow
\Diff^{\fr}(\TT^2)
\end{equation}
Even more, the continuous map (\ref{e12}) is an equivalence between group-objects in $\Spaces$.
\end{theorem}



\begin{cor}\label{t2}
The \emph{framed mapping class group of $\TT^2$} is canonically isomorphic with the braid group on 3 strands:
\[
\Braid_3
\xra{~\cong~}
{\sf MCG}^{\fr}(\TT^2)
~:=~
\pi_0\Bigl(
\Diff^{\fr}(\TT^2)
\Bigr)
~.
\]
\end{cor}


\begin{proof}
We explain the sequence of homomorphisms among groups:
\begin{eqnarray}
\nonumber
\Braid_3 
&
\xra{\cong}
&
\pi_0(\Braid_3) 
\\
\nonumber
&
\xla{\cong} 
&
\pi_0(\TT^2) \underset{\pi_0(\ref{e3})}\rtimes \pi_0(\Braid_3)
\\
\nonumber
&
\xla{\cong}
&
\pi_0(\TT^2 \underset{(\ref{e3})}\rtimes \Braid_3)
\\
\nonumber
&
\xra{\cong}
&
\pi_0\bigl(\Diff^{\fr}(\TT^2) \bigr)
~=:~
{\sf MCG}^{\fr}(\TT^2)
~.
\end{eqnarray}
The first isomorphism follows from the topological space $\Braid_3$ being endowed with the discrete topology.
The second isomorphism follows from $\pi_0(\TT^2)$ being a singleton, which is so because $\TT^2$ is connected.  
The third isomorphism is $[(p, \omega)] \mapsto ([p], [\omega]).$ The fourth map is the induced map on homotopy of (\ref{e12}), and as (\ref{e12}) is a homotopy equivalence its induced map on homotopy groups is an isomorphism.
\end{proof}




\begin{remark}\label{r2}
Corollary~\ref{t2} is compatible with the standard isomorphism $\SL_2(\ZZ) \cong {\sf MCG}^{\sf or}(\TT^2) =: \pi_0\Bigl(
\Diff^{\sf or}(\TT^2)
\Bigr)$, in the sense that the diagram among groups
\[
\xymatrix{
\Braid_3 \ar[rrrr]^-{(\ref{e1})} \ar[d]_-{\cong}
&&
&&
\SL_2(\ZZ) \ar[d]^-{\cong}
\\
{\sf MCG}^{\fr}(\TT^2)  \ar[rrrr]^-{\rm forget~structure}
&&
&&
{\sf MCG}^{\sf or}(\TT^2)
}
\]
commutes. 
Remark~\ref{r1} thusly grants a central extension among groups:
\begin{equation}\label{e17}
1
\longrightarrow
\ZZ
\longrightarrow
{\sf MCG}^{\fr}(\TT^2)
\longrightarrow
{\sf MCG}^{\sf or}(\TT^2)
\longrightarrow
1
~.
\end{equation}
\end{remark}




\begin{remark}\label{r3}
Let $(\cV,\ot)$ be a presentably symmetric monoidal $\infty$-category.\footnote{
For example, for $\Bbbk$ a commutative ring, take $(\cV,\ot) = \bigl({\sf Ch}_\Bbbk[\{{\sf quasi\text{-}isos}\}^{-1}],\underset{\Bbbk}\otimes^{\LL}\bigr)$ is the $\infty$-categorical localization of chain complexes over $\Bbbk$ on quasi-isomorphisms, with derived tensor product over $\Bbbk$.
More generally, for $R$ a commutative ring spectrum, take $(\cV,\ot) := (\Mod_R,\underset{R}\wedge)$ to be $R$-module spectra and smash product over $R$.
}
Let $A \in \Alg\Bigl( \Alg(\cV) \bigr)$ an associative algebra among associative algebras in $\cV$.
Consider its iterated Hochschild homology
\[
\sHH\bigl( \sHH( A ) \bigr)
~\in~
\cV
~.
\]
Through Dunn's additivity~(\cite{dunn??}), this object is canonically identified as factorization homology (in the sense of~\cite{old.fact}) over the framed torus:
\[
\int_{\TT^2} A
~\simeq~
\sHH\bigl( \sHH( A ) \bigr)
~\in~
\cV
~.
\]
There results an action of the topological group $\Diff^{\fr}(\TT^2)$ on $\sHH\bigl( \sHH( A ) \bigr)$.
Through Theorem~\ref{main.theorem} is identical with a morphism between group-objects in spaces:
\[
\TT^2 \underset{(\ref{e3})}\rtimes \Braid_3
\longrightarrow
\Aut_\cV\Bigl( 
\sHH\bigl( \sHH( A ) \bigr)
\Bigr)
~.
\]
\end{remark}




\begin{remark}\label{r4}
We follow up on Remark~\ref{r3}.
Let $A \in \CAlg(\Spectra)$ be a commutative ring spectrum, which forgets as an algebra among ring spectra.  
It is known~(\cite{??}) that the iterated cyclotomic trace map
\[
\sK\bigl( \sK( A ) \bigr) 
\longrightarrow
{\sf TC}\bigl( {\sf TC}( A ) )
\]
is not locally constant (in the argument $A$).  
Well, there is a canonical factorization of the iterated trace, from iterated $\sK$-theory:
\[
\xymatrix{
\Obj\Bigl( \Perf\bigl( \sK(A) \bigr) \Bigr) \ar[rr] \ar[dd]
&&
\sHH\Bigl( \sHH\bigl( \Perf\bigl( \Perf(A) \bigr) \bigr) \Bigr)^{{\sf h} ( \TT^2 \underset{(\ref{e3})}\rtimes \Braid_3) }  \ar[d]
&&
\sHH\bigl( \sHH( A ) \bigr)^{{\sf h} ( \TT^2 \underset{(\ref{e3})}\rtimes \Braid_3) } \ar[d] \ar[ll]^-{\simeq}
\\
&&
\sHH\Bigl( \sHH\bigl( \Perf\bigl( \Perf(A) \bigr)^{{\sf h} \TT}  \bigr) \Bigr)^{{\sf h} \TT}  \ar[d]
&&
\sHH\bigl( \sHH( A )^{{\sf h} \TT} \bigr)^{{\sf h} \TT} \ar[d] \ar[ll]^-{\simeq}
\\
\sK\bigl( \sK( A ) \bigr) \ar[rr]^-{{\sf tr}({\sf tr})}
\ar@{-->}[urr]^-{{\sf tr}^-({\sf tr}^-)}
\ar@(-,u)@{-->}[uurr]
&&
\sHH\Bigl( \sHH\bigl( \Perf\bigl( \Perf(A) \bigr) \bigr) \Bigr)
&&
\sHH\bigl( \sHH( A ) \bigr) \ar[ll]^-{\simeq}_-{\rm Morita~invariance}
.
}
\]
This suggests that the group $\Braid_3$, as it acts on $\TT^2$, could play a fundamental role in improving the iterated cyclotomic trace map
\[
\sK\bigl( \sK( A ) \bigr) 
\longrightarrow
{\sf TC}\bigl( {\sf TC}( A ) \bigr)
\]
toward being closer to locally constant (in the argument $A$).
\end{remark}



\section{Proof of Theorem~\ref{main.theorem}}\label{s1}
Here we prove Theorem~\ref{main.theorem}. First consider the following maps:

\begin{equation}\label{e201}
\Fr(\mathbb{T}^{2}) \rightarrow \Map(\mathbb{T}^{2}, \GL_{2}(\mathbb{R}), \qquad \psi \mapsto (p \mapsto (\phi_{0}^{-1} \circ \psi)_{p}).
\end{equation}

\begin{equation}\label{e202}
\Map(\mathbb{T}^{2}, \GL_{2}(\mathbb{R}) \rightarrow \Map_\ast \Bigl( \bigl( 0 \in \TT^2 \bigr) , \bigl( \uno \in \GL_2(\RR) \bigr) \Bigr) \times \GL_2(\RR),
\end{equation}

$$(\mathbb{T}^{2} \xrightarrow{f} GL_{2}(\mathbb{R})) \mapsto ((\mathbb{T}^{2}, 0) \xrightarrow{(f(0))^{-1}f} (GL_{2}(\mathbb{R}), Id), f(\textbf{0})).$$ \newline
%Let ${\sf{Sk}}_{1}(\TT^{2}) = \TT^{1}_{1} \vee \TT^{1}_{2}$ be the 1-skeleton of $\TT^{2}$ and $[\TT^{1}_{i}]$ be the corresponding generators of $\pi_1\bigl( 0 \in \TT^2 \bigr).$ Then define the map:

%\begin{equation} \label{e203}
%{\sf Homo}\Bigl( \pi_1\bigl( 0 \in \TT^2 \bigr) , \pi_1\bigl( \uno \in \GL_2(\RR) \bigr) \Bigr) \longrightarrow \ZZ \times \ZZ
%\end{equation}
%$$h \mapsto (h([\TT^{1}_{1}]), h([\TT^{1}_{2}])).$$


\begin{lemma}\label{t5}
Each of the continuous maps
\begin{eqnarray}\label{e21}
\Fr(\TT^2) 
&
\xra{~(\ref{e201})~}
&
\Map\bigl( \TT^2 , \GL_2(\RR) \bigr)
\\
\nonumber
&
\xra{~(\ref{e202})~}
&
\Map_\ast \Bigl( \bigl( 0 \in \TT^2 \bigr) , \bigl( \uno \in \GL_2(\RR) \bigr) \Bigr) \times \GL_2(\RR) 
\\
\nonumber
&
\xra{~\pi_1 \times \id_{\GL_2(\RR)}~}
&
{\sf Homo}\Bigl( \pi_1\bigl( 0 \in \TT^2 \bigr) , \pi_1\bigl( \uno \in \GL_2(\RR) \bigr) \Bigr) \times \GL_2(\RR) 
\\
\nonumber
&
\xra{(\ref{e203}) \times \id_{\GL_{2}(\RR)}}
&
\ZZ\times \ZZ \times \GL_2(\RR)
\end{eqnarray}
is a homotopy equivalence. 
In particular, the composite continuous map is a homotopy equivalence.  
\end{lemma}


\begin{proof} 
The map ~(\ref{e201})~ is a homeomorphism. It's a continuous map with a continuous inverse sending $(\mathbb{T}^{2} \xrightarrow{f} GL_{2}(\mathbb{R}))$ to the framing  \[ \begin{tikzcd}
\mathbb{T}^{2} \times \mathbb{R}^{2} \arrow{r}{F} \arrow{d} & \mathbb{T}^{2} \times \mathbb{R}^{2} \arrow{r}{\phi_{0}} \arrow{d}{} & T\mathbb{T}^{2} \arrow{d} \\
\mathbb{T}^{2} \arrow{r}{id} & \mathbb{T}^{2}  \arrow{r}{id} & \mathbb{T}^{2}
\end{tikzcd} \]
where the map $F$ is defined by $F(p, v) = (p, f(p)v).$
The map ~(\ref{e202})~ is also a homeomorphism. It's a continuous map with continuous inverse $$ \Map_\ast \Bigl( \bigl( 0 \in \TT^2 \bigr) , \bigl( \uno \in \GL_2(\RR) \bigr) \Bigr) \times \GL_2(\RR) \rightarrow \Map(\mathbb{T}^{2}, \GL_{2}(\mathbb{R})), \qquad (f, M) \mapsto Mf.$$ 

In general, for two topological groups $G$ and $H,$ passing to fundamental groups induces a bijection between $[K(G, 1), K(H, 1)]  \xra{\pi_{1}} {\sf homo}(G, H)$ where $[-,-]$ indicates homotopy classes of based maps. This is the homotopy inverse to the classifying space functor. Therefore \begin{equation}\label{e55} \Map_{\ast}\Bigl(K(G, 1), K(H, 1) \Bigr) \xra{\pi_{1}} {\sf homo}(G, H) \end{equation} is a homotopy equivalence. Because $\SS^{1} = SO(2) \hookrightarrow GL_{2}(\RR)$ is a homotopy equivalence onto the connected component containing $\uno$, the map $$\Map_\ast \Bigl( \bigl( 0 \in \TT^2 \bigr) , \bigl( \uno \in \GL_2(\RR) \bigr) \Bigr) \xra{\pi_{1}} {\sf Homo}\Bigl( \pi_1\bigl( 0 \in \TT^2 \bigr) , \pi_1\bigl( \uno \in \GL_2(\RR) \bigr) \Bigr) $$ factors through 
$$\Map_\ast \Bigl( \bigl( 0 \in \TT^2 \bigr) , \bigl( \uno \in \GL_2(\RR) \bigr) \Bigr) \xra{} \Map_\ast \Bigl( \bigl( 0 \in \TT^2 \bigr) , \bigl( \uno \in SO_2(\RR) \bigr) \Bigr) \xra{\pi_{1}} {\sf Homo}\Bigl( \pi_1\bigl( 0 \in \TT^2 \bigr) , \pi_1\bigl( \uno \in \GL_2(\RR) \bigr) \Bigr),$$ where the first map post composes with the Gram-Schmidt process and the second map is the homotopy equivalence $(\ref{e55})$ for $G = \ZZ^{2}$ and $H = \ZZ.$


Finally, the last map to consider is the identification
\begin{equation} \label{e203}
{\sf Homo}\Bigl( \pi_1\bigl( 0 \in \TT^2 \bigr) , \pi_1\bigl( \uno \in \GL_2(\RR) \bigr) \Bigr) \longrightarrow \ZZ \times \ZZ
\end{equation}
$$h \mapsto (h([\TT^{1}_{1}]), h([\TT^{1}_{2}]))$$ for generators $[\TT^{1}_{i}]$ of $\pi_1\bigl( 0 \in \TT^2 \bigr).$ This is a bijection with inverse $$\ZZ \times \ZZ \rightarrow {\sf Homo}\Bigl( \pi_1\bigl( 0 \in \TT^2 \bigr) , \pi_1\bigl( \uno \in \GL_2(\RR) \bigr) \Bigr), \qquad (a, b) \mapsto (\ZZ^{2} \xra{h_{(a, b)}} \ZZ)$$ where $h_{(a, b)}(p, q) = ap + bq.$ So we have that all maps are homotopy equivalences and therefore the composite will be a homotopy equivalence as well. 
\end{proof}

The diagram among topological spaces
\begin{equation}\label{e22}
\xymatrix{
\TT^2 \underset{(\ref{e2})}\rtimes \GL_2(\ZZ) \ar[rr]^-{(\ref{e7})} 
&&
\Diff(\TT^2) \ar[d]^-{(\ref{e10})}
&&
\\
&&
\Fr(\TT^2) \ar[rr]^-{(\ref{e21})}
&&
\ZZ\times \ZZ \times \GL_2(\RR)
~
.
}
\end{equation} 
determines a span among homotopy fibers:

\begin{equation}\label{e20}
\xymatrix{
{\sf hofib}_{(\ref{e21}(\varphi_0)}\Bigl(
\TT^2 \underset{(\ref{e2})}\rtimes \GL_2(\ZZ)
\xra{~(\ref{e22})~}
\ZZ\times \ZZ \times \GL_2(\RR)
\Bigr) \ar[d]^-{}
\\
{\sf hofib}_{(\ref{e21}(\varphi_0)}\Bigl(
\Diff(\TT^2)
\xra{~(\ref{e21})~ \circ ~(\ref{e10})~}
\ZZ\times \ZZ \times \GL_2(\RR)
\Bigr)
\\
{\sf hofib}_{\varphi_0}\Bigl(
\Diff(\TT^2)
\xra{~(\ref{e10})~}
\Fr(\TT^2)
\Bigr)
~=:~
\Diff^{\fr}(\TT^2)
~
\ar[u]^-{}
}
\end{equation}



\begin{observation}\label{t6}
Because the horizontal maps in~(\ref{e22}) is an equivalence, by Lemma (\ref{t70}) below, each of the maps in the span among topological spaces~(\ref{e20}) is a homotopy equivalence. Note that $(\ref{e21})\varphi_0) = (0, 0, \uno) \in \ZZ \times \ZZ \times \GL_{2}(\RR).$
We will compute $${\sf hofib}_{(0, 0, \uno)}\Bigl(
\Diff(\TT^2)
\xra{~(\ref{e21})~ \circ ~(\ref{e10})~}
\ZZ\times \ZZ \times \GL_2(\RR)
\Bigr)$$ below, allowing us to determine the homotopy type of $\Diff^{\fr}(\TT^2).$
\end{observation}


\begin{lemma}\label{t4}
There is a commutative diagram among topological spaces
\begin{equation}\label{e19}
\xymatrix{
\TT^2 \underset{(\ref{e2})}\rtimes \GL_2(\ZZ) \ar[rr]^-{(\ref{e7})} \ar[d]_-{\pr}
&&
\Diff(\TT^2) \ar[d]^-{(\ref{e10})}
\\
\GL_2(\ZZ) \ar[d]_-{\rm inclusion}
&&
\Fr(\TT^2) \ar[d]^-{(\ref{e21})}
\\
\GL_2(\RR) \ar[rr]^-{A\mapsto (0,0,A)}
&&
\ZZ\times \ZZ \times \GL_2(\RR)
.
}
\end{equation}
\end{lemma}


\begin{proof}
We have that $(p, A) \in \TT^2 \underset{(\ref{e2})}\rtimes \GL_2(\ZZ)$ maps to the linear diffeomorphism $f :=(q \mapsto Aq + p)$ under (\ref{e7}). Then under the orbit map (\ref{e10}) this linear diffeomorphism is mapped to the framing \[ \xymatrix{
\TT^{2} \times \RR^{2} \ar[d] \ar[rr]^-{f^{-1} \times id} && \TT^{2} \times \RR^{2} \ar[rr]^-{\varphi_{0}} 
\ar[d] && {\sT}\TT^{2} \ar[rr]^-{\sD f} \ar[d] && {\sT}\TT^{2} \ar[d] \\
\TT^{2} \ar[rr]^-{f^{-1}} && \TT^{2}  \ar[rr]^-{id} && \TT^{2} \ar[rr]^-{f} && \TT^{2}.}\] Now under the map (\ref{e21}) we see that this framing is first sent to the map from $\TT^{2}$ to $ GL_{2}(\mathbb{Z}):$ \begin{equation}\label{e298} q \mapsto \Bigr( \{q\} \times \RR^{2} \xra{f^{-1} \times \id} \{f^{-1}(q)\} \times \RR^{2} \xra{\varphi_{0}} {\sT}_{f^{-1}(q)} \TT^{2} \xra{{\sD} f} {\sT}_{q} \TT^{2} \xra{\varphi^{-1}_{0}} \{q\} \times \RR^{2} \Bigl). \end{equation} 
Recall though that the linear diffeomorphism $f$ is actually the composite $$\TT^{2} \xra{\quot^{-1}} \RR^{2}  \xra{A}  \RR^{2} \xra{\quot} \TT^{2} \xra{\trans_{p}} \TT^{2}$$ where we operate in a small neighborhood so ${\sf \quot^{-1}}$ is well defined.
Then by the chain rule ${{\sD} f}$ is the composite map
\[ \xymatrix{
{\sT} \TT^{2} \ar[d] \ar[rr]^-{{\sD} \quot^{-1}} && \sT \RR^{2} \ar[rr]^-{\sD A} 
\ar[d] && \sT \RR^{2} \ar[rr]^-{\sD \quot} \ar[d] && {\sT}\TT^{2} \ar[d] \ar[rr]^-{\sD \trans_{p}} && \ar[d]{\sT} \TT^{2} \\
\TT^{2} \ar[rr]^-{\quot^{-1}} && \RR^{2}  \ar[rr]^-{A} && \RR^{2} \ar[rr]^-{\quot} && \TT^{2} \ar[rr]^-{\trans_{p}} &&  \TT^{2}.} \] 
The map on the fibers of $\sD A$ is precisely $A,$ and the map on the fibers of $\sD \trans_{p}$ is the identity. So looking at (\ref{e298}) more closely, we see that for $(q, v) \in \{q\} \times \RR^{2}$ the leftmost map is the identity on $v$, and then is followed by: 
$$v \xra{\varphi_{0}|_{f^{-1}(q)}} \sD_{0}(\trans_{f^{-1}(q)} \circ \quot) (v) = (\sD_{0} \trans_{f^{-1}(q)} \circ \sD_{0} \quot) (v) \xra{\sD f|_{f^{-1}(q)}}$$  
$$\Big(\sD_{\quot(A(\widetilde{f^{-1}(q)}))} \trans_{p} \circ \sD_{A\widetilde{f^{-1}(q)}} \quot \circ \sD_{\widetilde{f^{-1}(q)}} A \circ \sD_{f^{-1}(q)} \quot^{-1} \circ \sD_{0} \trans_{f^{-1}(q)} \circ \sD_{0} \quot \Big) (v)$$
$$= \Big(\sD_{\quot(A(\widetilde{f^{-1}(q)}))} \trans_{p} \circ \sD_{A\widetilde{f^{-1}(q)}} \quot \circ \sD_{\widetilde{f^{-1}(q)}} A \Big) (v) \xra{\varphi_{0}^{-1}|_{q}}$$
$$\Big(\sD_{0}\quot^{-1} \circ \sD_{0}{\trans_{q}^{-1}} \circ \sD_{\quot(A(\widetilde{f^{-1}(q)}))} \trans_{p} \circ \sD_{A\widetilde{f^{-1}(q)}} \quot \circ \sD_{\widetilde{f^{-1}(q)}} A \Big) (v)=  (\sD_{\widetilde{f^{-1}(q)}}A)(v) = Av.$$ 
So we see under (\ref{e298}) the framing is sent to the constant map $$\TT^{2} \rightarrow \GL_{2}(\RR), \qquad q \mapsto A.$$
This constant map is then sent to $({\sf constant}_{\uno}, A) \in \Map_\ast \Bigl( \bigl( 0 \in \TT^2 \bigr) , \bigl( \uno \in \GL_2(\RR) \bigr) \Bigr) \times \GL_2(\RR)$ by $(\ref{e202}).$ As ${\sf constant}_{\uno}$ is null homotopic, its induced map on fundamental groups is the trivial map $\ZZ^{2} \xra{{\sf constant}_{0}} \ZZ$ which is $(0,0)$ as an element of $\ZZ^{2} \simeq {\sf Homo}\Bigl( \pi_1\bigl( 0 \in \TT^2 \bigr) , \pi_1\bigl( \uno \in \GL_2(\RR) \bigr) \Bigr).$
\newline \newline So we see that $(p, A) \in \TT^2 \underset{(\ref{e2})}\rtimes \GL_2(\ZZ)$ is mapped to $(0, 0, A)$ by (\ref{e7}) followed by (\ref{e10}) followed by (\ref{e21}) and therefore the diagram commutes.
\end{proof} 



\begin{prop}\label{t8}
There is a canonical null-homotopy of the composite homomorphism $\Braid_3 \xra{(\ref{e1})} \GL_2(\ZZ) \xra{\rm inclusion} \GL_2(\RR)$.  
The resulting homotopy-commutative diagram 
\begin{equation}\label{e24}
\xymatrix{
\Braid_3 \ar[rr]^-{(\ref{e1})} \ar[d]
&&
\GL_2(\ZZ) \ar[d]^-{\rm inclusion}
\\
\ast \ar[rr]^-{\lag \uno \rag}
&&
\GL_2(\RR)
}
\end{equation}
witnesses a homotopy-pullback.  
\end{prop}
\begin{proof}


We will first show that the following diagram commutes up to homotopy 
$$
\xymatrix{
\pi_{0}(\Omega \bigr( \SL_{2}(\RR)/\SL_{2}(\ZZ) \bigl)) && \ar[ll] 
\Omega \bigr( \SL_{2}(\RR)/\SL_{2}(\ZZ) \bigl) \ar[rr] \ar[d] && 
\SL_{2}(\ZZ) \ar[rr] \ar[d]&& 
\GL_{2}(\ZZ) \ar[d]
\\  && \ast \ar[rr]^-{\langle \uno \rangle} && \SL_{2}(\RR) \ar[rr] && \GL_{2}(\RR).
}$$

The right square is a homotopy pullback because $\SL_{2}(\RR)/\SL_{2}(\ZZ) \cong \GL_{2}(\RR)/\GL_{2}(\ZZ).$ The left square is a homotopy fiber because $\SL_{2}(\ZZ) \subset \SL_{2}(\RR),$ and for any closed subgroup $H \subset G,$ the homotopy fiber ${\sf hofib}_{e}(H \hookrightarrow G) \simeq \Omega(G/H).$


Now since $\pi_{n}(\SL_{2}(\ZZ)) = \pi_{n}(\SL_{2}(\RR)) = 0$ for $n > 1,$ the long exact sequence of homotopy groups associated to the fibration
$$
\xymatrix{
\Omega \bigr( \SL_{2}(\RR)/\SL_{2}(\ZZ) \bigl) \ar[rr]  && \SL_{2}(\ZZ)  \ar[d] \\
  && \SL_{2}(\RR), 
}
$$

$$ \hdots \rightarrow \pi_{n+1}(\SL_{2}(\RR)) \rightarrow \pi_{n}(\Omega \bigr( \SL_{2}(\RR)/\SL_{2}(\ZZ) \bigl)) \rightarrow \pi_{n}(\SL_{2}(\ZZ)) \rightarrow \pi_{n}(\SL_{2}(\RR)) \rightarrow \hdots,$$ implies that $\pi_{n}(\Omega \bigr( \SL_{2}(\RR)/\SL_{2}(\ZZ) \bigl)) = 0$ for all $n > 1.$ Therefore, each connected component of $\Omega \bigr( \SL_{2}(\RR)/\SL_{2}(\ZZ) \bigl)$ is contractible, and this contraction gives the homotopy equivalence $$\pi_{0}(\Omega \bigr( \SL_{2}(\RR)/\SL_{2}(\ZZ) \bigl)) \longleftarrow \Omega \bigr(\SL_{2}(\RR)/\SL_{2}(\ZZ) \bigl).$$

It is shown in \cite{mil} that $\SL_2(\RR) / \SL_2(\ZZ)$ is homeomorphic to the compliment of the trefoil, and it follows that $\pi_1( \SL_2(\RR) / \SL_2(\ZZ)) := \pi_{0}(\Omega \bigr( \SL_{2}(\RR)/\SL_{2}(\ZZ) \bigl)) \cong \Braid_{3}.$ Therefore we have that $${\sf hofib}_{\uno}(\GL_{2}(\ZZ) \rightarrow \GL_{2}(\RR)) \simeq \Omega \bigr( \SL_{2}(\RR)/\SL_{2}(\ZZ) \bigl) \simeq \Braid_{3}$$ and the diagram $(\ref{e24})$ is indeed a homotopy-pullback.
\end{proof}



\begin{lemma}\label{t7}
There is a homotopy-commutative diagram among topological spaces, in which each square is a homotpy-pullback:
\begin{equation}\label{e23}
\xymatrix{
\TT^2 
\underset{(\ref{e3})}\rtimes
\Braid_3
\ar[rr]^-{\id_{\TT^2}\rtimes ~(\ref{e1})~} \ar[d]_-{\pr}
&&
\TT^2 \underset{(\ref{e2})}\rtimes \GL_2(\ZZ) \ar[d]^-{\pr}
\\
\Braid_3 \ar[d] \ar[rr]^-{~(\ref{e1})~}
&&
\GL_2(\ZZ) \ar[d]^-{\rm inclusion}
\\
\ast \ar[rr]^-{\lag \uno \rag } \ar[d]
&&
\GL_2(\RR) \ar[d]^-{A\mapsto (0,0,A)}
\\
\ast \ar[rr]^-{\lag (0,0,\uno)\rag }
&&
\ZZ\times \ZZ \times \GL_2(\RR)
.
}
\end{equation}
\end{lemma}



\begin{proof}
The lower square is a homotopy-pullback because the bottom right vertical map is the inclusion of the path-component containing the image of the bottom horizontal map.
Proposition~\ref{t8} gives that the middle square is a homotopy pullback.
The upper square is a homotopy pullback because the induced map between fibers is non-canonically homeomorphic with the identity map $\TT^2 \xra{\id_{\TT^2}}\TT^2$, which is in particular a homotopy equivalence.  
\end{proof}





\begin{proof}[Proof of Theorem~\ref{main.theorem}]
For the composite map
\begin{equation}\label{e111}
\TT^{2} \rtimes \GL_{2}(\ZZ) \xra{\sf pr} \GL_{2}(\ZZ) \xra{\sf inclusion} \GL_{2}(\RR) \xra{A \mapsto (0,0,A)} \ZZ \times \ZZ \times \GL_{2}(\RR),
\end{equation}
we have the diagram
$$
\xymatrix{
\TT^2 
\underset{(\ref{e3})}\rtimes
\Braid_3
\ar[rr]^-{\id_{\TT^2}\rtimes ~(\ref{e1})~} \ar[d]
&&
\TT^2 \underset{(\ref{e2})}\rtimes \GL_2(\ZZ) \ar[d]^-{(\ref{e111})} \ar[rr]^-{(\ref{e7})}
&& 
\Diff(\TT^{2}) \ar[d]^-{(\ref{e10})}
\\
\ast \ar[rr]^-{(0, 0, \uno)}
&&
\ZZ \times \ZZ \times \GL_2(\RR) 
&&
\Fr(\TT^{2}) \ar[ll]^-{(\ref{e21})}
}$$
where the left square is the homotopy pullback from Lemma (\ref{t7}) and the right square is the commutative diagram from Lemma (\ref{t4}). As both (\ref{e7}) and (\ref{e21}) are homotopy equivalences, we have that the following diagram is a homotopy pullback:
$$
\xymatrix{
\TT^2 \underset{(\ref{e3})}\rtimes \Braid_3  \ar[rr]^-{(\ref{e8})} \ar[d]
&&
\Diff(\TT^2) \ar[d]^-{(\ref{e21}) \circ (\ref{e10})}
\\
\ast \ar[rr]^-{(0, 0, \uno)}
&&
\ZZ \times \ZZ \times \GL_2(\RR) 
.
}
$$
In particular we have that ${\sf hofib}_{(0, 0, \uno)}(\Diff(\TT^{2})\xra{(\ref{e21}) \circ (\ref{e10})} \ZZ \times \ZZ \times \GL_{2}(\RR)) \simeq \TT^2 \underset{(\ref{e3})}\rtimes \Braid_3,$ and then Observation (\ref{t6}) gives us an induced homotopy equivalence from ${\sf hofib}_{(0, 0, \uno)}(\Diff(\TT^{2})$ to $\Diff^{\sf fr}(\TT^{2}).$


\end{proof}



\section*{Miscellaneous Lemmas}

\begin{lemma} \label{t95}
The assignments
\[
\tau_{1,2} 
\mapsto
\begin{bmatrix} 1 & 1 \\ 0 & 1 \end{bmatrix}
~,\qquad
\tau_{2,3}
\mapsto 
\begin{bmatrix} 1 & 0 \\ -1 & 1 \end{bmatrix}
\]
uniquely extend as a group homomorphism
\[
\Braid_3 \xra{~(\ref{e1})~} \GL_2(\ZZ)
~.
\]
\end{lemma}
\begin{proof}
The given assignments uniquely define a homomorphism from the free group
$\Phi \colon \lag \tau_{1,2} , \tau_{2,3} \rag \to \GL_2(\ZZ)$.  
Given the presentation~(\ref{e15}) of the braid group, the existence and uniqueness of such a sought extension follows upon verifying an equality in $\GL_2(\ZZ)$:
$$\Phi(\tau_{1,2}\tau_{2,3}\tau_{1,2}) := \Phi(\tau_{1,2})\Phi(\tau_{2,3})\Phi(\tau_{1, 2}) = \Phi(\tau_{2,3}) \Phi(\tau_{1,2}) \Phi(\tau_{2,3}) =: \Phi(\tau_{2,3} \tau_{1,2} \tau_{2,3}).$$ Indeed  $$\Phi(\tau_{1,2})\Phi(\tau_{2,3})\Phi(\tau_{1, 2}) = \begin{bmatrix} 1 & 1 \\ 0 & 1 \end{bmatrix}   \begin{bmatrix} 1 & 0 \\ -1 & 1 \end{bmatrix}   \begin{bmatrix} 1 & 1 \\ 0 & 1 \end{bmatrix} =  \begin{bmatrix} 1 & 1 \\ 0 & 1 \end{bmatrix}  \begin{bmatrix} 1 & 1 \\ -1 & 0 \end{bmatrix} =  \begin{bmatrix} 0 & 1 \\ -1 & 0 \end{bmatrix}  $$ and 
$$\Phi(\tau_{2,3}) \Phi(\tau_{1,2}) \Phi(\tau_{2,3})  =  \begin{bmatrix} 1 & 0 \\ -1 & 1 \end{bmatrix}   \begin{bmatrix} 1 & 1 \\ 0 & 1 \end{bmatrix}   \begin{bmatrix} 1 & 0 \\ -1 & 1 \end{bmatrix} =  \begin{bmatrix} 1 & 0 \\ -1 & 1 \end{bmatrix}   \begin{bmatrix} 0 & 1 \\ -1 & 1 \end{bmatrix}  =  \begin{bmatrix} 0 & 1 \\ -1 & 0 \end{bmatrix}.$$
\end{proof}


\begin{lemma}\label{t70}
The diagram among topological spaces
$$\xymatrix{
A \ar[rr]^-{f} && B \ar[d]^-{g}\\
  &&  C 
}$$
induces a map $${\sf hofib}_{c_{0}}(A \xra{g \circ f} C) \longrightarrow {\sf hofib}_{c_{0}}(B \xra{g} C).$$ Furthermore, if $f$ is a homotopy equivalence then the induced map between homotopy fibers is a homotopy equivalence. Similarly, the diagram among topological spaces
$$\xymatrix{
B \ar[d]^-{g} && \\
C \ar[rr]^-{h} &&  D 
}$$
where $h(c_{0}) =  d_{0}$ induces a map $${\sf hofib}_{c_{0}}(B \xra{g} C) \longrightarrow {\sf hofib}_{d_{0}}(B \xra{h \circ g} D).$$ Furthermore, if $h$ is a homotopy equivalence then the induced map between homotopy fibers is a homotopy equivalence.
\end{lemma}
\begin{proof}
The natural map, $$A \times C^{I} \xra{f \times \id} B \times C^{I},$$ respects the homotopy fiber conditions. So the induced map $$A \times C^{I} \supset {\sf hofib}_{c_{0}}(A \xra{g \circ f} C) \ni (a , \gamma) \mapsto (f(a) , \gamma) \in {\sf hofib}_{c_{0}}(B \xra{g} C) \subset B \times C^{I}$$ is well defined. As $\id$ is a homotopy equivalence, if $f$ is a homotopy equivalence we have that the induced map is a homotopy equivalence. Similarly, we may consider the natural map $$B \times C^{I} \xra{\id \times h} B \times D^{I}.$$ Then for $h(c_{0}) = d_{0}$ we have that the induced map $${\sf hofib}_{c_{0}}(B \xra{g} C) \ni (b, \gamma) \mapsto (b, h \circ \gamma) \in {\sf hofib}_{d_{0}}(B \xra{h \circ g} D)$$ is well defined since for any $\gamma \in C^{I}$ where $\gamma(0) = c_{0}$ and $\gamma(1) = g(b),$ the path $h \circ \gamma$ will have $(h \circ \gamma)(0) = h(\gamma(0)) = d_{0}$ and $(h \circ \gamma)(1) = h(\gamma(1))= h(g(b)).$ Again, as we have a product map with $\id$ and $h$, if $h$ is a homotopy equivalence then the induced maps between homotopy fibers will be a homotopy equivalence as well.
\end{proof}


\begin{thebibliography}{9}
%\bibitem{alexander} J. W. Alexander, On the deformation of an n-cell, Proc. Nat. Acad. Sei. U.S.A. vol. 9 (1923) pp. 406-407.
\bibitem{ee} Earle, C. J.; Eells, J. The diffeomorphism group of a compact Riemann surface. Bull. Amer. Math. Soc. 73 (1967), no. 4, 557--559. https://projecteuclid.org/euclid.bams/1183528956
\bibitem{mil} J. Milnor, Introduction to Algebraic K-theory, Annals of Math. Study 72,
Princeton Univ. Press, 1971.
\bibitem{rolf} Rolfsen, D.; Knots and Links. Berkeley, Publish or Perish 1976
\end{thebibliography}


\end{document}